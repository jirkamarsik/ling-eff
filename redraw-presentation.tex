\documentclass{beamer}

\usepackage[utf8]{inputenc}
\usepackage{pgfpages}
\usepackage{stmaryrd}
\usepackage{amsmath}
\usepackage{graphicx}
\usepackage{tikz}
\usepackage{xcolor}

\usepackage[export]{adjustbox}
\setbeamercovered{transparent}

\hypersetup{pdfstartview={Fit}}
\def\limp {\mathbin{{-}\mkern-3.5mu{\circ}}}



\setbeamertemplate{navigation symbols}{}
\setbeamertemplate{footline}
  {\hfill {\normalsize \insertframenumber{}/\inserttotalframenumber{}}}

\definecolor{mygreen}{RGB}{30, 176, 37}

\definecolor{myColor}{rgb}{0.50,0.70,0.10}%
\definecolor{myColor2}{rgb}{0.50,0.20,0.80}%
%\definecolor{myColor}{rgb}{1.00,0.00,0.10}%
%\definecolor{myGreen}{rgb}{0.00,0.60,0.00}%
%\definecolor{myOrange}{rgb}{1.00,0.40,0.00}%
%\definecolor{myOtherOrange}{rgb}{0.90,0.50,0.00}%
%\definecolor{myBlue}{rgb}{0.20,0.20,0.70}%
%\definecolor{myGray}{rgb}{0.84,0.84,0.94}%
%\newcommand{\green}[1]{\textcolor{myGreen}{#1}}
%\newcommand{\blue}[1]{\textcolor{myBlue}{#1}} 
%\newcommand{\red}[1]{\textcolor{red}{#1}} 
%\newcommand{\marron}[1]{\textcolor{brown!65!black!80!white}{#1}}
%\newcommand{\myorange}[1]{\textcolor{myOrange}{#1}}
%\newcommand{\myotherorange}[1]{\textcolor{myOtherOrange}{#1}}
\newcommand{\mycolor}[1]{\textcolor{myColor}{#1}}
\newcommand{\mycolord}[1]{\textcolor{myColor2}{#1}}


  
\newcommand{\hastype}{\mathop{:}}

\newcommand{\dand}{\mathbin{\overline{\land}}}
\newcommand{\dnot}{\mathop{\overline{\lnot}}}
\newcommand{\dimpl}{\mathbin{\overline{\to}}}
\newcommand{\dexists}{\mathop{\overline{\exists}}}
\newcommand{\dforall}{\mathop{\overline{\forall}}}

\newcommand{\hsbind}{\mathbin{\texttt{>>=}}}
\newcommand{\hsrevbind}{\mathbin{\texttt{=<<}}}
\newcommand{\hsseq}{\mathbin{\texttt{>>}}}
\newcommand{\occons}{\mathbin{::}}

\newcommand{\statecps}[3]{\llbracket #3 \rrbracket^{#2}_{#1}}
\newcommand{\cps}[2]{\llbracket #2 \rrbracket^{#1}}

\newcommand{\sem}[1]{\llbracket #1 \rrbracket}
\newcommand{\intens}[1]{\overline{#1}}

\newcommand{\obj}[1]{\text{Obj}(#1)}
\newcommand{\inl}[1]{\text{inl}(#1)}
\newcommand{\inr}[1]{\text{inr}(#1)}
\newcommand{\id}[1]{\text{id}_{#1}}

\newcommand{\keyword}[1]{\texttt{#1}}
\newcommand{\effect}[1]{\textbf{#1}}
\newcommand{\semdom}[1]{\textbf{#1}}
\newcommand{\handle}[2]{\keyword{with}\ #1\ \keyword{handle}\ #2}


\newcommand{\highlight}[2]{\colorbox{#1}{$\displaystyle #2$}}


%% \setbeamercolor{block title}{fg=ngreen,bg=white} % Colors of the block titles
%% \setbeamercolor{block body}{fg=black,bg=white} % Colors of the body of blocks
%% \setbeamercolor{block alerted title}{fg=white,bg=dblue!70} % Colors of the highlighted block titles
%% \setbeamercolor{block alerted body}{fg=black,bg=dblue!10} % Colors of the body of highlighted blocks
%% % Many more colors are available for use in beamerthemeconfposter.sty












\title{Pragmatic Side Effects}
\author{Jirka Maršík and Maxime Amblard}
\institute[LORIA, Université de Lorraine, Inria]
{
  LORIA, UMR 7503, Université de Lorraine, CNRS, Inria, Campus Scientifique, \\
  F-54506 Vand\oe uvre-lès-Nancy, France
}
\date{20 March, 2015}







\begin{document}

\begin{frame}
  \titlepage
\end{frame}


\begin{frame}{Our Setting}
  Context:
  \begin{itemize}
  \item Montague semantics, using the $\lambda$ calculus
  \end{itemize}
  \vfill

  Objective:
  \begin{itemize}
  \item Increase the empirical coverage
  \end{itemize}
  \vfill

  Challenge:
  \begin{itemize}
  \item multiple sentences
    \begin{itemize}
    \item discourse phenomena
    \item pragmatics
    \end{itemize}
  \end{itemize}
\end{frame}


\begin{frame}{Example of Pragmasemantics}
  \framesubtitle{de Groote -- Type-Theoretic Dynamic Logic}
  \begin{columns}
    \begin{column}{0.5\textwidth}
      \begin{block}{Montague}
        \begin{align*}
          \sem{s} &= o \\
          \sem{n} &= \iota \to \sem{s} \\
          \sem{np} &= (\iota \to \sem{s}) \to \sem{s}
        \end{align*}
      \end{block}
    \end{column}
    \begin{column}{0.5\textwidth}
      \begin{block}{de Groote}
         \begin{align*}
          \sem{s} &= \gamma \to (\gamma \to o) \to o \\
          \sem{n} &= \iota \to \sem{s} \\
          \sem{np} &= (\iota \to \sem{s}) \to \sem{s}
        \end{align*}
      \end{block}
    \end{column}
  \end{columns}
  \vfill
  \pause
  $$
  \sem{\textit{He bought a car}} = \lambda e \phi.\ \exists
  x. car(x)\ \land\ bought(\texttt{sel}_{he}(e), x)\ \land\ \phi (x \occons e)
  $$
\end{frame}


\begin{frame}
  \begin{center}
    \vfill

    {\huge Drawing Inspiration from Programming Languages}

    \vfill

    \begin{quotation}
      There is in my opinion no important theoretical difference between
      natural languages and the programming languages of computer
      scientists.
    \end{quotation}
  \end{center}
\end{frame}


\begin{frame}{Side Effects in Programming Languages}
  \vfill
  Account for:
  \pause
  \vfill
  \begin{itemize}
  \item a program's interaction with the world of its users
    \begin{itemize}
    \item e.g., makings sounds, printing documents, moving robotic limbs\ldots
    \end{itemize}
  \end{itemize}
  \pause
  \vfill
  \begin{itemize}
  \item non-local interactions between parts of a program
    \begin{itemize}
    \item e.g., writing to and reading from variables, throwing and
      catching exceptions\ldots
    \end{itemize}
  \end{itemize}
  \vfill
  \pause
  \alert{Side effects align with pragmatics in their purpose.}
\end{frame}

\begin{frame}{Type Raising}
  Side effects and pragmatics align also in their theories.
  \pause
  \vfill
  Most famous example: Montague's type raising
  \begin{itemize}
  \item from entities to generalized quantifiers
  \item i.e., from $\iota$ to $(\iota \to o) \to o$
  \item e.g., $john$ becomes $\lambda P. P\ john$
  \end{itemize}
  \vfill
  \pause
  In computer science, discovered as continuations
  \begin{itemize}
  \item raising $\alpha$ to $(\alpha \to \omega) \to \omega$
  \item e.g., applying a function $f$ to two arguments $S$ and $O$ in
    continuation-passing style
    $$ \lambda P. S (\lambda x. O (\lambda y. P\ (f\ x\ y))) $$
  \end{itemize}
\end{frame}

\begin{frame}{Generalizing Denotations}
  ``Upgrading'' the types of denotations in order to keep a compositional
  semantics seems like a common strategy.
  \vfill
  \begin{tabular}{llr}
    Natural Languages & Prog. Languages & Type $\alpha$ becomes \\
    \hline
    Quantification & Control &
    $(\alpha \to \omega) \to \omega$ \\
    Anaphora & State &
    $\gamma \to \alpha \times \gamma$ \\
    Intensionality & Environment &
    $\delta \to \alpha$ \\
    Presuppositions & Exceptions &
    $\alpha \oplus \chi$ \\
    Questions & Non-determinism &
    $\alpha \to o$ \\
    Focus & &
    $\alpha \times (\alpha \to o)$ \\
    Expressives & Output &
    $\alpha \times \epsilon$ \\
    Prob. semantics & Prob. programming &
    $[\mathbb{R} \times \alpha]$ \\
  \end{tabular}
\end{frame}

\begin{frame}{How to Avoid Changing Denotations?}
  Different pragmasemantic phenomena, all in one theory \\
    $\rightarrow$ more and more elaborate types
  \vfill
  \pause
  We often have to change our minds on what is meaning
  \begin{itemize}
  \item old denotations $\rightarrow$ outdated
  \item denotations from other strands of work $\rightarrow$ incompatible
  \end{itemize}
  \pause
  \vfill
  Some solutions to this problem exist already in computer science.
\end{frame}

\newcommand{\includepicture}[1]{
    \includegraphics[scale=0.4]{dias/#1.eps}
}

\begin{frame}{Example from Computer Science}
  \begin{adjustbox}{center}
   \begin{tabular}{|c|c|c|} \hline
    \textbf{3} & \textbf{x + 3} & \textbf{print("hello")} \\ \hline
    $3$ & & \\ \hline
    \visible<2->{$\lambda s. \left<3, s\right>$} & \visible<2->{$\lambda s. \left<s("x") + 3, s\right>$} & \\ \hline
    \visible<3->{$\lambda s. \left<3, s, ""\right>$} & \visible<3->{$\lambda s. \left<s("x") + 3, s, ""\right>$} & \visible<3->{$\lambda s. \left<(), s, "hello"\right>$} \\ \hline
    \visible<4->{\includepicture{3}} & & \\ \hline
    \visible<5->{\includepicture{3}} & \visible<5->{\includepicture{x+3}} & \\ \hline
    \visible<6->{\includepicture{3}} & \visible<6->{\includepicture{x+3}} & \visible<6->{\includepicture{print}} \\ \hline
  \end{tabular}
  \end{adjustbox}
\end{frame}

\begin{frame}{Example from Linguistics}
  \begin{adjustbox}{center}
   \begin{tabular}{|c|c|c|} \hline
    \textbf{John} & \textbf{every boy} & \textbf{she} \\ \hline
    $j$ & & \\ \hline
    \visible<2->{$\lambda P. P j$} & \visible<2->{$\lambda P. \forall x. (boy\ x) \rightarrow (P\ x)$} & \\ \hline
    \visible<3->{$\lambda P e \phi. P j (j :: e) \phi$} &
    \visible<3->{\begin{tabular}{@{}c@{}}$\lambda P e \phi. [\forall x. (boy\ x) \rightarrow$ \\
      $P\ x\ (x :: e)\ (\lambda e'. \top)] \land \phi\ e$ \end{tabular}} &
    \visible<3->{$\lambda P e \phi. P (\texttt{sel}_{she}(e)) e \phi$} \\ \hline
    \visible<4->{\includepicture{john}} & & \\ \hline
    \visible<5->{\includepicture{john}} & \visible<5->{\includepicture{everyboy_static}} & \\ \hline
    \visible<6->{\includepicture{john_push}} & \visible<6->{\includepicture{everyboy}} & \visible<6->{\includepicture{she}} \\ \hline
  \end{tabular}
  \end{adjustbox}
\end{frame}

\begin{frame}{Reaping the Benefits of Stability}
  Consider the semantics of a relational noun like \textit{mother} in the
  construction \textit{the mother of X}.

  \pause

  \begin{align*}
    \sem{\textit{the mother of}} &= \lambda x.\ mother(x) \\
    \pause
    \sem{\textit{the mother of}} &= \lambda X P.\ X\ (\lambda
    x.\ P\ (mother(x))) \\
    \pause
    \sem{\textit{the mother of}} &= \lambda X P e \phi.\ X\ (\lambda x e'
    \phi'.\ P\ (mother(x))\ e'\ \phi')\ e\ \phi \\
  \end{align*}

  \pause

  Its denotation changes even though the meaning stays morally the same.
\end{frame}



\newcommand{\includepicturescale}[2]{
    \includegraphics[scale=#2]{dias/#1.eps}
}

\newcommand{\includepicturem}[1]{
    \includegraphics[scale=0.48]{dias/#1.eps}
}




\begin{frame}{Reaping the Benefits of Stability}
How does it work in our system?
\pause
$$
\sem{\textit{the mother of}} = \lambda x.\ mother(x)
$$
\vfill
\pause
\begin{adjustbox}{center}
  \begin{tabular}{ccc}
    \visible<3->{\includepicturem{john_push}} & \visible<4->{\includepicturem{everyboy}} & \visible<5->{\includepicturem{she}} \\
    & & \\
    $\visible<3->{\Downarrow}$ & $\visible<4->{\Downarrow}$ & $\visible<5->{\Downarrow}$ \\
    & & \\
    \visible<3->{\includepicturem{john_mother}} & \visible<4->{\includepicturem{everyboy_mother}} & \visible<5->{\includepicturem{she_mother}} \\
  \end{tabular}
\end{adjustbox}
\end{frame}

\begin{frame}{Reaping the Benefits of Stability}
  Our meaning for \textit{the mother of X} is agnostic about its
  argument. It works with simple, quantificational or dynamic meanings of
  \textit{X}.
  \vfill
  \pause
  This also holds for more involved meanings of the relational noun.
  \vfill
  \only<1-2>{\vspace{58mm}}
  \only<3>{
  \begin{itemize}
  \item dynamic \textit{mother}
   $$
    \sem{\textit{the mother of}} = \lambda x.\ \begin{array}{l}\includepicturescale{dynamic_mother}{0.45}\end{array}
   $$
  \end{itemize}}
  \only<4>{
  \begin{itemize}
  \item presuppositional \textit{mother}
  $$
    \sem{\textit{the mother of}} = \lambda x.\ \begin{array}{l}\includepicturescale{presuppositional_mother}{0.45}\end{array}
  $$
  \end{itemize}
  \vspace{13mm}
  }
\end{frame}

\begin{frame}{Algebraic Effects\ldots}

  We have been using a framework developed in PL research.

  \vfill
  
  In it:
  \begin{itemize}
  \item interacting with the context $=$ throwing an exception
  \item the exception contains a response for every possible outcome of the
    operation
  \end{itemize}
 
  \pause
  \vfill
  
  \begin{columns}
    \begin{column}{0.5\textwidth}
   Denotations are:
  \begin{itemize}
  \item algebraic expressions (drawn as trees)
  \item \textcolor{mygreen}{generators} $=$ values
  \item \textcolor{red}{operators} $=$ possible interactions with the context
  \item arity $=$ the number of possible outcomes
  \item type $= \mathcal{F}_{\textcolor{red}{\Sigma}}(\textcolor{mygreen}{\tau})$
  \end{itemize}
    \end{column}
    \begin{column}{0.5\textwidth}
      \only<2>{\includepicturescale{example}{0.4}}
    \end{column}
  \end{columns}
\end{frame}

\begin{frame}{\ldots and Handlers}
  Handlers give scope and interpretation to (some of) the effects in a
  computation.

  \begin{itemize}
  \item Practically, they are like exception handlers in programming languages.
  \item Technically, they are catamorphisms (folds) on the algebra of effects.
  \end{itemize}
  

  \pause
  \vfill 

  Examples:
  \begin{itemize}
  \item a tensed verb delimits quantification, creating a scope island
  \item logical negation blocks referent accessibility (as in DRT or TTDL)
  \item the common ground accomodates presuppositions if they have not been
    yet assumed
  \item hypotheseses can cancel presuppositions in their scope (\text{if
    \ldots, then \ldots})
  % \item shifted perspectives and quotation interpret speaker-dependent
  % expressions
  \end{itemize}
\end{frame}

\begin{frame}{Proof of Concept}
  \vfill
  We have built a small prototype to test and explore our approach.
  \vfill
  \begin{itemize}
  \item in-situ quantification
  \item discourse anaphora
  \item presuppositions (of referentials)
  \item their interactions
    \begin{itemize}
    \item e.g., binding problem
    \end{itemize}
  \end{itemize}
  \vfill
\end{frame}

\begin{frame}{Summary}
  \begin{itemize}
  \item perspective shift
    \begin{itemize}
    \item from denotations as complex objects \\
          to denotations as complex \textcolor{red}{processes} producing simple \textcolor{mygreen}{objects}
    \item focus on what meanings do, not on what they are
    \end{itemize}
    \vfill
    \pause
  \item content/context distinction
    \begin{itemize}
    \item \textcolor{mygreen}{objects} -- purely truth-conditional material
    \item \textcolor{red}{process} -- we dump the pragmatic wastebasket here
    \item placement of non-locality phenomena such as in-situ
      quantification is to our discretion
    \end{itemize}
    \vfill
    \pause
  \item easier to manage multiple effects
    \begin{itemize}
    \item our driving motivation (empirical coverage)
    \item stable denotations help avoid generalizing to the worst case
    \item captures parameters, mutable state, continuations, projections
      and their filtering/cancelling both flexibly and compositionally
      \begin{itemize}
      \item used in PLT research and functional programming too
      \end{itemize}
    \end{itemize}
  \end{itemize}
\end{frame}

\end{document}
