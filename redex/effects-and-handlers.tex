\documentclass{article}

\usepackage[utf8]{inputenc}
\usepackage{amsmath}
\usepackage{bussproofs}
\usepackage{stmaryrd}

\newcommand{\hsbind}{\mathbin{\texttt{>>=}}}
\newcommand{\ap}{\mathbin{|@|}}
\newcommand{\apl}{\mathbin{|@}}
\newcommand{\apr}{\mathbin{@|}}
\newcommand{\cons}{\mathbin{::}}
\newcommand{\cat}{\mathbin{+\mkern-10mu+}}

\newcommand{\abs}[1]{\textsc{#1}}
\newcommand{\obj}[1]{\textbf{#1}}
\newcommand{\sem}[1]{\llbracket #1 \rrbracket}
\newcommand{\lex}[2]{\sem{\abs{#1}} &:= #2}

\newcommand{\dand}{\mathbin{\overline{\land}}}
\newcommand{\dnot}{\mathop{\overline{\lnot}}}
\newcommand{\dor}{\mathop{\overline{\lor}}}
\newcommand{\dimpl}{\mathbin{\overline{\to}}}
\newcommand{\dexists}{\mathop{\overline{\exists}}}
\newcommand{\dforall}{\mathop{\overline{\forall}}}


\title{Effects and Handlers in Natural Language}
\author{Jirka Maršík}

\begin{document}

\maketitle



\section{The Calculus}

We will start by presenting a calculus suitable for describing the
denotations of natural language expressions. Our calculus will be an
extension of the simply-typed lambda calculus. We will enrich STLC with a
generic datatype for representing effectful computations alongside with
operations to produce and consume values of this type.


\subsection{Expressions}

Without further ado, we give the syntax of the expressions $e$ of our
language. In the definition below, $x$ ranges over variables from set
$\mathcal{X}$, $c$ over constants from signature $\Sigma$ and $OP$ over
operation symbols from signature $E$.

\begin{align*}
  e ::= \
  & e\ e \\
  & \lambda x.\ e \\
  & x \\
  & c \\
  & OP \\
  & \eta \\
  & \left[\mathcal{H}\ (OP\ e) \ldots\ (\eta\ e)\right]\\
  & \mathcal{C}
\end{align*}

The first 4 lines come directly from STLC with constants. On top of that,
we add constructors for effectful computations. An operation symbol $OP$
can be used to construct an effectful computation and $\eta$ can be used to
inject a pure value into the domain of computations.

$\mathcal{H}$ can be used to construct handlers. Handlers are assembled
from clauses, one of which is used to interpret pure computations and
others which interpret the different operation symbols found in
$E$. Finally, we have a special operator, $\mathcal{C}$, that lets us
leverage a certain interaction between $\lambda$-bindings and computations.


\subsection{Types}

We now give a syntax for the types of our calculus alongside with a typing
judgment. In the grammar below, $\nu$ ranges over atomic types from set
$\mathcal{T}$.

\begin{align*}
  \tau ::= \
  & \tau \to \tau \\
  & \nu \\
  & \mathcal{F}(\tau)
\end{align*}

Now onto the typing judgments. We will be using $\Gamma$ for contexts,
which map variables to types, $Sigma$ for signatures, which map constants
to types, and $E$ for effect signatures, which map operation symbols to an
input and an output type. Metavariables $M$, $N$ stand for expressions and
$\alpha$, $\beta$, $\gamma$\ldots\ stand for types.

  \def\labelSpacing{4pt}
  \begin{prooftree}
    \AxiomC{$\Gamma \vdash_{\Sigma,E} M : \alpha \to \beta$}
    \AxiomC{$\Gamma \vdash_{\Sigma,E} N : \alpha$}
    \RightLabel{[app]}
    \BinaryInfC{$\Gamma \vdash_{\Sigma,E} M N : \beta$}
  \end{prooftree}
  \begin{prooftree}
    \AxiomC{$\Gamma, x : \alpha \vdash_{\Sigma,E} M : \beta$}
    \RightLabel{[abs]}
    \UnaryInfC{$\Gamma \vdash_{\Sigma, E} \lambda x.\ M : \alpha \to \beta$}
  \end{prooftree}
  \begin{prooftree}
    \AxiomC{$x : \alpha \in \Gamma$}
    \RightLabel{[var]}
    \UnaryInfC{$\Gamma \vdash_{\Sigma,E} x : \alpha$}
  \end{prooftree}
  \begin{prooftree}
    \AxiomC{$c : \alpha \in \Sigma$}
    \RightLabel{[const]}
    \UnaryInfC{$\Gamma \vdash_{\Sigma,E} c : \alpha$}
  \end{prooftree}
  \begin{prooftree}
    \AxiomC{$OP : \alpha \to \beta \in E$}
    \RightLabel{[op]}
    \UnaryInfC{$\Gamma \vdash_{\Sigma,E} OP : \alpha \to (\beta \to
      \mathcal{F}(\gamma)) \to \mathcal{F}(\gamma)$}
  \end{prooftree}
  \begin{prooftree}
    \AxiomC{$\Gamma \vdash_{\Sigma,E} \eta : \alpha \to \mathcal{F}(\alpha)
      $\hskip 4pt [$\eta$]}
  \end{prooftree}
  \begin{prooftree}
    \AxiomC{$OP_i : \alpha_i \to \beta_i \in E$}
    \noLine
    \def\extraVskip{0pt}
    \UnaryInfC{$\Gamma \vdash_{\Sigma,E} M_i : \alpha_i \to (\beta_i \to
      \mathcal{F}(\delta)) \to \mathcal{F}(\delta)$}
    \noLine
    \UnaryInfC{$\Gamma \vdash_{\Sigma,E} M_\eta : \gamma \to \mathcal{F}(\delta)$}
    \def\extraVskip{2pt}
    \RightLabel{[$\mathcal{H}$]}
    \UnaryInfC{$\Gamma \vdash_{\Sigma,E} [\mathcal{H}\ (OP_i\ M_i)_{i \in I}\ (\eta\ M_\eta)] : \mathcal{F}(\gamma) \to \mathcal{F}(\delta)$}
  \end{prooftree}
  \begin{prooftree}
    \AxiomC{$\Gamma \vdash_{\Sigma,E} \mathcal{C} : (\alpha \to
      \mathcal{F}(\beta)) \to \mathcal{F}(\alpha \to \beta)$\hskip 4pt [$\mathcal{C}$]}
  \end{prooftree}

The typing rules mirror the syntax of expressions. Again, the first four
rules come from STLC with constants. The rules $op$ and $\eta$ let us
construct computations, expressions of type
$\mathcal{F}(\alpha)$. Computations can then be interpreted using handlers
whose type is given by the $\mathcal{H}$ rule. Finally, the $\mathcal{C}$
rule gives a type to the primitive operation of the same name, which
already gives us an idea of what it will do.


\subsection{Semantics}

We will now present a set of type-preserving rules that we will use to
evaluate/simplify expressions in our language.

\vspace{3mm}

\begin{tabular}{lr}
  $(\lambda x.\ M)\ N \rightarrow$ & [$\beta$] \\
  $M[x/N]$ & \\
  \\
  $[\mathcal{H}\ (OP_i\ M_i)\ldots\ (\eta\ M_\eta)]\ (\eta\ N) \rightarrow$ & [$\mathcal{H}$-$\eta$] \\
  $M_\eta\ N$ & \\
  \\
  $[\mathcal{H}\ (OP_i\ M_i)\ldots\ (\eta\ M_\eta)]\ (OP_i\ N_a\ N_k) \rightarrow$ & [$\mathcal{H}$-$OP$] \\
  $M_i\ N_a\ (\lambda x.\ [\mathcal{H}\ (OP_i\ M_i)\ldots\ (\eta\ M_\eta)]\ (N_k\ x))$ & where $x$ is fresh \\
  \\
  $[\mathcal{H}\ (OP_i\ M_i)\ldots\ (\eta\ M_\eta)]\ (OP\ N_a\ N_k) \rightarrow$ & [$\mathcal{H}$-$OP^*$] \\
  $OP\ N_a\ (\lambda x.\ [\mathcal{H}\ (OP_i\ M_i)\ldots\ (\eta\ M_\eta)]\ (N_k\ x))$ & where $x$ is fresh \\
  & and $OP \notin \{OP_i\}_i$ \\
  \\
  $\mathcal{C}\ (\lambda x.\ \eta\ M) \rightarrow$ & [$\mathcal{C}$-$\eta$] \\
  $\eta\ (\lambda x.\ M)$ & \\
  \\
  $\mathcal{C}\ (\lambda x.\ OP\ M_a\ M_k) \rightarrow$ & [$\mathcal{C}$-$OP$] \\
  $OP\ M_a\ (\lambda y.\ \mathcal{C}\ (\lambda x.\ M_k\ y))$ & where $y$ is fresh \\
  & and $x \notin FV(M_a)$
\end{tabular}

\vspace{3mm}

Besides the standard $\beta$-reduction rule, we have rules that define the
behavior of the handlers formed using $\mathcal{H}$ and of the
$\mathcal{C}$ operator.


\subsection{Sugar}

Here we will introduce a collection of generally useful syntactic shortcuts
and combinators for our calculus.

\begin{align*}
  \_ \circ \_ &: (\beta \to \gamma) \to (\alpha \to \beta) \to (\alpha \to \gamma) \\
  f \circ g &= \lambda x.\ f\ (g\ x) \\
  \_^* &: (\alpha \to \mathcal{F}(\beta)) \to (\mathcal{F}(\alpha) \to \mathcal{F}(\beta)) \\
  f^* &= [\mathcal{H}\ (\eta\ f)] \\
  \mathcal{F} &: (\alpha \to \beta) \to (\mathcal{F}(\alpha) \to \mathcal{F}(\beta)) \\
  \mathcal{F} &= \lambda f.\ (\eta \circ f)^* \\
  \_ \hsbind \_ &: \mathcal{F}(\alpha) \to (\alpha \to \mathcal{F}(\beta)) \to \mathcal{F}(\beta) \\
  M \hsbind N &= N^*\ M \\
  \\
  [\mathcal{H}\ (OP_i\ M_i)\ldots] &= [\mathcal{H}\ (OP_i\ M_i)\ldots\ (\eta\ \eta)]
\end{align*}

We will also make use of combinators for function application where one or
more of the arguments are computations.

\begin{align*}
  \_ \apl \_ &: \mathcal{F}(\alpha \to \beta) \to \alpha \to \mathcal{F}(\beta) \\
  F \apl x &= F \hsbind (\lambda x.\ \eta\ (f\ x)) \\
  \_ \apr \_ &: (\alpha \to \beta) \to \mathcal{F}(\alpha) \to \mathcal{F}(\beta) \\
  f \apr X &= X \hsbind (\lambda x.\ \eta\ (f\ x)) \\
  \_ \ap \_ &: \mathcal{F}(\alpha \to \beta) \to \mathcal{F}(\alpha) \to \mathcal{F}(\beta) \\
  F \ap X &= F \hsbind (\lambda f.\ X \hsbind (\lambda x.\ \eta\ (f\ x)))
\end{align*}


\section{Application to Natural Language}

\subsection{Initial Fragment}

We will start with an ACG-style fragment $\mathcal{G}_0 = \left<
\Sigma^a_0, \Sigma^o_0, \mathcal{L}_0 \right>$.

Abstract signature $\Sigma^a_0$:

\begin{align*}
  \abs{John}, \abs{Mary}, \abs{Alice} &: NP \\
  \abs{knows}, \abs{loves} &: NP \to NP \to S \\
  \abs{says} &: \overline{S} \to NP \to S \\
  \abs{that} &: S \to \overline{S} \\
\end{align*}

Object signature $\Sigma^o_0$:

\begin{align*}
  \obj{j}, \obj{m}, \obj{a} &: \iota \\
  \obj{know}, \obj{love} &: \iota \to \iota \to o \\
  \obj{say} &: \iota \to o \to o
\end{align*}

Lexicon $\mathcal{L}_0$:

\begin{align*}
  \lex{John}{\obj{j}} \\
  \lex{Mary}{\obj{m}} \\
  \lex{Alice}{\obj{a}} \\
  \lex{knows}{\lambda x y.\ \obj{know}\ y\ x} \\
  \lex{loves}{\lambda x y.\ \obj{love}\ y\ x} \\
  \lex{says}{\lambda s x.\ \obj{say}\ x\ s} \\
  \lex{that}{\lambda x.\ x}
\end{align*}


\subsection{Going to Effectful Semantics}

The technique used by our approach is to let every lexical item have an
effect on the interpretation of the discourse. We will thus interpret
abstract types as computations types ($S$ and $\overline{S}$ will
correspond to $\mathcal{F}(o)$ instead of $o$, $NP$ to $\mathcal{F}(\iota)$
instead of $\iota$ and $N$ to $\mathcal{F}(\iota \to o)$ instead of $\iota
\to o$).

We could rewrite our fragment $\mathcal{G}_0$ by hand to use computations,
but since the abstract signature $\Sigma^a_0$ is second-order, we can
easily automate the process. We define a raising function
$\mathrm{raise_\alpha}$. In the definitions below, variables such as
$a$ and $b$ range over atomic types.

\begin{align*}
  \mathrm{raise_a}(x) &= \eta\ x \\
  \mathrm{raise_{a \to \beta}}(f) &= \lambda X.\ X \hsbind (\lambda
  x. \mathrm{raise_\beta}(f\ x))
\end{align*}

We could now produce a lexicon $\mathcal{L}_1$ such that for every $c :
\alpha \in \Sigma^a_0$, we have $\mathcal{L}_1(c) =
\mathrm{raise_\alpha}(\mathcal{L}_0(c))$. There is just one hitch, we
would end up with $\mathcal{L}_1(\abs{loves}) = \lambda X Y.\ X \hsbind
(\lambda x.\ Y \hsbind (\lambda y.\ \eta\ (\obj{love}\ y\ x)))$ where $X$
corresponds to the object and $Y$ to the subject. However, we would
generally like the evaluation of the lexical items to be sequenced in the
same order in which the words appear in the source phrase (for reasons of
anaphora, in-situ quantification\ldots). In other words, we would like to
have $\mathcal{L}_1(\abs{loves}) = \lambda X Y.\ Y \hsbind (\lambda y.\ X
\hsbind (\lambda x.\ \eta\ (\obj{love}\ y\ x)))$.

There are several solutions to this problem. First off, we could go back
and change the abstract syntax so that arguments are supplied in the order
in which they appear in the surface form (this goes against the most common
analysis of transitive verbs in which they first merge with objects to form
verb phrases). Another reasonable solution might be to just change the
entries for lexical items such as transitive verbs manually without relying
on an automatic procedure for establishing the order of evaluation.

We will explore a third way that will use a general rule to raise all
denotations but which will cover the problematic case above
correctly. Since in categorial grammar, lexical items usually merge first
with their complements, which are often to the right, and then with their
arguments, which are often to the left. We will introduce
$\mathrm{raise^R_\alpha}$ that simulates right-to-left evaluation
instead of left-to-right. We start by unfolding the recursive definition of
$\mathrm{raise_\alpha}$ and then modifying it.

\begin{align*}
  \mathrm{raise_{a_1 \to\ \ldots\ \to a_n \to b}}(f) &= \lambda X_1 \ldots X_n.\ X_1 \hsbind (\lambda x_1.\ \ldots\ X_n \hsbind (\lambda x_n.\ \eta\ (f\ x_1 \ldots x_n))) \\
  \mathrm{raise^R_{a_1 \to\ \ldots\ \to a_n \to b}}(f) &= \lambda X_1 \ldots X_n.\ X_n \hsbind (\lambda x_n.\ \ldots\ X_1 \hsbind (\lambda x_1.\ \eta\ (f\ x_1 \ldots x_n))) \\
\end{align*}

We have $\mathrm{raise^R_{NP \to NP \to S}}(\abs{loves}) = \lambda X
Y.\ Y \hsbind (\lambda y.\ X \hsbind (\lambda
x.\ \eta\ (\obj{love}\ y\ x)))$.

We now define $\mathcal{G}_1 = \left< \Sigma^a_1, \Sigma^o_1, \mathcal{L}_1
\right>$, where $\Sigma^a_1 = \Sigma^a_0$, $\Sigma^o_1 = \Sigma^o_0$ and
where $\mathcal{L}_1$ is such that for every $c : \alpha \in \Sigma^a_1$,
we have $\mathcal{L}_1(c) =
\mathrm{raise^R_\alpha}(\mathcal{L}_0(c))$.


\subsection{Adding Logical Furniture}

We will now be enriching grammar by defining extensions, which will be
functions between grammars. The first extension we will define will add
some basic logical operators.

The signature $\Sigma^o_L$ lists the object constants we will be
adding:

\begin{align*}
  \lnot &: o \to o \\
  \land &: o \to o \to o \\
  \exists &: (\iota \to o) \to o
\end{align*}

We will also add their raised versions which operate on computations. This
will let further extensions modify them to add effects and/or handlers to
the logical operators used in all of the other extensions. Their
definitions are given in the lexicon $\mathcal{L}_L$ shown below:

\begin{align*}
  \dnot &: \mathcal{F}(o) \to \mathcal{F}(o) \\
  \dand, \dimpl, \dor &: \mathcal{F}(o) \to \mathcal{F}(o) \to \mathcal{F}(o) \\
  \dexists, \dforall &: (\iota \to \mathcal{F}(o)) \to \mathcal{F}(o) \\
  \dnot &= \mathrm{raise_{o \to o}}(\lnot) = \lambda X.\ X \hsbind (\lambda x.\ \eta\ (\lnot x)) \\
  \dand &= \mathrm{raise_{o \to o \to o}}(\land) = \lambda X Y.\ X \hsbind (\lambda x.\ Y \hsbind (\lambda y.\ \eta\ (x \land y))) \\
  \dexists &= \mathcal{F}(\exists) \circ \mathcal{C} = \lambda P.\ (\mathcal{C}\ P) \hsbind (\lambda p.\ \eta\ (\exists\ p)) \\
  \dimpl &= \lambda X Y.\ \dnot\ (X \dand (\dnot Y)) \\
  \dor &= \lambda X Y.\ \dnot\ ((\dnot X) \dand (\dnot Y)) \\
  \dforall &= \lambda P.\ \dnot\ (\dexists\ (\lambda x.\ \dnot\ (P\ x)))
\end{align*}

We will now define our first extension, $E_L$.\footnote{$\uplus$ stands for
  a union of signatures/lexicons where the second operand has precedence
  when it comes to assigning a type/image to an element which is in the
  domain of both of the operands.}

$$
E_{logic}(\left< \Sigma^a, \Sigma^o, \mathcal{L} \right>) = \left< \Sigma^a, \Sigma^o \uplus \Sigma^o_L, \mathcal{L} \uplus \mathcal{L}_L \right>
$$

We have introduced new constants into the domain of the lexicon which are
not part of the abstract signature. We are working with a variation of an
ACG lexicon that allows us to define object-level auxiliary terms. We can
convert such a lexicon to a classical ACG lexicon by taking the fixpoint of
the homomorphism induced by the lexicon and then restricting it to the
abstract constants (i.e.\ by replacing occurrences of the auxiliary
constants in the lexicon with their definitions from the lexicon).

This convenient facility will let extensions modify auxiliary terms and
thus indirectly modify all the terms that use these auxiliary terms.


\subsection{Common Nouns}

Our first objective will be to extend a grammar with quantificational
articles like \textit{every} and \textit{some}. However, this presupposes
that our grammar already has some common nouns to combine these articles
with. Since nouns themselves are not bound to the phenomenon of
quantification, we will introduce them in an isolated extension.

We will start by introducing some predicates at the object level (signature
$\Sigma^o_N$):

\begin{align*}
  \obj{man}, \obj{woman} &: \iota \to o
\end{align*}

Now for the abstract constants in signature $\Sigma^a_N$ (we treat
relativizers in our noun extension, since they form noun modifiers):

\begin{align*}
  \abs{man}, \abs{woman} &: N \\
  \abs{who}, \abs{whom} &: (NP \to S) \to N \to N
\end{align*}

Finally, we give an interpretation for the new abstract constants by
extending the lexicon with $\mathcal{L}_N$:

\begin{align*}
  \lex{man}{\eta\ \obj{man}} \\
  \lex{woman}{\eta\ \obj{woman}} \\
  \sem{\abs{who}} := \sem{\abs{whom}} &:= \lambda P N.\ N \hsbind (\lambda n.\ \mathcal{C}\ (\lambda x.\ (\eta\ (n\ x)) \dand (P\ (\eta\ x))))
\end{align*}

We have now finished our extension $E_N$.

$$
E_N(\left< \Sigma^a, \Sigma^o, \mathcal{L} \right>) = \left< \Sigma^a \uplus
\Sigma^a_N, \Sigma^o \uplus \Sigma^o_N, \mathcal{L} \uplus \mathcal{L}_N \right>
$$


\subsection{Quantification}

We will start our treatment of quantification by introducing an operation
symbol $SCOPE$ into our effect signature $E$. The operation will take
inputs of type $(\iota \to \mathcal{F}(o)) \to \mathcal{F}(o)$ and provide
outputs of type $\iota$ (i.e.\ $SCOPE : ((\iota \to \mathcal{F}(o)) \to
\mathcal{F}(o)) \to \iota \in E$). Our analysis will all revolve around
producing and handling this particular effect.

$$
E_Q(\left< \Sigma^a, \Sigma^o, \mathcal{L} \right>) = \left< \Sigma^a \uplus \Sigma^a_Q, \Sigma^o, \mathcal{L} \uplus \mathcal{L}_Q \right>
$$

First, we start by introducing the abstract constants for the new lexical
items (signature $\Sigma^a_Q$):

\begin{align*}
  \abs{every}, \abs{some}, \abs{a} &: N \to NP
\end{align*}

Now we can add/change the pertinant entries in the lexicon (lexicon $\mathcal{L}_Q$):

\begin{align*}
  \lex{every}{\lambda N.\ SCOPE\ (\lambda k.\ \dforall\ (\lambda x.\ (N \apl x) \dimpl (k\ x)))\ \eta} \\
  \sem{\abs{some}} := \sem{\abs{a}} &:= \lambda N.\ SCOPE\ (\lambda k.\ \dexists\ (\lambda x.\ (N \apl x) \dand (k\ x)))\ \eta \\
  \mathrm{SI} &: \mathcal{F}(o) \to \mathcal{F}(o) \\
  \mathrm{SI} &= [\mathcal{H}\ (SCOPE\ (\lambda c k.\ c\ k))] \\
  \lex{loves}{\lambda X Y.\ \mathrm{SI}\ (\mathcal{L}(\abs{loves})\ X\ Y)} \\
  \lex{knows}{\lambda X Y.\ \mathrm{SI}\ (\mathcal{L}(\abs{knows})\ X\ Y)} \\
  \lex{says}{\lambda S X.\ \mathrm{SI}\ (\mathcal{L}(\abs{says})\ S\ X)} \\
\end{align*}

We have given semantics to the new lexical items which make use of the
$SCOPE$ effect. Quantifier scope is influenced by other lexical items
resulting in so-called scope islands. We therefore define an auxiliary
term, a handler called $\mathrm{SI}$. Since tensed clauses form scope
islands, we insert $\mathrm{SI}$ into the existing lexical entries
for tensed verbs.

If we had a more fine-grained grammar which would provide a separate
lexical entry for the finite verb morpheme, we could simply add the
$\mathrm{SI}$ handler there without having to change all verbs.

A similar solution would work in an unlexicalized grammar in which, e.g.,
transitive verbs would be decomposed into a computation of some binary
relation of abstract type $TV$ and a combinator/rule for forming sentences
using transitive verbs of abstract type $NP \to TV \to NP \to S$. In that
case, $\mathrm{SI}$ would be inserted into the semantics of every
such rule which is to be considered a scope island.

\subsection{Dynamics}

We now present our final extension that will deal with pronouns and
anaphora. In our particular implementation, we will make appeal to three
new effects in our signature $E$: $FRESH$ of type $1 \to \iota$ for
introducing new discourse referents, $GET$ of type $1 \to \gamma$ for
accessing the current discourse context and $ASSERT$ of type $o \to 1$ for
asserting propositions in a context.

$$
E_D(\left< \Sigma^a, \Sigma^o, \mathcal{L} \right>) = \left< \Sigma^a \uplus \Sigma^a_D, \Sigma^o \uplus \Sigma^o_D, \mathcal{L} \uplus \mathcal{L}_D \right>
$$

We start with the constants for the new lexical items we will be treating
(signature $\Sigma^a_D$):

\begin{align*}
  \abs{him}, \abs{her} &: NP
\end{align*}

To implement dynamics, we will enrich our object language with constructs
for manipulating contexts and resolving anaphora (signature $\Sigma^o_D$):

\begin{align*}
  \mathrm{nil} &: \gamma \\
  \_ \cons \_ &: \iota \to \gamma \to \gamma \\
  \_ \cat \_ &: \gamma \to \gamma \to \gamma \\
  \mathrm{sel_{he}}, \mathrm{sel_{she}}  &: \gamma \to \iota \\
  * &: 1 \\
  \_ = \_ &: \iota \to \iota \to o
\end{align*}

We have a null discourse context $\mathrm{nil}$ and a way to
introduce entities into the context through $(\cons)$. We also have a
function for embedding contexts, $\cat$, which boils down to list
concatenation when we view contexts as lists of entities (this operator is
not crucial). The $\mathrm{sel}$ operators resolve anaphora and
retrieve antecedents from contexts. We also add a unit type and an equality
on individuals since they were absent so far.

Finally, we give the entries in the lexicon $\mathcal{L}_D$:

\begin{align*}
  \lex{him}{GET\ *\ (\lambda e.\ \eta\ (\mathrm{sel_{he}}\ e))} \\
  \lex{her}{GET\ *\ (\lambda e.\ \eta\ (\mathrm{sel_{she}}\ e))} \\
  \dexists &= \lambda P.\ FRESH\ *\ P \\
  \dnot &= \mathcal{L}(\dnot) \circ \mathrm{box} \\
  \mathrm{box} &: \mathcal{F}(o) \to \mathcal{F}(o) \\
  \mathrm{box} &= \lambda A.\ ([\mathcal{H}\ 
      (GET\ (\lambda u k.\ \eta\ (\lambda e.\ (GET\ *\ (\lambda e'.\ (k\ (e \cat e')) \hsbind (\lambda f.\ f\ e))))))\ \\
    & \hskip 1.8cm (FRESH\ (\lambda u k.\ \eta\ (\lambda e.\ \mathcal{L}(\dexists)\ (\lambda x.\ (k\ x) \hsbind (\lambda f.\ f\ (x \cons e))))))\ \\
    & \hskip 1.8cm (ASSERT\ (\lambda p k.\ \eta\ (\lambda e.\ (\eta\ p) \dand ((k\ *) \hsbind (\lambda f.\ f\ e)))))\ \\
    & \hskip 1.8cm (\eta\ (\lambda x.\ \eta\ (\lambda e.\ \eta\ x)))]\ A) \hsbind (\lambda f.\ f\ \mathrm{nil}) \\
  \mathrm{add} &: \iota \to \mathcal{F}(\iota) \\
  \mathrm{add} &= \lambda x.\ FRESH\ *\ (\lambda y.\ ASSERT\ (x = y)\ (\lambda u.\ \eta\ y)) \\
  \lex{john}{\mathcal{L}(\abs{john}) \hsbind \mathrm{add}} \\
  \lex{mary}{\mathcal{L}(\abs{mary}) \hsbind \mathrm{add}} \\
  \lex{alice}{\mathcal{L}(\abs{alice}) \hsbind \mathrm{add}} \\
\end{align*}

The semantics we give to the new pronouns uses the $GET$ effect to fetch
the discourse context and then searches for a suitable antecedent. We
change our definition of existential quantification to use the $FRESH$,
which, as we will later see, installs an existential quantifier at the
discourse level and adds the new variable into the context. However, we do
not want all new discourse referents to project to the top global
context. Namely we want dynamic effects to be blocked by negation and so we
insert a handler, $\mathrm{box}$, into the definition of negation.

Now to decipher the $\mathrm{box}$ handler.

First of all, we will look into its $\eta$ clause to see that the handler
interprets propositions $x : o$ as $\eta\ (\lambda e.\ \eta\ x) :
\mathcal{F}(\gamma \to \mathcal{F}(o))$\footnote{The outer $\mathcal{F}$ is
  due to the fact that the computation being handled could produce some
  other effects that are not handled by $\mathrm{box}$.}. The bodies of the
other clauses will therefore have this type too and we can see that they
all start with $\eta\ (\lambda e. \ldots)$ as well.

Next, we notice that there is a pattern that repeats itself in every
operation clause, $(k\ x) \hsbind (\lambda f.\ f\ e)$. This corresponds to
calling the continuation $k$ with two arguments, $x$ and $e$, and then
merging the effects of both of function calls. The slot of $x$ corresponds
to what the handler returns to the client of the operation and the slot of
$e$ corresponds to the updated value of the discourse context that the
handler tracks.

With these two observations out of the way, reading the handler becomes
easy. The $GET$ clause uses another $GET$ to find the discourse context
$e'$ in which the box is embedded and combines it with its own context $e$
to give the client a complete view of the context. The $FRESH$ clause wraps
the continuation in an existential quantifier, gives the quantified
variable to the client and adds it to the discourse context that it is
tracking. The $ASSERT$ clause just conjoins the supplied proposition to the
continuation.

We have given an interpretation into the type $\mathcal{F}(\gamma \to
\mathcal{F}(o))$ to be able to track the discourse context $\gamma$ within
the box. Once we have this interpretation in hand, we can apply it to
$\mathrm{nil}$ to start the process and reduce it to an $\mathcal{F}(o)$.

So far, we have covered how pronouns find their antecedents in the
discourse contexts and how entities end up being added to these contexts
through quantification. There is however still one class of lexical items
in our grammar that dynamic semantics wants to speak about and those are
the proper names. On top of just designating certain individuals, we want
them to have an effect on the discourse by introducing these individuals
into the context.

\subsubsection{Dynamic Indefinites}

We might ask ourselves why we introduced the $ASSERT$ effect to treat
proper names instead of directly making appeal to some $PUSH$ effect which
would add a given entity to the discourse context. The answer is that with
$ASSERT$ we can unify the treatment of proper names and indefinites in our
grammar.

Signature $\Sigma^a_A$:

\begin{align*}
  \abs{a} &: N \to NP
\end{align*}

Lexicon $\mathcal{L}_A$:

\begin{align*}
  \lex{a}{\lambda N.\ FRESH\ *\ (\lambda x.\ N \hsbind (\lambda n.\ ASSERT\ (n\ x)\ (\lambda u.\ \eta\ x)))}
\end{align*}

We can now define $E_A$ and $E_{D'}$:

\begin{align*}
E_A(\left< \Sigma^a, \Sigma^o, \mathcal{L} \right>) &= \left< \Sigma^a \uplus \Sigma^a_A, \Sigma^o, \mathcal{L} \uplus \mathcal{L}_A \right> \\
E_{D'} &= E_A \circ E_D
\end{align*}



\section{Putting It All Together}

We can now use our extensions to produce a variety of grammars. Below, we
will use $\mathcal{G}_{1'}$ to stand for $E_L(\mathcal{G}_1)$.

\begin{itemize}
\item $\mathcal{G}_1$ -- the initial fragment

  Can treat:
  \begin{itemize}
  \item John loves Mary.
  \item John says that Mary loves Alice.
  \end{itemize}

\item $\mathcal{G}_Q = E_Q(E_N(\mathcal{G}_{1'}))$

  Can treat:
  \begin{itemize}
  \item Every man loves a woman.
  \item John says that Mary loves a woman.
  \item John knows a man whom Alice says that Mary loves.
  \end{itemize}
  
\item $\mathcal{G}_D = E_D(\mathcal{G}_{1'})$

  Can treat:
  \begin{itemize}
  \item John says that Mary loves him.
  \end{itemize}

\item $\mathcal{G}_{D'} = E_N(E_A(\mathcal{G}_D)) = E_N(E_{D'}(\mathcal{G}_{1'}))$

  Can treat:
  \begin{itemize}
  \item John knows she loves a woman.
  \item A man says that he loves Mary.
  \item A man who loves Mary knows Mary.
  \end{itemize}

\item $\mathcal{G}_T = E_D(\mathcal{G}_Q) = E_Q(E_N(\mathcal{G}_D))$

  Can treat:
  \begin{itemize}
  \item Every man who loves a woman knows her.
  \end{itemize}
\end{itemize}


\section{Dealing with Scopal Ambiguities}

Sentences such as ``Every man loves a woman'' or ``Every owner of a siamese
cat loves a therapist'' are ambiguous. In ACGs, every abstract derivation
has exactly one object interpretation. In order to account for ambiguity,
we have two ways of proceeding. Either we introduce separate derivations to
account for the different readings or we make our interpretations
non-deterministic, underspecified. In the next two subsections, we will
explore both techniques.

\subsection{QR}

We propose a small extension which adds the $\abs{QR}$ operator to our
fragment. The goal of this operator is to let us project the effects
(e.g.\ scope-taking) of any $NP$ outside of its ``usual'' destination.

We give an update to the abstract signature, $\Sigma^a_{QR}$:

\begin{align*}
  \abs{QR} &: NP \to (NP \to S) \to S
\end{align*}

And we give a semantics to the $\abs{QR}$ operator in $\mathcal{L}_{QR}$:

\begin{align*}
  \lex{QR}{\lambda X k.\ \mathrm{SI}\ (X \hsbind (\lambda x.\ k\ (\eta\ x)))}
\end{align*}

Now, the extension is complete.

$$
E_{QR}(\left< \Sigma^a, \Sigma^o, \mathcal{L} \right>) = \left< \Sigma^a \uplus \Sigma^a_{QR}, \Sigma^o, \mathcal{L} \uplus \mathcal{L}_{QR} \right>
$$

We can now derive the inverse scope reading of ``Every man loves a woman''.

\begin{align*}
& \sem{\abs{QR}\ (\abs{a}\ \abs{woman})\ (\lambda
    x.\ \abs{loves}\ x\ (\abs{every}\ \abs{man}))} \\
\rightarrow^*\ & \mathrm{SI}\ (SCOPE\ (\lambda k.\ \dexists
x.\ (\eta\ (\obj{woman}\ x)) \dand (k\ x))\ (\lambda x.\ \eta\ (\forall
y.\ \obj{man}\ y \rightarrow \obj{love}\ y\ x))) \\
\rightarrow^*\ & (\lambda k.\ \dexists
x.\ (\eta\ (\obj{woman}\ x)) \dand (k\ x))\ (\lambda x.\ \eta\ (\forall
y.\ \obj{man}\ y \rightarrow \obj{love}\ y\ x)) \\
\rightarrow^*\ & \eta\ (\exists x.\ \obj{woman}\ x \land (\forall
y.\ \obj{man}\ y \rightarrow \obj{love}\ y\ x))
\end{align*}

The QR operator allows us to displace the scope of any quantifier
(actually, it can displace any kind of $NP$ effect, including dynamics, so
it can be used, e.g., to implement a kind of cataphora).

\subsection{Leaky Scope Islands}

We will now look at a solution that over-generates a little less and that
is based on a different strategy. Rather than having different derivations
that place the quantifier scope at different locations, we will weaken the
scope islands so that they may choose to let a quantifier escape.

In order to allow for choice and non-determinism, we will add a new effect,
$CHOOSE : 1 \to 2 \in E$, where $2$ is the Boolean type $1 +
1$\footnote{For this to work, we assume the presence of standard sum types
  (intuitionistic disjunction on types) in our calculus.}. We will also
introduce the following syntactic sugar for non-deterministic computations:

\begin{align*}
A + B = CHOOSE\ *\ (\lambda b.\ \text{case}\ b\ \text{of}\ &\operatorname{inl}(*) \Rightarrow A \\
|\ &\operatorname{inr}(*) \Rightarrow B)
\end{align*}

We would like certain quantifiers (e.g., indefinites) to be able to escape
scope islands. We will introduce an effect $SCOPE_P : ((\iota \to
\mathcal{F}(o)) \to \mathcal{F}(o)) \to \iota \in E$, which is a variant of
the $SCOPE$ effect to be used by quantifiers that ought to be able to
escape from scope islands.

We can now define our extension $E_{Q'}$, which builds upon $E_Q$:

$$
E_{Q'}(\left< \Sigma^a, \Sigma^o, \mathcal{L} \right>) =
\left< \Sigma^a, \Sigma^o, \mathcal{L} \uplus \mathcal{L}_{Q'} \right>
$$

where, $\mathcal{L}_{Q'}$, the changes we propose to the lexicon, are as
follows:

\begin{align*}
  \lex{every}{\lambda N.\ SCOPE\ (\lambda k.\ \mathrm{SL}\ (\dforall\ (\lambda x.\ (N \apl x) \dimpl (k\ x))))}\ \eta \\
  \sem{\abs{some}} := \sem{\abs{a}} &:= \lambda N.\ SCOPE_P\ (\lambda k.\ \mathrm{SL}\ (\dexists\ (\lambda x.\ (N \apl x) \dand (k\ x))))\ \eta \\
  \mathrm{SI} &= \mathcal{L}(\mathrm{SI}) \circ \mathrm{SL} \\
  \mathrm{SL} &: \mathcal{F}(o) \to \mathcal{F}(o) \\
  \mathrm{SL} &= [\mathcal{H}\ (SCOPE_P\ (\lambda c k.\ (c\ k) + (SCOPE_P\ c\ k)))] 
\end{align*}

The $\mathrm{SL}$ (Scope Location) handler introduces a possible place for
a quantifier to take scope.

We will now look again at the sentence ``Every man loves a woman''. First
off, we will introduce some handy shortcuts:

\begin{align*}
c_1 &= \lambda k. \dforall\ (\lambda x.\ (\eta\ (\obj{man}\ x)) \dimpl (k\ x)) \\
c_2 &= \lambda k.\ \dexists\ (\lambda x.\ (\eta\ (\obj{woman}\ x)) \dand
(k\ x)) \\
\sem{\abs{every}\ \abs{man}} &= SCOPE\ (\mathrm{SL} \circ c_1)\ \eta \\
\sem{\abs{a}\ \abs{woman}} &= SCOPE_P\ (\mathrm{SL} \circ c_2)\ \eta
\end{align*}


\begin{align*}
  & \abs{loves}\ (\abs{a}\ \abs{woman})\ (\abs{every}\ \abs{man}) \\
  \rightarrow^*\ & \mathrm{SI}\ (SCOPE\ (\mathrm{SL} \circ c_1)\ 
  (\lambda x.\ SCOPE_P\ (\mathrm{SL} \circ c_2)\
  (\lambda y.\ \eta\ (\obj{love}\ x\ y)))) \\
  \rightarrow^*\ & \mathrm{SL}\ (c_1\
  (\lambda x.\ \mathrm{SI}\ (SCOPE_P\ (\mathrm{SL} \circ c_2)\
  (\lambda y.\ \eta\ (\obj{love}\ x\ y))))) \\
  \rightarrow^*\ & \mathrm{SL}\ (c_1\
  (\lambda x.\ \mathrm{SL}\ (c_2\ 
  (\lambda y.\ \mathrm{SI}\ (\eta\ (\obj{love}\ x\ y)))) +
  SCOPE_P\ (\mathrm{SL} \circ c_2)\
  (\lambda y.\ \mathrm{SI}\ (\eta\ (\obj{love}\ x\ y))))) \\
  \rightarrow^*\ & \mathrm{SL}\ (c_1\
  (\lambda x.\ \mathrm{SL}\ (c_2\ 
  (\lambda y.\ \eta\ (\obj{love}\ x\ y))) +
  SCOPE_P\ (\mathrm{SL} \circ c_2)\
  (\lambda y.\ \eta\ (\obj{love}\ x\ y)))) \\
  \rightarrow^*\ & \mathrm{SL}\ (c_1\ (\lambda x.\ 
  \mathrm{SL}\ (\eta\ (\exists y.\ \obj{woman}\ y \land \obj{love}\ x\ y) +
  SCOPE_P\ (\mathrm{SL} \circ c_2)\
  (\lambda y.\ \eta\ (\obj{love}\ x\ y)))) \\
  \rightarrow^*\ & \mathrm{SL}\ (c_1\ (\lambda x.\ 
  (\eta\ (\exists y.\ \obj{woman}\ y \land \obj{love}\ x\ y) +
  SCOPE_P\ (\mathrm{SL} \circ c_2)\
  (\lambda y.\ \eta\ (\obj{love}\ x\ y)))) \\
  \rightarrow^*\ & \mathrm{SL}\ (\eta\ (\forall x.\ \obj{man}\ x
  \rightarrow (\exists y.\ \obj{woman}\ y \land \obj{love}\ x\ y)) \\
  & \hspace{7mm} + SCOPE_P\ (\mathrm{SL} \circ c_2)\ (\lambda y.\ 
  \eta\ (\forall x.\ \obj{man}\ x \rightarrow \obj{love}\ x\ y))) \\
  \rightarrow^*\ & \mathrm{SL}\ (\eta\ (\forall x.\ \obj{man}\ x
  \rightarrow (\exists y.\ \obj{woman}\ y \land \obj{love}\ x\ y))) \\
  +\ & \mathrm{SL}\ (SCOPE_P\ (\mathrm{SL} \circ c_2)\ (\lambda y.\
  \eta\ (\forall x.\ \obj{man}\ x \rightarrow \obj{love}\ x\ y))) \\
  \rightarrow^*\ & \eta\ (\forall x.\ \obj{man}\ x
  \rightarrow (\exists y.\ \obj{woman}\ y \land \obj{love}\ x\ y)) \\
  +\ & \mathrm{SL}\ (c_2\ (\lambda y.\ \eta\ (\forall x.\
  \obj{man}\ x \rightarrow \obj{love}\ x\ y))) \\
  +\ & SCOPE_P\ (\mathrm{SL} \circ c_2)\ (\lambda y.\ \eta\ (\forall x.\
  \obj{man}\ x \rightarrow \obj{love}\ x\ y))) \\
  \rightarrow^*\ & \eta\ (\forall x.\ \obj{man}\ x
  \rightarrow (\exists y.\ \obj{woman}\ y \land \obj{love}\ x\ y)) \\
  +\ & \mathrm{SL}\ (\eta\ (\exists y.\ \obj{woman}\ y \land (\forall x.\
  \obj{man}\ x \rightarrow \obj{love}\ x\ y))) \\
  +\ & SCOPE_P\ (\mathrm{SL} \circ c_2)\ (\lambda y.\ \eta\ (\forall x.\
  \obj{man}\ x \rightarrow \obj{love}\ x\ y)) \\
  \rightarrow^*\ & \eta\ (\forall x.\ \obj{man}\ x
  \rightarrow (\exists y.\ \obj{woman}\ y \land \obj{love}\ x\ y)) \\
  +\ & \eta\ (\exists y.\ \obj{woman}\ y \land (\forall x.\
  \obj{man}\ x \rightarrow \obj{love}\ x\ y)) \\
  +\ & SCOPE_P\ (\mathrm{SL} \circ c_2)\ (\lambda y.\ \eta\ (\forall x.\
  \obj{man}\ x \rightarrow \obj{love}\ x\ y)) \\
\end{align*}

\end{document}
