\documentclass{beamer}

\usepackage[utf8]{inputenc}
\usepackage{pgfpages}
\usepackage{stmaryrd}
\usepackage{amsmath}
\usepackage{listings}
\usepackage{graphicx}
\usepackage[cmtip,all]{xy}
\newcommand{\longsquiggly}{\xymatrix{{}\ar@{~>}[r]&{}}}

\hypersetup{pdfstartview={Fit}}

\setbeamertemplate{navigation symbols}{}
\setbeamertemplate{footline}
  {\hfill {\normalsize \insertframenumber{}/\inserttotalframenumber{}}}

  
\newcommand{\hastype}{\mathop{:}}

\newcommand{\dand}{\mathbin{\overline{\land}}}
\newcommand{\dnot}{\mathop{\overline{\lnot}}}
\newcommand{\dimpl}{\mathbin{\overline{\to}}}
\newcommand{\dexists}{\mathop{\overline{\exists}}}
\newcommand{\dforall}{\mathop{\overline{\forall}}}

\newcommand{\hsbind}{\mathbin{\texttt{>>=}}}
\newcommand{\hsrevbind}{\mathbin{\texttt{=<<}}}
\newcommand{\hsseq}{\mathbin{\texttt{>>}}}
\newcommand{\occons}{\mathbin{::}}

\newcommand{\statecps}[3]{\llbracket #3 \rrbracket^{#2}_{#1}}
\newcommand{\cps}[2]{\llbracket #2 \rrbracket^{#1}}

\newcommand{\sem}[1]{\llbracket #1 \rrbracket}
\newcommand{\intens}[1]{\overline{#1}}

\newcommand{\obj}[1]{\text{Obj}(#1)}
\newcommand{\inl}[1]{\text{inl}(#1)}
\newcommand{\inr}[1]{\text{inr}(#1)}
\newcommand{\id}[1]{\text{id}_{#1}}



\begin{document}

\title[Effects \& Semantics]{Field Report}
\author{Jiří Maršík}
\institute[INRIA Loria]
{
  Sémagramme \\
  INRIA Loria
}
\date[February 2014]{March 27, 2014}

\frame{\titlepage \setcounter{framenumber}{1}}

\begin{frame}
  \frametitle{Original Topic}

  Large-scale grammar\ldots
  \pause
  \begin{itemize}
  \item of French and English
    \pause
  \item with semantics and pragmatics (discourse structure)
    \pause
  \item accounting for explicit (and later implicit) discourse connectives
    \pause
  \item providing temporal and spatial representations
  \end{itemize}
\pause
\vfill
  To be evaluated...
\pause
  \begin{itemize}
  \item on annotated corpora (Annodis)
    \pause
  \item in international challenges
    \pause
  \item on cognitive observations (SLAM)
  \end{itemize}
\end{frame}

\begin{frame}
  \frametitle{Current Point of Research}

  \textbf{Problem:} Combining isolated treatments of semantic phenomena
  \pause
  \vfill
  non-compositionality at the syntax-semantics interface \\
  $\longsquiggly$ impure code generators
  \pause
  \vfill
  open set of sources of non-compositionality \\
  $\longsquiggly$ algebraic effects and handlers
  \vfill
  \pause
  handlers in syntax-semantics interface:
  \begin{itemize}
  \item DRS - internally dynamic, externally static connectives
  \item scope islands (e.g. tensed clauses)
  \item presupposition (Katia's thesis)
  \item option operator for implicit arguments
  \item overtly moved complementizers and interrogatives
  \item scope domains for events (Landman)
  \end{itemize}
\end{frame}

\begin{frame}
  \frametitle{Future Goals}

  Finish what I started:
  \begin{itemize}
  \item Commit to a suitable calculus with handlers
  \item Reductions ``behind a $\lambda$'' $\longsquiggly$ staging
  \item Crossover + inverse scope = problem for evaluation order
  \item Transform the handler calculus into plain $\lambda$-calculus
    \begin{itemize}
    \item Can this be linear or parsable?
    \end{itemize}
  \end{itemize}
  \vfill
  \pause
  Going further:
  \begin{itemize}
  \item Re-entrant contexts
  \item More phenomena
  \item Maybe something from the original topic?
  \item Writing my thesis!
  \end{itemize}
\end{frame}

\end{document}
