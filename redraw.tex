\documentclass[a4paper,11pt]{article}

\usepackage[utf8]{inputenc}
\usepackage{listings}
\usepackage{color}
\usepackage{amsmath}
\usepackage{stmaryrd}
\usepackage{verbatim}
\usepackage{graphicx}
\usepackage{natbib}
\usepackage{gb4e}
\usepackage{bussproofs}


\newcommand{\dand}{\mathbin{\overline{\land}}}
\newcommand{\dnot}{\mathop{\overline{\lnot}}}
\newcommand{\dimpl}{\mathbin{\overline{\to}}}
\newcommand{\dexists}{\mathop{\overline{\exists}}}
\newcommand{\dforall}{\mathop{\overline{\forall}}}

\newcommand{\hsbind}{\mathbin{\texttt{>>=}}}
\newcommand{\hsrevbind}{\mathbin{\texttt{=<<}}}
\newcommand{\hsseq}{\mathbin{\texttt{>>}}}
\newcommand{\occons}{\mathbin{::}}

\newcommand{\statecps}[3]{\llbracket #3 \rrbracket^{#2}_{#1}}
\newcommand{\cps}[2]{\llbracket #2 \rrbracket^{#1}}

\newcommand{\sem}[1]{\llbracket #1 \rrbracket}
\newcommand{\intens}[1]{\overline{#1}}

\newcommand{\keyword}[1]{\texttt{#1}}
\newcommand{\effect}[1]{\textbf{#1}}
\newcommand{\semdom}[1]{\textbf{#1}}
\newcommand{\handle}[2]{\keyword{with}\ #1\ \keyword{handle}\ #2}

\def\limp {\mathbin{{-}\mkern-3.5mu{\circ}}}


\title{Pragmatic Side Effects}
\author{Jirka Maršík \and Maxime Amblard}


\begin{document}

\maketitle

\section{Introduction}

In the quest to give a formal compositional semantics to natural languages,
semanticists have started turning their attention to phenomena that have
been also considered as parts of pragmatics (e.g., discourse anaphora and
presupposition projection). To account for these phenomena, the very kinds
of meanings assigned to words and phrases are often revisited. To be more
specific, in the prevalent paradigm of modeling natural language
denotations using the simply-typed lambda calculus (higher-order logic)
this means revisiting the types of denotations assigned to individual parts
of speech.

However, the lambda calculus also serves as a fundamental theory of
computation and in the study of computation, similar type shifts have been
employed to give a meaning to side effects. Side effects in programming
languages correspond to actions that go beyond the lexical scope of an
expression (a thrown exception might propagate throughout a program, a
variable modified at one point might later be read at an another) or even
beyond the scope of the program itself (a program might interact with the
outside world by e.g., printing documents, making sounds, operating robotic
limbs\ldots).

\section{Side Effects and Pragmatics}

We now explore some of the parallels between side effects of programming
languages and the pragma(seman)tic phenomena of linguistics.

\subsection{Parallel Functions}

We notice that pragmatics seems to do a similar service to semantics as
does the study of side effects in the field of programming language
semantics. Discourse anaphora is an example of an action whose effect
transcends the lexical scope of the expressions involved (the referent and
the referring expression), similar to the way a mutable store bridges the
gap between a variable write and read instruction. Presuppositions can be
seen as propagating through the structure of the discourse until they are
either validated by some established or hypothesized knowledge or
accomodated at the correct level much like an exception is propagating
throughout a program until it is caught by some handler. Finally,
pragmatics is interested in how a linguistic system interacts with the
world of its users similar to how programs interact with the world of their
users through side effects.

\subsection{Parallel Theories}

When semanticists turn their attention to phenomena whose effects go beyond
the scope of their syntactic domains (e.g., )

We have started exploring the viability of the methods used to treat side
effects in programming language semantics to the study of the phenomena
that prompt natural language semanticists to generalize the types of their
denotations in similar ways. Our motivation is to analyse the interplay of
the different effects studied by semanticists (dynamics, modality,
presuppositions\ldots).


\bibliography{paper/effects-paper}
\bibliographystyle{plainnat}

\end{document}
