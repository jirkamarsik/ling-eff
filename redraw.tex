\documentclass[a4paper,11pt]{article}

\usepackage[utf8]{inputenc}
\usepackage{listings}
\usepackage{color}
\usepackage{amsmath}
\usepackage{stmaryrd}
\usepackage{verbatim}
\usepackage{graphicx}
\usepackage{natbib}
\usepackage{gb4e}
\usepackage{bussproofs}


\newcommand{\dand}{\mathbin{\overline{\land}}}
\newcommand{\dnot}{\mathop{\overline{\lnot}}}
\newcommand{\dimpl}{\mathbin{\overline{\to}}}
\newcommand{\dexists}{\mathop{\overline{\exists}}}
\newcommand{\dforall}{\mathop{\overline{\forall}}}

\newcommand{\hsbind}{\mathbin{\texttt{>>=}}}
\newcommand{\hsrevbind}{\mathbin{\texttt{=<<}}}
\newcommand{\hsseq}{\mathbin{\texttt{>>}}}
\newcommand{\occons}{\mathbin{::}}

\newcommand{\statecps}[3]{\llbracket #3 \rrbracket^{#2}_{#1}}
\newcommand{\cps}[2]{\llbracket #2 \rrbracket^{#1}}

\newcommand{\sem}[1]{\llbracket #1 \rrbracket}
\newcommand{\intens}[1]{\overline{#1}}

\newcommand{\keyword}[1]{\texttt{#1}}
\newcommand{\effect}[1]{\textbf{#1}}
\newcommand{\semdom}[1]{\textbf{#1}}
\newcommand{\handle}[2]{\keyword{with}\ #1\ \keyword{handle}\ #2}

\def\limp {\mathbin{{-}\mkern-3.5mu{\circ}}}


\title{Pragmasemantics as a Two-Step Process}
\author{Jirka Maršík \and Maxime Amblard}


\begin{document}

\maketitle


In the quest to give a formal compositional semantics to natural languages,
semanticists have turned their attention to phenomena that have been
considered as parts of pragmatics, namely discourse anaphora and
presupposition projection. To account for these phenomena, the very kinds
of meanings assigned to words and phrases are revisited. To be more
specific, in the prevalent paradigm of modeling natural language
denotations using the simply-typed lambda calculus (higher-order logic)
this means revisiting the types of denotations assigned to individual parts
of speech.




\bibliography{paper/effects-paper}
\bibliographystyle{plainnat}

\end{document}
