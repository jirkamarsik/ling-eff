\chapter{Composing the Effects}
\label{chap:composing-effects}

In Chapters~\ref{chap:introducing-effects}
and~\ref{chap:dynamic-semantics}, we have seen $\calc$ analyses of
deixis~(\ref{sec:deixis}), conventional
implicature~\ref{sec:conventional-implicature},
quantification~(\ref{sec:quantification}), anaphora~(\ref{sec:banana-drt})
and presupposition~(\ref{sec:presuppositions}). All of those analyses
shared the same structure: constituents of atomic abstract type were
interpreted as $\calc$ computations.\footnote{Note though that in
  Chapter~\ref{chap:introducing-effects}, we had $\sem{S} = \FF_E(o)$ and
  after the shift to dynamic semantics in
  Chapter~\ref{chap:dynamic-semantics}, we switched to
  $\sem{S} = \FF_E(1)$.} The fact that there is a common underlying
structure present in all of these phenomena has already been discovered by
the pioneers of monads in natural language
semantics~\cite{shan2002monads,charlow2014semantics,giorgolo2015natural,barker2015monads}. However,
the point of expressing all of these phenomena in a single framework is not
just to have a uniform set of fragments, but to also be able to combine
those fragments into a wider picture.

Most recently, proposals for combining monads in natural language semantics
have started
appearing~\cite{charlow2014semantics,giorgolo2015natural,barker2015monads}. We
can divide them into two strategies:

\begin{itemize}
\item Constructing a ``supermonad'' that combines the structural elements
  of all the monads that we want to use.

  The denotations of constituents are then all computations within this
  monad (or their denotations are injected into this supermonad). The
  monad's $\hsbind$ operator serves as the universal glue, allowing us to
  combine any two meanings, regardless of the effects they use. This
  approach was adopted by Charlow in his
  dissertation~\cite{charlow2014semantics}.
\item Making all the necessary components to build semantic glue lexical
  items in the grammar.

  Instead of deriving a general monad that encompasses all of the necessary
  monadic structure, we include the $\eta$ and $\hsbind$ of every monad we
  want to use into our grammar. Different lexical entries will use
  different (combinations of) monads and it is one of the duties of parsing
  to find the necessary semantic glue which combines these meanings in a
  sound way. This approach was proposed by Giorgolo and Asudeh in their
  ESSLLI 2015 course~\cite{giorgolo2015natural} and by Charlow in Barker's
  ESSLLI 2015 course~\cite{barker2015monads}.
\end{itemize}

Since ACGs separate the object-level types (where we use monads and
semantic types) from the abstract-level types (where we use linear
implications and syntactic types and where we control the set of valid
syntactic structures), the latter approach becomes impractical. If we
wanted to add the monadic combinators $\eta$ and $\hsbind$ and give them
their correct types, we would need to change the logic of the
abstract-level type system, e.g.\ by including a modality for every
monad. In doing this, we would no longer be working with ACGs, but with
some other type-logical grammar formalism, and we would lose the benefit of
existing results for ACGs (e.g.\ on the complexity of
parsing~\cite{kanazawa2007parsing}). Our approach is an instance of the
former method. Our ``supermonad'' is the free monad $\FF_E$
from~\ref{ssec:monad} and~\ref{ssec:free-monad}. In contrast to the
approach used in Charlow's dissertation~\cite{charlow2014semantics}, we
explore the use of effects and handlers instead of monad transformers and
we perform all computation in a formally defined object language $\calc$,
which is an extension of the simply-typed lambda calculus.

In~\ref{sec:presuppositions}, we have already seen how to combine our
treatment of anaphora with a treatment of presuppositions. We will extend
the fragment from Chapter~\ref{chap:dynamic-semantics} with the effects
from Chapter~\ref{chap:introducing-effects} using the same process that we
have used when adding presuppositions:

\begin{enumerate}
\item We will translate any closed handlers used in lexical entries into
  open handlers. While closed handlers can often be simpler to compute
  with, they are no longer applicable in the presence of other effects. For
  an example of this process, look in~\ref{ssec:revising-dynamic-handler},
  where we translate the $\BOX$ handler, which is used in entries that make
  use of dynamic negation, into an open handler.
\item We will update the existing entries in the grammar(s) to reflect any
  interactions between the lexical entries and phenomena present in both
  treatments. In the case of presupposition, we have made it so that
  contexts can cancel
  presuppositions~(\ref{ssec:cancelling-presuppositions}) and accommodate
  presuppositions~(\ref{ssec:ambiguous-accommodation}) by modifying the
  $\BOX$ (and later $\DBOX$) handler used in dynamic negation.
\end{enumerate}


\minitoc


\section{The Dynamic Kernel}

As per the methodology described in~\ref{sec:methodology}, we will start
with a basic ``seed'' grammar into which we will incorporate treatments of
diverse phenomena. The grammar that we will start with will be a small
dynamic fragment, very much like the one
in~\ref{sec:banana-drt}.\footnote{We could have also started with a more
  basic fragment without any effects and add dynamics separately. However,
  adding dynamics would force us to change almost all of the fragment:
  every individual meaning would need to add its referent to the context
  and every propositional meaning would need to add the proposition to the
  context. Furthermore, as we have seen
  in~\ref{ssec:truth-conditions-side-effects}, we would also change the
  type of sentence interpretations from $\FF_E(o)$ to $\FF_E(1)$. A very
  similar translation from a static grammar to a dynamic one is described
  in~\cite{lebedeva2012expression}~(Definition~4.27).} While
in~\ref{sec:banana-drt}, we tried to keep our dynamic grammar as close to
possible to the presentation of DRT in Kamp and Reyle's
textbook~\cite{kamp1993discourse}, the grammar that we will use here will
be more in line with the categorial tradition. Instead of using
non-lexicalized rules such as $\abs{trans} : VP \limp NP \limp NP \limp S$
and $\abs{common} : CN \limp N$ (which were standing in for the DRT
construction rules $\crlitv$ and $\crlin$), we will be using lexicalized
items such as $\abs{loves} : NP \limp NP \limp S$ and $\abs{man} : N$.

We first give the lexical items in our grammar and their syntactic types
(i.e.\ the abstract signature).

\begin{align*}
  \abs{she}, \abs{he}, \abs{it} &: NP \\
  \abs{a} &: N \limp NP \\
  \abs{man}, \abs{woman}, \abs{Porsche}, \abs{Mercedes} &: N \\
  \abs{loves}, \abs{owns}, \abs{fascinates} &: NP \limp NP \limp S \\
  \abs{not-the-case} &: S \limp S \\
  \abs{and}, \abs{if-then}, \abs{either-or} &: S \limp S \limp S \\
  \abs{\_.\_} &: D \limp S \limp D \\
  \abs{nil} &: D
\end{align*}

This grammar is about dynamic semantics and anaphora. Therefore, it
contains indefinites (indefinite article + common nouns) and pronouns to
showcase the introduction and retrieval of discourse referents. It also
contains entries for verbs which let use these noun phrases inside of
sentences. One of the key issues of dynamic semantics is how anaphora
interacts with logical operators. For these purposes, we include in our
grammar lexical items corresponding to constructions that mimic the logical
operators of negation, conjunction, implication and disjunction. Finally,
dynamic semantics studies how anaphora works across sentences, through a
discourse. We treat discourse as a list of sentences. We introduce an
atomic abstract type $D$ of discourses, an empty discourse $\abs{nil} : D$
and a discourse extension operator $\abs{\_.\_} : D \limp S \limp D$.

We will now give a semantic interpretation to the abstract language
generated by these items. First, we define the object signature that will
be the target of our interpretation.

\begin{align*}
  \top, \bot &: o \\
  \lnot &: o \to o \\
  (\_ \land \_), (\_ \to \_), (\_ \lor \_) &: o \to o \to o \\
  \exists, \forall &: (\iota \to o) \to o \\
  \_ = \_ &: \iota \to \iota \to o \\
  \obj{man}, \obj{woman}, \obj{Porsche}, \obj{Mercedes} &: \iota \to o \\
  \obj{love}, \obj{own}, \obj{fascinate} &: \iota \to \iota \to o \\
  \nil &: \gamma \\
  \_ \cons \_ &: \iota \to \gamma \to \gamma \\
  \_ \cons \_ &: o \to \gamma \to \gamma \\
  \_ \cat \_ &: \gamma \to \gamma \to \gamma \\
  \selhe, \selshe, \selit &: \gamma \to \iota
\end{align*}

We will have a type $o$ of propositions, which will be built out of
first-order logic (FOL) formulas. We include constants for all FOL
constructors: tautology ($\top$), contradiction ($\bot$), negation
($\lnot$), conjunction ($\land$), implication ($\to$), disjunction
($\lor$), existential quantification ($\exists$), universal quantification
($\forall$), equality on terms ($=$), unary predicates (corresponding to
common nouns) and binary predicates (corresponding to transitive verbs). We
aim to treat anaphoric binding but not anaphora resolution, and so we
introduce constants for operations that will perform anaphora resolution:
the oracles $\selhe$, $\selshe$ and $\selit$.\footnote{In
  Chapter~\ref{chap:dynamic-semantics}, we presented a simplified account
  of DRT that omitted gender features. We add them back in a limited form
  to make the examples more comprehensible.} The anaphora resolution
operators work on contexts, which contain all the knowledge in the common
grounds and all individuals available for discussion. These contexts, of
type $\gamma$, are built up using $\nil$, $\cons$ and $\cat$, where $\cons$
is overloaded to work both for individuals and propositions.

Now that we have defined the object signature into which we want to
interpret our abstract language, we are ready to lay down the
lexicon. We interpret the atomic abstract types as computations:

\begin{align*}
  \sem{NP} &= \FF_E(\iota) \\
  \sem{N} &= \FF_E(\iota \to o) \\
  \sem{S} &= \FF_E(1) \\
  \sem{D} &= \FF_E(1)
\end{align*}

The effect signature $E$ is the DRT effect signature
from~\ref{ssec:dynamic-negation}

\begin{align*}
  E = \{\ &\typedop{get}{1}{\gamma}, \\
          &\typedop{introduce}{1}{\iota}, \\
          &\typedop{assert}{o}{1}\ \}
\end{align*}

The interpretation of the constants in our abstract signature is given
next:

\begin{align*}
  \lex{she}{\begin{aligned}[t]
      &\app{\op{get}}{\star}{(\lam{e}{ \\
      &\etaE{(\selshe(e))}})}
    \end{aligned}} \\
  \lex{he}{\begin{aligned}[t]
      &\app{\op{get}}{\star}{(\lam{e}{ \\
      &\etaE{(\selhe(e))}})}
    \end{aligned}} \\
  &\vdots \\
  \lex{a}{\begin{aligned}[t]
      \lam{N}{&\app{\op{introduce}}{\star}{(\lam{x}{ \\
              &N \hsbind (\lam {n}{ \\
              &\app{\op{assert}}{(n(x))}{(\lam{\_}{ \\
              &\etaE{x}})}})})}}
    \end{aligned}} \\
  \lex{man}{\etaE{\obj{man}}} \\
  \lex{woman}{\etaE{\obj{woman}}} \\
  &\vdots \\
  \lex{loves}{\lam{O S}{(\obj{love} \apr S \aplr O) \hsbind \op{assert}!}} \\
  \lex{owns}{\lam{O S}{(\obj{own} \apr S \aplr O) \hsbind \op{assert}!}} \\
  &\vdots \\
  \lex{not-the-case}{\lam{A}{\dnot A}} \\
  \lex{and}{\lam{A B}{A \dand B}} \\
  \lex{if-then}{\lam{A B}{A \dimp B}} \\
  \lex{either-or}{\lam{A B}{A \dor B}} \\
  \lex{\_.\_}{\lam{D S}{D \hsbind (\lam{\_}{S})}} \\
  \lex{nil}{\etaE{\star}}
\end{align*}

The interpretations are almost the same as the ones given
in~\ref{sec:banana-drt}, with the following changes:

\begin{itemize}
\item We use the simplified entry for pronouns, derived
  in~\ref{ssec:algebraic-drt}, to which we add gender markings.
\item We also give interpretations to the two new constants, $\_.\_$ and
  $\abs{nil}$. We interpret the empty discourse $\abs{nil}$ as a discourse
  which contributes nothing, a trivial computation that immediately returns
  the dummy value $\star$. The discourse extension operator $\_.\_$ is
  interpreted the same as the dynamic conjunction $\dand$, by chaining the
  evaluation of its constituents.
\end{itemize}

In the interpretations, we make use of \emph{dynamic} logical operators
that work with propositions of type $\FF_E(1)$, most of which we have seen
in Chapter~\ref{chap:dynamic-semantics}. Below, we give the definition for
the complete set of first-order dynamic logical operators:

\begin{align*}
  A \dand B &= A \hsbind (\lam{\_}{B}) \\
  \dnot A &= \ap{\BOX}{A} \hsbind (\lam{a}{\ap{\op{assert}!}{(\lnot a)}}) \\
  \dexists x.\,A &= \app{\op{introduce}}{\star}{(\lam{x}{A})} \\
  A \dimp B &= \dnot (A \dand \dnot B) \\
  A \dor B &= \dnot (\dnot A \dand \dnot B) \\
  \dforall x.\,A &= \dnot (\dexists x.\,\dnot A)
\end{align*}

From their definitions, we can glean some of the dynamic characteristics of
these operators:

\begin{itemize}
\item In $A \dand B$, the effects of $A$ combine with and scope over the
  effects of $B$. The discourse referents introduced by $A$ are therefore
  accessible in $B$, i.e.\ $\dand$ is an \emph{internally dynamic}
  operator.
\item The existential quantifier
  $\dexists : (\iota \to \FF_E(1)) \to \FF_E(1)$ uses the $\op{introduce}$
  operation to scope over its continuation. This in combination with the
  previous fact allows us to derive the key equation of Dynamic Predicate
  Logic~\cite{groenendijk1991dynamic}
  $(\dexists x.\,A) \dand B = (\dexists x.\,A \dand B)$.\footnote{The
    equation follows from a single reduction using the $\op{op}.\hsbind$
    rule~(Property~\ref{prop:bind-rules}).}
\item The $\dnot A$ dynamic negation uses the $\BOX$ handler to interpret
  the dynamic operations in $A$ and therefore stop their projection. This
  makes $\dnot$ an \emph{externally static operator} since the dynamic
  effects of its argument do not project (i.e.\ are not accessible) out of
  the resulting proposition.
\item The last three operators are all headed by $\dnot$ and are therefore
  all externally static.
\item The left conjunct in the definition of $A \dor B$ is negated and
  $\dor$ is therefore internally static (discourse contributions of $A$ are
  not accessible in $B$). On the other hand, the left conjunct in the
  definition of $A \dimp B$ is not negated and $\dimp$ is therefore
  internally dynamic (as in the example ``\emph{If John owns a car$_1$,
    then it$_1$ is cheap}'').
\end{itemize}

The final piece of the puzzle is the $\BOX$ handler, which we have defined
in~\ref{ssec:revising-dynamic-handler} and which we repeat here.

\begin{align*}
  &\BOX :\FF_{E \uplus E_\DRT}(1) \to \FF_{E \uplus \{\op{get}\}}(o) \\
  &\BOX = \lam{A}{(\ap{\begin{aligned}[t]\lban
  &\onto{\op{get}}{(\lam{\_ k}{\etaE{(\lam{e}{\app{\op{get}}{\star}{(\lam{e'}{\ap{k}{(e \cat e')} \apll e})}})}})}, \\
  &\onto{\op{introduce}}{(\lam{\_ k}{\etaE{(\lam{e}{\existsr x.\ \ap{k}{x} \apll (x \cons e)})}})}, \\
  &\onto{\op{assert}}{(\lam{p k}{\etaE{(\lam{e}{p \andr (\ap{k}{\star} \apll (p \cons e))})}})}, \\
  &\onto{\eta}{(\lam{\_}{\etaE{(\lam{e}{\top})}})}\rban}{A}) \apll \nil
    \end{aligned}} \\
  \\
  &\_ \apll \_ : \FF_E(\alpha \to \FF_E(\beta)) \to \alpha \to \FF_E(\beta) \\
  &F \apll x = F \hsbind (\lam{f}{\ap{f}{x}}) \\
  &\existsr : (\iota \to \FF_E(o)) \to  \FF_E(o) \\
  &\existsr P = \exists \apr (\ap{\CC}{P})
\end{align*}

We have described the evolution of this handler in
Chapter~\ref{chap:dynamic-semantics}, so we will not go through the details
again. We will highlight just one thing, in connection to the definition of
$\dexists$ above. The $\dexists$ uses the $\op{introduce}$ and the
reduction rule $\op{op}.\hsbind$ tells us that operations project out of
computations
($\app{\op{op}}{M_\petitp}{(\lam{x}{M_\petitc})} \hsbind N
\to_{\op{op}.\hsbind} \app{\op{op}}{M_\petitp}{(\lam{x}{M_\petitc \hsbind
    N})}$). The $\BOX$ handler replaces the $\op{introduce}$ operation with
an existential quantifier. Using $\op{introduce}$ thus has the effect of
installing an existential quantifier at the scope of the nearest enclosing
box, much like the DRT construction rule for indefinites, $\crid$ (see
Figure~\ref{fig:crid}), introduces a discourse referent to the enclosing
DRS.

The objective of our semantics is to assign truth-conditions to
sentences. These truth-conditions are expressed as propositions, terms of
type $o$. As we progress, we will be adding more and more effects to
implement a compositional semantics for ``non-compositional'' phenomena
such as anaphora, presupposition and implicature. We will define a handler
that will strip away this extra structure and give us the truth-conditions
of a sentence in some default context.

\begin{align*}
  \EMPTY &: \FF_{E \uplus \{\op{get}\}}(\alpha) \to \FF_E(\alpha) \\
  \EMPTY &= \banana{\onto{\op{get}}{(\lam{\_ k}{\ap{k}{\nil}})}} \\
  \TOP &: \FF_{E \uplus E_\DRT}(1) \to \FF_{E}(o) \\
  \TOP &= \EMPTY \comp \BOX \\
  \\
  \cherry \comp \TOP &: \FF_{E_\DRT}(1) \to o
\end{align*}

The $\EMPTY$ handler evaluates a meaning in the empty context $\nil$,
interpreting away the $\op{get}$ operation. On the other hand, the $\BOX$
handler interprets away the $\op{introduce}$ and $\op{assert}$ operations
and so by composing them, we can interpret away all the effects in
$E_\DRT$. The resulting handler, $\TOP$, plays the role of a top-most
(top-level) DRS: it is a box in an empty context (i.e.\ there is no other
DRS that is accessible from this one).

Furthermore, if we look at the special case of the type of $\TOP$ when
$E = \emptyset$, we get a pure computation (type $\FF_\emptyset(o)$) as the
result. This means we can use the $\cherry$ operator to get at the result
directly. The $\cherry \comp \TOP$ combinator (pronounced ``cherry on
top'') gives us a formal way to associate a proposition to the denotation
of a sentence or discourse (remember that $\sem{S} = \sem{D} = \FF_E(1)$
with $E$ currently being $E_\DRT$).


\section{Adding Effects}
\label{sec:adding-effects}

We will now integrate the effects that we have seen in
Chapters~\ref{chap:introducing-effects} and~\ref{chap:dynamic-semantics}
into our dynamic grammar. We will start with the effect that we have
already seen interact with anaphora in Section~\ref{sec:presuppositions}.


\subsection{Adding Presupposition}
\label{ssec:adding-presupposition}

We will be enriching our fragment with the following referring expressions:
proper names such as \emph{John} and \emph{Mary}, possessive constructions
expressing ownership or other relations (\emph{$X$'s car}, \emph{$X$'s
  children}, \emph{$X$'s best friend}\ldots) and definite descriptions
(\emph{the car}). Here are the new entries into our abstract signature:

\begin{align*}
  \abs{John}, \abs{Mary} &: NP \\
  \abs{poss} &: NP \limp N \limp NP \\
  \abs{children}, \abs{best-friend} &: NP \limp NP \\
  \abs{the} &: N \limp NP
\end{align*}

In order to give a meaning to these constructions, we will need some extra
structure in our model. We will therefore add the following into our
object signature:

\begin{align*}
  \obj{John}, \obj{Mary} &: \iota \to o \\
  \obj{own}, \obj{children}, \obj{best-friend} &: \iota \to \iota \to o \\
  \selP &: (\iota \to o) \to \gamma \to \iota
\end{align*}

We represent proper names in our models as predicates. The idea behind a
predicate such as $\obj{John}$ is that $\ap{obj{John}}{x}$ should be true
for any $x$ which is called John.

Now we can describe how we extend our lexicon from abstract terms to object
terms so that it covers the new constructions. The interpretation of the
abstract types will stay the same, we will only change the effect signature
to include the following two operations:

\begin{align*}
\op{presuppose} &: (\iota \to o) \rightarrowtail \iota \\
\op{amb} &: 1 \rightarrowtail 2
\end{align*}

We extend the lexicon to the new constructions, using the new
$\op{presuppose}$ operation.

\begin{align*}
  \lex{John}{\ap{\op{presuppose}!}{\obj{John}}} \\
  \vdots \\
  \lex{poss}{\lam{X N}{X \hsbind (\lam{x}{N \hsbind (\lam{n}{\ap{\op{presuppose}!}{(\lam{y}{\ap{n}{y} \land \app{\obj{own}}{x}{y}})}})})}} \\
  \lex{children-of}{\lam{X}{X \hsbind (\lam{x}{\ap{\op{presuppose}!}{(\lam{y}{\app{\obj{children}}{y}{x}})}})}} \\
  \vdots \\
  \lex{the}{\lam{P}{P \hsbind (\lam{p}{\ap{\op{presuppose}!}{p}})}}
\end{align*}

These entries are the same as the ones we have seen in
Section~\ref{sec:presuppositions}. Besides introducing interpretations for
the new lexical items, we will also modify some of the existing
interpretations or combinators to reflect the interactions between the
existing effects and the effect being added. In this chapter, whenever we
will revise the interpretations of existing lexical items or the
definitions of existing combinators, we will use the $\coloneqq$ symbol and
if the right-hand side of the definition will make use of any of the
symbols being redefined, those symbols are meant to refer to the existing
definition.

\begin{align*}
  \TOP &: \FF_{E \uplus E_\DRT \uplus \{\op{presuppose}, \op{amb}\}}(1) \to \FF_{E}(o) \\
  \TOP &\coloneqq \search \comp \TOP \comp \accommodate \comp \useFind \\
       &= \search \comp \EMPTY \comp \BOX \comp \accommodate \comp \useFind \\
  \BOX &: \FF_{E \uplus E_\DRT \uplus \{\op{presuppose}\}}(1) \to \FF_{E \uplus \{\op{get}, \op{presuppose}, \op{amb}\}}(o) \\
  \BOX &\coloneqq \BOX \comp \maybeAccommodate \comp \useFind \\
\end{align*}

The changes proposed above account for the following features of
presuppositions:

\begin{itemize}
\item we add $\useFind$ to all boxes ($\BOX$), including also the topmost
  one ($\TOP$), so that presuppositions of expressions referring to
  entities already available in the context do not trigger presuppositions
\item we add $\maybeAccommodate$ to the $\BOX$ handler because a
  presupposition can be accommodated in any DRS on the projection line from
  the point where the presupposition was triggered
\item we add $\accommodate$ to the $\TOP$ handler because we want any
  presuppositions that have been neither cancelled nor (locally)
  accommodated to be accommodated at the top level
\item we add $\search$ to the $\TOP$ handler so that the proposition that
  we recover is the most preferred available reading of the sentence
  (w.r.t.\ the presupposition accommodation ambiguity
  in~\ref{ssec:ambiguous-accommodation})
\end{itemize}

Below, we give the definition of the handlers for the new $\op{presuppose}$
which account for the ways a presupposition can be eliminated: global
accommodation ($\accommodate$), local accommodation ($\maybeAccommodate$)
or cancellation ($\useFind$).

\begin{align*}
  \accommodate &: \FF_{E \uplus \{\op{presuppose}\}}(\alpha) \to \FF_E(\alpha) \\
  \accommodate &= \banana{\onto{\op{presuppose}}{(\lam{P k}{
    \app{\op{introduce}}{\star}{(\lam{x}{\app{\op{assert}}{(\ap{P}{x})}{(\lam{\_}{\ap{k}{x}})}})}})}} \\
  \maybeAccommodate &: \FF_{E \uplus \{\op{presuppose}\}}(\alpha) \to
                      \FF_{E \uplus \{\op{presuppose},\op{amb}\}}(\alpha) \\
  \maybeAccommodate &= \banana{\onto{\op{presuppose}}{(\lam{P k}{
    \app{\op{presuppose}}{P}{k} +
    \app{\op{introduce}}{\star}{(\lam{x}{\app{\op{assert}}{(\ap{P}{x})}{(\lam{\_}{\ap{k}{x}})}})}})}} \\
  \useFind &: \FF_{E \uplus \{\op{presuppose}\}}(\alpha) \to \FF_{E \uplus \{\op{get}, \op{presuppose}\}}(\alpha) \\
  \useFind &= \banana{\onto{\op{presuppose}}{(\lam{P k}{\ap{\find}{P} \hsbind k})}}
\end{align*}

These all come from Section~\ref{sec:presuppositions}, as well as the
definitions of $\find$ and $+$ given below.

\begin{align*}
  \find &: (\iota \to o) \to \FF_{E \uplus \{\op{get}, \op{presuppose}\}}(\iota) \\
  \find &= \lam{P}{\app{\op{get}}{\star}{(\lam{e}{\case{\selP(P, e)}{x}{\etaE{x}}{\_}{\ap{\op{presuppose}!}{P}}})}} \\
  \_+\_ &: \FF_{E \uplus \op{amb}}(\alpha) \to \FF_{E \uplus \op{amb}}(\alpha) \to \FF_{E \uplus \op{amb}}(\alpha) \\
  M + N &= \app{\op{amb}}{\star}{(\lam{b}{\ifte{b}{M}{N}})}
\end{align*}

Finally, we give the handler for the $\op{amb}$ effect.

\begin{align*}
  \search &: \FF_{E \uplus \{\op{amb}\}}(\alpha) \to \FF_E(\alpha) \\
  \search &= \banana{\onto{\op{amb}}{(\lam{\_ k}{\ap{k}{\true};\ \ap{k}{\false}})}}
\end{align*}

This uses the $M; N$ notation, whose typing and reduction rules were given
in Definition~\ref{def:semi-typing-rule} and
Definition~\ref{def:semi-reduction-rule}, respectively.


\subsection{Adding Conventional Implicature}
\label{ssec:adding-conventional-implicature}




\section{ACG Transformers Introducing Effects}
\subsection{Quantification}

\subsubsection{Common Nouns}

Our first objective will be to extend a grammar with quantificational
articles like \textit{every} and \textit{some}. However, this presupposes
that our grammar already has some common nouns to combine these articles
with. Since nouns themselves are not bound to the phenomenon of
quantification, we will introduce them in an isolated extension.

We will start by introducing some predicates at the object level (signature
$\Sigma^o_N$):

\begin{align*}
  \obj{man}, \obj{woman} &: \iota \to o
\end{align*}

Now for the abstract constants in signature $\Sigma^a_N$ (we treat
relativizers in our noun extension, since they form noun modifiers):

\begin{align*}
  \abs{man}, \abs{woman} &: N \\
  \abs{who}, \abs{whom} &: (NP \to S) \to N \to N
\end{align*}

Finally, we give an interpretation for the new abstract constants by
extending the lexicon with $\mathcal{L}_N$:

\begin{align*}
  \lex{man}{\eta\ \obj{man}} \\
  \lex{woman}{\eta\ \obj{woman}} \\
  \sem{\abs{who}} := \sem{\abs{whom}} &:= \lambda P N.\ N \hsbind (\lambda n.\ \mathcal{C}\ (\lambda x.\ (\eta\ (n\ x)) \dand (P\ (\eta\ x))))
\end{align*}

We have now finished our extension $E_N$.

$$
E_N(\left< \Sigma^a, \Sigma^o, \mathcal{L} \right>) = \left< \Sigma^a \uplus
\Sigma^a_N, \Sigma^o \uplus \Sigma^o_N, \mathcal{L} \uplus \mathcal{L}_N \right>
$$


\subsubsection{Quantifiers}

We will start our treatment of quantification by introducing an operation
symbol $SCOPE$ into our effect signature $E$. The operation will take
inputs of type $(\iota \to \mathcal{F}(o)) \to \mathcal{F}(o)$ and provide
outputs of type $\iota$ (i.e.\ $SCOPE : ((\iota \to \mathcal{F}(o)) \to
\mathcal{F}(o)) \to \iota \in E$). Our analysis will all revolve around
producing and handling this particular effect.

$$
E_Q(\left< \Sigma^a, \Sigma^o, \mathcal{L} \right>) = \left< \Sigma^a \uplus \Sigma^a_Q, \Sigma^o, \mathcal{L} \uplus \mathcal{L}_Q \right>
$$

First, we start by introducing the abstract constants for the new lexical
items (signature $\Sigma^a_Q$):

\begin{align*}
  \abs{every}, \abs{some}, \abs{a} &: N \to NP
\end{align*}

Now we can add/change the pertinant entries in the lexicon (lexicon $\mathcal{L}_Q$):

\begin{align*}
  \lex{every}{\lambda N.\ SCOPE\ (\lambda k.\ \dforall\ (\lambda x.\ (N \apl x) \dimpl (k\ x)))\ \eta} \\
  \sem{\abs{some}} := \sem{\abs{a}} &:= \lambda N.\ SCOPE\ (\lambda k.\ \dexists\ (\lambda x.\ (N \apl x) \dand (k\ x)))\ \eta \\
  \mathrm{SI} &: \mathcal{F}(o) \to \mathcal{F}(o) \\
  \mathrm{SI} &= [\mathcal{H}\ (SCOPE\ (\lambda c k.\ c\ k))] \\
  \lex{loves}{\lambda X Y.\ \mathrm{SI}\ (\mathcal{L}(\abs{loves})\ X\ Y)} \\
  \lex{knows}{\lambda X Y.\ \mathrm{SI}\ (\mathcal{L}(\abs{knows})\ X\ Y)} \\
  \lex{says}{\lambda S X.\ \mathrm{SI}\ (\mathcal{L}(\abs{says})\ S\ X)} \\
\end{align*}

We have given semantics to the new lexical items which make use of the
$SCOPE$ effect. Quantifier scope is influenced by other lexical items
resulting in so-called scope islands. We therefore define an auxiliary
term, a handler called $\mathrm{SI}$. Since tensed clauses form scope
islands, we insert $\mathrm{SI}$ into the existing lexical entries
for tensed verbs.

If we had a more fine-grained grammar which would provide a separate
lexical entry for the finite verb morpheme, we could simply add the
$\mathrm{SI}$ handler there without having to change all verbs.

A similar solution would work in an unlexicalized grammar in which, e.g.,
transitive verbs would be decomposed into a computation of some binary
relation of abstract type $TV$ and a combinator/rule for forming sentences
using transitive verbs of abstract type $NP \to TV \to NP \to S$. In that
case, $\mathrm{SI}$ would be inserted into the semantics of every
such rule which is to be considered a scope island.

