\chapter{Introducing the Effects}

\setcounter{exx}{0}

In this chapter, we will take a miniature fragment and extend it in
different directions, studying various linguistic phenomena. The common
point in all of these analyses is that they will all rely on the notion of
computations introduced in $\banana{\lambda}$.

We will start by first taking the initial fragment and lift its usual
semantics into the domain of computations~(\ref{sec:lifting-semantics}), so
as to be compatible with the development in the following sections. We will
then consider the semantics of deictic pronouns~(\ref{sec:deixis}),
appositives~(\ref{sec:conventional-implicature}) and quantificational noun
phrases~(\ref{sec:quantification}). After a tour of these analyses, we take
a step back and reflect on the methodological process that underlied their
development~(\ref{sec:methodology}). We then end the chapter with the
analysis of a more involved phenomenon: dynamicity and anaphora.

\minitoc


\section{Lifting Semantics into Computations}
\label{sec:lifting-semantics}

Let us start with a very tiny fragment with proper names to name
individuals and verbs to act as predicates over these individuals:

\begin{align*}
  \abs{John}, \abs{Mary} &: NP \\
  \abs{loves} &: NP \limp NP \limp S
\end{align*}

In this tiny fragment of proper names and predicates, we could interpret
noun phrases as individuals ($\sem{NP} = \iota$) and sentences as
propositions ($\sem{S} = o$). Provided we have some constants
$\obj{j} : \iota$ and $\obj{m} : \iota$ and a predicate
$\obj{love} : \iota \to \iota \to o$, we can give the following semantics
to these items:

\begin{align*}
  \lex{John}{\obj{j}} \\
  \lex{Mary}{\obj{m}} \\
  \lex{loves}{\lam{o s}{\app{\obj{love}}{s}{o}}}
\end{align*}

This interpretation works fine for simple sentences such as \emph{John
  loves Mary}.

However, the denotations that we will assign to noun phrases that are
deictic or quantificational will not fit into the type $\iota$. Instead, we
will interpret these constituents as \emph{computations} producing
individuals, type $\FF_E(\iota)$. In order to satisfy the homomorphism
property of ACGs and to have a sound syntax-semantics interface, we will
need to lift the denotations of these basic constants and predicates into
computations. This is very much like the case when one introduces
quantified noun phrases and switches from using generalized quantifiers
instead of simple individuals as denotations of noun phrases.

We will want linguistic expressions to denote computations. One systematic
way to achieve that is to say that the atomic types of our abstract
syntactic signature should be interpreted as computations. We will write
$\sem{-}_\petitv$ for the semantic interpretation using simple values and
$\sem{-}_\petitc$ for the semantic interpretation using computations. On
the type level, we will define
$\sem{\alpha}_\petitc = \FF_E(\sem{\alpha}_\petitv)$. By applying this to
the common Montagovian interpretation, we get:

\begin{align*}
  \sem{S}_\petitv &= o & \sem{S}_\petitc &= \FF_E(o) \\
  \sem{NP}_\petitv &= \iota & \sem{NP}_\petitc &= \FF_E(\iota) \\
  \sem{N}_\petitv &= \iota \to o & \sem{N}_\petitc &= \FF_E(\iota \to o)
\end{align*}

To lift the denotations of noun phrases from $\sem{NP}_\petitv = \iota$ to
$\sem{NP}_\petitc = \FF_E(\iota)$, it suffices to use $\eta$ to inject
$\iota$ inside of $\FF_E(\iota)$. This goes the same for any other lexical
item whose abstract type is an atomic type. For syntactic constructors that
take arguments, such as verbs or adjectives, we will chain the computations
of their arguments and apply the meaning of the constructor to the meaning
of the results of these computations. We will limit ourselves to syntactic
constructors of \emph{second-order type}, i.e.\ abstract constants whose
type is $a_1 \limp \cdots a_n \limp b$ where all $a_i$ and $b$ are atomic
types. If we use higher-order syntactic constructors, we will give them a
bespoke semantics.

\begin{align*}
  \liftl_\alpha &: \sem{\alpha}_\petitv \to \sem{\alpha}_\petitc \\
  \liftl_a(x) &= \etaE{x} \\
  \liftl_{a \limp \beta}(f) &= \lam{X}{X \hsbind (\lam{x}{\liftl_\beta(\ap{f}{x})})}
\end{align*}

This particular schema chains the evaluation of its arguments from
left-to-right (as can be seen by looking at the expanded non-recursive
definition below). While indexicality is order-independent, some of the
effects that we will introduce later (such as anaphora) are
order-dependent. The order that we would like to reflect in the evaluation
is the linear lexical order in which elements appear in the spoken/written
form of the sentence. Since in categorial grammars of English, it is often
the case that operators first take their complements from the right and
then apply to their argument on the left (e.g.\ transitive verbs or
relative pronouns of type $(NP \limp S) \limp N \limp N$), we will often
like to chain the evaluation of the arguments in the order opposite to the
one in which we receive them. This will give rise to the $\liftr$
transformation.

\begin{align*}
  \liftl_{a_1 \limp \cdots \limp a_n \limp b}(f) &= \lam{X_1 \ldots X_n}{X_1 \hsbind (\lam{x_1}{\ldots\ X_n \hsbind (\lam{x_n}{\etaE{(\ap{f}{x_1\ \ldots\ x_n})}})})} \\
  \liftr_{a_1 \limp \cdots \limp a_n \limp b}(f) &= \lam{X_1 \ldots X_n}{X_n \hsbind (\lam{x_n}{\ldots\ X_1 \hsbind (\lam{x_1}{\etaE{(\ap{f}{x_1\ \ldots\ x_n})}})})} \\
\end{align*}

With these in hand, we can now lift the interpretations of our simple
fragment into computations:

\begin{align*}
\sem{\abs{John}}_\petitc &= \liftl_{NP}(\obj{j}) \\
&= \etaE{\obj{j}} \\
\sem{\abs{Mary}}_\petitc &= \liftl_{NP}(\obj{m}) \\
&= \etaE{\obj{m}} \\
\sem{\abs{loves}}_\petitc &= \liftr_{NP \limp NP \limp S}(\lam{o s}{\app{\obj{love}}{s}{o}}) \\
&= \lam{O S}{S \hsbind (\lam{s}{O \hsbind (\lam{o}{\app{\obj{love}}{s}{o}})})} \\
&= \lam{O S}{\obj{love} \apr S \aplr O} \\
&= \lam{O S}{\etaE{\obj{love}} \aplr S \aplr O}
\end{align*}

In the lexical entry for $\abs{loves}$, we can express the series of binds
using the application operators from~\ref{ssec:composing-functions}. The
idea behind the notation is that you are supposed to put double brackets on
a side whenever the argument on that side is a computation. In our example,
$S$ and $O$ are both computations and $\obj{love}$ is a pure function
term. We can expand this term so that it is a bit more regular by making
$\obj{love}$ into a computation as well. We can then a notice a connection
between the denotations in $\sem{-}_\petitv$ and the denotations in
$\sem{-}_\petitc$. The $\sem{-}_\petitc$ denotations are just the
$\sem{-}_\petitv$ denotations where every constant $\obj{c}$ was replaced
with $\etaE{\obj{c}}$ and every application $\ap{f}{x}$ was replaced with
$f \aplr x$. Note that $\eta$ and $\aplr$ make up the applicative functor
$\FF_E$ (see~\ref{ssec:applicative-functor}).

We can prove that adding this computation layer does not affect the
predictions that this semantics makes.

\begin{lemma}\label{lem:second-order-no-abstractions}
  (\demph{Second-order terms hide no abstractions})

  Every $\beta$-normal term $\Gamma \vdash M : \tau$ of second-order type
  $\tau$ in the simply-typed lambda calculus is of the form
  $\lam{x_1 \ldots x_n}{N}$ where $N$ contains no abstractions, provided
  that any constants used have only a second-order type and any variables
  present in $\Gamma$ have a first-order (i.e.\ atomic) type.
\end{lemma}

\begin{proof}
  Let $M = \lam{x_1 \ldots x_n}{N}$ where $N$ is not an abstraction ($n$
  might be $0$). We will need to prove that $N$ contains no abstractions.

  We will use proof by contradiction. Let us assume that $N$ contains some
  abstractions. We order the subterms of $N$ left-to-right, depth-first.
  Let $N' = \lam{x}{N''}$ be the first abstraction we find in this
  traversal. $N'$ must be a proper subterm of $N$ since $N$ is known not to
  be an abstraction.
  
  We now consider the contexts in which $N'$ can occur:

  \begin{itemize}
  \item $N'$ occurs as the function in an application $\ap{N'}{A}$
    
    This is in contradiction with $M$ being a $\beta$-normal form since we
    have an abstraction in function position, i.e.\ a $\beta$-redex.

  \item $N'$ occurs as the argument in an application $\ap{F}{N'}$
    
    Since we know that the function that we are applying to $N'$ does not
    contain any abstractions since $N'$ is the first abstraction we
    encountered in the left-to-right ordering of subterms. Furthermore,
    $N'$ is the first abstraction in a depth-first ordering and so the only
    (free) variables that occur in $F$ are the variables $x_1 \ldots x_n$
    and the variables in $\Gamma$, all of which are of first-order
    type. $F$ is therefore not a variable (since it must have a functional
    type). $F$ is either a constant or an application. Furthermore, if $F$
    is an application, we can apply the same reasoning and by induction
    conclude that $F$ must be a constant $c$ applied to some $n$ arguments,
    $F = \ap{c}{A_1\ \ldots\ A_n}$, with $n$ possibly being $0$. However,
    since $c$ is of second-order type, all of its arguments are of
    first-order type. This is in contradiction with $N'$ being an
    abstraction.

  \item $N'$ occurs as the body of an abstraction $\lam{y}{N'}$
    
    This is in contradiction with $N'$ being the first abstraction we
    encounter in the depth-first order.
  \end{itemize}
\end{proof}

\begin{corollary}\label{cor:second-order-trees}
  (\demph{Second order terms are trees})
  
  Assuming all constants are of second-order type, every closed term
  $\vdash M : \nu$ of atomic type $\nu$ has a $\beta$-normal form
  $\ap{c}{M_1 \ldots M_n}$ where $M_i$ are also closed terms of atomic
  types.
\end{corollary}

\begin{proof}
  From Lemma~\ref{lem:second-order-no-abstractions}, we know that the
  normal form of $M$ is of the shape $\lam{x_1 \ldots x_n}{N}$ where $N$
  contains no abstraction. Furthermore, we know that $M$ is of an atomic
  type, therefore $n = 0$ and the normal form is just $N$. Since $N$ is
  closed, it contains no free variables, and since it is abstraction-free,
  it contains no bound variables either. Note that this is also the case
  for every subterm of $N$.
  
  $N$ is composed entirely of constants and applications. If an $N$ is an
  application, then the function is either a constant or some smaller
  application composed entirely of constants and applications. By
  induction, we thus show that $M$'s normal form, $N$, is of the shape
  $\ap{c}{M_1\ \ldots\ M_n}$. Since $c$ has a second-order type, all of its
  arguments must have a first-order, atomic, type.
\end{proof}

\begin{observation}
  (\demph{Conservativity of lifting})
  
  Let $\left< \Sigma_\petita, \Sigma_\petito, \sem{-}_\petitv, S \right>$
  be an ACG where every constant $c : \tau \in \Sigma_\petita$ is of at
  most second-order type ($\tau = a_1 \limp \ldots \limp a_n \limp b$ where
  all $a_i$ and $b$ are atomic types) and let
  $\left< \Sigma_\petita, \Sigma_\petito, \sem{-}_\petitc, S \right>$ be
  the ACG whose lexicon $\sem{-}_\petitc$ satisfies the following
  conditions:

  \begin{itemize}
  \item there exists some effect signature $E$ such that for every abstract
    atomic type $\tau \in \Sigma_\petita$,
    $\sem{\tau}_\petitc = \FF_E(\sem{\tau}_\petitv)$
  \item for every abstract constant $c : \tau \in \Sigma_\petita$,
    $\sem{c}_\petitc = \lift_\tau(\sem{c}_\petitv)$ where $\lift$ is either
    $\liftl$ or $\liftr$\footnote{This result is analogous to Barker's
      Simulation Theorem for the Continuation Schema
      in~\cite{barker2002continuations}. As in Barker's theorem, this
      result holds not only for $\liftl$ and $\liftr$ but to liftings that
      arbitrarily permute the evaluation order of their arguments.}
  \end{itemize}
  
  Then for every closed well-typed abstract term
  $\vdash_{\Sigma_\petita} M : \nu$ where $\nu$ is an atomic abstract type
  from $\Sigma_\petita$, we have:

  $$
  \sem{M}_\petitc = \etaE{(\sem{M}_\petitv)}
  $$
\end{observation}

\begin{proof}
  From Corollary~\ref{cor:second-order-trees}, we know that $M$ can be
  $\beta$-converted to a term of the form $\ap{c}{M_1\ \ldots\ M_n}$ where
  all $M_i$ are also closed abstract terms of atomic type. Let
  $a_1 \limp \ldots a_n \limp \nu$ be the type of $c$ in $\Sigma_\petita$.

  We will proceed by induction on the structure of this normal form:

  \begin{itemize}
  \item $M = c$ with $c : \nu \in \Sigma_\petita$
    
    In that case, we have the desired property by definition of
    $\sem{-}_\petitc$.
  
  \item $M = \ap{c}{M_1\ \ldots\ M_n}$ with $n > 0$

    First, we apply $\sem{-}_\petitc$ to $M$ and make use of the induction
    hypothesis for $M_i$:

    \begin{align*}
      \sem{\ap{c}{M_1\ \ldots\ M_n}}_\petitc
      &= \ap{\sem{c}_\petitc}{\sem{M_1}_\petitc\ \ldots\ \sem{M_n}_\petitc} \\
      &= \ap{\lift_{a_1 \limp \ldots a_n \limp \nu}(\sem{c}_\petitv)}
            {(\etaE{\sem{M_1}_\petitv})\ \ldots\ (\etaE{\sem{M_n}_\petitv})}
    \end{align*}
    
    If $\lift$ is $\liftl$, then:

    \begin{align*}
      &\ap{\liftl_{a_1 \limp \ldots a_n \limp \nu}(\sem{c}_\petitv)}
         {(\etaE{\sem{M_1}_\petitv})\ \ldots\ (\etaE{\sem{M_n}_\petitv})} \\
      =\ &\ap{(\lam{X_1 \ldots X_n}{X_1 \hsbind (\lam{x_1}{\ldots\ 
                                   X_n \hsbind (\lam{x_n}{
                   \etaE{(\ap{\sem{c}_\petitv}{x_1\ \ldots\ x_n})}})})})}
            {(\etaE{\sem{M_1}_\petitv})\ \ldots\ (\etaE{\sem{M_n}_\petitv})} \\
      =\ &(\etaE{\sem{M_1}_\petitv}) \hsbind (\lam{x_1}{\ldots\ 
         (\etaE{\sem{M_n}_\petitv}) \hsbind (\lam{x_n}{
              \etaE{(\ap{\sem{c}_\petitv}{x_1\ \ldots\ x_n})}})}) \\
      =\ &\etaE{(\ap{\sem{c}_\petitv}{\sem{M_1}_\petitv\ \ldots\ \sem{M_n}_\petitv})} \\
      =\ &\etaE{\sem{\ap{c}{M_1\ \ldots\ M_n}}_\petitv} \\
    \end{align*}
    
    Similarly, if $\lift$ is $\liftr$, then:
    
    \begin{align*}
      &\ap{\liftr_{a_1 \limp \ldots a_n \limp \nu}(\sem{c}_\petitv)}
         {(\etaE{\sem{M_1}_\petitv})\ \ldots\ (\etaE{\sem{M_n}_\petitv})} \\
      =\ &\ap{(\lam{X_1 \ldots X_n}{X_n \hsbind (\lam{x_n}{\ldots\ 
                                   X_1 \hsbind (\lam{x_1}{
                   \etaE{(\ap{\sem{c}_\petitv}{x_1\ \ldots\ x_n})}})})})}
            {(\etaE{\sem{M_1}_\petitv})\ \ldots\ (\etaE{\sem{M_n}_\petitv})} \\
      =\ &(\etaE{\sem{M_n}_\petitv}) \hsbind (\lam{x_n}{\ldots\ 
         (\etaE{\sem{M_1}_\petitv}) \hsbind (\lam{x_1}{
              \etaE{(\ap{\sem{c}_\petitv}{x_1\ \ldots\ x_n})}})}) \\
      =\ &\etaE{(\ap{\sem{c}_\petitv}{\sem{M_1}_\petitv\ \ldots\ \sem{M_n}_\petitv})} \\
      =\ &\etaE{\sem{\ap{c}{M_1\ \ldots\ M_n}}_\petitv} \\
    \end{align*}
  \end{itemize}
\end{proof}

\begin{corollary}
  For every term $M$ in the abstract language of a second-order ACG with an
  atomic distinguished type $S$, we have:

  $$
  \sem{M}_\petitc = \etaE{(\sem{M}_\petitv)}
  $$
\end{corollary}


\section{Deixis}
\label{sec:deixis}

The first phenomenon that we will speak about is
\emph{deixis}~\cite{levinson2004deixis}. Deictic expressions is the class
of expressions that depend on the time and place of the utterance, the
speaker and the addressee and any kind of pointing/presenting the speaker
might be doing to draw the attention of the addressee. These expressions
include personal pronouns, temporal expressions, tenses, demonstratives and
others. All of these are characterized by their dependence on the
extra-linguistic context. In this section, we will restrict our attention
to a very limited subset of these expressions: singular first-person
pronouns (\emph{I}, \emph{me}).

\begin{align*}
\abs{I} &: NP \\
\abs{me} &: NP
\end{align*}

The meanings that we assign to expressions in natural languages must
reflect this context-sensitivity: the truth-conditions of \emph{Mary loves
  me} change when it is pronounced by John and when by
Peter. Montague~\cite{montague1973proper} achieved this by having the
meaning of every expression depend on a point of reference: a pair of a
possible world and a moment in time (i.e.\ the modal \emph{where} and the
\emph{when} of the utterance). To model the first-person pronouns, we will
need to have our meanings depend on the identity of the speaker.

In the case of a deictic expression like the first-person pronoun, we have
an expression whose referent cannot be determined solely from its form and
the meaning of its parts. We will need to reach out into the context and it
is for this that we will be using the \emph{operations} in
$\banana{\lambda}$. The first-person pronoun has an interaction with its
context, which consists of asking the context for the identity of the
speaker. For this kind of interaction with the context, we will introduce
an operation symbol, $\op{speaker}$. We will also fix the symbol's input
and output types. The input type represents the information and/or the
parameters that the denotation of the first-person pronoun or any other
expression necessitating the identity of the speaker will need to provide
to the context. Since we have no information or parameter to give to the
context, we will use the trivial input type $1$, whose only value is
$\star$. The output type represents the information that the context will
provide us in return. We are interested in the identity of the speaker and
so the type of this information will be the type of individuals, $\iota$.

We can now model the meaning of a first-person pronoun as a computation of
type $\FF_E(\iota)$ that interacts with the context and produces a referent
of type $\iota$. The effect signature $E$ can be any signature provided
that $\typedop{speaker}{1}{\iota} \in E$.

\begin{align*}
  \lex{I}{\app{\op{speaker}}{\star}{(\lam{x}{\etaE{x}})}} \\
  &= \ap{\op{speaker}!}{\star} \\
  \lex{me}{\ap{\op{speaker}!}{\star}}
\end{align*}

The denotations of $\abs{I}$ and $\abs{me}$ demand the context for the
identity of the speaker $x$ using operation $\op{speaker}$ and then declare
that $x$ to be the referent of the pronoun. Taking the output of an
operation and then immediately returning it as the result of the
computation will be a common pattern and so we use the $\op{speaker}!$
shorthand introduced in~\ref{ssec:operations-and-handlers}.

Now the question is how to use the denotation given above to build meanings
of sentences containing first-person pronouns, e.g.\ \emph{Mary loves
  me}. In Montague's use of points of reference, Montague introduces an
intermediate language of intensional logic~\cite{montague1973proper}. When
Montague then gives a interpretation to this language, the point of
reference at which an expression is to be evaluated is passed through to
its subexpressions. We will be making use of the lifted semantics
introduced in the previous section. The chaining of the computations will
serve our purpose of propagating the indexicality of the noun phrase to the
sentence containing (or vice versa, the propagation of the sentence's
speaker to the noun phrase contained within).

With the interpretations given before, we can now analyse the following
sentences:

\begin{exe}
  \ex John loves Mary. \label{ex:trivial}
  \ex John loves me. \label{ex:deixis}
\end{exe}

whose meanings we can calculate as:

\NoChapterPrefix
\begin{align}
  \sem{\app{\abs{loves}}{\abs{Mary}}{\abs{John}}} & \tto 
  \etaE{(\app{\obj{love}}{\obj{j}}{\obj{m}})} \\
  \sem{\app{\abs{loves}}{\abs{me}}{\abs{John}}} & \tto
  \app{\op{speaker}}{\star}{(\lam{x}{\etaE{(\app{\obj{love}}{\obj{j}}{x})}})}
\end{align}
\ChapterPrefix

For~\eqref{ex:trivial}, we get a pure computation that produces the
proposition $\app{\obj{love}}{\obj{j}}{\obj{m}}$, which is the same
proposition that the sentence denoted before we modified the fragment to
use computations. However, in the case of~\eqref{ex:deixis}, we do not have
any single proposition as the denotation. Instead, we have a request to
identify the speaker of the utterance and then we have a different
proposition $\app{\obj{love}}{\obj{j}}{x}$ for each possible speaker
$x$. The truth conditions of this sentence can only be found by considering
some hypothetical speaker $s$. Given such a speaker, we could resolve all
of the requests for the speaker's identity and since our effect signature
$E$ does not contain any other operations, arrive at the desired
truth-conditions. This function that will interpret the $\op{speaker}$
operation symbols will be a handler.

\begin{align*}
  \withSpeaker &: \iota \to \FF_{\{\typedop{speaker}{1}{\iota}\} \uplus E}(\alpha) \to \FF_E(\alpha) \\
  \withSpeaker &= \lam{s}{\banana{\onto{\op{speaker}}{(\lam{x k}{\ap{k}{s}})}}}
\end{align*}

The type tells us that $\ap{\withSpeaker}{s}$ is a handler for the
$\op{speaker}$ operation. It takes any computation of type
$\FF_{\{\typedop{speaker}{1}{\iota}\} \uplus E}(\alpha)$ and gives back a
computation of type $\FF_E(\alpha)$, in which $\op{speaker}$ will not be
used. Since $\op{speaker}$ is the only operation in our effect signature,
by applying this handler to the denotation of a sentence in our fragment,
we get a denotation of type $\FF_\emptyset(o)$, which is isomorphic to
$o$\footnote{The two directions of the isomorphism are given by
  $\cherry : \FF_\emptyset(\alpha) \to \alpha$ and
  $\eta : \alpha \to \FF_\emptyset(\alpha).$}.

$$
  \app{\withSpeaker}{s}{\sem{\app{\abs{loves}}{\abs{me}}{\abs{John}}}} \tto
  \etaE{(\app{\obj{love}}{\obj{j}}{s})}
$$


\subsection{Quotations}

Up to now, we could have assumed that at the object level, we have a
constant $\obj{speaker}$ standing in for the speaker. After adding a
constant to our logical signature, our new models would have
interpretations for symbols in the original signature and for the
$\obj{speaker}$ constant, i.e.\ our models would become descriptions of the
world paired with some deictic index. However, removing the notion of a
context and making the speaker be a part of the model would make it
difficult to analyze the difference between the following two sentences:

\begin{exe}
  \ex John said Mary loves me. \label{ex:indirect-speech}
  \ex John said ``Mary loves me''. \label{ex:direct-speech}
\end{exe}

In our setting, we can model this kind of behavior since the $\withSpeaker$
is not a meta-level operation, but it is a term in our calculus like any
other. We can therefore have lexical entries for direct and indirect speech
that will interact differently with deictic expressions.

\begin{align*}
  \abs{said}_{\abs{is}} &: S \limp NP \limp S \\
  \abs{said}_{\abs{ds}} &: S \limp NP \limp S
\end{align*}

We have two lexical items that correspond to the use of $\abs{said}$ in
both direct speech and indirect speech. They differ both in surface
realization (both in prosody and punctuation) and semantic
interpretation\footnote{Here we model the truth-conditions of direct speech
  similarly to those of indirect speech: a relation between a speaker and a
  proposition. A more accurate account of direct speech would relate the
  speaker to the exact utterance that is being attributed to
  him~\cite{shan2010character}.
  %% TODO: Read Kaplan1989 and Shan2010 more closely.
}.

\begin{align*}
  \sem{\abs{said}_{\abs{is}}} &= \lam{C S}{\obj{say} \apr S \aplr C} \\
  &= \lam{C S}{S \hsbind (\lam{s}{\ap{\obj{say}}{s} \apr C})} \\
  \sem{\abs{said}_{\abs{ds}}}
  &= \lam{C S}{S \hsbind (\lam{s}{\ap{\obj{say}}{s} \apr (\app{\withSpeaker}{s}{C})})}
\end{align*}

The indirect speech use of \emph{said} has the same kind of lexical entry
as the transitive verb \emph{loves}. In the direct speech entry, we would
like to bind the speaker within the complement clause to the referent of
the subject. We will therefore want to wrap the complement clause in a
handler for $\op{speaker}$. However, the handler $(\ap{\withSpeaker}{s})$
needs to know the referent $s$ of the subject $S$. We will need to first
evaluate $S$ and bind its result to $s$. To highlight the fact that the two
entries differ only in the use of the $(\ap{\withSpeaker}{s})$ handler, we
have expanded the entry for direct speech into the same form. Also note
that in this solution, we have had to use $\hsbind$ and we cannot get by
with only $\eta$ and $\aplr$, which would be the case if we were to use
applicative functor which is not a monad.

We can now plug in this new entry and compute the meanings
of~\eqref{ex:indirect-speech} and~\eqref{ex:direct-speech}.

\NoChapterPrefix
\begin{align}
  \sem{\app{\abs{said}_{\abs{is}}}{(\app{\abs{loves}}{\abs{me}}{\abs{Mary}})}{\abs{John}}}
  & \tto \app{\op{speaker}}{\star}{(\lam{x}{\etaE{(\app{\obj{say}}{\obj{j}}{(\app{\obj{love}}{\obj{m}}{x})})}})} \\
  \sem{\app{\abs{said}_{\abs{ds}}}{(\app{\abs{loves}}{\abs{me}}{\abs{Mary}})}{\abs{John}}}
  & \tto \etaE{(\app{\obj{say}}{\obj{j}}{(\app{\obj{love}}{\obj{m}}{\obj{j}})})}
\end{align}
\ChapterPrefix

In the meaning of~\eqref{ex:indirect-speech}, the $\op{speaker}$ operation
projects outside the complement clause and we end up with another
speaker-dependent proposition. On the other hand,
in~\eqref{ex:direct-speech}, the dependence on the identity of the speaker
has been discharged by the handler contained in the denotation of
$\abs{said}_{\abs{ds}}$.


\subsection{Algebraic Considerations}
\label{ssec:algebraic-deixis}

One of the traditions from which the technique of effects and handlers
originates is the study of algebraic effects by Plotkin, Power, Pretnar and
Hyland~\cite{hyland2006combining,plotkin2009handlers,pretnar2010logic,plotkin2013handling}.
The semantics of a system of operations is not given by handlers but by a
system of equations. Instead of writing a handler which would interpret two
computations as the same object, we would give equations that let us prove
the two computations equivalent. This perspective can give us some
insights.

Assuming that the $\op{speaker}$ operation is only ever handled by the
$\withSpeaker$ handler, the following equations become admissible:

\begin{align*}
  \app{\op{speaker}}{\star}{(\lam{x}{\app{\op{speaker}}{\star}{(\lam{y}{M(x,y)})}})}
  &= \app{\op{speaker}}{\star}{(\lam{x}{M(x,x)})} \\
  M &= \app{\op{speaker}}{\star}{(\lam{x}{M})}
\end{align*}

$M$ is a metavariable ranging over computations of type
$\FF_{\{ \typedop{speaker}{1}{\iota} \}}(\alpha)$\footnote{For a
  formalization of the use of metavariables, see~\ref{ssec:crs}
  or~\cite{klop1993combinatory}.}. We can check by reduction that whenever
$M_1 = M_2$ via the above equations, then $(\app{\withSpeaker}{s}{M_1})$
and $(\app{\withSpeaker}{s}{M_2})$ are convertible in
$\banana{\lambda}$. The insight we get from these equations is that asking
for the speaker is an idempotent operation: if we ask twice within the same
computation, we are guaranteed to get the same answer. It also tells us
that asking for the speaker has no other effect than to make available the
identity of the speaker: we can add a request for the current speaker and
if we do not use the answer, this addition will not change anything.

The motivation behind the choice of exactly these two equations is
normalization. Denotations in
$\FF_{\{ \typedop{speaker}{1}{\iota} \}}(\alpha)$ are formed by a series of
$\op{speaker}$ operations followed by a value of type $\alpha$. We can use
the first equation to collapse all of the $\op{speaker}$ operations into
one, or if there were no $\op{speaker}$ operations we can use the second
equation to include one. This means that we can see every value of type
$\FF_{\{ \typedop{speaker}{1}{\iota} \}}(\alpha)$ as being equal to one
which is of the shape
$\app{\op{speaker}}{\star}{(\lam{x}{\etaE{(M(x))}})}$, i.e.\ a family which
to every $x : \iota$ assigns an $M(x) : \alpha$. This connects us back to
the treatment of deixis in Montague's approach~\cite{montague1973proper}
where meanings are functions which assign to every point of reference $x$
some referent $M(x)$. This normalization is imlemented by the
$\withSpeaker$ handler. If we flip its arguments, getting
($\lam{M s}{\app{\withSpeaker}{s}{M}}$), we get a function of type
$\FF_{\{ \typedop{speaker}{1}{\iota} \}}(\alpha) \to (\iota \to \alpha)$.


\section{Conventional Implicature}
\label{sec:conventional-implicature}

We have seen an example of an expression asking the context for some
missing information. We will now look at a phenomenon which incurs
communication in the opposite direction. \emph{Conventional
  implicatures}~\cite{potts2005logic} are parts of the entailed meaning
which are not at-issue, i.e.\ are not being asserted, simply mentioned. One
of the distinguishing signs is that they project out of logical contexts
such as negation, disjunction or implication. Typical examples include
supplements such as nominal appositives and supplementary relative clauses,
and expressives such as epithets.

Here, we will deal with supplements, namely nominal appositives and
supplementary relative clauses. We will assume abstract constants for the
(supplementary) relative pronoun, the appositive construction and a
relational noun\footnote{We are working in a minimal fragment without
  determiners. Relational nouns will let use some meaningful noun phrases
  as nominal appositives in the examples to come.} with the following
types:

\begin{align*}
  \abs{who}_{\abs{s}} &: (NP \limp S) \limp NP \limp NP \\
  \abs{appos} &: NP \limp NP \limp NP \\
  \abs{best-friend} &: NP \limp NP
\end{align*}

The point of our modeling is to show that the conventional implicatures
engendered by the supplements project out of all sorts of logical
context. We will therefore also consider a fragment that contains syntactic
constructions for negation (``it is not the case that X''), implication
(``if X, then Y'') and disjunction (``either X, or Y'').

\begin{align*}
  \abs{not-the-case} &: S \limp S \\
  \abs{if-then} &: S \limp S \limp S \\
  \abs{either-or} &: S \limp S \limp S
\end{align*}

In our fragment, a noun phrase contributes both its referent and also
possibly some conventional implicature. If we would model this fact by
interpreting NPs as pairs, we would be forced to revisit all the other
lexical entries to take this change into account (e.g.\ transitive verbs
such as $\abs{loves}$ would need to explicitly aggregate the conventional
implicatures of both its subject and object). Rather than do that, we will
make use of $\banana{\lambda}$ and model this interaction with the context
as an operation. When a linguistic expression wants to conventionally
implicate something, it will use the $\op{implicate}$ operation. The
expression will need to communicate what exactly it wants to
implicate. This will be a proposition and so the input type of
$\op{implicate}$ will be the type $o$ of propositions. We do not need to
collect any information from the context and so the output type will be
$1$. Using this operation, we can now give denotations to expressions that
are about to generate conventional implicatures:

\begin{align*}
  \sem{\abs{who}_{\abs{s}}} &= \lam{C X}{X \hsbind (\lam{x}{\op{implicate} \apr (\ap{C}{(\etaE{x})}) \apl (\lam{z}{\etaE{x}})})} \\
  \lex{appos}{\lam{Y X}{X \hsbind (\lam{x}{\op{implicate} \apr (x \eqr Y) \apl (\lam{z}{\etaE{x}})})}} \\
  \lex{best-friend}{\lam{X}{\obj{best-friend} \apr X}}
\end{align*}

In both of the supplement constructions, we first evaluate the head NP $X$
to get its referent $x$. We use $x$ twice: once to construct the
implicature and once to produce the referent of the entire complex noun
phrase. The implicature constructed by the relative clause is the clause
with its gap filled in by an expression that refers to the same referent as
the head noun $X$. The implicature of the appositive is a statement of
equality between the referents of the two noun phrases\footnote{The
  operator $\eqr$ is the variant of $=$ that expects its right-hand side
  argument to be a computation and whose result is therefore a computation
  too.}. The first argument that we are applying $\op{implicate}$ to in
both entries is a computation and hence the $\apr$ and $\apl$ operators
surrounding it. Finally, we also give the semantics to the relation noun
``X's best friend'' by assuming that we have function
$\obj{best-friend} : \iota \to \iota$ in the model.

We will want to show that these denotations project through the operators
that make up the logical structure of a sentence/discourse. Since the
conventional implicature mechanism is implemented using operations, we can
get this behavior for free by just using the standard operators.

\begin{align*}
  \lex{not-the-case}{\liftl_{S \limp S}(\lnot)} \\
  &= \lam{X}{\lnot \apr X} \\
  \lex{if-then}{\liftl_{S \limp S \limp S}(\to)} \\
  &= \lam{X Y}{X \implr Y} \\
  \lex{either-or}{\liftl_{S \limp S \limp S}(\lor)} \\
  &= \lam{X Y}{X \orlr Y}
\end{align*}

We can now look at several examples of conventional implicatures buried
inside logical operators:

\begin{exe}
  \ex Either John loves Sarah, or Mary, John's best friend, loves John. \label{ex:either-or}
  \ex If it is not the case that John, whom Sarah loves, loves Sarah then Mary loves John. \label{ex:if-not}
\end{exe}

We expect~\eqref{ex:either-or} to implicate that Mary is John's best friend
and~\eqref{ex:if-not} to implicate that Sarah loves John. If we compute
their denotations, we find out that these actually are the propositions
that the two sentences try to implicate.

\NoChapterPrefix
\begin{align}
  & \sem{\app{\abs{either-or}}{(\app{\abs{loves}}{\abs{Jane}}{\abs{John}})}{(\app{\abs{loves}}{\abs{John}}{(\app{\abs{appos}}{(\ap{\abs{best-friend}}{\abs{John}})}{\abs{Mary}})})}} \nonumber \\
  & \tto \app{\op{implicate}}{(\obj{m} = \ap{\obj{best-friend}}{\obj{j}})}{(\lam{z}{\etaE{(\app{\obj{love}}{\obj{j}}{\obj{s}} \lor \app{\obj{love}}{\obj{m}}{\obj{j}})}})} \\
  & \sem{\app{\abs{if-then}}{(\ap{\abs{not-the-case}}{(\app{\abs{loves}}{\abs{Sarah}}{\abs{John}})})}{(\app{\abs{loves}}{\abs{John}}{\abs{Mary}})}} \nonumber \\
  & \tto \app{\op{implicate}}{(\app{\obj{love}}{\obj{s}}{\obj{j}})}{(\lam{z}{\etaE{(\lnot(\app{\obj{love}}{\obj{j}}{\obj{s}}) \to \app{\obj{love}}{\obj{m}}{\obj{j}})}})}
\end{align}
\ChapterPrefix

To go full circle and fulfill the empirical criterion that the meaning
of~\eqref{ex:either-or} should entail that Mary is John's best friend, we
will need to translate the term containing $\op{implicate}$ operations into
a proposition. This will be another question of interpreting operation
symbols and so we will use a handler.

\begin{align*}
  \accommodate &: \FF_{\{\typedop{implicate}{o}{1}\}}(o) \to o \\
  \accommodate &= \bbanana{\onto{\op{implicate}}{(\lam{i k}{i \land \ap{k}{\star}})}}
\end{align*}

This handler applies only to computations that produce propositions. It
collects all of the implicatures and conjoins them with the (at-issue)
proposition. We used a closed handler to make the presentation slightly
simpler (no need to mention computation types inside the handler). In
Chapter~\ref{chap:composing-effects}, we will introduce an open handler
which does the same but works inside arbitrary effect signatures.

If we apply the $\accommodate$ handler to the denotations
of~\eqref{ex:either-or} and~\eqref{ex:if-not}, then we can verify that we
get propositions that do entail the intended implicatures.

\begin{align*}
  & \ap{\accommodate}{\sem{\app{\abs{either-or}}{(\app{\abs{loves}}{\abs{Jane}}{\abs{John}})}{(\app{\abs{loves}}{\abs{John}}{(\app{\abs{appos}}{(\ap{\abs{best-friend}}{\abs{John}})}{\abs{Mary}})})}}} \nonumber \\
  & \tto (\obj{m} = \ap{\obj{best-friend}}{\obj{j}}) \land (\app{\obj{love}}{\obj{j}}{\obj{s}} \lor \app{\obj{love}}{\obj{m}}{\obj{j}}) \\
  & \ap{\accommodate}{\sem{\app{\abs{if-then}}{(\ap{\abs{not-the-case}}{(\app{\abs{loves}}{\abs{Sarah}}{\abs{John}})})}{(\app{\abs{loves}}{\abs{John}}{\abs{Mary}})}}} \nonumber \\
  & \tto (\app{\obj{love}}{\obj{s}}{\obj{j}}) \land (\lnot(\app{\obj{love}}{\obj{j}}{\obj{s}}) \to \app{\obj{love}}{\obj{m}}{\obj{j}})
\end{align*}


\subsection{Algebraic Considerations}
\label{ssec:algebraic-ci}

As in~\ref{ssec:algebraic-deixis}, we will look for equations that are
admissible w.r.t.\ the $\accommodate$ handler (equations such equivalent
computations will be interpreted with the same value) with an eye towards
deriving some canonical form.

\begin{align*}
  \app{\op{implicate}}{a}{(\lam{x}{\app{\op{implicate}}{b}{(\lam{y}{M(x,y)})}})}
  &= \app{\op{implicate}}{(a \land b)}{(\lam{x}{M(x,x)})} \\
  M &= \app{\op{implicate}}{\top}{(\lam{x}{M})}
\end{align*}

The equations are quite similar to those we have seen with
deixis\footnote{The output type of $\op{implicate}$ is the unit type $1$
  whose only value is $\star$. We can therefore replace $y$ with $x$ since
  they are guaranteed to be equal. For simplicity, we could also assume
  that $M$ is fresh for $x$ and $y$.}. Here, $M$ ranges over computations
of type $\FF_{\{ \typedop{implicate}{o}{1} \}}(\alpha)$. Since we know that
our only handler, $\accommodate$, ends up combining all the implicatures
using conjunction, we can collapse two implicatures into a single
implicature. Furthermore, implicating the tautology $\top$ has no
observable effect. By making an appeal to the conjunction operator, we can
make use of the equations governing the operations on propositions and
derive equations for,e.g., the commutativity or idempotence of
$\op{implicate}$:

\begin{align*}
  \app{\op{implicate}}{a}{(\lam{x}{\app{\op{implicate}}{b}{(\lam{y}{M(x,y)})}})}
  &= \app{\op{implicate}}{(a \land b)}{(\lam{x}{M(x,x)})} \\
  &= \app{\op{implicate}}{(b \land a)}{(\lam{x}{M(x,x)})} \\
  &= \app{\op{implicate}}{b}{(\lam{x}{\app{\op{implicate}}{a}{(\lam{y}{M(y,x)})}})}
\end{align*}

A denotation in $\FF_{\{ \typedop{implicate}{o}{1} \}}(\alpha)$ is a
sequence of implicated propositions terminated with a value of type
$\alpha$. This is very much like the result of the \emph{parsetree
  interpretation} in~\cite{potts2005logic} (Definition~2.50), which is an
$(n+1)$-tuple of the interpretation of some term of type $\alpha$ together
with the propositions which are the interpretations of the $n$ conventional
implicatures embedded within the term.

Instead of constructing an $n$-tuple of the implicated propositions, our
handler conjoins all the propositions into a single conjunction. This lets
us admit the above equations, which tell us that a computation in
$\FF_{\{ \typedop{implicate}{o}{1} \}}(\alpha)$ is always equivalent to one
of the form $\app{\op{implicate}}{p}{(\lam{x}{\etaE{M}})}$, where $p$ is a
proposition and $M$ is a value of type $\alpha$. For the case of
$\FF_{\{ \typedop{implicate}{o}{1} \}}(o)$ in particular, the domain in
which we interpret sentences, we have that the denotation is a pair of
propositions: the implicated content and the at-issue content.

As in~\ref{ssec:algebraic-deixis}, we can construct a handler which maps a
computation to this canonical representation:

\begin{align*}
  \operatorname{accommodate'}
&: \FF_{\{\typedop{implicate}{o}{1}\}}(\alpha) \to (o \times \alpha) \\
  \operatorname{accommodate'}
&= \bbanana{\onto{\op{implicate}}{(\lam{i k}{\left< i \land \ap{\pi_1}{(\ap{k}{\star})}, \ap{\pi_2}{(\ap{k}{\star})} \right>})},\ 
            \onto{\eta}{(\lam{x}{\left< \top, x \right>})}}
\end{align*}


\section{Quantification}
\label{sec:quantification}

Next, we turn our attention to in-situ quantification. This is the
phenomenon of quantified noun phrases such as \emph{every man} or \emph{a
  woman} acting as quantifiers over the sentence in which they appear. One
of the mechanisms which is used to model the inversion from being an
argument of a verb inside a sentence to taking the verb and the whole
sentence as an argument yourself are
continuations~\cite{de2001type,barker2002continuations}.



We will be working with the following abstract syntax:

\begin{align*}
  \abs{every}, \abs{a} &: N \limp NP \\
  \abs{man}, \abs{woman} &: N
\end{align*}

The challenge will be how to fit in the denotations of \emph{every man} or
\emph{some woman} into the type $\sem{NP}_\petitc = \FF_E(\iota)$. We
cannot find a unique referent for either of these noun phrases and we need
to take into account their quantificational effect on the meaning of the
whole sentence. We will make use of the operations and handlers present in
$\banana{\lambda}$, as we have done in all the previous section of this
Chapter. We will use an operation symbol $\op{scope}$ of type
$((\iota \to o) \to o) \rightarrowtail \iota$, for which we can give two
different motivations:

\begin{itemize}
\item We know that continuations are a useful technique for dealing with
  quantification and in Chapter~\ref{chap:continuations}, we have seen how
  to encode continuations in $\banana{\lambda}$. The type that we have
  given to the $\shifto$ operator there was
  $((\delta \to \FF_E(\omega)) \to \FF_E(\omega)) \rightarrowtail
  \delta$. We can specialize this to our case. The type $\omega$ of
  observations will be the type $o$ of propositions, as it will be over
  propositions that our quantifiers will scope. The type $\delta$ of
  expressions at which we will shift will be the type $\iota$ of
  individuals as that is the type of referents for noun phrases. Finally,
  in this section we will be dealing only with quantification and no other
  effects and therefore the signature $E$ will be empty and we can also go
  ahead replace $\FF_\emptyset(\omega)$ with $\omega$. This leaves with the
  type $((\iota \to o) \to o) \rightarrowtail \iota$.
\item Like in the previous sections, we can also figure out the input and
  output types of $\op{scope}$ by considering what information can the
  quantificational noun phrase offer to its context and what does it expect
  in return. We know that we can model the meaning of quantificational noun
  phrases as generalized quantifiers~\cite{montague1973proper}: e.g.\
  \emph{every man} becomes
  $\lam{P}{\forall x.\ \ap{\obj{man}}{x} \to \ap{P}{x}}$ of type
  $(\iota \to o) \to o$. We also want quantificational noun phrases to be
  like other noun phrases in that they should behave as if their referents
  were individuals of type $\iota$. Inside the denotation, we therefore
  have a generalized quantifier of type $(\iota \to o) \to o$ and we would
  like to trade it for an individual of type $\iota$, leading us to the
  type for $\op{scope} : ((\iota \to o) \to o) \rightarrowtail \iota$. We
  give a generalized quantifier to the context and the context will have
  that quantifier scope over the sentence. In return, we get the variable
  of type $\iota$ that is being quantified over and that stands for the
  referent of the noun phrase.
\end{itemize}

With this operation $\op{scope}$, we can now give a semantics to our
determiners:

\begin{align*}
  \lex{every}{\lam{N}{\ap{\op{scope}!}{(\lam{c}{\forall x.\ (\ap{\SI}{(N \apl x)}) \to \ap{c}{x}})}}} \\
  \lex{a}{\lam{N}{\ap{\op{scope}!}{(\lam{c}{\exists x.\ (\ap{\SI}{(N \apl x)}) \land \ap{c}{x}})}}} \\
  \lex{man}{\etaE{\obj{man}}} \\
  \lex{woman}{\etaE{\obj{woman}}} \\
  \SI &= \bbanana{\onto{\op{scope}}{(\lam{c k}{\ap{c}{k}})}}
\end{align*}

The denotations of $\abs{every}$ and $\abs{a}$ are both making use of the
new $\op{scope}$ operation. If we look at the meaning of
$\sem{\ap{\abs{every}}{\abs{man}}} = \app{\op{scope}}{(\lam{c}{\forall x.\
    \ap{\obj{man}}{x} \to \ap{c}{x}})}{(\lam{x}{\etaE{x}})}$, we see that
it is composed of two parts: some logical material
$\forall x.\ \ap{\obj{man}}{x} \to \ap{c}{x}$ that is to scope over the
enclosing context $c$ and a placeholder NP denotation $\etaE{x}$ for some
variable $x$. This is very much like Cooper
storage~\cite{cooper1979montague}. We store the request to scope certain
material over the sentence and we keep a variable as a placeholder
denotation for the NP. Before we proceed to explain this fragment any
further, we will consider the following example:

\begin{exe}
  \ex Every man loves a woman. \label{ex:quantifiers}
\end{exe}

and its denotation:

\NoChapterPrefix
\begin{align}
& \sem{\app{\abs{loves}}{(\ap{\abs{a}}{\abs{woman}})}{(\ap{\abs{every}}{\abs{man})}}} \nonumber \\
&\tto \app{\op{scope}}{(\lam{c}{\forall x.\ \ap{\obj{man}}{x} \to \ap{c}{x}})}{(\lam{x}{\app{\op{scope}}{(\lam{c}{\exists y.\ \ap{\obj{woman}}{y} \land \ap{c}{y}})}{(\lam{y}{\etaE{(\app{\obj{love}}{x}{y})}})}})}
\end{align}
\ChapterPrefix

We have $\etaE{(\app{\obj{love}}{x}{y})}$ as the kernel of the meaning,
with both $(\lam{c}{\forall x.\ \ap{\obj{man}}{x} \to \ap{c}{x}})$ and
$(\lam{c}{\exists y.\ \ap{\obj{woman}}{y} \land \ap{c}{y}})$ scheduled to
scope over it. We see that as in Cooper's approach, the scope material that
we stored in the NPs gets carried over to the meaning of the entire
clause. Now we need an analogue to Cooper's retrieval procedure that can
take this scope material and apply it to the meaning of the nucleus. This
role is carried out by the handler $\SI$, short for Scope Island.

\begin{align*}
& \ap{\SI}{\sem{\app{\abs{loves}}{(\ap{\abs{a}}{\abs{woman}})}{(\ap{\abs{every}}{\abs{man}})}}} \\
&\tto \forall x.\ \ap{\obj{man}}{x} \to (\exists y.\ \ap{\obj{woman}}{y} \land \app{\obj{love}}{x}{y})
\end{align*}

The handler $\SI$ will also end up being part of our
denotations. Quantifiers can only take scope up to the limit of the nearest
enclosing scope island and that is a constraint that we can encode using
this handler. We could, for example, implement the constraint that tensed
clauses should form scope islands by including the $\SI$ handler in the
denotations of the constructors of tensed clauses.

\begin{align*}
  \lex{loves}{\lam{O S}{\ap{(\eta \circ \SI)}{(\obj{love} \apr S \aplr O)}}}
\end{align*}

Here, we apply not only $\SI$, but $\eta \circ \SI$, to the sentence
denotation. This is because our type for interpreting sentences, $\sem{S}$,
is $\FF_E(o)$ and since in this section, we are using a closed handler
(type $\FF_E(o) \to o$), we have to follow with the injection $\eta$ (type
$o \to \FF_E(o)$)\footnote{We could also just as well choose $\sem{S} = o$
  but we want to be consistent with the other treatments we have shown so
  far and the ones we will see later.}.

Using this new lexical entry for $\abs{loves}$, we can directly compute a
reading for Sentence~\ref{ex:quantifiers}:

\addtocounter{equation}{-1}
\NoChapterPrefix
\begin{align}
& \sem{\app{\abs{loves}}{(\ap{\abs{a}}{\abs{woman}})}{(\ap{\abs{every}}{\abs{man}})}} \nonumber \\
&\tto \etaE{(\forall x.\ \ap{\obj{man}}{x} \to (\exists y.\ \ap{\obj{woman}}{y} \land \app{\obj{love}}{x}{y}))}
\end{align}
\ChapterPrefix

Finally, we will address the use of $\SI$ in the denotations of
$\abs{every}$ and $\abs{a}$. We will note that we interpret constituents of
type $N$ with the type $\FF_E(\iota \to o)$. However, in the fragment so
far, we have only seen pure nouns, nouns whose denotation is of the form
$\etaE{M}$. Nevertheless, we can imagine complex constituents of type $N$
which do have a quantificational effect (e.g.\ \emph{owner of a cat}).

\begin{align*}
  \abs{owner-of} &: NP \limp N \\
  \lex{owner-of}{\liftl_{NP \limp N}(\lam{y x}{\app{\obj{own}}{x}{y}})} \\
  &= \lam{Y}{Y \hsbind (\lam{y}{\etaE{(\lam{x}{\app{\obj{own}}{x}{y}})}})}
\end{align*}

The relational noun \emph{owner} does not contribute any quantificational
effect, but the complex noun \emph{owner of $Y$} inherits any effects of
$Y$. We will see this in the denotation of \emph{owner of a cat}:

\begin{align*}
&\sem{\ap{\abs{owner-of}}{(\ap{\abs{a}}{\abs{cat}})}} \\
&\tto \ap{\sem{\abs{owner-of}}}{(\app{\op{scope}}
    {(\lam{c}{\exists y.\ \ap{\obj{cat}}{y} \land \ap{c}{y}})}
    {(\lam{y}{\etaE{y}})})} \\
&\tto (\lam{y x}{\app{\obj{own}}{x}{y}}) \apr (\app{\op{scope}}
    {(\lam{c}{\exists y.\ \ap{\obj{cat}}{y} \land \ap{c}{y}})}
    {(\lam{y}{\etaE{y}})}) \\
&\tto \app{\op{scope}}{(\lam{c}{\exists y.\ \ap{\obj{cat}}{y} \land \ap{c}{y}})}
    {(\lam{y}{\etaE{(\ap{(\lam{y x}{\app{\obj{own}}{x}{y}})}{y})}})} \\
&\to_\beta \app{\op{scope}}{(\lam{c}{\exists y.\ \ap{\obj{cat}}{y} \land \ap{c}{y}})}
                      {(\lam{y}{\etaE{(\lam{x}{\app{\obj{own}}{x}{y}})}})}
\end{align*}

We can look at the result through the analogy to Cooper storage. We started
with the meaning of \emph{a cat}, which stored the generalized quantifier
$\lam{c}{\exists y.\ \ap{\obj{cat}}{y} \land \ap{c}{y}}$ in the storage and
whose semantic placeholder was the variable $y$. We then applied the
function $\lam{y x}{\app{\obj{own}}{x}{y}}$ to this semantic placeholder
and we arrived at $\lam{x}{\app{\obj{own}}{x}{y}}$ with the storage still
holding the quantifier
$\lam{c}{\exists y.\ \ap{\obj{cat}}{y} \land \ap{c}{y}}$.

And so we have seen that constituents of type $N$ can have a
quantificational effect in much the same way as other constituents, such as
those of type $NP$. In the generalized quantifier for indefinites,
$(\lam{c}{\exists y.\ \ap{n}{y} \land \ap{c}{y}})$, we make use of the
meaning $n$ of the restrictor noun. By looking at the type of
$\op{scope} : ((\iota \to o) \to o) \rightarrowtail \iota$, we see that we
cannot pass a computation with quantificational effects as an argument to
$\op{scope}$\footnote{In that case, we would need to have
  $\typedop{scope}{((\iota \to \FF_E(o)) \to o)}{\iota} \in E$. This is
  problematic since $E$ is itself used as a part of a type contained in
  $E$. The type of $\op{scope}$ would therefore not be neither finite nor
  well-founded.}. So instead, we discharge the quantificational potential
of the noun in the restrictor of the generalized quantifier using the $\SI$
handler\footnote{This is very much like the situation we have seen towards
  the end of Chapter~\ref{chap:continuations}. The $\SI$ handler is our
  $\reset$. We can only pass the type check if we use $\shift$/$\reset$
  (Section~\ref{sec:considering-types}). We encode $\shift$ using $\shifto$
  by using a $\reset$ (in our case $\SI$) inside the argument to $\shifto$
  (in our case $\op{scope}$), as in
  Section~\ref{sec:turning-to-shift}.}. We will look at the kind of
readings this leads to by considering the classic example
from~\cite{burchardt2004computational}:

\begin{exe}
  \ex Every owner of a siamese cat loves a therapist. \label{ex:siamese}
\end{exe}

Assuming the inclusion of the common nouns \emph{siamese cat} and
\emph{therapist} in our fragment (with semantics analogous to the ones for
\emph{man} and \emph{woman}), we can compute the following interpretation:

\NoChapterPrefix
\begin{align}
& \sem{\app{\abs{loves}}{(\ap{\abs{a}}{\abs{therapist}})}{(\ap{\abs{every}}{(\ap{\abs{owner-of}}{(\ap{\abs{a}}{\abs{siamese-cat}})})})}} \nonumber \\
&\tto \etaE{(\forall x.\ (\exists y.\ \ap{\obj{siamese-cat}}{y} \land \app{\obj{own}}{x}{y}) \to (\exists z.\ \ap{\obj{therapist}}{z} \land \app{\obj{love}}{x}{z}))}
\end{align}
\ChapterPrefix


\subsection{Quantifier Ambiguity}
\label{ssec:quantifier-ambiguity}

The lexical entry that we have considered for the determiner
\emph{every}\footnote{Though the same applies to the lexical entry for the
  indefinite article \emph{a} as well.} so far was:

$$
\sem{\abs{every}} = \lam{N}{\ap{\op{scope}!}{(\lam{c}{\forall x.\
      (\ap{\SI}{(N \apl x)}) \to \ap{c}{x}})}}
$$

However, if we followed the analogy to $\shift$ and $\reset$ (i.e.\
wrapping the entire body of the function that is the argument to
$\op{scope}$/$\shifto$ in $\SI$/$\reset$, see
Section~\ref{sec:turning-to-shift}), we might have tried the following:

$$
\sem{\abs{every'}} = \lam{N}{\ap{\op{scope}!}{(\lam{c}{\ap{\SI}{(N \hsbind
        (\lam{n}{\etaE{(\forall x.\ \ap{n}{x} \to \ap{c}{x})}}))}})}}
$$

Here, we put the $\SI$ handler above the quantifier introduced by the
semantics of the determiner. For reasons of readability, we have also
pulled the only impure part, $N$, in front of the expression using the
$\hsbind$ operator. This lexical entry puts the scope of any quantifiers
found in the restrictor noun over the scope of the quantifier contribued by
the determiner. If we try using this lexical entry when computing the
meaning of Sentence~\ref{ex:siamese}, we find another possible reading:

\addtocounter{equation}{-1}
\NoChapterPrefix
\begin{align}
& \sem{\app{\abs{loves}}{(\ap{\abs{a}}{\abs{therapist}})}{(\ap{\abs{every'}}{(\ap{\abs{owner-of}}{(\ap{\abs{a}}{\abs{siamese-cat}})})})}} \nonumber \\
&\tto \etaE{(\exists y.\ \ap{\obj{siamese-cat}}{y} \land (\forall x.\ \app{\obj{own}}{x}{y} \to (\exists z.\ \ap{\obj{therapist}}{z} \land \app{\obj{love}}{x}{z})))}
\end{align}
\ChapterPrefix

Both of the readings we have seen are actually valid interpretations of
this sentence; quantifiers are a notorious source of ambiguities. In the
ACG framework, a term from the abstract signature, the tectogrammatic
structure of a sentence, is mapped by a deterministic interpretation
function into a term in the object signature, in the case of semantics, its
sense or denotation. The conventional approach to dealing with ambiguities
in ACGs~\cite{de2001towards,pogodalla2007generalizing} is to make it so
that the ambiguous sentence has multiple corresponding abstract terms,
i.e.\ tectogrammatic analyses\footnote{Alternatively, we could make it so
  that a single object term corresponds to multiple readings. A natural way
  to deal with that would be to introduce the effect of non-determinism
  (signature $\typedop{choose}{1}{2}$).
%% TODO?: Maybe show the approach that relies on weakly-typed lambda-banana?
}. We can therefore imagine that our fragment contains both $\abs{every}$
and $\abs{every'}$, one giving narrow scope to resrictor quantifiers and
the other giving them wide scope (the same would be the case for other
determiners such as $\abs{a}$).

Note that Sentence~\ref{ex:quantifiers} is ambiguous as well. The
existential quantifier in \emph{a woman} can take either wide or narrow
scope (we call it wide scope when the object quantifier scopes over the
subject quantifier and we call it narrow when it scopes under). The
relative scope of the quantifiers is given by the order in which they
appear in the computation: to get the object in wide scope, we would need
to chain the computations of the subject and object by putting object
first, subject last.

\begin{align*}
\lex{loves}{\lam{O S}{\ap{(\eta \circ \SI)}{(\obj{love} \apr S \aplr O)}}} \\
&= \lam{O S}{\ap{(\eta \circ \SI)}{(S \hsbind (\lam{s}{O \hsbind
  (\lam{o}{\etaE{(\app{\obj{love}}{s}{o})}})}))}} \\
\lex{loves'}{\lam{O S}{\ap{(\eta \circ \SI)}{(O \hsbind (\lam{o}{S \hsbind
  (\lam{s}{\etaE{(\app{\obj{love}}{s}{o})}})}))}}}
\end{align*}

Using the lexical entry, we can now derive the other reading for
Sentence~\ref{ex:quantifiers}:

\addtocounter{equation}{-2}
\NoChapterPrefix
\begin{align}
& \sem{\app{\abs{loves}}{(\ap{\abs{a}}{\abs{therapist}})}{(\ap{\abs{every'}}{(\ap{\abs{owner-of}}{(\ap{\abs{a}}{\abs{siamese-cat}})})})}} \nonumber \\
&\tto \etaE{(\exists y.\ \ap{\obj{siamese-cat}}{y} \land (\forall x.\ \app{\obj{own}}{x}{y} \to (\exists z.\ \ap{\obj{therapist}}{z} \land \app{\obj{love}}{x}{z})))}
\end{align}
\ChapterPrefix
\addtocounter{equation}{1}

as well as two more readings for Sentence~\ref{ex:siamese}:

\addtocounter{equation}{-1}
\NoChapterPrefix
\begin{align}
& \sem{\app{\abs{loves'}}{(\ap{\abs{a}}{\abs{therapist}})}{(\ap{\abs{every}}{(\ap{\abs{owner-of}}{(\ap{\abs{a}}{\abs{siamese-cat}})})})}} \nonumber \\
&\tto \etaE{(\exists z.\ \ap{\obj{therapist}}{z} \land (\forall x.\ (\exists y.\ \ap{\obj{siamese-cat}}{y} \land \app{\obj{own}}{x}{y}) \to \app{\obj{love}}{x}{z}))} \\
\addtocounter{equation}{-1}
& \sem{\app{\abs{loves'}}{(\ap{\abs{a}}{\abs{therapist}})}{(\ap{\abs{every'}}{(\ap{\abs{owner-of}}{(\ap{\abs{a}}{\abs{siamese-cat}})})})}} \nonumber \\
&\tto \etaE{(\exists z.\ \ap{\obj{therapist}}{z} \land (\exists y.\ \ap{\obj{siamese-cat}}{y} \land (\forall x.\ \app{\obj{own}}{x}{y} \to \app{\obj{love}}{x}{z})))}
\end{align}
\ChapterPrefix

However, Sentence~\ref{ex:siamese} has one more reading that we will not be
able to get at using this technique.

$$
\exists y.\ \ap{\obj{siamese-cat}}{y} \land (\exists z.\ \ap{\obj{therapist}}{z} \land (\forall x.\ \app{\obj{own}}{x}{y} \to \app{\obj{love}}{x}{z}))
$$

This reading is equivalent to the last reading,
$\sem{\app{\abs{loves'}}{(\ap{\abs{a}}{\abs{therapist}})}{(\ap{\abs{every'}}{(\ap{\abs{owner-of}}{(\ap{\abs{a}}{\abs{siamese-cat}})})})}}$. The
only difference is that this time, \emph{a siamese cat} has scope over
\emph{a therapist}. This distinction would be important in sentences
involving a different quantifier then the existential coming from the
indefinite, such as in the sentence ``\emph{Every researcher of a company
  saw most samples}''~\cite{burchardt2004computational}. In our fragment,
we cannot reproduce this reading because the order in which the quantifiers
take place goes like this: \emph{a siamese cat} from the subject, \emph{a
  therapist} from the object and \emph{every owner} from the
subject. However, we have no simple way to crack open the chain of
quantifications from the subject and insert the quantification from the
object in the middle. We will therefore turn to a more robust and elegant
solution which will get us the readings we want and will not profilerate
duplicate entries such as $\abs{every}$/$\abs{every'}$ and
$\abs{loves}$/$\abs{loves'}$.


\subsubsection{Quantifier Raising}

We will make use of the fact that ACGs are a categorial grammar formalism
and their syntactic structures are not just constituency trees, but
derivations which can use hypothetical reasoning. That is to say, the terms
in the abstract signature which represent the tectogrammatic structure can
contain $\lambda$-binders and variables. We will use this to implement
Montague's treatment of quantifiers~\cite{montague1973proper}, called
``quantifying in'' or ``quantifier raising''.

Montague's technique can be explained by introducing by saying that
sentences such as ``\emph{every man loves a woman}'' can be paraphrased as
``\emph{a woman, every man loves her}''. The quantifier that is to be
raised is replaced by a pronoun/variable. Later, we use a construction that
lets us form a sentence by combining a noun phrase with a sentence in which
a variable is to be bound. We can licence this kind of construction in our
ACG by adding the following (unlexicalised) abstract constant:

\begin{align*}
  \abs{QR} &: NP \limp (NP \limp S) \limp S \\
  \lex{QR}{\lam{X k}{X \hsbind (\lam{x}{\ap{k}{(\etaE{x})}})}}
\end{align*}

The type of $\abs{QR}$ can also be read as a kind of type lifting operator
on NPs: it takes an $NP$ and gives back an $(NP \limp S) \limp S$ (the
syntactic type for quantificational noun phrases used
in~\cite{pogodalla2007generalizing}). As for the semantics, we can regard
it as a variation on the term $\lam{X k}{\ap{k}{X}}$: instead of passing
$X$ directly to $k$ though, we first evaluate it to get its referent $x$
and then we pass to $k$ a trivial computation that always this referent
$x$.

We will demonstrate $\abs{QR}$ by deriving the object wide scope reading of
Sentence~\ref{ex:quantifiers}:

\addtocounter{equation}{-2}
\NoChapterPrefix
\begin{align}
& \ap{\SI}{\sem{(\app{\abs{QR}}{(\ap{\abs{a}}{\abs{woman}})}{(\lam{y}{\app{\abs{loves}}{y}{(\ap{\abs{every}}{\abs{man}})}})})}} \nonumber \\
&\tto \ap{\SI}{(\app{\op{scope}}
  {(\lam{c}{\exists y.\ \ap{\obj{woman}}{y} \land \ap{c}{y}})}
  {(\lam{y}{\forall x.\ \etaE{(\ap{\obj{man}}{x} \rightarrow \app{\obj{love}}{x}{y})}})})} \nonumber \\
&\tto \ap{(\lam{c}{\exists y.\ \ap{\obj{woman}}{y} \land \ap{c}{y}})}
         {(\lam{y}{\forall x.\ \ap{\obj{man}}{x} \rightarrow \app{\obj{love}}{x}{y}})} \nonumber \\
&\tto \exists y.\ \ap{\obj{woman}}{y} \land (\forall x.\ \ap{\obj{man}}{x} \rightarrow \app{\obj{love}}{x}{y})
\end{align}
\ChapterPrefix
\addtocounter{equation}{1}

We can also generate all five readings of Sentence~\ref{ex:siamese},
without incurring the overgeneration due to Cooper's
storage~\cite{burchardt2004computational} thanks to the ACG type
system\footnote{The overgenerating case is due to variable escaping its
  scope. However, such a term with a free occurrence of a variable would
  not be a well-typed closed term and so such a derivation does not
  exist.}.

\addtocounter{equation}{-1}
\NoChapterPrefix
\begin{align}
& \sem{\app{\abs{loves}}{(\ap{\abs{a}}{\abs{therapist}})}{(\ap{\abs{every}}{(\ap{\abs{owner-of}}{(\ap{\abs{a}}{\abs{siamese-cat}})})})}} \nonumber \\
&\tto \etaE{(\forall x.\ (\exists y.\ \ap{\obj{siamese-cat}}{y} \land \app{\obj{own}}{x}{y}) \to (\exists z.\ \ap{\obj{therapist}}{z} \land \app{\obj{love}}{x}{z}))} \\
\addtocounter{equation}{-1}
& \ap{\SI}{\sem{\app{QR}{(\ap{\abs{a}}{\abs{siamese-cat}})}{(\lam{y}{(\app{\abs{loves}}{(\ap{\abs{a}}{\abs{therapist}})}{(\ap{\abs{every}}{(\ap{\abs{owner-of}}{y})})})})}}} \nonumber \\
&\tto \exists y.\ \ap{\obj{siamese-cat}}{y} \land (\forall x.\ \app{\obj{own}}{x}{y} \to (\exists z.\ \ap{\obj{therapist}}{z} \land \app{\obj{love}}{x}{z})) \\
\addtocounter{equation}{-1}
& \ap{\SI}{\sem{\app{QR}{(\ap{\abs{a}}{\abs{therapist}})}{(\lam{z}{(\app{\abs{loves}}{z}{(\ap{\abs{every}}{(\ap{\abs{owner-of}}{(\ap{\abs{a}}{\abs{siamese-cat}})})})})})}}} \nonumber \\
&\tto \exists z.\ \ap{\obj{therapist}}{z} \land (\forall x.\ (\exists y.\ \ap{\obj{siamese-cat}}{y} \land \app{\obj{own}}{x}{y}) \to \app{\obj{love}}{x}{z}) \\
\addtocounter{equation}{-1}
& \ap{\SI}{\sem{\app{QR}{(\ap{\abs{a}}{\abs{siamese-cat}})}{(\lam{y}{(\app{QR}{(\ap{\abs{a}}{\abs{therapist}})}{(\lam{z}{(\app{\abs{loves}}{z}{(\ap{\abs{every}}{(\ap{\abs{owner-of}}{y})})})})})})}}} \nonumber \\
&\tto \exists y.\ \ap{\obj{siamese-cat}}{y} \land (\exists z.\ \ap{\obj{therapist}}{z} \land (\forall x.\ \app{\obj{own}}{x}{y} \to \app{\obj{love}}{x}{z})) \\
\addtocounter{equation}{-1}
& \ap{\SI}{\sem{\app{QR}{(\ap{\abs{a}}{\abs{therapist}})}{(\lam{z}{(\app{QR}{(\ap{\abs{a}}{\abs{siamese-cat}})}{(\lam{y}{(\app{\abs{loves}}{z}{(\ap{\abs{every}}{(\ap{\abs{owner-of}}{y})})})})})})}}} \nonumber \\
&\tto \exists z.\ \ap{\obj{therapist}}{z} \land (\exists y.\ \ap{\obj{siamese-cat}}{y} \land (\forall x.\ \app{\obj{own}}{x}{y} \to \app{\obj{love}}{x}{z}))
\end{align}
\ChapterPrefix

The QR operator allows us to displace the scope of any quantifier
\footnote{Actually, it can displace any kind of $NP$ effect, including
  dynamics, and so it can be used, e.g., to implement a kind of
  cataphora. This will turn out to be a bug, not a feature, because of
  crossover constraints~\cite{shan2006explaining}, to be treated in
  Chapter~\ref{chap:composing-effects}.}. We might now risk overgenerating
by letting quantifiers leak outside of scope islands. However, this can be
remedied at the level of the abstract type signature. Pogodalla and
Pompigne~\cite{pogodalla2012controlling} show how to use dependent types in
the abstract types of an ACG to enforce a scope island constraint ---
namely that quantificational noun phrases should not take scope outside of
the nearest enclosing tensed clause.


\subsection{Algebraic Considerations}
\label{ssec:algebraic-quantification}

Unlike deixis and conventional implicature, we will not derive many useful
admissible equations for $\op{scope}$. If we try to collapse two uses of
$\op{scope}$ into a single one, we can succeed only partially:

$$
\app{\op{scope}}{f_1}{(\lam{x}{\app{\op{scope}}{f_2}{(\lam{y}{M(x,y)})}})}
= \app{\op{scope}}{(\lam{c}{\ap{f_1}{(\lam{x}{\ap{f_2}{(\lam{y}{\ap{c}{(x,y)}})}})}})}{(\lam{(x,y)}{M(x,y)})}
$$

We can compose the two quantifiers $f_1$ and $f_2$, but then we are
quantifying over pairs of individuals $(x,y)$, which is not something that
we have planned to do with $\op{scope}$, whose output type is $\iota$, the
type of (single) individuals. Therefore, the above would not even
type-check correctly. However, we could define a rule which shows us how
pure computations that yield individuals correspond to computations using
$\op{scope}$.

$$
\etaE{M} = \ap{\op{scope}!}{(\lam{c}{\ap{c}{M}})}
$$

$M$ ranges over values of type $\iota$. This equation shows why in
$\banana{\lambda}$, we are not obliged to raise the denotations of our
non-quantificational noun phrases (such as the proper names $\abs{John}$
and $\abs{Mary}$) into generalized quantifiers: the pure individuals
$\etaE{\obj{j}}$ and $\etaE{\obj{m}}$ behave exactly the same as the
generalized quantifiers $\lam{c}{\ap{c}{\obj{j}}}$ and
$\lam{c}{\ap{c}{\obj{m}}}$, respectively. This means that a sentence can
mix the lexical entries for proper nouns from~\ref{sec:lifting-semantics}
with the new lexical entries for quantificational noun phrases from this
section in a sound way, without violating the homomorphism property of the
ACG.

\begin{exe}
  \ex John loves a man. \label{ex:mixed-nps}
\end{exe}

\vspace{-8mm}

\NoChapterPrefix
\begin{align}
& \sem{\app{\abs{loves}}{(\ap{\abs{a}}{\abs{man}})}{\abs{John}}} \nonumber \\
&\tto \etaE{(\exists x.\ \ap{\obj{man}}{x} \land \app{\obj{love}}{\obj{j}}{x})}
\end{align}
\ChapterPrefix

Since we do not have any useful laws to simplify the denotations in
$\FF_{\{ \typedop{scope}{((\iota \to o) \to o)}{\iota} \}}(\alpha)$, the
canonical representations will be a hierarchy of generalized quantifiers,
one scoping over the other, with a value of type $\alpha$ at the
bottom. Previously, we have drawn analogy to Cooper storage. However, in
Cooper storage, the quantifiers are not stored hierarchically, but
side-by-side, so that any quantifier can be retrieved. However, this can
lead to generating undesired meanings in which variables escape from the
scopes of their intended binders. Our approach is closer to \emph{Keller
  storage}~\cite{keller1988nested}, also known as nested Cooper storage. In
Keller storage, quantifiers can be stored both side-by-side and embedded:
any of the quantifiers stored by side-by-side can be retrieved, but
whenever a quantifier is retrieved, all of its embedded quantifiers are
retrieved at the same time. Our representation lacks the side-by-side mode
of composition. The chain of $\op{scope}$ operations in an
$\FF_{\{ \typedop{scope}{((\iota \to o) \to o)}{\iota} \}}(\alpha)$
denotation corresponds to a series of embedded quantifiers in Keller
storage: we cannot retrieve a quantifier without first retrieving the
quantifiers which precede it in the chain. This is exemplified by this
chain from Sentence~\ref{ex:siamese} where the $y$ in the second quantifier
is bound by the first quantifier:

$$
\app{\op{scope}}{(\lam{c}{\exists y.\ \ap{\obj{cat}}{y} \land \ap{c}{y}})}
{(\lam{y}{\app{\op{scope}}{(\lam{c}{\forall x.\ \app{\obj{own}}{x}{y} \to \ap{c}{x}})}
{(\lam{x}{\etaE{x}})}})}
$$

This means that our representation behaves like Keller storage in that it
prevents scope extrusion (variables escaping out of the scope of their
intended binders). However, it lacks the basic feature of Cooper (and
Keller) storage that is the side-by-side storing of quantifiers to enable
ambiguity. In~\ref{ssec:quantifier-ambiguity}, we have dealt with the
ambiguity issue by use of Montague's ``quantifying in''.


\section{Methodology}
\label{sec:methodology}

We will now take a step back and identify the methodology we have used to
analyse these three phenomena: deixis, conventional implicature and
quantification.


\subsection{Using Computation Types}
\label{ssec:methodology-computation-types}

At the basis of any linguistic modelling that uses $\banana{\lambda}$ is
the notion of a computation: all of the extensions to the simply-typed
lambda calculus in $\banana{\lambda}$ deal with computation types, types of
the form $\FF_E(\alpha)$. We will want to use computations in the semantic
interpretations of our lexical items.

If we start from some existing Montagovian semantics (interpreting
sentences as propositions, noun phrases as individuals, nouns as sets),
there is a question of where to introduce the computation types. In our
approach, we choose to make the interpretation of every atomic abstract
type a computation. This has roughly the effect of making every constituent
a computation and gives a bit more meaning to the notion of an atomic
type\footnote{Meaning that this strategy constrains the choice of atomic
  types, which is otherwise somewhat arbitrary.}. Barker's continuization
approach~\cite{barker2002continuations}, which replaces the NP
interpretation type $\iota$ with a ``computation type''
$(\iota \to o) \to o$, is a similar strategy. Other possible strategies
include:

\begin{itemize}
\item Turning every atomic semantic type ($\iota$, $o$, \ldots) into a
  computation

  This idea is well-established in formal semantics. Examples include de
  Groote's Montagovian treatment of anaphora~\cite{de2006towards} ($o$ is
  replaced with $\gamma \to (\gamma \to o) \to o$), Lebedeva's extension of
  this formalism~\cite{lebedeva2012expression} (furthermore replaces
  $\iota$ with $\gamma \to \iota$), Ben-Avi and
  Winter's~\cite{ben2007semantics} intensionalization procedure (replaces
  $o$ with $\sigma \to o$) and de Groote and Kanazawa's
  variation~\cite{de2013note} (also replaces $\iota$ with
  $\sigma \to \iota$).

\item Turning every semantic function type $\alpha \to \beta$ into the type
  $\alpha \to \FF_E(\beta)$

  This corresponds to interpreting a call-by-value language using the monad
  $\FF_E$~\cite{moggi1991notions,wadler1992essence}. Techniques that use a
  call-by-value impure language for their semantic entries fall in this
  category as well. Examples include Shan's use of $\shift$ and
  $\reset$~\cite{shan2004delimited,shan2005linguistic} and our previous
  attempts~\cite{marsik2014algebraic} using the Eff
  language~\cite{bauer2012programming}.

\item Turning every semantic function type $\alpha \to \beta$ into the type
  $\FF_E(\alpha) \to \FF_E(\beta)$

  This is very similar to the idea above but instead of call-by-value, it
  lets us get a call-by-name interpretation. Call-by-name side effects in
  natural language semantics have been proposed by
  Kiselyov~\cite{kiselyov2008call}.
\end{itemize}

This palette of strategies ranges the possibility space in a tradeoff
between flexibility and simplicity. Inserting computations into more and
more places gives us more expressivity but this comes at the price of the
system's simplicity:

\begin{itemize}
\item A noun denotation of type $\FF_E(\iota \to o)$ has some effect and
  then yields a pure predicate of type $\iota \to o$.
\item A noun denotation of type $\iota \to \FF_E(o)$ might have different
  effects, depending on the semantic argument that it will be applied
  to. As a consequence, this effect becomes available only when we apply
  the denotation to some argument of type $\iota$\footnote{We could use the
    $\CC$ operator of $\banana{\lambda}$ to get a value of the above type
    $\FF_E(\iota \to o)$. However, this partial operation succeeds only if
    the effects do not actually depend on the argument. If that is the
    case, we might as well use $\FF_E(\iota \to o)$ directly.}.
\item A noun denotation of type $\FF_E(\iota) \to \FF_E(o)$ might evaluate
  its argument first, or it might perform some other effects, or it might
  handle some of the operations used in the computation of its argument.
\end{itemize}

Furthermore, in a calculus such as $\banana{\lambda}$, where order of
evaluation is controlled manually using monadic combinators, having an
overabundance of computation types clouds the terms with uninteresting
plumbing and enlarges the possibility space to a point that facilitates ad
hoc solutions. In the analyses presented in this manuscript, we have found
that the simplest strategy which is sufficiently expressive for our
purposes is the one which introduces a computation type into the
interpretation of every atomic abstract type and so we stick with this
strategy throughout the whole manuscript. However, the techniques developed
here can be also used in the other settings.

We can now take some base grammar that will serve as our starting point in
investigating some phenomenon. In our case, this is one of the smallest
fragments imaginable: transitive verbs and names. We presuppose a semantics
for this fragment that uses computations. We either build it from scratch
or we lift an existing semantics as we did
in~\ref{sec:lifting-semantics}. In our case, this base semantics is very
simple, it does not use generalized quantifiers or dynamic logics, since we
can treat these phenomena using effects. Again, we use the simplest
possible types for the computations. For example for noun phrases, we do
not use generalized quantifiers, since we can do quantification as an
effect, we use just simple individuals. We cannot go simpler than that
since verbs still need to know what the referents of their subjects are and
objects, to what their predicates should be applied. Similarly for
sentences, we will use simple propositions instead of dynamic propositions
since we will treat dynamicity as an effect. Again, it might seem that a
proposition is the bare minimum that a sentence must denote because that is
the product that we are interested in. However, in analysis of dynamic
semantics, it might make sense to model sentences as having no referent,
only side effects that contribute to some knowledge base.

\subsection{Choosing an Effect Signature}
\label{ssec:choosing-effect-signature}




\section{Dynamic Semantics}


\subsection{Anaphora}


\subsection{Presuppositions}


\subsection{Double Negation}


