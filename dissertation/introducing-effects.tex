\chapter{Introducing the Effects}

\setcounter{exx}{0}

\minitoc

\section{Deixis}

The first phenomenon that we will speak about is
\emph{deixis}~\cite{levinson2004deixis}. Deictic expressions is the class
of expressions that depend on the time and place of the utterance, the
speaker and the addressee and any kind of pointing/presenting the speaker
might be doing to draw the attention of the addressee. These expressions
include personal pronouns, temporal expressions, tenses, demonstratives and
others. All of these are characterized by their dependence on the
extra-linguistic context. In this section, we will restrict our attention
to a very limited subset of these expressions: singular first-person
pronouns (\emph{I}, \emph{me}, \emph{my}).

The meanings that we assign to expressions in natural languages must
reflect this context-sensitivity: the truth-conditions of \emph{Mary loves
  me} change when it is pronounced by John and when by
Peter. Montague~\cite{montague1973proper} achieved this by having the
meaning of every expression depend on a point of reference: a pair of a
possible world and a moment in time (i.e.\ the modal \emph{where} and the
\emph{when} of the utterance). To model the first-person pronouns, we will
need to have our meanings depend on the identity of the speaker.

In the case of a deictic expression like the first-person pronoun, we have
an expression whose referent cannot be determined solely from its form and
the meaning of its parts. We will need to reach out into the context and it
is for this that we will be using the \emph{operations} in
$\banana{\lambda}$. The first-person pronoun has an interaction with its
context, which consists of asking the context for the identity of the
speaker. For this kind of interaction with the context, we will introduce
an operation symbol, $\op{speaker}$. We will also fix the symbol's input
and output types. The input type represents the information and/or the
parameters that the denotation of the first-person pronoun or any other
expression necessitating the identity of the speaker will need to provide
to the context. Since we have no information or parameter to give to the
context, we will use the trivial input type $1$, whose only value is
$\star$. The output type represents the information that the context will
provide us in return. We are interested in the identity of the speaker and
so the type of this information will be the type of individuals, $\iota$.

We can now model the meaning of a first-person pronoun as a computation of
type $\FF_E(\iota)$ that interacts with the context and produces a referent
of type $\iota$. The effect signature $E$ can be any signature provided
that $\typedop{speaker}{1}{\iota} \in E$.

\begin{align*}
  \lex{I}{\app{\op{speaker}}{\star}{(\lam{x}{\etaE{x}})}} \\
  &= \ap{\op{speaker}!}{\star} \\
  \lex{me}{\ap{\op{speaker}!}{\star}}
\end{align*}

The denotations of $\abs{I}$ and $\abs{me}$ demand the context for the
identity of the speaker $x$ using operation $\op{speaker}$ and then declare
that $x$ to be the referent of the pronoun. Taking the output of an
operation and then immediately returning it as the result of the
computation will be a common pattern and so we use the $\op{speaker}!$
shorthand introduced in~\ref{ssec:operations-and-handlers}.

Now the question is how to use the denotation given above to build meanings
of sentences containing first-person pronouns, e.g.\ \emph{Mary loves
  me}. In Montague's use of points of reference, Montague introduces an
intermediate language of intensional logic~\cite{montague1973proper}. When
Montague then gives a interpretation to this language, the point of
reference at which an expression is to be evaluated is passed through to
its subexpressions. We will have to do a similar technical step and state
how do the meanings of expressions change when they are no longer simple
values but computations.


\subsection{Raising the Semantics into Computations}

Imagine that in our tiny fragment, apart from $\abs{I} : NP$ and $\abs{me}
: NP$, whose meaning we have given above, we also have:

\begin{align*}
  \abs{John}, \abs{Mary} &: NP \\
  \abs{loves} &: NP \limp NP \limp S
\end{align*}

In this tiny fragment of proper names and predicates, we could easily
imagine interpreting noun phrases by individuals ($\sem{NP} = \iota$) and
sentences as propositions ($\sem{S} = o$). Provided we have some constants
$\obj{j} : \iota$ and $\obj{m} : \iota$ and a predicate
$\obj{love} : \iota \to \iota \to o$, we can give the following semantics
to these items:

\begin{align*}
  \lex{John}{\obj{j}} \\
  \lex{Mary}{\obj{m}} \\
  \lex{loves}{\lam{o s}{\app{\obj{love}}{s}{o}}}
\end{align*}

This interpretation works fine for simple sentences such as \emph{John
  loves Mary}. However, the type of the denotations we gave to $\abs{I}$
and $\abs{me}$ is not compatible with the types of denotations we assume
here. The context-dependence of first-person pronouns have made us use a
computation as a denotation for some of the noun phrases. In order to
satisfy the homomorphism property of ACGs and to have a sound
syntax-semantics interface, we will need to lift the denotations of simple
denotations into computations. This is very much like the case when one
introduces quantified noun phrases and switches from using generalized
quantifiers instead of simple individuals as denotations of noun phrases.

We will want linguistic expressions to denote computations. One systematic
way to achieve that is to say that the base types of our abstract syntactic
signature should be interpreted as computations. We will write
$\sem{-}_\petitv$ for the semantic interpretation using simple values and
$\sem{-}_\petitc$ for the semantic interpretation using computations. On
the type level, we will define
$\sem{\alpha}_\petitc = \FF_E(\sem{\alpha}_\petitv)$. By applying this to
the common Montagovian interpretation, we get:

\begin{align*}
  \sem{S}_\petitv &= o & \sem{S}_\petitc &= \FF_E(o) \\
  \sem{NP}_\petitv &= \iota & \sem{NP}_\petitc &= \FF_E(\iota) \\
  \sem{N}_\petitv &= \iota \to o & \sem{N}_\petitc &= \FF_E(\iota \to o)
\end{align*}

To raise the denotations of noun phrases from $\sem{NP}_\petitv = \iota$ to
$\sem{NP}_\petitc = \FF_E(\iota)$, it suffices to use $\eta$ to inject
$\iota$ inside of $\FF_E(\iota)$. This goes the same for any other lexical
item whose abstract type is a base type. For syntactic constructors that
take arguments, such as verbs or adjectives, we will chain the computations
of their arguments and apply the meaning of the constructor to the meaning
of the results of these computations. We will limit ourselves to syntactic
constructors of \emph{second-order type}, i.e.\ abstract constants whose
type is $a_1 \limp \cdots a_n \limp b$ where $a_i$ and $b$ are base
types. If we use higher-order syntactic constructors, we will give them a
bespoke semantics.

\begin{align*}
  \raisel_\alpha &: \sem{\alpha}_\petitv \to \sem{\alpha}_\petitc \\
  \raisel_a(x) &= \etaE{x} \\
  \raisel_{a \limp \beta}(f) &= \lam{X}{X \hsbind (\lam{x}{\raisel_\beta(\ap{f}{x})})}
\end{align*}

This particular schema chains the evaluation of its arguments from
left-to-right (as can be seen by looking at the expanded non-recursive
definition below). While indexicality is order-independent, some of the
effects that we will introduce later (such as anaphora) are
order-dependent. The order that we would like to reflect in the evaluation
is the linear lexical order in which elements appear in the spoken/written
form of the sentence. Since in categorial grammars of English, it is often
the case that operators first take their complements from the right and
then apply to their argument on the left (e.g.\ transitive verbs or
relative pronouns of type $(NP \limp S) \limp N \limp N$), we will often
like to chain the evaluation of the arguments in the order opposite to the
one in which we receive them. This will give rise to the $\raiser$
transformation.

\begin{align*}
  \raisel_{a_1 \limp \cdots \limp a_n \limp b}(f) &= \lam{X_1 \ldots X_n}{X_1 \hsbind (\lam{x_1}{\ldots\ X_n \hsbind (\lam{x_n}{\etaE{(\ap{f}{x_1 \ldots x_n})}})})} \\
  \raiser_{a_1 \limp \cdots \limp a_n \limp b}(f) &= \lam{X_1 \ldots X_n}{X_n \hsbind (\lam{x_n}{\ldots\ X_1 \hsbind (\lam{x_1}{\etaE{(\ap{f}{x_1 \ldots x_n})}})})} \\
\end{align*}

With these in hand, we can now raise the interpretations of our simple
fragment into computations:

\begin{align*}
\sem{\abs{John}}_\petitc &= \raisel_{NP}(\obj{j}) \\
&= \etaE{\obj{j}} \\
\sem{\abs{Mary}}_\petitc &= \raisel_{NP}(\obj{m}) \\
&= \etaE{\obj{m}} \\
\sem{\abs{loves}}_\petitc &= \raiser_{NP \limp NP \limp S}(\lam{o s}{\app{\obj{love}}{s}{o}}) \\
&= \lam{O S}{S \hsbind (\lam{s}{O \hsbind (\lam{o}{\app{\obj{love}}{s}{o}})})} \\
&= \lam{O S}{\obj{love} \apr S \aplr O} \\
&= \lam{O S}{\etaE{\obj{love}} \aplr S \aplr O}
\end{align*}

In the lexical entry for $\abs{loves}$, we can express the series of binds
using the application operators from~\ref{ssec:composing-functions}. The
idea behind the notation is that you are supposed to put double brackets on
a side whenever the argument on that side is a computation. In our example,
$S$ and $O$ are both computations and $\obj{love}$ is a pure function
term. We can expand this term so that it is a bit more regular by making
$\obj{love}$ into a computation as well. We can then a notice a connection
between the denotations in $\sem{-}_\petitv$ and the denotations in
$\sem{-}_\petitc$. The $\sem{-}_\petitc$ denotations are just the
$\sem{-}_\petitv$ denotations where every constant $\obj{c}$ was replaced
with $\etaE{\obj{c}}$ and every application $\ap{f}{x}$ was replaced with
$f \aplr x$. Note that $\eta$ and $\aplr$ make up the applicative functor
$\FF_E$ (see~\ref{ssec:applicative-functor}).


\subsection{Deixis in an Example}

After our digression into raising semantics to computations, we can go back
to our example with first-person pronouns. With the interpretations given
before, we can now analyse the following sentences:

\begin{exe}
  \ex John loves Mary. \label{ex:trivial}
  \ex John loves me. \label{ex:deixis}
\end{exe}

whose meanings we can calculate as:

\NoChapterPrefix
\begin{align}
  \sem{\app{\abs{loves}}{\abs{Mary}}{\abs{John}}} & \tto 
  \etaE{(\app{\obj{love}}{\obj{j}}{\obj{m}})} \\
  \sem{\app{\abs{loves}}{\abs{me}}{\abs{John}}} & \tto
  \app{\op{speaker}}{\star}{(\lam{x}{\etaE{(\app{\obj{love}}{\obj{j}}{x})}})}
\end{align}
\ChapterPrefix

For~\eqref{ex:trivial}, we get a pure computation that produces the
proposition $\app{\obj{love}}{\obj{j}}{\obj{m}}$, which is the same
proposition that the sentence denoted before we modified the fragment to
use computations. However, in the case of~\eqref{ex:deixis}, we do not have
any single proposition as the denotation. Instead, we have a request to
identify the speaker of the utterance and then we have a different
proposition $\app{\obj{love}}{\obj{j}}{x}$ for each possible speaker
$x$. The truth conditions of this sentence can only be found by considering
some hypothetical speaker $s$. Given such a speaker, we could resolve all
of the requests for the speaker's identity and since our effect signature
$E$ does not contain any other operations, arrive at the desired
truth-conditions. This function that will interpret the $\op{speaker}$
operation symbols will be a handler.

\begin{align*}
  \withSpeaker &: \iota \to \FF_{\{\typedop{speaker}{1}{\iota}\} \uplus E}(\alpha) \to \FF_E(\alpha) \\
  \withSpeaker &= \lam{s}{\banana{\onto{\op{speaker}}{(\lam{x k}{\ap{k}{s}})}}}
\end{align*}

The type tells us that $\ap{\withSpeaker}{s}$ is a handler for the
$\op{speaker}$ operation. It takes any computation of type
$\FF_{\{\typedop{speaker}{1}{\iota}\} \uplus E}(\alpha)$ and gives back a
computation of type $\FF_E(\alpha)$, in which $\op{speaker}$ will not be
used. Since $\op{speaker}$ is the only operation in our effect signature,
by applying this handler to the denotation of a sentence in our fragment,
we get a denotation of type $\FF_\emptyset(o)$, which is isomorphic to
$o$\footnote{The two directions of the isomorphism are given by
  $\cherry : \FF_\emptyset(\alpha) \to \alpha$ and
  $\eta : \alpha \to \FF_\emptyset(\alpha).$}.

$$
  \app{\withSpeaker}{s}{\sem{\app{\abs{loves}}{\abs{me}}{\abs{John}}}} \tto
  \etaE{(\app{\obj{love}}{\obj{j}}{s})}
$$


\subsection{Quotations}

Up to now, we could have assumed that at the object level, we have a
constant $\obj{speaker}$ standing in for the speaker. After adding a
constant to our logical signature, our new models would have
interpretations for symbols in the original signature and for the
$\obj{speaker}$ constant, i.e.\ our models would become descriptions of the
world paired with some deictic index. However, removing the notion of a
context and making the speaker be a part of the model would make it
difficult to analyze the difference between the following two sentences:

\begin{exe}
  \ex John said Mary loves me. \label{ex:indirect-speech}
  \ex John said ``Mary loves me''. \label{ex:direct-speech}
\end{exe}

In our setting, we can model this kind of behavior since the $\withSpeaker$
is not a meta-level operation, but it is a term in our calculus like any
other. We can therefore have lexical entries for direct and indirect speech
that will interact differently with deictic expressions.

\begin{align*}
  \abs{said}_{\abs{is}} &: S \limp NP \limp S \\
  \abs{said}_{\abs{ds}} &: S \limp NP \limp S
\end{align*}

We have two lexical items that correspond to the use of $\abs{said}$ in
both direct speech and indirect speech. They differ both in surface
realization (both in prosody and punctuation) and semantic interpretation.

\begin{align*}
  \sem{\abs{said}_{\abs{is}}} &= \lam{C S}{\obj{say} \apr S \aplr C} \\
  &= \lam{C S}{S \hsbind (\lam{s}{\ap{\obj{say}}{s} \apr C})} \\
  \sem{\abs{said}_{\abs{ds}}}
  &= \lam{C S}{S \hsbind (\lam{s}{\ap{\obj{say}}{s} \apr (\app{\withSpeaker}{s}{C})})}
\end{align*}

The indirect speech use of \emph{said} has the same kind of lexical entry
as the transitive verb \emph{loves}. In the direct speech entry, we would
like to bind the speaker within the complement clause to the referent of
the subject. We will therefore want to wrap the complement clause in a
handler for $\op{speaker}$. However, the handler $(\ap{\withSpeaker}{s})$
needs to know the referent $s$ of the subject $S$. We will need to first
evaluate $S$ and bind its result to $s$. To highlight the fact that the two
entries differ only in the use of the $(\ap{\withSpeaker}{s})$ handler, we
have expanded the entry for direct speech into the same form. Also note
that in this solution, we have had to use $\hsbind$ and we cannot get by
with only $\eta$ and $\aplr$, which would be the case if we were to use
applicative functor which is not a monad.

We can now plug in this new entry and compute the meanings
of~\eqref{ex:indirect-speech} and~\eqref{ex:direct-speech}.

\NoChapterPrefix
\begin{align}
  \sem{\app{\abs{said}_{\abs{is}}}{(\app{\abs{loves}}{\abs{me}}{\abs{Mary}})}{\abs{John}}}
  & \tto \app{\op{speaker}}{\star}{(\lam{x}{\etaE{(\app{\obj{say}}{\obj{j}}{(\app{\obj{love}}{\obj{m}}{x})})}})} \\
  \sem{\app{\abs{said}_{\abs{ds}}}{(\app{\abs{loves}}{\abs{me}}{\abs{Mary}})}{\abs{John}}}
  & \tto \etaE{(\app{\obj{say}}{\obj{j}}{(\app{\obj{love}}{\obj{m}}{\obj{j}})})}
\end{align}
\ChapterPrefix

In the meaning of~\eqref{ex:indirect-speech}, the $\op{speaker}$ operation
projects outside the complement clause and we end up with another
speaker-dependent proposition. On the other hand,
in~\eqref{ex:direct-speech}, the dependence on the identity of the speaker
has been discharged by the handler contained in the denotation of
$\abs{said}_{\abs{ds}}$.


\section{Conventional Implicature}

We have seen an example of an expression asking the context for some
missing information. We will now look at a phenomenon which incurs
communication in the opposite direction. \emph{Conventional
  implicatures}~\cite{potts2005logic} are parts of the entailed meaning
which are not at-issue, i.e.\ are not being asserted, simply mentioned. One
of the distinguishing signs is that they project out of logical contexts
such as negation, disjunction or implication. Typical examples include
supplements such as nominal appositives and supplementary relative clauses,
and expressives such as epithets.

Here, we will deal with supplements, namely nominal appositives and
supplementary relative clauses. We will assume abstract constants for the
(supplementary) relative pronoun, the appositive construction and a
relational noun\footnote{We are working in a minimal fragment without
  determiners. Relational nouns will let use some meaningful noun phrases
  as nominal appositives in the examples to come.} with the following
types:

\begin{align*}
  \abs{who}_{\abs{s}} &: (NP \limp S) \limp NP \limp NP \\
  \abs{appos} &: NP \limp NP \limp NP \\
  \abs{best-friend} &: NP \limp NP
\end{align*}

The point of our modeling is to show that the conventional implicatures
engendered by the supplements project out of all sorts of logical
context. We will therefore also consider a fragment that contains syntactic
constructions for negation (``it is not the case that X''), implication
(``if X, then Y'') and disjunction (``either X, or Y'').

\begin{align*}
  \abs{not-the-case} &: S \limp S \\
  \abs{if-then} &: S \limp S \limp S \\
  \abs{either-or} &: S \limp S \limp S
\end{align*}

In our fragment, a noun phrase contributes both its referent and also
possibly some conventional implicature. If we would model this fact by
interpreting NPs as pairs, we would be forced to revisit all the other
lexical entries to take this change into account (e.g.\ transitive verbs
such as $\abs{loves}$ would need to explicitly aggregate the conventional
implicatures of both its subject and object). Rather than do that, we will
make use of $\banana{\lambda}$ and model this interaction with the context
as an operation. When a linguistic expression wants to conventionally
implicate something, it will use the $\op{implicate}$ operation. The
expression will need to communicate what exactly it wants to
implicate. This will be a proposition and so the input type of
$\op{implicate}$ will be the type $o$ of propositions. We do not need to
collect any information from the context and so the output type will be
$1$. Using this operation, we can now give denotations to expressions that
are about to generate conventional implicatures:

\begin{align*}
  \sem{\abs{who}_{\abs{s}}} &= \lam{C X}{X \hsbind (\lam{x}{\op{implicate} \apr (\ap{C}{(\etaE{x})}) \apl (\lam{z}{\etaE{x}})})} \\
  \lex{appos}{\lam{Y X}{X \hsbind (\lam{x}{\op{implicate} \apr (x \eqr Y) \apl (\lam{z}{\etaE{x}})})}} \\
  \lex{best-friend}{\lam{X}{\obj{best-friend} \apr X}}
\end{align*}

In both of the supplement constructions, we first evaluate the head NP $X$
to get its referent $x$. We use $x$ twice: once to construct the
implicature and once to produce the referent of the entire complex noun
phrase. The implicature constructed by the relative clause is the clause
with its gap filled in by an expression that refers to the same referent as
the head noun $X$. The implicature of the appositive is a statement of
equality between the referents of the two noun phrases\footnote{The
  operator $\eqr$ is the variant of $=$ that expects its right-hand side
  argument to be a computation and whose result is therefore a computation
  too.}. The first argument that we are applying $\op{implicate}$ to in
both entries is a computation and hence the $\apr$ and $\apl$ operators
surrounding it. Finally, we also give the semantics to the relation noun
``X's best friend'' by assuming that we have function
$\obj{best-friend} : \iota \to \iota$ in the model.

We will want to show that these denotations project through the operators
that make up the logical structure of a sentence/discourse. Since the
conventional implicature mechanism is implemented using operations, we can
get this behavior for free by just using the standard operators.

\begin{align*}
  \lex{not-the-case}{\raisel_{S \limp S}(\lnot)} \\
  &= \lam{X}{\lnot \apr X} \\
  \lex{if-then}{\raisel_{S \limp S \limp S}(\to)} \\
  &= \lam{X Y}{X \implr Y} \\
  \lex{either-or}{\raisel_{S \limp S \limp S}(\lor)} \\
  &= \lam{X Y}{X \orlr Y}
\end{align*}

We can now look at several examples of conventional implicatures buried
inside logical operators:

\begin{exe}
  \ex Either John loves Sarah, or Mary, John's best friend, loves John. \label{ex:either-or}
  \ex If it is not the case that John, whom Sarah loves, loves Sarah then Mary loves John. \label{ex:if-not}
\end{exe}

We expect~\eqref{ex:either-or} to implicate that Mary is John's best friend
and~\eqref{ex:if-not} to implicate that Sarah loves John. If we compute
their denotations, we find out that these actually are the propositions
that the two sentences try to implicate.

\NoChapterPrefix
\begin{align}
  & \sem{\app{\abs{either-or}}{(\app{\abs{loves}}{\abs{Jane}}{\abs{John}})}{(\app{\abs{loves}}{\abs{John}}{(\app{\abs{appos}}{(\ap{\abs{best-friend}}{\abs{John}})}{\abs{Mary}})})}} \nonumber \\
  & \tto \app{\op{implicate}}{(\obj{m} = \ap{\obj{best-friend}}{\obj{j}})}{(\lam{z}{\etaE{(\app{\obj{love}}{\obj{j}}{\obj{s}} \lor \app{\obj{love}}{\obj{m}}{\obj{j}})}})} \\
  & \sem{\app{\abs{if-then}}{(\ap{\abs{not-the-case}}{(\app{\abs{loves}}{\abs{Sarah}}{\abs{John}})})}{(\app{\abs{loves}}{\abs{John}}{\abs{Mary}})}} \nonumber \\
  & \tto \app{\op{implicate}}{(\app{\obj{love}}{\obj{s}}{\obj{j}})}{(\lam{z}{\etaE{(\lnot(\app{\obj{love}}{\obj{j}}{\obj{s}}) \to \app{\obj{love}}{\obj{m}}{\obj{j}})}})}
\end{align}
\ChapterPrefix

To go full circle and fulfill the empirical criterion that the meaning
of~\eqref{ex:either-or} should entail that Mary is John's best friend, we
will need to translate the term containing $\op{implicate}$ operations into
a proposition. This will be another question of interpreting operation
symbols and so we will use a handler.

\begin{align*}
  \accommodate &: \FF_{\{\typedop{implicate}{o}{1}\}}(o) \to o \\
  \accommodate &= \bbanana{\onto{\op{implicate}}{(\lam{i k}{i \land \ap{k}{\star}})}}
\end{align*}

This handler applies only to computations that produce propositions. It
collects all of the implicatures and conjoins them with the (at-issue)
proposition. We used a closed handler to make the presentation slightly
simpler (no need to mention computation types inside the handler). In
Chapter~\ref{chap:composing-effects}, we will introduce an open handler
which does the same but works inside arbitrary effect signatures.

If we apply the $\accommodate$ handler to the denotations
of~\eqref{ex:either-or} and~\eqref{ex:if-not}, then we can verify that we
get propositions that do entail the intended implicatures.

\begin{align*}
  & \ap{\accommodate}{\sem{\app{\abs{either-or}}{(\app{\abs{loves}}{\abs{Jane}}{\abs{John}})}{(\app{\abs{loves}}{\abs{John}}{(\app{\abs{appos}}{(\ap{\abs{best-friend}}{\abs{John}})}{\abs{Mary}})})}}} \nonumber \\
  & \tto (\obj{m} = \ap{\obj{best-friend}}{\obj{j}}) \land (\app{\obj{love}}{\obj{j}}{\obj{s}} \lor \app{\obj{love}}{\obj{m}}{\obj{j}}) \\
  & \ap{\accommodate}{\sem{\app{\abs{if-then}}{\ap{\abs{not-the-case}}{(\app{\abs{loves}}{\abs{Sarah}}{\abs{John}})}}{(\app{\abs{loves}}{\abs{John}}{\abs{Mary}})}}} \nonumber \\
  & \tto (\app{\obj{love}}{\obj{s}}{\obj{j}}) \land (\lnot(\app{\obj{love}}{\obj{j}}{\obj{s}}) \to \app{\obj{love}}{\obj{m}}{\obj{j}})
\end{align*}


\section{Quantification}

Next, we turn our attention to in-situ quantification. This is the
phenomenon of quantified noun phrases such as \emph{every man} or \emph{a
  woman} acting as quantifiers over the sentence in which they appear. One
of the mechanisms which is used to model the inversion from being an
argument of a verb inside a sentence to taking the verb and the whole
sentence as an argument yourself are
continuations~\cite{de2001type,barker2002continuations}.



We will be working with the following abstract syntax:

\begin{align*}
  \abs{every}, \abs{a} &: N \limp NP \\
  \abs{man}, \abs{woman} &: N
\end{align*}

The challenge will be how to fit in the denotations of \emph{every man} or
\emph{some woman} into the type $\sem{NP}_\petitc = \FF_E(\iota)$. We
cannot find a unique referent for either of these noun phrases and we need
to take into account their quantificational effect on the meaning of the
whole sentence. We will make use of the operations and handlers present in
$\banana{\lambda}$, as we have done in all the previous section of this
Chapter. We will use an operation symbol $\op{scope}$ of type
$((\iota \to o) \to o) \rightarrowtail \iota$, for which we can give two
different motivations:

\begin{itemize}
\item We know that continuations are a useful technique for dealing with
  quantification and in Chapter~\ref{chap:continuations}, we have seen how
  to encode continuations in $\banana{\lambda}$. The type that we have
  given to the $\shifto$ operator there was
  $((\delta \to \FF_E(\omega)) \to \FF_E(\omega)) \rightarrowtail
  \delta$. We can specialize this to our case. The type $\omega$ of
  observations will be the type $o$ of propositions, as it will be over
  propositions that our quantifiers will scope. The type $\delta$ of
  expressions at which we will shift will be the type $\iota$ of
  individuals as that is the type of referents for noun phrases. Finally,
  in this section we will be dealing only with quantification and no other
  effects and therefore the signature $E$ will be empty and we can also go
  ahead replace $\FF_\emptyset(\omega)$ with $\omega$. This leaves with the
  type $((\iota \to o) \to o) \rightarrowtail \iota$.
\item Like in the previous sections, we can also figure out the input and
  output types of $\op{scope}$ by considering what information can the
  quantificational noun phrase offer to its context and what does it expect
  in return. We know that we can model the meaning of quantificational noun
  phrases as generalized quantifiers~\cite{montague1973proper}: e.g.\
  \emph{every man} becomes
  $\lam{P}{\forall x.\ \ap{\obj{man}}{x} \to \ap{P}{x}}$ of type
  $(\iota \to o) \to o$. We also want quantificational noun phrases to be
  like other noun phrases in that they should behave as if their referents
  were individuals of type $\iota$. Inside the denotation, we therefore
  have a generalized quantifier of type $(\iota \to o) \to o$ and we would
  like to trade it for an individual of type $\iota$, leading us to the
  type for $\op{scope} : ((\iota \to o) \to o) \rightarrowtail \iota$. We
  give a generalized quantifier to the context and the context will have
  that quantifier scope over the sentence. In return, we get the variable
  of type $\iota$ that is being quantified over and that stands for the
  referent of the noun phrase.
\end{itemize}

With this operation $\op{scope}$, we can now give a semantics to our
determiners:

\begin{align*}
  \lex{every}{\lam{N}{\ap{\op{scope}!}{(\lam{c}{\forall x.\ (\ap{\SI}{(N \apl x)}) \to \ap{c}{x}})}}} \\
  \lex{a}{\lam{N}{\ap{\op{scope}!}{(\lam{c}{\exists x.\ (\ap{\SI}{(N \apl x)}) \land \ap{c}{x}})}}} \\
  \lex{man}{\etaE{\obj{man}}} \\
  \lex{woman}{\etaE{\obj{woman}}} \\
  \SI &= \bbanana{\onto{\op{scope}}{(\lam{c k}{\ap{c}{k}})}}
\end{align*}

The denotations of $\abs{every}$ and $\abs{a}$ are both making use of the
new $\op{scope}$ operation. If we look at the meaning of
$\sem{\ap{\abs{every}}{\abs{man}}} = \app{\op{scope}}{(\lam{c}{\forall x.\
    \ap{\obj{man}}{x} \to \ap{c}{x}})}{(\lam{x}{\etaE{x}})}$, we see that
it is composed of two parts: some logical material
$\forall x.\ \ap{\obj{man}}{x} \to \ap{c}{x}$ that is to scope over the
enclosing context $c$ and a placeholder NP denotation $\etaE{x}$ for some
variable $x$. This is very much like Cooper
storage~\cite{cooper1979montague}. We store the request to scope certain
material over the sentence and we keep a variable as a placeholder
denotation for the NP. Before we proceed to explain this fragment any
further, we will consider the following example:

\begin{exe}
  \ex Every man loves a woman. \label{ex:quantifiers}
\end{exe}

\NoChapterPrefix
\begin{align}
& \sem{\app{\abs{loves}}{(\ap{\abs{a}}{\abs{woman}})}{(\ap{\abs{every}}{\abs{man})}}} \nonumber \\
&\tto \app{\op{scope}}{(\lam{c}{\forall x.\ \ap{\obj{man}}{x} \to \ap{c}{x}})}{(\lam{x}{\app{\op{scope}}{(\lam{c}{\exists y.\ \ap{\obj{woman}}{y} \land \ap{c}{y}})}{(\lam{y}{\etaE{(\app{\obj{love}}{x}{y})}})}})}
\end{align}
\ChapterPrefix

We have $\etaE{(\app{\obj{love}}{x}{y})}$ as the kernel of the meaning,
with both $(\lam{c}{\forall x.\ \ap{\obj{man}}{x} \to \ap{c}{x}})$ and
$(\lam{c}{\exists y.\ \ap{\obj{woman}}{y} \land \ap{c}{y}})$ scheduled to
scope over it. We see that as in Cooper's approach, the scope material that
we stored in the NPs gets carried over to the meaning of the entire
clause. Now we need an analogue to Cooper's retrieval procedure that can
take this scope material and apply it to the meaning of the nucleus. This
role is carried out by the handler $\SI$, short for Scope Island.

\begin{align*}
& \ap{\SI}{\sem{\app{\abs{loves}}{(\ap{\abs{a}}{\abs{woman}})}{(\ap{\abs{every}}{\abs{man})}}}} \\
&\tto \etaE{(\forall x.\ \ap{\obj{man}}{x} \to (\exists y.\ \ap{\obj{woman}}{y} \land \app{\obj{love}}{x}{y}))}
\end{align*}




\subsection{Quantifier Ambiguity}

Sentences such as ``Every man loves a woman'' or ``Every owner of a siamese
cat loves a therapist'' are ambiguous. In ACGs, every abstract derivation
has exactly one object interpretation. In order to account for ambiguity,
we have two ways of proceeding. Either we introduce separate derivations to
account for the different readings or we make our interpretations
non-deterministic, underspecified. In the next two subsections, we will
explore both techniques.

\subsubsection{QR}

We propose a small extension which adds the $\abs{QR}$ operator to our
fragment. The goal of this operator is to let us project the effects
(e.g.\ scope-taking) of any $NP$ outside of its ``usual'' destination.

We give an update to the abstract signature, $\Sigma^a_{QR}$:

\begin{align*}
  \abs{QR} &: NP \to (NP \to S) \to S
\end{align*}

And we give a semantics to the $\abs{QR}$ operator in $\mathcal{L}_{QR}$:

\begin{align*}
  \lex{QR}{\lambda X k.\ \mathrm{SI}\ (X \hsbind (\lambda x.\ k\ (\eta\ x)))}
\end{align*}

Now, the extension is complete.

$$
E_{QR}(\left< \Sigma^a, \Sigma^o, \mathcal{L} \right>) = \left< \Sigma^a \uplus \Sigma^a_{QR}, \Sigma^o, \mathcal{L} \uplus \mathcal{L}_{QR} \right>
$$

We can now derive the inverse scope reading of ``Every man loves a woman''.

\begin{align*}
& \sem{\abs{QR}\ (\abs{a}\ \abs{woman})\ (\lambda
    x.\ \abs{loves}\ x\ (\abs{every}\ \abs{man}))} \\
\rightarrow^*\ & \mathrm{SI}\ (SCOPE\ (\lambda k.\ \dexists
x.\ (\eta\ (\obj{woman}\ x)) \dand (k\ x))\ (\lambda x.\ \eta\ (\forall
y.\ \obj{man}\ y \rightarrow \obj{love}\ y\ x))) \\
\rightarrow^*\ & (\lambda k.\ \dexists
x.\ (\eta\ (\obj{woman}\ x)) \dand (k\ x))\ (\lambda x.\ \eta\ (\forall
y.\ \obj{man}\ y \rightarrow \obj{love}\ y\ x)) \\
\rightarrow^*\ & \eta\ (\exists x.\ \obj{woman}\ x \land (\forall
y.\ \obj{man}\ y \rightarrow \obj{love}\ y\ x))
\end{align*}

The QR operator allows us to displace the scope of any quantifier
(actually, it can displace any kind of $NP$ effect, including dynamics, so
it can be used, e.g., to implement a kind of cataphora).

\subsubsection{Leaky Scope Islands}

We will now look at a solution that over-generates a little less and that
is based on a different strategy. Rather than having different derivations
that place the quantifier scope at different locations, we will weaken the
scope islands so that they may choose to let a quantifier escape.

In order to allow for choice and non-determinism, we will add a new effect,
$CHOOSE : 1 \to 2 \in E$, where $2$ is the Boolean type $1 +
1$\footnote{For this to work, we assume the presence of standard sum types
  (intuitionistic disjunction on types) in our calculus.}. We will also
introduce the following syntactic sugar for non-deterministic computations:

\begin{align*}
A + B = CHOOSE\ *\ (\lambda b.\ \text{case}\ b\ \text{of}\ &\operatorname{inl}(*) \Rightarrow A \\
|\ &\operatorname{inr}(*) \Rightarrow B)
\end{align*}

We would like certain quantifiers (e.g., indefinites) to be able to escape
scope islands. We will introduce an effect $SCOPE_P : ((\iota \to
\mathcal{F}(o)) \to \mathcal{F}(o)) \to \iota \in E$, which is a variant of
the $SCOPE$ effect to be used by quantifiers that ought to be able to
escape from scope islands.

We can now define our extension $E_{Q'}$, which builds upon $E_Q$:

$$
E_{Q'}(\left< \Sigma^a, \Sigma^o, \mathcal{L} \right>) =
\left< \Sigma^a, \Sigma^o, \mathcal{L} \uplus \mathcal{L}_{Q'} \right>
$$

where, $\mathcal{L}_{Q'}$, the changes we propose to the lexicon, are as
follows:

\begin{align*}
  \lex{every}{\lambda N.\ SCOPE\ (\lambda k.\ \mathrm{SL}\ (\dforall\ (\lambda x.\ (N \apl x) \dimp (k\ x))))}\ \eta \\
  \sem{\abs{some}} := \sem{\abs{a}} &:= \lambda N.\ SCOPE_P\ (\lambda k.\ \mathrm{SL}\ (\dexists\ (\lambda x.\ (N \apl x) \dand (k\ x))))\ \eta \\
  \mathrm{SI} &= \mathcal{L}(\mathrm{SI}) \circ \mathrm{SL} \\
  \mathrm{SL} &: \mathcal{F}(o) \to \mathcal{F}(o) \\
  \mathrm{SL} &= [\mathcal{H}\ (SCOPE_P\ (\lambda c k.\ (c\ k) + (SCOPE_P\ c\ k)))] 
\end{align*}

The $\mathrm{SL}$ (Scope Location) handler introduces a possible place for
a quantifier to take scope.

We will now look again at the sentence ``Every man loves a woman''. First
off, we will introduce some handy shortcuts:

\begin{align*}
c_1 &= \lambda k. \dforall\ (\lambda x.\ (\eta\ (\obj{man}\ x)) \dimp (k\ x)) \\
c_2 &= \lambda k.\ \dexists\ (\lambda x.\ (\eta\ (\obj{woman}\ x)) \dand
(k\ x)) \\
\sem{\abs{every}\ \abs{man}} &= SCOPE\ (\mathrm{SL} \circ c_1)\ \eta \\
\sem{\abs{a}\ \abs{woman}} &= SCOPE_P\ (\mathrm{SL} \circ c_2)\ \eta
\end{align*}


\begin{align*}
  & \sem{\abs{loves}\ (\abs{a}\ \abs{woman})\ (\abs{every}\ \abs{man})} \\
  \rightarrow^*\ & \mathrm{SI}\ (SCOPE\ (\mathrm{SL} \circ c_1)\ 
  (\lambda x.\ SCOPE_P\ (\mathrm{SL} \circ c_2)\
  (\lambda y.\ \eta\ (\obj{love}\ x\ y)))) \\
  \rightarrow^*\ & \mathrm{SL}\ (c_1\
  (\lambda x.\ \mathrm{SI}\ (SCOPE_P\ (\mathrm{SL} \circ c_2)\
  (\lambda y.\ \eta\ (\obj{love}\ x\ y))))) \\
  \rightarrow^*\ & \mathrm{SL}\ (c_1\
  (\lambda x.\ \mathrm{SL}\ (c_2\ 
  (\lambda y.\ \mathrm{SI}\ (\eta\ (\obj{love}\ x\ y)))) +
  SCOPE_P\ (\mathrm{SL} \circ c_2)\
  (\lambda y.\ \mathrm{SI}\ (\eta\ (\obj{love}\ x\ y))))) \\
  \rightarrow^*\ & \mathrm{SL}\ (c_1\
  (\lambda x.\ \mathrm{SL}\ (c_2\ 
  (\lambda y.\ \eta\ (\obj{love}\ x\ y))) +
  SCOPE_P\ (\mathrm{SL} \circ c_2)\
  (\lambda y.\ \eta\ (\obj{love}\ x\ y)))) \\
  \rightarrow^*\ & \mathrm{SL}\ (c_1\ (\lambda x.\ 
  \mathrm{SL}\ (\eta\ (\exists y.\ \obj{woman}\ y \land \obj{love}\ x\ y) +
  SCOPE_P\ (\mathrm{SL} \circ c_2)\
  (\lambda y.\ \eta\ (\obj{love}\ x\ y)))) \\
  \rightarrow^*\ & \mathrm{SL}\ (c_1\ (\lambda x.\ 
  (\eta\ (\exists y.\ \obj{woman}\ y \land \obj{love}\ x\ y) +
  SCOPE_P\ (\mathrm{SL} \circ c_2)\
  (\lambda y.\ \eta\ (\obj{love}\ x\ y)))) \\
  \rightarrow^*\ & \mathrm{SL}\ (\eta\ (\forall x.\ \obj{man}\ x
  \rightarrow (\exists y.\ \obj{woman}\ y \land \obj{love}\ x\ y)) \\
  & \hspace{7mm} + SCOPE_P\ (\mathrm{SL} \circ c_2)\ (\lambda y.\ 
  \eta\ (\forall x.\ \obj{man}\ x \rightarrow \obj{love}\ x\ y))) \\
  \rightarrow^*\ & \mathrm{SL}\ (\eta\ (\forall x.\ \obj{man}\ x
  \rightarrow (\exists y.\ \obj{woman}\ y \land \obj{love}\ x\ y))) \\
  +\ & \mathrm{SL}\ (SCOPE_P\ (\mathrm{SL} \circ c_2)\ (\lambda y.\
  \eta\ (\forall x.\ \obj{man}\ x \rightarrow \obj{love}\ x\ y))) \\
  \rightarrow^*\ & \eta\ (\forall x.\ \obj{man}\ x
  \rightarrow (\exists y.\ \obj{woman}\ y \land \obj{love}\ x\ y)) \\
  +\ & \mathrm{SL}\ (c_2\ (\lambda y.\ \eta\ (\forall x.\
  \obj{man}\ x \rightarrow \obj{love}\ x\ y))) \\
  +\ & SCOPE_P\ (\mathrm{SL} \circ c_2)\ (\lambda y.\ \eta\ (\forall x.\
  \obj{man}\ x \rightarrow \obj{love}\ x\ y))) \\
  \rightarrow^*\ & \eta\ (\forall x.\ \obj{man}\ x
  \rightarrow (\exists y.\ \obj{woman}\ y \land \obj{love}\ x\ y)) \\
  +\ & \mathrm{SL}\ (\eta\ (\exists y.\ \obj{woman}\ y \land (\forall x.\
  \obj{man}\ x \rightarrow \obj{love}\ x\ y))) \\
  +\ & SCOPE_P\ (\mathrm{SL} \circ c_2)\ (\lambda y.\ \eta\ (\forall x.\
  \obj{man}\ x \rightarrow \obj{love}\ x\ y)) \\
  \rightarrow^*\ & \eta\ (\forall x.\ \obj{man}\ x
  \rightarrow (\exists y.\ \obj{woman}\ y \land \obj{love}\ x\ y)) \\
  +\ & \eta\ (\exists y.\ \obj{woman}\ y \land (\forall x.\
  \obj{man}\ x \rightarrow \obj{love}\ x\ y)) \\
  +\ & SCOPE_P\ (\mathrm{SL} \circ c_2)\ (\lambda y.\ \eta\ (\forall x.\
  \obj{man}\ x \rightarrow \obj{love}\ x\ y)) \\
\end{align*}

\begin{align*}
  & \sem{\abs{who}\ (\abs{knows}\ (\abs{a}\ \abs{cat}))\ \abs{man}} \\
  \rightarrow^*\ & \mathcal{C}\ (\lambda x.\ (\eta\ (\obj{man}\ x)) \dand
  \sem{\abs{knows}\ (\abs{a}\ \abs{cat})\ (\eta\ x)}) \\
  \rightarrow^*\ & \mathcal{C}\ (\lambda x.\ (\eta\ (\obj{man}\ x)) \dand
  (\mathrm{SI}\ (SCOPE_P\ c_{cat}\ (\lambda y.\ \eta\ (\obj{know}\ x\ y))))) \\
  \rightarrow^*\ & \mathcal{C}\ (\lambda x.\ (\eta\ (\obj{man}\ x)) \dand
  ((c_{cat}\ (\lambda y.\ \eta\ (\obj{know}\ x\ y))) +
  (SCOPE_P\ c_{cat}\ (\lambda y.\ \eta\ (\obj{know}\ x\ y)))) \\
  \rightarrow^*\ & \mathcal{C}\ (\lambda x.\ (\eta\ (\obj{man}\ x)) \dand
  ((\eta\ (\exists y.\ (\obj{cat}\ y \land \obj{know}\ x\ y))) +
  (SCOPE_P\ c_{cat}\ (\lambda y.\ \eta\ (\obj{know}\ x\ y)))) \\
  \rightarrow^*\ & \eta\ (\lambda x.\ \obj{man}\ x \land
  (\exists y.\ \obj{cat}\ y \land \obj{know}\ x\ y)) \\
  +\ & SCOPE_P\ c_{cat}\ (\lambda y.\ \eta\ (\lambda x.\ (\obj{man}\ x \land \obj{know}\ x\ y))) \\
\end{align*}

\begin{align*}
  & \sem{\abs{every}\ (\abs{who}\ (\abs{knows}\ (\abs{a}\ \abs{cat}))\ \abs{man})} \\
\rightarrow^*\ & SCOPE\ (\lambda k.\ \mathrm{SL}\ ((\dforall x.\
  \eta\ (\obj{man}\ x \land (\exists y.\ \obj{cat}\ y \land
  \obj{know}\ x\ y)) \dimp (k\ x))) \\
  & \hspace{25mm} + SCOPE_P\ c_{cat}\ (\lambda y.\ \dforall x.\ (\eta\ (\obj{man}\ x
  \land \obj{know}\ x\ y)) \dimp (k\ x)))\ \eta \\
\rightarrow^*\ & SCOPE\ (\lambda k.\ (\mathrm{SL}\ (\dforall x.\
  \eta\ (\obj{man}\ x \land (\exists y.\ \obj{cat}\ y \land
  \obj{know}\ x\ y)) \dimp (k\ x))) \\
  & \hspace{18mm} + (\mathrm{SL}\ (SCOPE_P\ c_{cat}\ (\lambda y.\ \dforall
  x.\ (\eta\ (\obj{man}\ x \land \obj{know}\ x\ y)) \dimp (k\ x))))\ \eta \\
\rightarrow^*\ & SCOPE\ (\lambda k.\ (\mathrm{SL}\ (\dforall x.\
  \eta\ (\obj{man}\ x \land (\exists y.\ \obj{cat}\ y \land
  \obj{know}\ x\ y)) \dimp (k\ x))) \\
  & \hspace{18mm} + (c_{cat}\ (\lambda y.\ (\mathrm{SL}\ (\dforall
  x.\ (\eta\ (\obj{man}\ x \land \obj{know}\ x\ y)) \dimp (k\ x))))) \\
  & \hspace{18mm} + (SCOPE_P\ c_{cat}\ (\lambda y.\ \mathrm{SL}\ (\dforall
  x.\ (\eta\ (\obj{man}\ x \land \obj{know}\ x\ y)) \dimp (k\ x)))))\ \eta \\
\rightarrow^*\ & SCOPE\ (\lambda k.\ (\mathrm{SL}\ (\dforall x.\
  \eta\ (\obj{man}\ x \land (\exists y.\ \obj{cat}\ y \land
  \obj{know}\ x\ y)) \dimp (k\ x))) \\
  & \hspace{18mm} + (\mathrm{SL}\ (\dexists y.\ (\eta\ (\obj{cat}\ y)) \dand (\mathrm{SL}\ \dforall
  x.\ (\eta\ (\obj{man}\ x \land \obj{know}\ x\ y)) \dimp (k\ x)))) \\
  & \hspace{18mm} + (SCOPE_P\ c_{cat}\ (\lambda y.\ \mathrm{SL}\ (\dforall
  x.\ (\eta\ (\obj{man}\ x \land \obj{know}\ x\ y)) \dimp (k\ x)))))\ \eta \\
\end{align*}

\begin{align*}
  & \sem{\abs{loves}\ (\abs{a}\ \abs{woman})\ (\abs{every}\ (\abs{who}\ (\abs{knows}\ (\abs{a}\ \abs{cat}))\ \abs{man}))} \\
  \rightarrow^*\ & \mathrm{SI}\ (SCOPE\ (\mathrm{SL} \circ c_{man})\ 
  (\lambda x.\ SCOPE_P\ (\mathrm{SL} \circ c_{woman})\
  (\lambda z.\ \eta\ (\obj{love}\ x\ z)))) \\
  \rightarrow^*\ & \mathrm{SL}\ (c_{man}\ 
  (\lambda x.\ \mathrm{SI}\ (SCOPE_P\ (\mathrm{SL} \circ c_{woman})\
  (\lambda z.\ \eta\ (\obj{love}\ x\ z))))) \\
  \rightarrow^*\ & \mathrm{SL}\ (c_{man}\ 
  (\lambda x.\ (\mathrm{SL}\ (c_{woman}\ (\lambda
  z.\ \mathrm{SI}\ (\eta\ (\obj{love}\ x\ z))))) \\
  & \hspace{20mm} + (SCOPE_P\ (\mathrm{SL} \circ c_{woman})\
  (\lambda z.\ \mathrm{SI}\ (\eta\ (\obj{love}\ x\ z)))))) \\
  \rightarrow^*\ & \mathrm{SL}\ (c_{man}\ 
  (\lambda x.\ (\mathrm{SL}\ (c_{woman}\ (\lambda
  z.\ \eta\ (\obj{love}\ x\ z)))) \\
  & \hspace{20mm} + (SCOPE_P\ (\mathrm{SL} \circ c_{woman})\
  (\lambda z.\ \eta\ (\obj{love}\ x\ z))))) \\
\end{align*}



\section{Dynamic Semantics}


\subsection{Anaphora}


\subsection{Presuppositions}


\subsection{Double Negation}


