\section{Evaluating the Resulting Fragment}

We can now use our extensions to produce a variety of grammars. Below, we
will use $\mathcal{G}_{1'}$ to stand for $E_L(\mathcal{G}_1)$.

\begin{itemize}
\item $\mathcal{G}_1$ -- the initial fragment

  Can treat:
  \begin{itemize}
  \item John loves Mary.
  \item John says that Mary loves Alice.
  \end{itemize}

\item $\mathcal{G}_Q = E_Q(E_N(\mathcal{G}_{1'}))$

  Can treat:
  \begin{itemize}
  \item Every man loves a woman.
  \item John says that Mary loves a woman.
  \item John knows a man whom Alice says that Mary loves.
  \end{itemize}
  
\item $\mathcal{G}_D = E_D(\mathcal{G}_{1'})$

  Can treat:
  \begin{itemize}
  \item John says that Mary loves him.
  \end{itemize}

\item $\mathcal{G}_{D'} = E_N(E_A(\mathcal{G}_D)) = E_N(E_{D'}(\mathcal{G}_{1'}))$

  Can treat:
  \begin{itemize}
  \item A man says that he loves Mary.
  \item A man who loves Mary knows Mary.
  \end{itemize}

\item $\mathcal{G}_T = E_D(\mathcal{G}_Q) = E_Q(E_N(\mathcal{G}_D))$

  Can treat:
  \begin{itemize}
  \item Every man who loves a woman knows her.
  \end{itemize}
\end{itemize}
