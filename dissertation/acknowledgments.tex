I would like to thank my supervisors, whose contributions have been essential to
my work. Maxime Amblard was always eager to spend his time listening to all the
developments in my work, including numerous dead ends, and providing help and
support every step of the way. Philippe de Groote had a large impact on my work
by sharing his insight and experience with the lambda calculus, leading to some
of the most important developments in the first part of this thesis. I would
also like to thank the other members of the jury, who have all gone above and
beyond their duty and devoted their time to studying my work: Chris Barker, Hugo
Herbelin, Myriam Quatrini, Christina Unger, Laurent Vigneron.

The research presented in this manuscript was carried out within the Sémagramme
team at Inria. One of the key factors behind the quality of this work was the
friendly atmosphere at the Sémagramme team fostered by its researchers,
engineers, PhD students and interns: Clément Beysson, Sarah Blind, Vu Anh Duc,
Bruno Guillaume, Nicolas Lefebvre, Pierre Ludmann, Aleksandre Maskharashvili,
Guy Perrier, Sylvain Pogodalla, Sai Qian, Stéphane Tiv. My gratitude also
extends to all the people that made the day-to-day working conditions at Loria
very pleasant, notably the excellent staff at the restaurant, the administrative
teams handling all the paperwork (Aurélie Aubry, Céline Simon, François Thaveau)
and the many friends and colleagues that I got to play music with or chat during
lunch breaks: Mihai Andries, Benoît Chappet de Vangel, Émilie Colin, Sylvain
Contassot-Vivier, Baldwin Dumortier, Mariia Fedotenkova, Iñaki Fernández Pérez,
Francesco Giovannini, Arseniy Gorin, Meysam Hashemi, Adrien Krähenbühl, Cecilia
Lindig-Leon, Jérémy Miranda, Aybüke Özgün, Motaz Saad, Yann Salaun, Evangelia
Tsiontsiou, Jan Volec.

My decision to stay in Nancy and do a PhD with the Sémagramme team was in one
part due to an excellent introduction into the fields of formal grammar and
computational semantics that I had during my Master studies there, taught by
Maxime Amblard, Philippe de Groote, Bruno Guillaume, Guy Perrier and Sylvain
Pogodalla. The other part that made me enjoy my Master studies in Nancy so much
were my amazing classmates, talented individuals who were always eager to go to
bars on weekdays and organize study group sessions on weekends: Bruno
Andriamiarina Miharimanana, Georgy Boytchev, Caroline Ceballos, Tatiana
Ekeinhor-Komi, Nicolas Feuillet, Natalia Korchagina, Imane Nkairi, Gabin
Personeni, Wei Qiu, Camille Sauder.

During my PhD studies, I drew a lot of inspiration and motivation from my
regular visits to the European Summer School of Logic, Language and Information.
Among the many bright minds that I have had the time to discuss and spend time
with, I would like to mention Lasha Abzianidze, Eleri Aedmaa, Pepijn Kokke and
Stellan Petersson.

Before ending this section, I would also like to thank the people who have
enriched my life outside of the office. I am grateful to Marie-George Alle-Hans
for her great teaching skills during the university dancing classes. I am
forever in debt to Nelly Madigou for carefully teaching me the basics of cello
playing. I was very lucky to find and join the Orchestre Symphonique
Universitaire de Lorraine, where I got to discover the joys of orchestral
playing. I also wish to thank the many friends and flatmates who have kept me
company during my stay in Nancy: Aurélie Boizard, Orianne Cabaret, Marie Gantet,
Marine Larrière, Jordan Medic, Dorine Petit, Gautier Vacheron.

Finally, I wish to send the biggest thanks to my parents and family who have
supported me during my long studies abroad.