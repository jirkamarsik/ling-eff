\chapter{Properties}


\section{Algebraic Properties}

Here we will introduce a collection of generally useful syntactic shortcuts
and combinators for our calculus.

\begin{align*}
  \_ \circ \_ &: (\beta \to \gamma) \to (\alpha \to \beta) \to (\alpha \to \gamma) \\
  f \circ g &= \lambda x.\ f\ (g\ x) \\
  \_^* &: (\alpha \to \mathcal{F}(\beta)) \to (\mathcal{F}(\alpha) \to \mathcal{F}(\beta)) \\
  f^* &= [\mathcal{H}\ (\eta\ f)] \\
  \mathcal{F} &: (\alpha \to \beta) \to (\mathcal{F}(\alpha) \to \mathcal{F}(\beta)) \\
  \mathcal{F} &= \lambda f.\ (\eta \circ f)^* \\
  \_ \hsbind \_ &: \mathcal{F}(\alpha) \to (\alpha \to \mathcal{F}(\beta)) \to \mathcal{F}(\beta) \\
  M \hsbind N &= N^*\ M \\
  \\
  [\mathcal{H}\ (OP_i\ M_i)\ldots] &= [\mathcal{H}\ (OP_i\ M_i)\ldots\ (\eta\ \eta)]
\end{align*}

We will also make use of combinators for function application where one or
more of the arguments are computations.

\begin{align*}
  \_ \apl \_ &: \mathcal{F}(\alpha \to \beta) \to \alpha \to \mathcal{F}(\beta) \\
  F \apl x &= F \hsbind (\lambda x.\ \eta\ (f\ x)) \\
  \_ \apr \_ &: (\alpha \to \beta) \to \mathcal{F}(\alpha) \to \mathcal{F}(\beta) \\
  f \apr X &= X \hsbind (\lambda x.\ \eta\ (f\ x)) \\
  \_ \ap \_ &: \mathcal{F}(\alpha \to \beta) \to \mathcal{F}(\alpha) \to \mathcal{F}(\beta) \\
  F \ap X &= F \hsbind (\lambda f.\ X \hsbind (\lambda x.\ \eta\ (f\ x)))
\end{align*}


\section{Subject Reduction}

\section{Termination}

\section{Confluence}

