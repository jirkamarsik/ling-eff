\chapter{Introduction to Formal Semantics}
\label{chap:intro-fs}

Semantics is the study of the meaning of language. In this Chapter, we will
review the very basics of a school of formal semantics which originated
with Richard Montague in the early
70's~\cite{montague1970english,montague1970universal,montague1973proper}~(\ref{sec:montague}). We
then present a formalism that embodies the principles of montagovian
semantics, the Abstract Categorial
Grammars~\cite{de2001towards}~(\ref{sec:acg}). In
Chapter~\ref{chap:dynamic-semantics}, we will be analyzing anaphora in
$\calc$ by emulating theories of dynamic semantics. To that end, we briefly
present dynamic semantics by introducing two of its incarnations: Discourse
Representation Theory~\cite{kamp1993discourse}~(\ref{sec:drt}) and
Type-Theoretic Dynamic Logic~\cite{de2006towards}~(\ref{sec:ttdl}).

\minitoc


\section{Montague Semantics}
\label{sec:montague}




\section{Abstract Categorial Grammars}
\label{sec:acg}




\section{Discourse Representation Theory}
\label{sec:drt}




\section{Type-Theoretic Dynamic Logic}
\label{sec:ttdl}
