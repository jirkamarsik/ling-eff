\chapter{Introduction to Formal Semantics}
\label{chap:intro-fs}

Semantics is the study of the meaning of language. In this Chapter, we will
review the very basics of a school of formal semantics which originated
with Richard Montague in the early
70's~\cite{montague1970english,montague1970universal,montague1973proper}~(\ref{sec:montague}). We
then present a formalism that embodies the principles of montagovian
semantics, the Abstract Categorial
Grammars~\cite{de2001towards}~(\ref{sec:acg}). In
Chapter~\ref{chap:dynamic-semantics}, we will be analyzing anaphora in
$\calc$ by emulating theories of dynamic semantics. To that end, we briefly
present dynamic semantics by introducing two of its incarnations: Discourse
Representation Theory~\cite{kamp1993discourse}~(\ref{sec:drt}) and
Type-Theoretic Dynamic Logic~\cite{de2006towards}~(\ref{sec:ttdl}).

\minitoc


\section{Montague Semantics}
\label{sec:montague}

When studying semantics, one of the verifiable predictions that we can make
is whether the contents of one utterance entails the contents of
another. This issue already preoccupied Aristotle, who addressed the
problem in his study of syllogisms. Using his theory, Aristotle could
systematically predict that the contents of
Example~\ref{ex:syllogism-hypothesis} entail the contents of
Example~\ref{ex:syllogism-conclusion}, i.e.\ ``if
(\ref{ex:syllogism-hypothesis}), then (\ref{ex:syllogism-conclusion})'' is
a valid argument.

\begin{exe}
  \ex Every man is mortal. Socrates is a man. \label{ex:syllogism-hypothesis}
  \ex Socrates is mortal. \label{ex:syllogism-conclusion}
\end{exe}

In the 20th century, mathematical logic studied similar properties on
formal artificial languages, leading to the developments of new ideas such
as model theory and Tarski's definition of
truth~\cite{sep-tarski-truth,tarski1986arithmetical}. Montague then argues
that natural languages deserve the same formal treatment as the artificial
languages of logic and mathematics:

\begin{quote}
  There is in my opinion no important theoretical difference between
  natural languages and the artificial languages of logicians; indeed, I
  consider it possible to comprehend the syntax and semantics of both kinds
  of language within a single natural and mathematically precise theory.

  \begin{flushright}
  Universal Grammar~\cite{montague1970universal}
  \end{flushright}
\end{quote}

In formal logic, the formulas of the artificial languages are defined
inductively, by a series of construction rules. The definition of truth is
then inductive on the structure of the formula: for every rule that lets us
form a logical formula, there is a rule which tells us how to compute its
truth value in a model. In his 


\section{Abstract Categorial Grammars}
\label{sec:acg}




\section{Discourse Representation Theory}
\label{sec:drt}




\section{Type-Theoretic Dynamic Logic}
\label{sec:ttdl}
