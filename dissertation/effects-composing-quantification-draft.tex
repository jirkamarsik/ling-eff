\subsection{Quantification}

\subsubsection{Common Nouns}

Our first objective will be to extend a grammar with quantificational
articles like \textit{every} and \textit{some}. However, this presupposes
that our grammar already has some common nouns to combine these articles
with. Since nouns themselves are not bound to the phenomenon of
quantification, we will introduce them in an isolated extension.

We will start by introducing some predicates at the object level (signature
$\Sigma^o_N$):

\begin{align*}
  \obj{man}, \obj{woman} &: \iota \to o
\end{align*}

Now for the abstract constants in signature $\Sigma^a_N$ (we treat
relativizers in our noun extension, since they form noun modifiers):

\begin{align*}
  \abs{man}, \abs{woman} &: N \\
  \abs{who}, \abs{whom} &: (NP \to S) \to N \to N
\end{align*}

Finally, we give an interpretation for the new abstract constants by
extending the lexicon with $\mathcal{L}_N$:

\begin{align*}
  \lex{man}{\eta\ \obj{man}} \\
  \lex{woman}{\eta\ \obj{woman}} \\
  \sem{\abs{who}} := \sem{\abs{whom}} &:= \lambda P N.\ N \hsbind (\lambda n.\ \mathcal{C}\ (\lambda x.\ (\eta\ (n\ x)) \dand (P\ (\eta\ x))))
\end{align*}

We have now finished our extension $E_N$.

$$
E_N(\left< \Sigma^a, \Sigma^o, \mathcal{L} \right>) = \left< \Sigma^a \uplus
\Sigma^a_N, \Sigma^o \uplus \Sigma^o_N, \mathcal{L} \uplus \mathcal{L}_N \right>
$$


\subsubsection{Quantifiers}

We will start our treatment of quantification by introducing an operation
symbol $SCOPE$ into our effect signature $E$. The operation will take
inputs of type $(\iota \to \mathcal{F}(o)) \to \mathcal{F}(o)$ and provide
outputs of type $\iota$ (i.e.\ $SCOPE : ((\iota \to \mathcal{F}(o)) \to
\mathcal{F}(o)) \to \iota \in E$). Our analysis will all revolve around
producing and handling this particular effect.

$$
E_Q(\left< \Sigma^a, \Sigma^o, \mathcal{L} \right>) = \left< \Sigma^a \uplus \Sigma^a_Q, \Sigma^o, \mathcal{L} \uplus \mathcal{L}_Q \right>
$$

First, we start by introducing the abstract constants for the new lexical
items (signature $\Sigma^a_Q$):

\begin{align*}
  \abs{every}, \abs{some}, \abs{a} &: N \to NP
\end{align*}

Now we can add/change the pertinant entries in the lexicon (lexicon $\mathcal{L}_Q$):

\begin{align*}
  \lex{every}{\lambda N.\ SCOPE\ (\lambda k.\ \dforall\ (\lambda x.\ (N \apl x) \dimpl (k\ x)))\ \eta} \\
  \sem{\abs{some}} := \sem{\abs{a}} &:= \lambda N.\ SCOPE\ (\lambda k.\ \dexists\ (\lambda x.\ (N \apl x) \dand (k\ x)))\ \eta \\
  \mathrm{SI} &: \mathcal{F}(o) \to \mathcal{F}(o) \\
  \mathrm{SI} &= [\mathcal{H}\ (SCOPE\ (\lambda c k.\ c\ k))] \\
  \lex{loves}{\lambda X Y.\ \mathrm{SI}\ (\mathcal{L}(\abs{loves})\ X\ Y)} \\
  \lex{knows}{\lambda X Y.\ \mathrm{SI}\ (\mathcal{L}(\abs{knows})\ X\ Y)} \\
  \lex{says}{\lambda S X.\ \mathrm{SI}\ (\mathcal{L}(\abs{says})\ S\ X)} \\
\end{align*}

We have given semantics to the new lexical items which make use of the
$SCOPE$ effect. Quantifier scope is influenced by other lexical items
resulting in so-called scope islands. We therefore define an auxiliary
term, a handler called $\mathrm{SI}$. Since tensed clauses form scope
islands, we insert $\mathrm{SI}$ into the existing lexical entries
for tensed verbs.

If we had a more fine-grained grammar which would provide a separate
lexical entry for the finite verb morpheme, we could simply add the
$\mathrm{SI}$ handler there without having to change all verbs.

A similar solution would work in an unlexicalized grammar in which, e.g.,
transitive verbs would be decomposed into a computation of some binary
relation of abstract type $TV$ and a combinator/rule for forming sentences
using transitive verbs of abstract type $NP \to TV \to NP \to S$. In that
case, $\mathrm{SI}$ would be inserted into the semantics of every
such rule which is to be considered a scope island.
