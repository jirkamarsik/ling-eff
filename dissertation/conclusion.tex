\chapter*{Conclusion}

\section{Summary of Results}

In Part~\ref{part:calculus}, we have introduced $\calc$, a formal calculus
of effects and handlers. Its definition is given in
Chapter~\ref{chap:definitions}. The calculus can be compared to several
existing calculi:

\begin{itemize}
\item System F (i.e.\ the polymorphic lambda calculus or the second-order
  lambda calculus)

  $\calc$ extends the simply-typed lambda calculus with computation types
  $\FF_E(\alpha)$. Computations are algebraic expressions and as such can
  be expressed as inductive data types.\footnote{An inductive type is a
    recursive type with positive
    constructors. In~\ref{ssec:termination-for-idts}, we have seen that a
    computation type $\FF_E(\alpha)$ has positive constructors $\eta$ and
    $\op{op}$ for every $\op{op} \in E$.} Inductive data types, along with
  the sums and products that we add to the calculus in
  Section~\ref{sec:sums-and-products}, can be expressed in System
  F~\cite{wadler1990recursive}.

  In $\calc$, a computation of type $\FF_E(\alpha)$ can also be given the
  type $\FF_{E \uplus E'}(\alpha)$, where $E \uplus E'$ is an extension of
  $E$. However, in the direct encoding of $\calc$ into System F, for every
  $E'$ 
\end{itemize}

\section{Future Work}
