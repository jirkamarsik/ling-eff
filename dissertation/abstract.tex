\AbstractsOnEvenPage
\DontNumberAbstractPages

\begin{ThesisAbstract}
\vspace*{-1cm}

  \begin{FrenchAbstract}

    \begin{french}
    Ces travaux s’intéressent à la modélisation formelle de la sémantique
    des langues naturelles. Pour cela, nous suivons le principe de
    compositionalité qui veut que le sens d’une expression complexe soit
    une fonction du sens de ses parties. Ces fonctions sont généralement
    formalisées à l’aide du $\lambda$-calcul. Cependant, ce principe est
    remis en cause par certains usages de la langue, comme les pronoms
    anaphoriques ou les présuppositions. Ceci oblige à soit abandonner la
    compositionalité, soit modifier les structures du sens. Dans le premier
    cas, le sens n’est alors plus obtenu par un calcul qui correspond à des
    fonctions mathématiques, mais par un calcul dépendant du contexte, ce
    qui le rapproche des langages de programmation qui manipulent leur
    contexte avec des effets de bord. Dans le deuxième cas, lorsque les
    structures de sens sont ajustées, les nouveaux sens ont tendance à
    avoir une structure de monade. Ces dernières sont elles-mêmes largement
    utilisées en programmation fonctionnelle pour coder des effets de bord,
    que nous retrouvons à nouveau. Par ailleurs, s’il est souvent possible
    de proposer le traitement d’un unique phénomène, composer plusieurs
    traitements s’avère être une tâche complexe. Nos travaux proposent
    d’utiliser les résultats récents autour des langages de programmation
    pour parvenir à combiner ces modélisations par les effets de bord.

    Pour cela, nous étendons le $\lambda$-calcul avec une monade qui
    implémente les \textit{effects} et les \textit{handlers}, une technique
    récente dans l’étude des effets de bord. Dans la première partie de la
    thèse, nous démontrons les propriétés fondamentales de ce calcul
    (préservation de type, confluence et terminaison). Dans la seconde
    partie, nous montrons comment utiliser le calcul pour le traitement de
    plusieurs phénomènes linguistiques: deixis, quantification, implicature
    conventionnelle, anaphore et présupposition. Enfin, nous construisons
    une unique grammaire qui gère ces phénomènes et leurs interactions.
    
    \KeyWords{sémantique formelle, compositionalité, effets de bord,
      monades, grammaires catégorielles abstraites, sémantique dynamique.}
    \end{french}
  \end{FrenchAbstract}

  \begin{EnglishAbstract}
    In formal semantics, researchers assign meanings to sentences of a
    natural language. This work is guided by the principle of
    compositionality: the meaning of an expression is a function of the
    meanings of its parts. These functions are often formalized using the
    $\lambda$-calculus. However, there are areas of language which
    challenge the notion of compositionality, e.g.\ anaphoric pronouns or
    presupposition triggers. These force researchers to either abandon
    compositionality or adjust the structure of meanings.  In the first
    case, meanings are derived by processes that no longer correspond to
    pure mathematical functions but rather to context-sensitive procedures,
    much like the functions of a programming language that manipulate their
    context with side effects. In the second case, when the structure of
    meanings is adjusted, the new meanings tend to be instances of the same
    mathematical structure, the monad. Monads themselves being widely used
    in functional programming to encode side effects, the common theme that
    emerges in both approaches is the introduction of side
    effects. Furthermore, different problems in semantics lead to different
    theories which are challenging to unite. Our thesis claims that by
    looking at these theories as theories of side effects, we can reuse
    results from programming language research to combine them.

    This thesis extends $\lambda$-calculus with a monad of computations. The
    monad implements effects and handlers, a recent technique in the study
    of programming language side effects.  In the first part of the thesis,
    we prove some of the fundamental properties of this calculus: subject
    reduction, confluence and termination. Then in the second part, we
    demonstrate how to use the calculus to implement treatments of several
    linguistic phenomena: deixis, quantification, conventional implicature,
    anaphora and presupposition. In the end, we build a grammar that
    features all of these phenomena and their interactions.

    \KeyWords{formal semantics, compositionality, side effects, monads,
      abstract categorial grammars, dynamic semantics.}
  \end{EnglishAbstract}
\end{ThesisAbstract}
