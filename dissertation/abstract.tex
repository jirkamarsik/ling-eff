\begin{ThesisAbstract}
  \begin{FrenchAbstract}
    Dans la sémantique formelle, les chercheurs attribuent des sens à des
    phrases d'un langage naturel. Ce travail est guidé par le principe
    méthodologique de la compositionalité: le sens d'une expression
    complexe est une fonction des sens de ses constituants. Ces fonctions
    de sens sont souvent formalisé en utilisant le
    lambda-calcul. Cependant, il existe des parties de langage qui
    remettent en cause la notion de compositionnalité, des exemples communs
    étant les pronoms anaphoriques et les déclencheurs des
    présuppositions. Ceux-ci forcent les chercheurs à soit ajuster la
    structure des sens ou abandonner la compositionalité au sens strict du
    terme. Dans le premier cas, les nouveaux sens ont tendance à avoir la
    structure d'une monade, alors que dans le second cas, les sens sont
    dérivés avec des fonctions aux effets de bord. Les monades étant
    elles-mêmes largement utilisés dans la programmation fonctionnelle pour
    coder des effets de bord, le thème sous-jacent, commun à ces deux
    approches, est l'introduction d'effets de bord. En outre, les divers
    problèmes dans la sémantique conduisent à des diverses théories qui
    sont difficiles à unir. Notre thèse prétend que, en regardant ces
    théories comme des théories d'effets de bord, on peut réutiliser les
    résultats de la recherche dans les langages de programmation pour les
    combiner.

    Cette thèse étend le lambda-calcul avec une monade des computations. La
    monade implémente les effets et les handlers, une technique récente
    dans l'étude des effets de bord du langage de programmation. Dans la
    première partie de la thèse, nous démontrons quelques propriétés
    fondamentales de ce calcul: la réduction du sujet, la confluence et la
    terminaison. Ensuite, dans la deuxième partie, on montre comment
    utiliser le calcul pour mettre en œuvre des traitements de plusieurs
    phénomènes linguistiques: deixis, quantification, implicature
    conventionnelle, anaphore et présupposition. En fin de compte, on
    construit une grammaire qui couvre tous ces phénomènes et leurs
    interactions.

    \KeyWords{sémantique formelle, compositionalité, effets de bord,
      monades, grammaires catégorielles abstraites, sémantique dynamique.}
  \end{FrenchAbstract}

  \begin{EnglishAbstract}
    In formal semantics, researchers assign meanings to sentences of a
    natural language. This work is guided by the methodological principle
    of compositionality: the meaning of a complex expression is a function
    of the meanings of its constituents. These meaning functions are often
    formalized using the lambda calculus. However, there are areas of
    language which challenge the notion of compositionality, common
    examples being anaphoric pronouns and presupposition triggers. These
    force researchers to either adjust the structure of meanings or to
    abandon compositionality in the strict sense of the word. In the former
    case, the new meanings tend to have the structure of a monad, whereas
    in the latter, meanings are derived by functions with side
    effects. Monads themselves being widely used in functional programming
    to encode side effects, the common theme that emerges in both
    approaches is the introduction of side effects. Furthermore, different
    problems in semantics lead to different theories which are challenging
    to unite. Our thesis claims that by looking at these theories as
    theories of side effects, we can reuse results from programming
    language research to combine them.

    This thesis extends lambda calculus with a monad of computations. The
    monad implements effects and handlers, a recent technique in the study
    of programming language side effects.  In the first part of the thesis,
    we prove some of the fundamental properties of this calculus: subject
    reduction, confluence and termination. Then in the second part, we
    demonstrate how to use the calculus to implement treatments of several
    linguistic phenomena: deixis, quantification, conventional implicature,
    anaphora and presupposition. In the end, we build a grammar that covers
    all of these phenomena and their interactions.

    \KeyWords{formal semantics, compositionality, side effects, monads,
      abstract categorial grammars, dynamic semantics.}
  \end{EnglishAbstract}
\end{ThesisAbstract}
