\documentclass[11pt]{lettre}

\usepackage[utf8]{inputenc}
\usepackage[francais]{babel}

\makeatletter
\newcommand*{\NoRule}{\renewcommand*{\rule@length}{0}}
\makeatother

\begin{document}
\begin{letter}{M. le Président de l'Université de Lorraine \\
               s/c \\
               M. le Directeur de l'école doctorale IAEM}
\NoRule
\name{Jiří Maršík}
\signature{Jirka Maršík \\ Doctorant au Loria}
\address{Jiří Maršík \\ Loria \\ Campus Scientifique \\ BP 239 \\ 54506 Vandoeuvre-lès-Nancy}
\lieu{Vandœuvre-lès-Nancy}
\telephone{06 38 01 13 84}
\email{jiri.marsik@univ-lorraine.fr}
\nofax

\def\concname{Objet :~} % On définit ici la commande 'objet'
\conc{Demande de dérogation --- rédaction et soutenance d'une thèse en anglais}

\opening{Monsieur le président,}

Je sollicite de votre haute bienveillance l’autorisation de rédiger et de
soutenir ma thèse de doctorat en anglais.

Les raisons pour lesquels je voudrais rédiger et soutenir ma thèse en
anglais:

\begin{itemize}
\item Je ne maîtrise pas le langue français autant pour pouvoir écrire
  toute une thèse dedans sans que la qualité ne dégrade trop.
\item On envisage à inviter des rapporteurs non-francophones qui donc ne
  pourraient pas lire et juger le travail correctement.
\item Le manuscrit traitera une problématique étudié aussi par des
  chercheurs en dehors des pays francophones. Une thèse en français aurait
  empêché ces chercheurs à accéder à mon travail.
\end{itemize}

Cette demande est faite en accord avec mes directeurs de thèse, Ph.\ de
Groote et M.\ Amblard.

\closing{En vous remerciant de l’attention que vous voudrez bien porter à
  ma demande, je vous prie, Monsieur le Président, d’agréer mes salutations
  respectueuses.}

\end{letter}
\end{document}
