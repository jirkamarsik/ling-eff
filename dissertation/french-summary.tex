\chapter*{Résumé en français}
\label{chap:french-summary}

La motivation de ces travaux de thèse est de proposer des traductions
automatiques d'énoncés en langue naturelle en des représentations logiques. 
L'intérêt est double, d'un côté de permettre aux linguistes de définir une sémantique
formelle de la langue et de l'autre côté de permettre aux informaticiens de les 
utiliser pour, par exemple, créer des systèmes de raisonnement automatique sur des données 
exprimées en langue naturelle.

De nombreux travaux de recherche cherchent à définir un système transformant 
des énoncés en anglais ou en d'autres langues en représentation sémantique. Mais ils se heurtent à plusieurs difficultés.
Par exemple dans une phrase en anglais, on peut identifier plusieurs phénomènes 
linguistiques et leur associer des théories qui en rendent compte.

\begin{exe}
  \exr{ex:intro} She might still be dating that idiot.
\end{exe}

Cet exemple est constitué :
\begin{enumerate}
  \item \label{item:french-first-feature} d'expressions anaphoriques, comme
    le pronom \emph{she}. On sait qu'il est possible de traduire des phrases avec des anaphores
    dans des structures de la Discourse Representation
    Theory~(DRT)~\cite{kamp1993discourse}, des formules de la Dynamic Predicate
    Logic~(DPL)~\cite{groenendijk1991dynamic} ou des $\lambda$-termes avec des
     continuations~\cite{de2006towards}.
  \item de l'auxiliaire modal \emph{might} qu'on peut traduire comme un
    opérateur d'une logique modale ou une quantification existentielle sur des
    mondes possibles.
  \item du temps présent progressif dans \emph{be dating}. Comme avec
    l'auxiliaire modal, on peut le traduire en un opérateur d'une logique
    temporelle ou introduire une quantification existentielle sur des
    intervalles de temps pendant lesquels le fait de \emph{dating} a lieu et postuler
    que le moment d'énonciation de la phrase se trouve dans cet intervalle.
  \item du déclencheur de présupposition \emph{still} qui implique que le sujet
    et l'objet sortaient déjà ensemble. Un mécanisme permettant de projeter
    cette présupposition en dehors de la portée de n'importe quel opérateur
    logique est alors nécessaire.\footnote{Dans le présent, on peut déduire que
      les deux sont \emph{peut-être} en train de sortir ensemble (grâce à
      l'auxiliaire modal \emph{might}), mais avec la présupposition, on peut
      déduire que dans le passé, ils ont \emph{dû} sortir ensemble.} On peut
    utiliser la stratégie de Lebedeva~\cite{lebedeva2012expression} qui utilise
    une levé d'exception pour projeter la présupposition.
  \item \label{item:french-last-feature} de l'épithète expressif \emph{idiot}.
    En suivant la théorie des implicatures conventionnelles de
    Potts~\cite{potts2005logic}, élaborée par Gutzmann~\cite{gutzmann2015use},
    on introduit une deuxième dimension du sens dans laquelle on va noter
    l'attitude négative de l'interlocuteur envers l'objet.
\end{enumerate}

Les descriptions ci-dessus peuvent sembler être une bonne première approximation de 
l'exemple. Il est possible de les reprendre et intuitivement construire une représentation 
logique raisonnable du contenu informationnel. Mais comment formaliser ce processus 
complexe ? La plupart de ces théories utilisent leur propre système de représentation, leurs propres
définitions, notations et opérations.

La DRT introduit un encodage spécifique des formules logiques et propose un
algorithme qui les construits progressivement à partir de la phrase~\cite{kamp1993discourse}. La logique des implicatures conventionnelles de
Potts introduit des formules logiques bidimensionnelles et définit ses propres
façons de les combiner~\cite{potts2005logic}. Les traitements compositionnels
de l'intensionalité ou du temps ont tendance à utiliser le $\lambda$-calcul
simplement typé~\cite{ben2007semantics,de2013note}, comme c'est également le cas
avec le traitement des anaphores de de Groote~\cite{de2006towards}. Pour son traitement
des présuppositions, Lebedeva utilise une version modifiée du calcul de de Groote
qui inclut aussi les exceptions~\cite{lebedeva2012expression}.

Il semble clair que pour rendre compte des phénomènes présentés dans les descriptions~\ref{item:french-first-feature} à~\ref{item:french-last-feature}, il est pertinent de proposer un cadre universel basé sur un langage formel.

\section*{Les monades}

Notre langage universel est fondé sur le $\lambda$-calcul. Grâce au travail très
influent de Richard Montague~\cite{montague1973proper}, le $\lambda$-calcul est
un formalisme répandu en sémantique compositionnelle formelle.\footnote{Il est
  généralement motivé par le principe de compositionnalité de Frege qui dit que
  le sens d'une expression complexe doit être déterminé par le sens de ses
  constituants (c.-à-d.\ être une fonction de). Si un sens complexe est une
  fonction d'autres sens, il est logique d'utiliser un calcul sur les fonctions,
  comme le $\lambda$-calcul.} Beaucoup de systèmes pour la sémantique utilisent
le $\lambda$-calcul et les autres ont tendance à se traduire en $\lambda$-calcul
(par exemple la $\lambda$-DRT~\cite{kuschert1995type} ou le traitement de
dynamicité à partir des continuations de de Groote~\cite{de2006towards}).

Cependant, même si nous avons plusieurs théories qui sont toutes formalisées
dans le $\lambda$-calcul, cela ne signifie pas nécessairement qu'elles sont
compatibles ou que nous savons les combiner. Une théorie de l'intensionalité
pourrait déclarer que les phrases sont traduites en termes de type
$\sigma \to o$, le type des fonctions des mondes possibles vers les valeurs de
vérité. D'un autre côté, une analyse des expressives suggère que des phrases
devraient correspondre à des termes de type $o \times \epsilon$, le type des
paires composées de valeurs de vérité (le contenu propositionnel) et de marqueurs
expressifs (le contenu expressif). Les deux théories seraient compatibles au
niveau du calcul utilisé mais pas au niveau des termes. Une fonction
traitant des propositions intensionnelles ne seraient pas directement
applicables à une proposition expressive.

Pour poursuivre notre recherche d'uniformité et de compatibilité des opérations
sémantiques, nous avons examiné les termes et les types utilisés par les
traitements sémantiques à partir du $\lambda$-calcul et nous avons essayé de trouver
une structure sous-jacente commune. Nous remarquons que l'ensemble de ces 
approches partagent les caractéristiques suivantes :

\begin{enumerate}
\item \label{item:french-type-transformation} Les types de certaines 
  dénotations sont élargies. Par exemple, lorsqu'il s'agit de quantificateurs,
  le type des dénotations des syntagmes nominaux va de $\iota$ (des individus) à
  $(\iota \to o) \to o$ (des quantificateurs généralisés sur les individus);
  dans la sémantique intensionnelle, le type des dénotations des phrases va de $o$
  (des valeurs de vérité) à $\sigma \to o$ (des fonctions des mondes possibles
  vers des valeurs de vérité, c.-à-d.\ des ensembles de mondes possibles) ; et
  avec les expressives, le type des dénotations des phrases va de $o$ à $o \times
  \epsilon$ (des valeurs de vérité associées à des marqueurs expressifs).
\item \label{item:french-monad-eta} Il existe un processus pouvant transformer des
  dénotations de l'ancien type dans des dénotations du nouveau type. Dans
  l'exemple avec les quantificateurs, il s'agit de la fameuse opération de type 
  <<\textit{type raising}>>. Dans l'exemple avec les intensions,
  il s'agit du combinateur $\textbf{K}$ qui transforme une valeur de vérité en
  une fonction constante qui attribue cette valeur de vérité à tous les mondes,
  (une intension rigide). Dans l'exemple avec des expressifs, c'est la fonction
  qui couple une proposition avec un marqueur expressif neutre/vide.
\item Puis il y a d'autres termes de type étendu qui ne peuvent pas être obtenus
  en utilisant la fonction de \textit{type raising} évoqué ci-dessus ; ce sont
  ceux pour lequel nous avons élargi le type. Les syntagmes nominaux
  quantificationnel tels que \emph{everyone} ne sont pas les résultats d'un
  \textit{type raising} sur un terme de type d'individu. Les propositions
  intensionnelles telles que \emph{Hespherus is Phosphorus} ont des extensions
  qui varient d'un monde à l'autre. Les expressives tel que le diminutif
  \emph{Wolfie} indiquent un individu et portent également un marqueur expressif
  qui transmet l'attitude du locuteur envers le référent.
\item \label{item:french-monad-mu} Enfin, ces approches ont aussi une stratégie 
  de composition des dénotations plus petites en des dénotations plus grandes
  et de traitement de la complexité ajoutée par les types plus élaborés. Lorsqu'on
  applique un verbe transitif à un sujet et à un objet quantificationnel, on
  laisse l'un (souvent le sujet) prendre la portée et ensuite on laisse l'autre
  prendre la sienne. Lorsque nous appliquons le verbe aux arguments
  intensionnels, nous passons le monde où nous évaluons la phrase au sujet et
  à l'objet. Quand on l'applique à des arguments expressifs, on applique le verbe
  aux référents du sujet et de l'objet et on recueille le contenu expressif des deux.
\end{enumerate}

Ce type de structure est en réalité très commun dans la programmation fonctionnelle et dans
la sémantique dénotationnelle des langages de programmation. C'est la structure
d'un \emph{foncteur applicatif}~\cite{mcbride2008applicative}. Les exemples
ci-dessus sont également des instances d'une structure plus spéciale appelée une
\emph{monade}~\cite{moggi1991notions}.

Nous n'entrons pas dans cette section dans les détails de la définition d'une monade que l'on peut retrouver dans ce document,
mais nous allons néanmoins en donner un aperçu. Une monade est un triplet $(T, \eta,
\hsbind)$ où $T$ est une fonction sur les types (l'expansion des types que nous
avons vu dans la section~\ref{item:french-type-transformation}), $\eta$ est une façon de
transformer des valeurs simples en valeurs étendues (les fonctions de \textit{type raising}
dans~\ref{item:french-monad-eta}) et $\hsbind$ nous donne un moyen général de
combiner des valeurs de ce type étendu (semblable aux exemples données
dans~\ref{item:french-monad-mu}).\footnote{Cette façon de présenter une monade (un
  constructeur de type, $\eta$ et $\hsbind$) est particulière à la programmation
  fonctionnelle. Notez que cette présentation diffère de celle utilisée dans la
  théorie des catégories, qui remplace $\hsbind$ par une transformation
  naturelle $\mu$ \cite{mac1978categories}.} Le triplet doit également satisfaire
certaines propriétés algébriques qui garantissent que la composition des
fonctions sur des types étendus est associative et que la fonction de \emph{type raising}
sert d'unité pour cet opérateur de composition.

Les analyses discutées ci-dessus sont toutes des exemples de monades. La
prévalence des monades dans la sémantique de la langue a déjà été
mise en avant par Shan dans~\cite{shan2002monads}. Cependant, le défi consiste à
combiner l'utilisation de plusieurs monades en même temps.


\section*{Les effets de bord linguistiques}

Les monades apparaissent souvent dans la sémantique dénotationnelle des langages
de programmation pour tenir compte des notions de calcul communément appelées
\emph {les effets de bord}~\cite{moggi1991notions}. Nous pouvons nous baser sur
cette correspondance et considérer la structure monadique en langue naturelle
comme des effets de bord linguistiques. Cette analogie a été poursuivie par
Shan~\cite{shan2005linguistic,shan2005thesis} et
Kiselyov~\cite{kiselyov2008call} et est présente dans les travaux récents sur
les monades dans la sémantique de la langue
naturelle~\cite{giorgolo2012monads,charlow2014semantics}. Cependant, l'idée
remonte avant que les monades soient introduites en informatique. Dans leur
article de 1977, Hobbs et Rosenschein prennent une perspective computationnelle
sur la logique intensionnelle de Montague~\cite{montague1973proper} : les
intensions correspondent aux programmes et les extensions correspondent aux
valeurs. Un programme peut accéder aux variables globales qui décrivent l'état
du monde.\footnote{La dépendance à un environnement d'un certain type $\sigma$
  est un effet de bord qui peut être décrit en utilisant la monade <<reader>>.
  Cette monade transforme le type $\alpha$ vers le type $\sigma \to \alpha$.
  Cette transformation de types est exactement la même que celle décrite par les
  théories d'intensionalisation~\cite{ben2007semantics,de2013note}.} Les
opérateurs $\uparrow$ et $\downarrow$, qui traduisent les expressions dénotant
les extensions et les expressions dénotant les intensions, correspondent
respectivement aux opérateurs $\texttt{quote} $ et $\texttt{eval}$ dans le
langage Lisp.

L'idée de traiter les expressions linguistiques comme des actions ou des
programmes avec des effets est également pertinente pour la sémantique
dynamique, qui traite le sens d'une phrase comme des instructions mettant à
jour le <<common ground>> (ou un autre contexte
linguistique).\footnote{L'utilisation des monades pour encoder des effets
  dynamiques (l'anaphore) remonte à 2009 et aux travaux de Giorgolo et
  Unger~\cite{giorgolo2009coreference,unger2012dynamic}.} La sémantique
dynamique et l'anaphore sont parfois classées comme appartenant à la fois à la
sémantique et à la pragmatique. C'est également le cas pour d'autres phénomènes
que nous traitons comme des effets de bord dans cette thèse: la deixis, la
présupposition, l'implicature conventionnelle. La pragmatique étudie la façon
dont un langage se situe dans la communauté de ses utilisateurs, c.-à-d.\ comment
il est réellement utilisé par ses locuteurs pour atteindre leurs objectifs. Il
ne serait donc pas surprenant que la pragmatique corresponde bien aux effets de
bord des langages de programmation, car les effets de bord eux-mêmes concernent
la façon dont les programmes peuvent interagir avec le monde de leurs
utilisateurs (par exemple, en faisant apparaître des choses sur l'écran ou en
récupérant des entrées fournies par l'utilisateur).


\section*{Les effets et les handlers}

En prenant les différentes structures monadiques de la sémantique de la langue
naturelle comme des effets de bord, nous pouvons appliquer des théories qui les 
combinent pour définir un formalisme capable d'exprimer différents aspects du langage 
en même temps. Les effets et les handlers sont pour nous un tel
cadre théorique. Ici, les programmes sont interprétés comme des
séquences d'instructions (ou plus généralement comme des arbres de
décision).\footnote{Plus précisément, nous interprétons des programmes dans une
  monade libre~\cite{swierstra2008data}.} Les instructions sont des symboles
appelés des \emph{opérations}, qui représentent les différents effets, les
différentes façons dont les programmes peuvent interagir avec leur contexte.
Dans notre application à la sémantique de la langue naturelle, voici quelques
exemples d'opérations qui figurent dans nos définitions, ainsi que leur
sémantique :\footnote{Les opérations ne sont que des symboles et n'ont donc pas
  un sens inhérent.}

\begin{itemize}
\item $\op{introduce}$ introduit un nouveau référent de discours dans le
  contexte. C'est l'opération utilisée par des syntagmes nominaux comme
  l'indéfini \emph{a man}.
\item $\ap{\op{presuppose}}{P}$ présuppose l'existence d'une entité satisfaisant
  le prédicat $P$. Elle est utilisée par des descriptions définies \emph{the $P$}
  et par des noms propres.
\item $\ap{\op{implicate}}{i}$ indique que $i$ est une implicature
  conventionnelle. Cette opération est utilisée par des constructions
  appositives telles que \emph {John, who is my neighbor}.
\item $\op{speaker}$ demande au contexte l'identité du locuteur. Il est utilisé
  par le pronom de première personne pour trouver son référent.
\end{itemize}

Le calcul de la dénotation d'une expression linguistique est décomposé en ces
opérations. Lorsque les expressions se combinent pour former des phrases et des
discours, ces opérations finissent par être concaténées dans un grand programme
qui effectue une série d'interactions avec son contexte. C'est à ce point que
les handlers sont utilisés. Un \emph{handler} est un interpréteur qui donne une
définition aux symboles d'opération présents dans un programme. Les handlers peuvent être
rendus modulaires\footnote{De la même manière que les monades peuvent être
  transformées en <<monad transformers>> (<<monad morphisms>>) puis
  composées~\cite{shan2002monads,wu2015transformers}.} afin que l'interprète
global de toutes les interactions avec le contexte puisse être défini comme la
composition de plusieurs petits handlers, chacun traitant un aspect différent du
langage (la dynamique, les implicatures, la deixis\ldots).

Lorsque nous utilisons les effets et les handlers, nous commençons par énumérer
l'ensemble de toutes les interactions possibles que les programmes (c.-à-d.\ les
expressions linguistiques dans notre application) peuvent avoir avec leurs
contextes. Ensuite, nous pouvons interpréter les expressions linguistiques comme
des séquences d'instructions. Enfin, nous écrivons des handlers qui
mettent en œuvre ces instructions et produisent une représentation sémantique
appropriée. Cette approche suit donc de près le mantra donné par Lewis:

\begin{quote}
  \begin{english}
    \textit{In order to say what a meaning is, we may first ask what a meaning
      does and then find something that does that.}
  \end{english}

  Afin de dire ce que c'est le sens, nous pouvons d'abord demander ce que le
  sens fait et puis trouver quelque chose qui le fait.

  \begin{flushright}
    General Semantics, David Lewis~\cite{lewis1970general}
  \end{flushright}
\end{quote}

Nous pouvons faire remonter les origines des effets et des handlers à deux moments. L'un
est le travail de Cartwright et Felleisen sur les Extensible Denotational
Language Specifications~\cite{cartwright1994extensible}, dans lequel une
technique de développement de la sémantique est présentée qui fait que lorsqu'un
langage (de programmation) est étendu avec des nouvelles constructions (et des
nouveaux effets de bord), les dénotations restent compatibles et peuvent être
réutilisées. L'autre précurseur est le travail de Hyland, de Plotkin et de Power
sur les effets algébriques~\cite{hyland2006combining}, une technique catégorique
pour étudier les programmes avec des effets de bord, qui fut ensuite étendu par
Plotkin et Pretnar pour inclure les
handlers~\cite{plotkin2009handlers,pretnar2010logic,plotkin2013handling}. La
technique a gagné en popularité ces dernières années (depuis 2012). Elle
trouve des applications à la fois dans l'encodage des effets dans les langages
de programmation fonctionnels
purs~\cite{kiselyov2013extensible,kiselyov2015freer,kammar2013handlers,brady2013programming}
et dans la conception des langages de
programmation~\cite{bauer2012programming,lindley2016dobedobedo,dolan2015effective,kiselyov2016eff,hillerstrom2016compiling}.
Notre thèse explore l'applicabilité des effets et des handlers à la sémantique
de la langue naturelle.

\section*{Résumé des résultats}
\label{sec:french-summary-of-results}

Après avoir introduit la problématique dans laquelle se situe
ces travaux, nous revenons sur une description du contenu du manuscrit.

Dans la partie~\ref{part:calculus}, nous introduisons $\calc$, un calcul
formel qui étend le $\lambda$-calcul simplement typé (STLC) avec des effets et
des handlers.

La définition de $\calc$ est donnée dans le chapitre~\ref{chap:definitions}.
$\calc$ introduit une nouvelle famille de types dans STLC, les types des
calculs, et des nouveaux termes, qui comprennent des constructeurs et des
destructeurs des calculs. Nous avons donné un système de types au calcul qui
étend celui de STLC et une sémantique de réduction qui combine les réductions de
STLC $\beta$ et $\eta$ avec les définitions des nouveaux symboles. Dans ce
chapitre, nous maintenons deux perspectives sur la signification de ces termes :
les calculs peuvent être considérés comme des programmes qui interagissent avec
un système à travers un ensemble d'opérations spéciales ou ils peuvent être
considérés comme des expressions algébriques construites à partir d'une signature
algébrique infinie.

Dans le chapitre~\ref{chap:examples}, nous avons donné un exemple d'utilisation
du calcul $\calc$. En plus d'introduire les notations et les
réductions du calcul, l'exemple sert d'aperçu du genre d'ingénierie
linguistique que nous faisons plus tard dans le manuscrit. Au cours du chapitre,
nous développons une sémantique compositionnelle pour un langage
informatique simple avec des erreurs et des variables. Ceci nous permet de
démontrer la modularité de l'utilisation de notre monade de calcul, car nous
pouvons ajouter des variables au langage sans avoir à modifier la sémantique
des autres constructions.

La contribution principale de la partie~\ref{part:calculus} se trouve dans le
chapitre~\ref{chap:properties}, dans lequel nous avons développé la métathéorie
de $\calc$. Dans la section~\ref{sec:derived-rules}, les concepts qui sont
primitifs dans d'autres langues (les handlers clos et l'opérateur $\hsbind$) ont
été définis dans $\calc$ et leurs règles de typage et de réduction ont été
dérivées à partir de celles de $\calc$. Dans la
section~\ref{sec:algebraic-properties}, nous avons ensuite relié le calcul à la
théorie des monades en identifiant une monade dans la catégorie dans laquelle
nous interprétons $\calc$ avec notre \emph{sémantique dénotationnelle}. Dans la
section~\ref{sec:type-soundness}, nous avons prouvé la \emph{préservation des
  types} (<<subject reduction>>) dans $\calc$. Ce résultat donne une cohérence
entre le système des types de $\calc$ et sa sémantique de réduction,
garantissant que les types sont conservés par la réduction. Ceci est complété par
une preuve de \emph{progression}, qui indique que les termes qui n'utilisent
aucun des opérateurs partiels et qui ne peuvent plus être réduits doivent avoir
une forme spécifique.

Nous avons enchaîné avec la preuve d'une autre propriété fondamentale : la
\emph{normalisation forte}. Sa preuve a été divisée en deux parties :
\emph{confluence} (prouvée dans la section~\ref{sec:confluence}) et
\emph{terminaison} (prouvée dans la section~\ref{sec:termination}). Les preuves
de confluence et de terminaison procèdent par des stratégies similaires : prouver
la propriété pour le calcul sans la $\eta$-réduction en appliquant un résultat
général, puis étendre la propriété au calcul complet. Dans le cas de la
confluence, le résultat général est la confluence des systèmes de réduction
combinatoires orthogonaux~\cite{klop1993combinatory}. Dans le cas de la
terminaison, nous nous appuyons sur deux techniques : la terminaison de la
relation de réduction dans les systèmes de type de données inductives qui
valident le schéma général~\cite{blanqui2000termination} et l'étiquetage
sémantique d'ordre supérieur~\cite{hamana2007higher}, ce qui nous permet
d'utiliser notre sémantique dénotationnelle pour étiqueter les termes de notre
calcul afin de valider le schéma général.

Andrej Bauer a fait l'analogie dans laquelle les effets et les handlers sont aux
continuations délimitées ce que les boucles while ou les constructions
if-then-else sont aux instructions du type \textit{goto}~\cite{bauer2012lambda}. Les
continuations elles-mêmes se sont avérées être un outil efficace dans la
sémantique de la langue naturelle~\cite{de2001type,barker2002continuations,shan2005linguistic,de2006towards,
  barker2006continuations,barker2014continuations}. Dans le
chapitre~\ref{chap:continuations}, nous avons montré comment $\calc$ peut
simuler des continuations délimitées, à savoir les opérateurs de contrôle
délimitée $\shift$/$\reset$. Nous avons présenté un $\lambda$-calcul avec appel
par valeur et les opérateurs $\shift$ et $\reset$ et simulé ses types et ses
réductions dans $\calc$.

Dans la partie~\ref{part:natural-language}, nous démontrons les applications
de $\calc$ aux problèmes de modélisation du sens des énoncés de la langue
naturelle.

Après avoir examiné les bases de la sémantique formelle dans le
chapitre~\ref{chap:intro-fs}, nous avons montré comment les calculs dans $\calc$
peuvent être utilisés pour donner une sémantique compositionnelle à plusieurs
phénomènes linguistiques ``non compositionnels". Dans le
chapitre~\ref{chap:introducing-effects}, nous avons décrit comment introduire
des calculs dans une sémantique compositionnelle tout en préservant les
dénotations assignées par la sémantique dans la
section~\ref{sec:lifting-semantics}. Nous avons ensuite présenté des analyses de
plusieurs phénomènes linguistiques dans $\calc$: la deixis
(section~\ref{sec:deixis}), l'implicature conventionnelle à la Potts
(section~\ref{sec:conventional-implicature}) et la quantification à la Montague
(section~\ref{sec:quantification}). Nous avons ensuite décrit explicitement la
méthodologie utilisée pour trouver le genre d'analyses que nous avons présentées
dans le chapitre~\ref{chap:introducing-effects} afin d'encourager les chercheurs
à élaborer d'autres analyses dans le même cadre.

Nous avons consacré le chapitre~\ref{chap:dynamic-semantics} à un phénomène
particulièrement complexe : la dynamique. Dans les sections~\ref{sec:drt-as-pl}
et~\ref{sec:banana-drt}, nous avons montré comment une analyse $\calc$ de la
dynamique peut être extraite à partir de la Discourse Representation Theory.
Cela donne un moyen de gérer la dynamique dans $\calc$ et renforcer l'idée que
les effets et les handlers sont des mécanismes appropriés pour traiter la langue
naturelle. Nous avons également montré comment interpréter les calculs $\calc$
comme les propositions dynamiques de la Type-Theoretic Dynamic Logic
(TTDL)~\cite{de2006towards}. Dans sa thèse~\cite{lebedeva2012expression},
Lebedeva a étendu TTDL avec des exceptions pour traiter les présuppositions et
dans la section~\ref{sec:presuppositions}, nous avons intégré l'analyse de
présuppositions de Lebedeva dans notre analyse $\calc$ de la dynamique (nous
avons comparé notre adaptation avec l'original dans
\ref{ssec:comparison-ttdl}). Dans la section~\ref{sec:double-negation},
nous avons considéré une autre extension de TTDL pour la double
négation~\cite{qian2014accessibility} et nous avons montré que le \textit{type
  raising} des dénotations qui y est utilisé ne peut pas être simulé dans
$\calc$.

Dans le chapitre~\ref{chap:composing-effects}, nous montrons comment l'utilisation d'effets et de handlers permet de
combiner le traitement de phénomènes différents dans une seule grammaire. Nous avons commencé
avec la grammaire dynamique développée dans le
chapitre~\ref{chap:dynamic-semantics}, répété dans les
sections~\ref{sec:dynamic-kernel} et~\ref{sec:adding-presuppositions}. Nous
avons ensuite étendu cette grammaire avec les implicatures
conventionnelles~(\ref{sec:adding-conventional-implicature}), la
deixis~(\ref{sec:adding-deixis}) et la
quantification~(\ref{sec:adding-quantification}) avec peu ou aucune modification
de la sémantique originale. Nous terminons le chapitre en esquissant une
analyse des subordonnées restrictives et leurs interactions avec les
présuppositions dans la section~\ref{sec:relative-clauses}.