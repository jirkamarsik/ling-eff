\chapter{Connections}

In this chapter, we will draw some connections to related formalisms.

\begin{itemize}
\item In~\ref{sec:continuations}, we will see how the semantics of our
  calculus echoes those of calculi that offer control operators for
  delimited continuations.
\item In \ref{sec:monads}, \ldots
\item In \ref{sec:translation-to-coc}, \ldots
\item In \ref{sec:reason-for-c}, \ldots
\end{itemize}

\section{Continuations}
\label{sec:continuations}

We will try to show the connection between the operational semantics of
operators for delimited control and the reduction semantics of our
calculus. The nature of control operators is that their evaluation depends
on their context. In order for a language with such control operators to be
deterministic, it must have a fixed evaluation order. So in order to set up
the stage for our study of delimited control, we will start by simulating
the call-by-value $\lambda$-calculus in $\banana{\lambda}$.

\subsection{Introducing Call-by-Value}

First, we introduce some key notions of call-by-value and ordered
evaluation.

We will single out some of the expressions in the $\lambda$-calculus and
call them values.

\begin{align*}
  V ::= &\ \lam{x}{M} \\
   | \, &\ x \\
  M, N ::= &\ V \\
   | \, &\ (\ap{M}{N})
\end{align*}

The idea behind this distinction is that values ($V$) are terms that have
already been reduced/evaluated. This distinction will become useful in
definining the notion of an \emph{evaluation context} $C$:

\begin{align*}
  C ::= &\ [] \\
  | \, &\ (\ap{C}{M}) \\
  | \, &\ (\ap{V}{C})
\end{align*}

We make use of evaluation contexts by replacing the placeholder $[]$ by
another term $M$, an operation we write down as $C[M]$. We can now define
the semantics:

$$
C[\ap{(\lam{x}{M})}{V}] \to_\beta C[\subst{M}{x}{V}]
$$

Here we see that we only substitute \emph{values} for the variables in a
$\lambda$-abstraction. Also note that we can only perform reductions inside
an evaluation context. Given our definition of $C$, this enforces a
left-to-right evaluation order.

\subsection{Simulating Call-by-Value}
\label{ssec:simulating-cbv}

We first present the translation from call-by-value $\lambda$-calculus
($\lambda_v$) to $\banana{\lambda}$ and then we elaborate on it.

\begin{align*}
  \sem{x} &= \ap{\eta}{x} \\
  \sem{\lam{x}{M}} &= \ap{\eta}{(\lam{x}{\sem{M}})} \\
  \sem{\ap{M}{N}} &= \sem{M} \hsbind (\lam{m}{\sem{N} \hsbind (\lam{n}{\ap{m}{n}})})
\end{align*}

An expression of $\lambda_v$ is modelled in $\banana{\lambda}$ as a
computation. The values form a special case since they are all interpreted
as pure computations, terms of the form $(\ap{\eta}{M})$ for some $M$. In
interpreting an application $(\ap{M}{N})$, we first evaluate $M$ and then
$N$, reflecting the behavior we have defined for $\lambda_v$ above.

Before we show that this translation is indeed a faithful one, we will
discuss the types of the interpretations to get a better understanding of
the structures involved.

$\lambda_v$ can be typed with the type system of the simply-typed
$\lambda$-calculus. A well-typed $\lambda_v$ term will then yield a
well-typed $\banana{\lambda}$ term since our translation satisifies a
property of the following form\footnote{The proof of the property is a
  routine proof by induction on the structure of the $\lambda_v$ term.}:

$$
\Gamma \vdash M : \alpha
\quad \Rightarrow \quad
\sem{\Gamma} \vdash \sem{M} : \FF_E(\sem{\alpha})
$$

We see that we model $\lambda_v$ expressions of type $\alpha$ using
computations that yield values of type $\sem{\alpha}$. The type-level
translation $\sem{.}$ is defined below. It is responsible for translating
the function type so that it takes values but produces computations (since
the body of a $\lambda$-abstraction can in general be any expression and
the denotation of an expression is a computation). In the formulas below,
$\nu$ stands for an atomic type, $\emptyset$ is the empty environment.

\begin{align*}
  \sem{\alpha \to \beta} &= \sem{\alpha} \to \FF_E(\sem{\beta}) \\
  \sem{\nu} &= \nu \\
  \sem{\Gamma, x: \alpha} &= \sem{\Gamma}, x: \sem{\alpha} \\
  \sem{\emptyset} &= \emptyset
\end{align*}

To show that our translation simulates the behavior of $\lambda_v$, we will
prove that any reduction chain $M \tto N$ in $\lambda_v$ gives rise to a
reduction chain $\sem{M} \tto \sem{N}$ in $\banana{\lambda}$. We start by
proving $\sem{\ap{(\lam{x}{M})}{V}} \tto \sem{\subst{M}{x}{V}}$.

\begin{align*}
  \sem{\ap{(\lam{x}{M})}{V}}
&= \sem{\lam{x}{M}} \hsbind (\lam{m}{\sem{V} \hsbind (\lam{n}{\ap{m}{n}})}) \\
&= (\ap{\eta}{(\lam{x}{\sem{M}})}) \hsbind (\lam{m}{(\ap{\eta}{v}) \hsbind (\lam{n}{\ap{m}{n}})}) \\
&\to_{\eta.\hsbind,\beta} (\ap{\eta}{v}) \hsbind (\lam{n}{\ap{(\lam{x}{\sem{M}})}{n}}) \\
&\to_{\eta.\hsbind,\beta} \ap{(\lam{x}{\sem{M}})}{v} \\
&\to_{\beta} \subst{\sem{M}}{x}{v} \\
&= \sem{\subst{M}{x}{V}}
\end{align*}

where $\sem{V} = \ap{\eta}{v}$. We first expand the definition of $\sem{.}$
for the application, the abstraction and the argument value (we note that
$\sem{V}$ is always equal to $\ap{\eta}{v}$ for some $v$). Since we have
two occurrences of a pure computation being piped into a bind
($(\ap{\eta}{x}) \hsbind k$), we can simplify using $\eta.\hsbind$ and
$\beta$. On line 5, we finally get to the point where we perform the
$\banana{\lambda}$ $\beta$-reduction that actually corresponds to the
$\lambda_v$ $\beta$-reduction that we are modelling.

Finally, on line 6, we push the substitution under the $\sem{.}$
operator. This is valid, since any free $x$ in $\sem{M}$ must have
originated as a translation of a free $x$ in $M$. In the first expression,
such an $x$ would get interpreted as $\ap{\eta}{x}$ and then $x$ would get
replaced with $v$ to get $\ap{\eta}{v}$. In the second expression, the $x$
would first get replaced by $V$ and then $V$ would be interpreted as
$\ap{\eta}{v}$. In both cases, we get the same result.

We have proven that our simulation preserves the reduction
$\ap{(\lam{x}{M})}{V} \to \subst{M}{x}{V}$. However, it might seem that our
work is not over since this is just a special case of the rule in
$\lambda_v$, which licenses the reduction $C[\ap{(\lam{x}{M})}{V}] \to
C[\subst{M}{x}{V}]$. Nevertheless, our calculus is pure and lets us perform
reductions in any syntactic context (see the notion of syntactic closure
from~\ref{sec:reductions}). This means that whenever we have $\sem{M} \to
\sem{N}$, we also always have $\sem{C}[\sem{M}] \to
\sem{C}[\sem{N}]$\footnote{We have not defined the $\sem{.}$ interpretation
  operator for contexts. Contexts are simply terms with a $[]$ inside so we
  just extend the $\sem{.}$ for terms with the clause $\sem{[]} = []$.},
which is the same as $\sem{C[M]} \to \sem{C[N]}$.

We have shown that for any reduction chain $M \tto N$ in $\lambda_V$, we
have a corresponding chain in $\sem{M} \tto \sem{N}$ in
$\banana{\lambda}$. Can we get anything stronger?

The inverse is not true since the reductions in $\banana{\lambda}$ are more
fine-grained. For a pair $M \to N$ and the corresponding chain $\sem{M} \to
\ldots \to M_i \to \ldots \to \sem{N}$, the intermediate terms $M_i$ might
not correspond to any particular term in $\lambda_v$.

For normalizing calculi, the preservation of reduction chains upon
translation entails that whenever $M = N$, then also $\sem{M} = \sem{N}$
(where $=$ stands for having the same normal form). However, in our case,
the inverse implication is not true. Our simulation can sometime simplify
things ``ahead of time'' compared to $\lambda_v$. As an example,
$\lam{x}{(\ap{(\lam{y}{y})}{x})}$ and $\lam{x}{x}$ do not have the same
normal form in $\lambda_v$ (they are both already values) but their
translations into $\banana{\lambda}$ both reduce to
$\ap{\eta}{(\lam{x}{\ap{\eta}{x}})}$. In a way, the reduction rules of
$\banana{\lambda}$ can reduce certain expressions earlier than $\lambda_v$
if it is safe to do so without affecting the semantics.

We have also established that the translation preserves types. This means
that we have a way of simulating simply-typed $\lambda_v$ in our typed
$\banana{\lambda}$ and a way of simulating untyped $\lambda_v$ in an
untyped version of $\banana{\lambda}$.

\subsection{Introducing Control Operators}

Now that we have introduced how a $\lambda$-calculus with a notion of
evaluation order is to be interpreted in $\banana{\lambda}$, we are ready
to introduce control operators. Our claim is that effects and handlers are
very close to delimited continuations. The pair of operators that resembles
the behavior of handlers the most are the operators $\shifto$ and
$\reseto$.

\begin{align*}
  V ::= &\ \lam{x}{M} \\
   | \, &\ x \\
  M, N ::= &\ V \\
   | \, &\ (\ap{M}{N}) \\
   | \, &\ (\ap{\shifto}{M}) \\
   | \, &\ (\ap{\reseto}{M})
\end{align*}

The terms of this new calculus, $\lambda_{\shifto}$, are just the terms of
$\lambda_v$ extended with the operators $\shifto$ and $\reseto$. Before we
give the reduction rules, we will also have to refine our notion of an
evaluation context.

\begin{align*}
  C ::= &\ [] \\
   | \, &\ (\ap{C}{M}) \\
   | \, &\ (\ap{V}{C}) \\
   | \, &\ (\ap{\shifto}{C}) \\
   | \, &\ (\ap{\reseto}{C}) \\
  F ::= &\ [] \\
   | \, &\ (\ap{F}{M}) \\
   | \, &\ (\ap{V}{F}) \\
   | \, &\ (\ap{\shifto}{F})
\end{align*}

For evaluation contexts ($C$), we add rules saying that when evaluating
applications of either $\shifto$ or $\reseto$, we can evaluate their
arguments. We also introduce a notion of an \emph{evaluation frame}
($F$). Similarly to how values are a subset of terms, evaluation frames are
a subset of evaluation contexts. A frame is a context which does not embed
$[]$ inside a $\reseto$.

\vspace{4mm}
\begin{tabular}{>{$}r<{$} >{$}c<{$} >{$}l<{$}}
  C[\ap{(\lam{x}{M})}{V}] & \to_\beta & C[\subst{M}{x}{V}] \\
  C[\ap{\reseto}{V}] & \to_{\reseto} & C[V] \\
  C[\ap{\reseto}{(F[\ap{\shifto}{V}])}] & \to_{\shifto} & C[\ap{V}{(\lam{x}{\ap{\reseto}{(F[x])}})}]
\end{tabular}
\vspace{4mm}

We keep the reduction rule $\beta$ and we add two new rules for $\reseto$
and $\shifto$. The rule for $\reseto$ makes $\reseto$ look redundant but
its importance shows up in the rule for $\shifto$. In the $\shifto$ rule,
we have an application of $\shifto$ buried inside the context
$C[\ap{\reseto}{F}]$. This kind of context corresponds to a context which
embeds $[]$ inside at least one $\reseto$. The $F$ then corresponds to the
frame which separates the $\shifto$ from the nearest enclosing
$\reseto$. The only role of $\reseto$ is thus to serve as a kind of marker
to delimit the context/continuation $F$ of $\shifto$. The argument to
$\shifto$ then receives this continuation $F$, composed with $\reseto$, as
its argument\footnote{You may already start to see similarities with the
  $\banana{\op{op}}$ rule of $\banana{\lambda}$ and why we chose $\shifto$
  and $\reseto$ in particular.}.

\subsection{Simulating $\shifto$ and $\reseto$}

To simulate $\lambda_{\shifto}$ in $\banana{\lambda}$, all we have to do is
extend the simulation of $\lambda_v$ with interpretations of the two new
syntactic forms:

\begin{align*}
  \sem{\ap{\shifto}{M}} &= \sem{M} \hsbind (\lam{m}{\ap{\op{shift0}!}{m}}) \\
  \sem{\ap{\reseto}{M}} &= \ap{\resetobanana}{\sem{M}}
\end{align*}

The above translation also gives us a general template for simulating
effectful calculi in $\banana{\lambda}$. Amongst the impure operators of a
calculus, we identify those that manipulate a context (raising an
exception, modifying a variable, reading a dynamically bound value,
accessing the continuation\ldots) and those that establish a context
(exception handlers, transactions, binders for dynamic variables,
continuation delimiters such as $\texttt{prompt}$ or
$\texttt{reset}$\ldots). Operators that manipulate the context are
translated into operations in $\banana{\lambda}$,
i.e.\ $\sem{\ap{\texttt{op}}{M}} = \sem{M} \hsbind
(\lam{m}{\ap{\op{op}!}{m}})$. Operators that establish a context are
translated into handlers, i.e.\ $\sem{\ap{\texttt{op}}{M}} =
\ap{\banana{\ldots}}{\sem{M}}$.

To show that this simulation is faithful, we will prove that $M \to N$ in
$\lambda_{\shifto}$ implies $\sem{M} \ttoffrom \sem{N}$ in
$\banana{\lambda}$. $\sem{M} \ttoffrom \sem{N}$ says that we can go from
$\sem{M}$ to $\sem{N}$ through a series of reductions and
expansions\footnote{We will need the expansions when ``unevaluating'' some
  of the monadic binds that have been introduced by our translations.}. In
this case, our simulation property will prove that equivalence given by the
calculus' equational theory is preserved\footnote{Such a one-sided property
  is also satisfied by trivial interpretations so the property itself does
  not do the interpretation justice. In other words, this interpretation is
  complete but not sound (consider already the $\lambda_v$ example towards
  the end of \ref{ssec:simulating-cbv}). \TODO{Is there some better
    characterization of this kind of simulation? Check out Plotkin's
    seminal paper on Call-by-value, call-by-name and the
    lambda-calculus.}}, i.e.\ $M = N$ in $\lambda_\shifto$ implies $\sem{M}
= \sem{N}$ in $\banana{\lambda}$ (where $X = Y$ is to be read as $X
\ttoffrom Y$).

We have three reduction rules to tackle: $\to_\beta$, $\to_\reseto$ and
$\to_\shifto$.  We have proven the case of $\to_\beta$
in~\ref{ssec:simulating-cbv} and that proof still holds in this extended
interpretation. We also reuse our observation
from~\ref{ssec:simulating-cbv} that in order to prove $\sem{C[M]} \tto
\sem{C[N]}$, it is enough to prove $\sem{M} \tto \sem{N}$. The case of
$\to_\reseto$ is a simple one so we will deal with that first:

\begin{align*}
  \sem{\ap{\reseto}{V}}
  &= \ap{\resetobanana}{\sem{V}} \\
  &= \ap{\resetobanana}{(\ap{\eta}{v})} \\
  &\to_{\banana{\eta}} \ap{\eta}{v} \\
  &= \sem{V}
\end{align*}

where $\sem{V} = \ap{\eta}{v}$. As we have said
in~\ref{ssec:operations-and-handlers}, an (open) handler without an
explicit clause for $\eta$ is presumed to handle $\eta$ with $\eta$. In
this case, the $\lambda_\shifto$ reduction ends up corresponding to exactly
one $\banana{\eta}$ reduction in $\banana{\lambda}$.

Now, let's deal with the last reduction rule, $\to_\shifto$.

\begin{align*}
  \sem{\ap{\reseto}{(F[\ap{\shifto}{V}])}}
  &= \ap{\resetobanana}{\sem{F[\ap{\shifto}{V}]}} \\
  &\ttoffrom \ap{\resetobanana}{(\app{\op{shift0}}{v}{(\lam{x}{\sem{F[x]}})})} \\
  &\to_{\banana{\op{op}}} \app{(\lam{c k}{\ap{c}{k}})}{v}{(\lam{x}{\ap{\resetobanana}{\sem{F[x]}}})} \\
  &\to_{\beta,\beta} \ap{v}{(\lam{x}{\ap{\resetobanana}{\sem{F[x]}}})} \\
  &= \ap{v}{(\lam{x}{\sem{\ap{\reseto}{(F[x])}}})} \\
  &\from_{\beta,\eta.\hsbind} (\ap{\eta}{(\lam{x}{\sem{\ap{\reseto}{(F[x])}}})}) \hsbind (\lam{n}{\ap{v}{n}}) \\
  &= \sem{\lam{x}{\ap{\reseto}{(F[x])}}} \hsbind (\lam{n}{\ap{v}{n}}) \\
  &\from_{\beta,\eta.\hsbind} (\ap{\eta}{v}) \hsbind (\lam{m}{\sem{\lam{x}{\ap{\reseto}{(F[x])}}} \hsbind (\lam{n}{\ap{m}{n}})}) \\
  &= \sem{V} \hsbind (\lam{m}{\sem{\lam{x}{\ap{\reseto}{(F[x])}}} \hsbind (\lam{n}{\ap{m}{n}})}) \\
  &= \sem{\ap{V}{(\lam{x}{\ap{\reseto}{(F[x])}})}}
\end{align*}

where $\sem{V} = \ap{\eta}{v}$. Lines 2 and 3 are the crucial lines. On
line 2, we use an upcoming lemma that will show that through a series of
reductions and expansions, we can go from $\sem{F[\ap{\shifto}{V}]}$ to
$(\app{\op{shift0}}{v}{(\lam{x}{\sem{F[x]}})})$ where $\sem{V} =
\ap{\eta}{v}$. Since we have moved $\op{shift0}$ to the head of the
handler's argument, we can apply the $\banana{\op{op}}$ rule on line
3. Since the handler clause is basically the identity function, it
disappears on line 4 after two $\beta$-reductions.

From then on, we perform a series of expansions while trying to push the
interpretation operator $\sem{.}$ outwards from just $F[x]$ to the entire
expression. The expansions used above are a reversal of a common idiom we
have used before. We used to go from $(\ap{\eta}{M}) \hsbind (\lam{m}{N})$
to $\subst{N}{m}{M}$ using $\eta.\hsbind$ and then a
$\beta$-reduction. Here, we go the other way from $\subst{N}{m}{M}$ to
$(\ap{\eta}{M}) \hsbind (\lam{m}{N})$ using a $\beta$-expansion and the
derived expansion $\eta.\hsbind$ (lines 6 and 8).

\subsubsection*{Contexts $=$ Continuations: Proving the Lemma}

All that is left to show is a proof of the lemma we mentioned above:

$$
\sem{F[\ap{\shifto}{V}]} \ttoffrom (\app{\op{shift0}}{v}{(\lam{x}{\sem{F[x]}})})
$$

where $\sem{V} = \ap{\eta}{v}$. This lemma not only allows us to prove the
simulation property that is the focus of this section, but it also gives us
more insight into $\banana{\lambda}$. If we read it from right to left and
slightly generalizing, it tells us how to think of terms of the form
$\app{\op{op}}{x}{k}$. They represent computations where the next point of
evaluation is a contextually dependent operation $\op{op}$. $x$ is the
operation's argument and $k$ captures the context in which $\op{op}$ is
being used inside the computation.

Our proof will proceed by induction on the structure of $F$. We will start
with the base case, $F = []$.

\begin{align*}
  \sem{F[\ap{\shifto}{V}]}
  &= \sem{\ap{\shifto}{V}} \\
  &= \sem{V} \hsbind (\lam{m}{\ap{\op{shift0}!}{m}}) \\
  &\to_{\eta.\hsbind,\beta} \ap{\op{shift0}!}{v} \\
  &= \ap{(\lam{x}{\app{\op{shift0}}{x}{\eta}})}{v} \\
  &\to_\beta \app{\op{shift0}}{v}{\eta} \\
  &\from_{\eta.\to} \app{\op{shift0}}{v}{(\lam{x}{\ap{\eta}{x}})} \\
  &= \app{\op{shift0}}{v}{(\lam{x}{\sem{x}})} \\
  &= \app{\op{shift0}}{v}{(\lam{x}{\sem{F[x]}})} \\
\end{align*}

The individual steps are pretty self-explanatory. The steps that might be
unfamiliar occur on lines 4 and 6. On line 4, we expand the definition of
the exclamation mark defined in~\ref{ssec:operations-and-handlers} and on
line 6, we perform an $\eta$-expansion.

Next case, $F = (\ap{F'}{M})$:

\begin{align*}
  \sem{F[\ap{\shifto}{V}]}
  &= \sem{\ap{(F'[\ap{\shifto}{V}])}{M}} \\
  &= \sem{F'[\ap{\shifto}{V}]} \hsbind (\lam{m}{\sem{M} \hsbind (\lam{n}{\ap{m}{n}})}) \\
  &\ttoffrom (\app{\op{shift0}}{v}{(\lam{x}{\sem{F'[x]}})}) \hsbind (\lam{m}{\sem{M} \hsbind (\lam{n}{\ap{m}{n}})}) \\
  &\to_{\op{op}.\hsbind} \app{\op{shift0}}{v}{(\lam{x}{\sem{F'[x]} \hsbind (\lam{m}{\sem{M} \hsbind (\lam{n}{\ap{m}{n}})})})} \\
  &= \app{\op{shift0}}{v}{(\lam{x}{\sem{\ap{(F'[x])}{M}}})} \\
  &= \app{\op{shift0}}{v}{(\lam{x}{\sem{F[x]}})} \\
\end{align*}

Again, the steps are quite mechanical. Line 3 uses the induction hypothesis
and on line 4, we see the derived $\op{op}.\hsbind$ rule introduced in
Chapter~\ref{chap:properties}\footnote{\TODO{Make this reference more
    precise once the Properties chapter has been written.}} pushing the
$\hsbind$ inside the continuation.

Case $F = (\ap{V'}{F'})$:

\begin{align*}
  \sem{F[\ap{\shifto}{V}]}
  &= \sem{\ap{V'}{(F'[\ap{\shifto}{V}])}} \\
  &= \sem{V'} \hsbind (\lam{m}{\sem{F'[\ap{\shifto}{V}]} \hsbind (\lam{n}{\ap{m}{n}})}) \\
  &\to_{\eta.\hsbind,\beta} \sem{F'[\ap{\shifto}{V}]} \hsbind (\lam{n}{\ap{v'}{n}}) \\
  &\ttoffrom (\app{\op{shift0}}{v}{(\lam{x}{\sem{F'[x]}})}) \hsbind (\lam{n}{\ap{v'}{n}}) \\
  &\to_{\op{op}.\hsbind} \app{\op{shift0}}{v}{(\lam{x}{\sem{F'[x]} \hsbind (\lam{n}{\ap{v'}{n}})})}\\
  &\from_{\beta,\eta.\hsbind} \app{\op{shift0}}{v}{(\lam{x}{\sem{V'} \hsbind (\lam{m}{\sem{F'[x]} \hsbind (\lam{n}{\ap{m}{n}})})})}\\
  &= \app{\op{shift0}}{v}{(\lam{x}{\sem{\ap{V'}{(F'[x])}}})} \\
  &= \app{\op{shift0}}{v}{(\lam{x}{\sem{F[x]}})}
\end{align*}

where $\sem{V'} = \ap{\eta}{v'}$. This proof is very similar to the one for
the case before. We have just two extra steps, on lines 3 and 6, where we
first push the $\sem{V'}$ ($= \ap{\eta}{v'}$) in through the $\hsbind$
using $\eta.\hsbind$ and $\beta$ and then we pull it out in a different
context by reversing the process.

Finally, the last case, where $F = (\ap{\shifto}{F'})$:

\begin{align*}
  \sem{F[\ap{\shifto}{V}]}
  &= \sem{\ap{\shifto}{(F'[\ap{\shifto}{V}])}} \\
  &= \sem{F'[\ap{\shifto}{V}]} \hsbind (\lam{m}{\ap{\op{shift0}!}{m}}) \\
  &\ttoffrom (\app{\op{shift0}}{v}{(\lam{x}{\sem{F'[x]}})}) \hsbind (\lam{m}{\ap{\op{shift0}!}{m}}) \\
  &\to_{\op{op}.\hsbind} \app{\op{shift0}}{v}{(\lam{x}{\sem{F'[x]} \hsbind (\lam{m}{\ap{\op{shift0}!}{m}})})} \\
  &= \app{\op{shift0}}{v}{(\lam{x}{\sem{\ap{\shifto}{(F'[x])}}})} \\
  &= \app{\op{shift0}}{v}{(\lam{x}{\sem{F[x]}})}
\end{align*}

And this case is just as simple as the $F = (\ap{F'}{M})$ one. This
concludes our proof of this lemma. Note that we did not include a case for
$F = (\ap{\reseto}{F'})$. Such a context is not a frame, since it embeds
$[]$ inside a $\reseto$. We can also check that our property would no
longer hold in this case, since the $\shifto$ coming from $F'$ would get
handled by the $\reseto$.

By proving this lemma, we have also finished our proof of the fact that
whenever we have $M \ttoffrom N$ in $\lambda_\shifto$, we also have
$\sem{M} \ttoffrom \sem{N}$ in $\banana{\lambda}$.

\subsubsection*{Considering Types}




\subsection{Other Control Operators}


\TODO{
  \begin{itemize}
  \item Diagram
  \item Lemmas, properties, proofs done right (syntactically)
  \item Move Quantifier Ambiguity to Part 2
  \end{itemize}
}

\section{Monads}
\label{sec:monads}


\section{Partial Translation to the Calculus of Constructions}
\label{sec:translation-to-coc}


\section{Reason for $\mathcal{C}$}
\label{sec:reason-for-c}

