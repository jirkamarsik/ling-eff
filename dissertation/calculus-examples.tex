\chapter{Examples}
\label{chap:examples}

We have given a formal definition of our calculus but we have not shown how
to use it to model side effects and what are the benefits. In this chapter,
we will present an example problem and as we explore it deeper, we will see
emerge some of the properties behind our calculus.

\minitoc

\section{Introducing Our Running Example}

We will try to build a calculator. Given some arithmetic expression, we
would like to reduce it to a simple number. 

$$
\app{\sume}{(\ap{\lite}{1})}{(\app{\prode}{(\ap{\lite}{2})}{(\ap{\lite}{4})})}
\quad \to \quad 9
$$

In the above, expressions are formed using the constructors $\sume$,
$\prode$ and $\lite$ which stand for sums, products and literals,
respectively. Expressions in our calculator have the type $exp$, the types
of the constructors (formation rules) are given below\footnote{The type
  $\NN$ is the type of natural numbers. Since our examples will deal with
  numbers, we will treat naturals as terms of our calculus that have the
  type $\NN$. Their syntax and equational theory are both standard so we
  will not repeat it here.}:

\begin{align*}
  \sume &: exp \to exp \to exp \\
  \prode &: exp \to exp \to exp \\
  \lite &: \NN \to exp
\end{align*}

The exercise is similar to our enterprise of formal semantics. We have a
language (fragment) with a formal grammar and we try to find a systematic
way of identifying sentences of this language with their senses or
references. Our methodology will also be the same of formal semantics: a
composition of meanings that mirrors the syntactic structure.

Before we go on, a quick technical remark about the $exp$ type and its
encoding. A common approach is to model $exp$ as an inductive
algebraic data type\footnote{This would be something called an
  \emph{initial encoding} of a language.} and then have our calculator be a
function from $exp$ to $\NN$ or some other domain of
interpretation. Our calculus does not directly provide inductive types but
the $\FF$ type constructor is built on top of them and since it is
parameterizable by an effect signature, it can be used to implement a
variety of inductive types. In this particular scenario, we could use
$\FF_{\{\typedop{sum}{1}{2},\typedop{prod}{1}{2}\}}(\NN)$ as
$exp$. The notion of a closed handler ($\bbanana{}$, see
$\ref{ssec:operations-and-handlers}$) would then give use the induction
principle over these arithmetic expression, i.e.\ a way to compositionally
compute some value from the arithmetic expression.

We will not be using the encoding described above. The point of this
chapter is to make the reader familiar with the notion of using the
$\FF$-types to model computations. We will therefore restrict their use to
computations.

We can avoid introducing any kind of type for encoding arithmetic
expressions. The reduction step from an arithmetic expression to a simple
number can be realized by adding equations that ``define'' the functions
$\sume$, $\prode$, $\lite$ and the type $exp$. We thus take $exp$ to be the
name of the type of our desired interpretations and we take $\sume$,
$\prode$ and $\lite$ to be some functions that compute directly with these
interpretations\footnote{This is known as a \emph{final encoding} of a
  language.}. The task of the semanticist is to fix some specific
interpretation for the abstract type $exp$ and figuring out the definitions
of the abstract functions $\sume$, $\prode$ and $\lite$\footnote{As we will
  see in Chapter~\ref{chap:intro-fs} when we introduce them, this is
  exactly the idea behind abstract categorial grammars
  \cite{de2001towards}.}. Within our tiny fragment, this is quite easy to
do:

\begin{align*}
  exp &= \NN \\
  \sume &= \lam{x y}{x + y} \\
  \prode &= \lam{x y}{x \times y} \\
  \lite &= \lam{x}{x}
\end{align*}

Applying these equations to the term
$\app{\sume}{(\ap{\lite}{1})}{(\app{\prode}{(\ap{\lite}{2})}{(\ap{\lite}{4})})}$
and then reducing according to the reduction rules of our calculus
(alongside with rules/equations to perform arithmetic operations), we get
the result, $9$.


\section{Adding Errors}

We will now try expanding our fragment by adding integer division.

\begin{align*}
  \dive &: exp \to exp \to exp
\end{align*}

We could be tempted to treat this the same way and simply give this
semantics to $\dive$:

\begin{align*}
  \dive &= \lam{x y}{x / y}
\end{align*}

However, this assumes that we have a division operator that defines the
division $x/y$ for all naturals $x$ and $y$. This is not the case when $y =
0$. Defining $x/0$ to be some specific natural $d$ is not satisfactory
since then we cannot distinguish between the cases when $x/y$ is $d$ and
when $x/y$ is not defined. Fortunately, this is what sum types are very
useful for. We will have the type of $x/y$ be $\NN + 1$, $x/0$ will reduce
to $(\ap{\inr}{\star})$ and $x/y$ for $y \neq 0$ will reduce to
$(\ap{\inl}{q})$ where $q$ is the greatest natural such that $q y \le
x$.

\subsection{Raising the Type of $exp$}
\label{ssec:raising-the-type-of-exp}

Now the cases when $x/y$ is and is not defined are clearly
delimited\footnote{Realizing that $x/y$ is not defined for $y = 0$ and then
  changing the type and behavior of the underlying division operator is
  very much like semanticists taking into account that expressions such as
  \emph{the King of France} need not have a reference and then changing the
  underlying model by switching from functions to
  relations.}. Nevertheless, we just made ourselves a new problem. The
result of a division is something of type $\NN + 1$ and so in our
interpretation, $exp$ will correspond to $\NN + 1$. This is in line with
our intuition that arithmetic expressions which contain division can
sometimes be undefined. However, the implications of having $exp = \NN + 1$
begin to sting when we consider the cases of $\sume$ and
$\prode$\footnote{As a convention, we use $\_$ as a variable name for
  variables whose values are never used.}.

\begin{align*}
  exp &= \NN + 1 \\
  \dive &= \lam{X Y}{\case{X}{x}{\case{Y}{y}{x / y}{\_}{\ap{\inr}{\star}}}{\_}{\ap{\inr}{\star}}} \\ 
  \sume &= \lam{X Y}{\case{X}{x}{\case{Y}{y}{\ap{\inl}{(x + y)}}{\_}{\ap{\inr}{\star}}}{\_}{\ap{\inr}{\star}}} \\ 
  \prode &= \lam{X Y}{\case{X}{x}{\case{Y}{y}{\ap{\inl}{(x \times y)}}{\_}{\ap{\inr}{\star}}}{\_}{\ap{\inr}{\star}}} \\ 
  \lite &= \lam{x}{\ap{\inl}{x}} \\
        &\to_{\eta} \inl
\end{align*}


the calculation only if both of the operands successfully yield a
natural\footnote{If you are unfamiliar with the notation used in the terms
  above, consult Section~\ref{sec:sums-and-products} where we introduced
  sums and products into our calculus.}. In $\sume$, $\prode$ and $\lite$,
we also have to wrap the result in $\inl$ in order to go from $\NN$ to $\NN
+ 1$\footnote{We $\eta$ reduce the definition of $\lite$ for
  convenience.}. In $\dive$, we do not wrap the result using $\inl$ since
division already produces a value of type $\NN + 1$. All this seems a heavy
price to pay just to include division.

\subsection{Refactoring with Monads}
\label{ssec:refactoring-with-monads}

We can make this solution look a little bit better. We can introduce a
function on types and a pair of combinators that will allow us to be a
little less repetitive.

\begin{align*}
  T_\bot(\alpha) &= \alpha + 1 \\
  \eta_\bot &: \alpha \to T_\bot(\alpha) \\
  \eta_\bot &= \inl \\
  (\hsbind_\bot) &: T_\bot(\alpha) \to (\alpha \to T_\bot(\beta)) \to T_\bot(\beta) \\
  (\hsbind_\bot) &= \lam{X k}{\case{X}{x}{\ap{k}{x}}{\_}{\ap{\inr}{\star}}}
\end{align*}

With these in our hands, we can now straighten out our interpretation.

\begin{align*}
  exp &= T_\bot(\NN) \\
  \dive &= \lam{X Y}{X \hsbind_\bot (\lam{x}{Y \hsbind_\bot (\lam{y}{x / y})})} \\
  \sume &= \lam{X Y}{X \hsbind_\bot (\lam{x}{Y \hsbind_\bot (\lam{y}{\ap{\eta_\bot}{(x + y)}})})} \\
  \prode &= \lam{X Y}{X \hsbind_\bot (\lam{x}{Y \hsbind_\bot (\lam{y}{\ap{\eta_\bot}{(x \times y)}})})} \\
  \lite &= \lam{x}{\ap{\eta_\bot}{x}}
\end{align*}

This pattern that we uncovered in the type $\NN + 1$ and the terms defining
$\dive$, $\sume$, $\prode$ and $\lite$ is not incidental. The triple
$\left<T_\bot, \eta_\bot, \hsbind_\bot \right>$ forms a monad\footnote{This
  way of presenting a monad (a type constructor, $\eta$ and $\hsbind$) is
  particular to functional programming. Note that this presentation differs
  from the category-theoretical one which replaces $\hsbind$ with a natural
  transformation $\mu$ \cite{mac1978categories}.}. This formulation in
terms of monadic operations will allow us to transition more easily into
our proposed solution since, as we will show
in~\ref{sec:algebraic-properties}, $\left< \FF_E, \eta, \hsbind \right>$
also forms a monad.

\subsection{Interpreting Expressions as Computations}

We will now explore an alternative solution which takes advantage of the
$\FF$-types in our calculus. We will interpret expressions as computations
of naturals, $exp = \FF_E(\NN)$. Computations will have the ability to fail
by using the operation $\typedop{fail}{1}{0}$. The input type $1$ means
that there is only one way to fail (i.e.\ $\op{fail}$ does not distinguish
failure states) and the output type $0$ means that there is no continuation
(formally there is a dummy continuation that accepts the impossible type
$0$). There is a natural way to generalize this approach by using
$\typedop{error}{\chi}{0}$ instead. Computations can now terminate by
throwing exceptions of type $\chi$, allowing us to distinguish failure
states. To identify the division-by-zero failure state, we will introduce
$\dbzelong : \chi$. A computation that uses the $\op{error}$ operation to
throw a division-by-zero exception would look like the following:

\begin{align*}
  \dbze &: \FF_{\{\typedop{error}{\chi}{0}\}}(\alpha) \\
  \dbze &= \app{\op{error}}{\dbzelong}{(\lam{o}{\absurd{o}})}
\end{align*}

The continuation uses the [empty] rule to turn the $o$ of type $0$ into
something of type $\FF_E(\alpha)$.

\begin{align*}
  exp &= \FF_{\{\typedop{error}{\chi}{0}\}}(\NN) \\
  \dive &= \lam{X Y}{X \hsbind (\lam{x}{Y \hsbind (\lam{y}{\case{(x / y)}{z}{\ap{\eta}{z}}{\_}{\dbze}})})} \\
  \sume &= \lam{X Y}{X \hsbind (\lam{x}{Y \hsbind (\lam{y}{\ap{\eta}{(x + y)}})})} \\
  \prode &= \lam{X Y}{X \hsbind (\lam{x}{Y \hsbind (\lam{y}{\ap{\eta}{(x \times y)}})})} \\
  \lite &= \lam{x}{\etaE{x}}
\end{align*}

Let us compare this with the last set of definitions:

\begin{itemize}
\item We replaced the $\left< T_\bot, \eta_\bot, \hsbind_\bot \right>$ monad
  with the $\left< \FF_{\{\typedop{error}{\chi}{0}\}}, \eta, \hsbind \right>$ monad.
\item Since the type of $x / y$ is not the same as the type of our
  interpretations, we need to translate from the type $\NN + 1$ to the type
  $\FF_{\{\typedop{error}{\chi}{0}\}}(\NN)$ by case analysis.
\end{itemize}

The advantage to using the $\FF_E$ monad instead of the $T_\bot$ monad is
that the $\FF_E$ can be extended to handle other kinds of effects besides
exceptions whereas the $T_\bot$ monad need would need to be replaced by a
different one. We will see an example of this later
in~\ref{sec:enriching-the-context-with-variables}. For now, we still have
something more to explore regarding errors.

\subsection{Handling Errors}

We can now use division in our small calculator language. However, division
can make the evaluation of an expression fail and yield no useful
result. We could thus ask for a way to speculatively evaluate a
subexpression and if it fails, recover by providing some default value and
carry on evaluating the rest of the expression.

We add a new construction into our language:

$$
\trye : exp \to exp \to exp
$$

Its intended meaning is to return the value of its second
argument. However, if the second argument fails to evaluate, the value of
the first argument should be used instead.

$$
\app{\trye}{(\ap{\lite}{42})}{(\app{\dive}{(\ap{\lite}{1})}{(\ap{\lite}{0})})}
\quad \to \quad \ap{\eta}{42}
$$

Our task now is to give a formal semantics to $\trye$.

$$
\trye = \lam{X Y}{\ap{\banana{\onto{\op{error}}{(\lam{e k}{X})}}}{Y}}
$$

Instead of passing $Y$ through the $\hsbind$ operator, which basically says
to do whatever $Y$ would do (i.e.\ fail whenever $Y$ fails), we apply a
handler to $Y$\footnote{Feeding it into $\hsbind$ or applying a handler to
  it are the two most common ways we will be using computations.}. This
handler is very simple, it replaces any failed computation with the
computation $X$ (the backup computation that should yield the default
value). It behaves like the exception handlers of common programming
languages, whenever the computation $Y$ would throw an $\op{error}$, we
handle it by running computation $X$ instead.

\subsection{Examples of Errors}

We now have enough interesting material to play around with a little bit
and see if (and why) it really works as it should. We will take an
expression large enough to contain all the features which we added and we
will evaluate it piece by piece.

$$
\app{\sume}{\litec{5}}{\tryec{\litec{0}}{\prodec{\litec{3}}{\divec{\litec{2}}{\litec{0}}}}}
$$

Let's start with $\app{\dive}{\litec{2}}{\litec{0}}$.

\begin{align*}
  \app{\dive}{\litec{2}}{\litec{0}}
  &= \app{\dive}{(\ap{\eta}{2})}
                {(\ap{\eta}{0})} \\
  &= \app{(\lam{X Y}{X \hsbind (\lam{x}{Y \hsbind (\lam{y}
          {\case{(x / y)}{z}{\ap{\eta}{z}}{\_}{\dbze}})})})}
                {(\ap{\eta}{2})} {(\ap{\eta}{0})} \\
  &\to_{\beta,\beta} (\ap{\eta}{2}) \hsbind (\lam{x}{(\ap{\eta}{0}) \hsbind (\lam{y}
          {\case{(x / y)}{z}{\ap{\eta}{z}}{\_}{\dbze}})}) \\
  &= \ap{(\lam{x}{(\ap{\eta}{0}) \hsbind (\lam{y}
          {\case{(x / y)}{z}{\ap{\eta}{z}}{\_}{\dbze}})})^*}{(\ap{\eta}{2})} \\
  &= \ap{\banana{\onto{\eta}{(\lam{x}{(\ap{\eta}{0}) \hsbind (\lam{y}
          {\case{(x / y)}{z}{\ap{\eta}{z}}{\_}{\dbze}})})}}}{(\ap{\eta}{2})} \\
  &\to_{\banana{\eta}} \ap{(\lam{x}{(\ap{\eta}{0}) \hsbind (\lam{y}
          {\case{(x / y)}{z}{\ap{\eta}{z}}{\_}{\dbze}})})}{2} \\
  &\to_{\beta} (\ap{\eta}{0}) \hsbind (\lam{y}
          {\case{(2 / y)}{z}{\ap{\eta}{z}}{\_}{\dbze}}) \\
  &\to_{\ldots} \ap{(\lam{y}{\case{(2 / y)}{z}{\ap{\eta}{z}}{\_}{\dbze}})}{0} \\
  &\to_{\beta} \case{(2 / 0)}{z}{\ap{\eta}{z}}{\_}{\dbze} \\
  &\to_{/} \case{(\ap{\inr}{\star})}{z}{\ap{\eta}{z}}{\_}{\dbze} \\
  &\to_{\beta.+_2} \dbze \\
\end{align*}

The reductions state which rule was used to perform the step. The equality
on line 2 is due to our interpretation of $\dive$. Those on lines 4 and 5
come from the definitions of $\hsbind$ and $^*$
in~\ref{ssec:composing-functions}. The step on line 8 is just a repeat of
the steps on lines 4 to 6. This sequence of steps will be quite common and
we will from now on refer to it as the reduction rule
$\eta.\hsbind$\footnote{This rule will be formally introduced in
  Chapter~\ref{chap:properties}. It says that whenever we have
  $(\ap{\eta}{x}) \hsbind k$, we can reduce this to $\ap{k}{x}$.}

The result of this first elaboration was not surprising: dividing 2 by 0
throws a $\dbzelong$ exception. We will now try to see what happens when
this faulty expression appears within another expression.

\begin{align*}
  \app{\prode}{\litec{3}}{\dbze}
  &= \app{\prode}{(\ap{\eta}{3})}{\dbze} \\
  &= \app{(\lam{X Y}{X \hsbind (\lam{x}{Y \hsbind (\lam{y}{\ap{\eta}
          {(x \times y)}})})})}
      {(\ap{\eta}{3})}{\dbze} \\
  &\to_{\beta,\beta} (\ap{\eta}{3}) \hsbind (\lam{x}{\dbze \hsbind (\lam{y}{\ap{\eta}
          {(x \times y)}})}) \\
  &\to_{\eta.\hsbind} \ap{(\lam{x}{\dbze \hsbind (\lam{y}{\ap{\eta}
          {(x \times y)}})})}{3} \\
  &\to_{\beta} \dbze \hsbind (\lam{y}{\ap{\eta} {(3 \times y)}}) \\
  &= (\app{\op{error}}{\dbzelong}{(\lam{o}{\absurd{o}})}) \hsbind (\lam{y}{\ap{\eta} {(3 \times y)}}) \\
  &= \ap{(\lam{y}{\ap{\eta} {(3 \times y)}})^*}{(\app{\op{error}}{\dbzelong}{(\lam{o}{\absurd{o}})})} \\
  &= \ap{\banana{\onto{\eta}{(\lam{y}{\ap{\eta} {(3 \times y)}})}}}{(\app{\op{error}}{\dbzelong}{(\lam{o}{\absurd{o}})})} \\
  &\to_{\banana{\op{op}'}} \app{\op{error}}{\dbzelong}{(\lam{z}{\ap{\banana{\onto{\eta}{(\lam{y}{\ap{\eta} {(3 \times y)}})}}}{(\ap{(\lam{o}{\absurd{o}})}{z})}})} \\
  &\to_{\beta} \app{\op{error}}{\dbzelong}{(\lam{z}{\ap{\banana{\onto{\eta}{(\lam{y}{\ap{\eta} {(3 \times y)}})}}}{(\absurd{z})}})} \\
  &\simeq \dbze
\end{align*}

Line 2 is due to our interpretation of $\prode$. Line 4 is due to the
$\eta.\hsbind$ rule that we have demonstrated in the last example. Lines 6,
7 and 8 expand the definitions of $\dbze$, $\hsbind$ and $^*$. On line 11,
we equate the term with $\dbze$. The terms differ in the continuation but
since the continuation can never be called\footnote{Technical aside: To be
  precise, we should define $\dbze$ to be a class of terms of the shape
  $\app{\op{error}}{\dbzelong}{M}$ for any term $M$. The only operations we
  will ever perform on $\dbze$ in these examples will be congruent with the
  equivalence relation of our calculus refined by adding equalities between
  terms of the shape $\app{\op{error}}{\dbzelong}{M}$. Therefore, so long
  as we end up removing the $\dbze$ terms from our result, we can treat
  them as equivalent during our calculations.}, we can consider them equal.

We have seen the $\dbzelong$ exception propagate. Now let's see what
happens when it hits a $\trye$ construction.

\begin{align*}
  \app{\trye}{\litec{0}}{\dbze}
  &= \app{\trye}{(\ap{\eta}{0})}{\dbze} \\
  &= \app{(\lam{X Y}{\ap{\banana{\onto{\op{error}}{(\lam{e k}{X})}}}{Y}})}{(\ap{\eta}{0})}{\dbze} \\
  &\to_{\beta,\beta} \ap{\banana{\onto{\op{error}}{(\lam{e k}{\ap{\eta}{0}})}}}{\dbze} \\
  &= \ap{\banana{\onto{\op{error}}{(\lam{e k}{\ap{\eta}{0}})}}}{(\app{\op{error}}{\dbzelong}{(\lam{o}{\absurd{o}})})} \\
  &\to_{\banana{\op{op}}} \app{(\lam{e k}{\ap{\eta}{0}})}{\dbzelong}{(\lam{z}{\ap{\banana{\onto{\op{error}}{(\lam{e k}{\ap{\eta}{0}})}}}{(\ap{(\lam{o}{\absurd{o}})}{z})}})} \\
  &\to_{\beta,\beta} \ap{\eta}{0}
\end{align*}

Line 2 is our interpretation of $\trye$. Line 4 is the definition of
$\dbze$.

We see that since the embedded expression failed to evaluate (its sense was
$\dbze$), we have evaluated the literal 0 instead. We now have an actual
number again and we can try feeding it into another operation.

\begin{align*}
  \app{\sume}{\litec{5}}{(\ap{\eta}{0})}
  &= \app{\sume}{(\ap{\eta}{5})}{(\ap{\eta}{0})} \\
  &= \app{(\lam{X Y}{X \hsbind (\lam{x}{Y \hsbind (\lam{y}{\ap{\eta}{(x + y)}})})})}{(\ap{\eta}{5})}{(\ap{\eta}{0})} \\
  &\to_{\beta,\beta} (\ap{\eta}{5}) \hsbind (\lam{x}{(\ap{\eta}{0}) \hsbind (\lam{y}{\ap{\eta}{(x + y)}})}) \\
  &\to_{\eta.\hsbind} \ap{(\lam{x}{(\ap{\eta}{0}) \hsbind (\lam{y}{\ap{\eta}{(x + y)}})})}{5} \\
  &\to_{\beta} (\ap{\eta}{0}) \hsbind (\lam{y}{\ap{\eta}{(5 + y)}}) \\
  &\to_{\eta.\hsbind} \ap{(\lam{y}{\ap{\eta}{(5 + y)}})}{0} \\
  &\to_{\beta} \ap{\eta}{(5 + 0)} \\
  &\to_+ \ap{\eta}{5}
\end{align*}

Line 2 is our interpretation of $\sume$. The others are reductions of the
kind we have seen before.

Finally, we see the simplest case when an operator just reaches into the
results of both operands using $\to_{\eta.\hsbind}$ and carries out its
operation. This also concludes our evaluation of the expression we have
presented at the beginning of this subsection. We can therefore now
conclude that:

$$
\app{\sume}{\litec{5}}{\tryec{\litec{0}}{\prodec{\litec{3}}{\divec{\litec{2}}{\litec{0}}}}}
\quad \to \quad \ap{\eta}{5}
$$


\section{Enriching the Context with Variables}
\label{sec:enriching-the-context-with-variables}

Sometimes, when writing down arithmetic expressions, it becomes useful to
introduce variables for intermediate expressions and then build up the
final result using those. Let's try and add such a facility to our
calculator.

\begin{align*}
  \lete &: var \to exp \to exp \to exp \\
  \vare &: var \to exp \\
  \bar{x}, \bar{y}, \bar{z},\ldots &: var
\end{align*}

$\lete$ binds a variable to the result of the first expression; this
variable stays available during the evaluation of the second
expression. $\vare$ then lets us use the variable in place of an
expression. We also introduce terms for the different variables that can be
used in our calculator. They look the same as the variables in our calculus
but with a bar on top.

Here is a term that uses the new constructions:

\begin{align*}
& \letecn{x}{\sumec{\litec{2}}{\litec{3}}}
 {\\ & \hspace{9mm} \letec{y}{\prodec{\varec{x}}{\varec{x}}}
 {\\ & \hspace{19mm} \prodec{\varec{y}}{\litec{2}}}} \quad \to \quad \ap{\eta}{50}
\end{align*}

In order to give a semantics to $\lete$ and $\vare$, we will augment the
set of operations in the computations we use to interpret the
expressions. We want computations to be able to ask for the values of
variables and so we will introduce $\typedop{get}{var}{\NN}$. In our new
interpretation, we will have $exp =
\FF_{\{\typedop{error}{\chi}{0},\typedop{get}{var}{\NN}\}}(\NN)$. Though it
might seem that we are changing our domain of interpretation and will thus
need to redo our existing interpretations, it is actually not the case. The
type that we associated with $exp$ before could have been more precisely
written as $\FF_{\{\typedop{error}{\chi}{0}\} \uplus E}(\NN)$, where $E$ is
a free variable ranging over effect signatures. What we are doing is simply
specifying $E = \{\typedop{get}{var}{\NN}\} \uplus E'$, where $E'$ is again
free.

\begin{align*}
  \lete &= \lam{v X Y}{X \hsbind
    (\lam{x}{\ap{\banana{\onto{\op{get}}{(\lam{u k}{\ifte{u = v}{(\ap{k}{x})}{(\app{\op{get}}{u}{k})}})}}}{Y}})} \\
  \vare &= \op{get}!
\end{align*}

In the implementation above, we also rely on the presence of an equality
predicate on the $var$ type:

 $$
  (=) : var \to var \to 2
 $$

And that is all we need. We can put these two combinators side-by-side with
the existing ones like $\dive$ and $\trye$ and we will have a semantics for
a calculator with exceptions and variables.

\subsection{Example with Variables}

To get some practice with the calculus, we will evaluate a simple
expression with multiple variables, step by step.

$$
\letecn{x}{\litec{1}}{\letec{y}{\litec{2}}{\sumec{\varec{x}}{\varec{y}}}}
$$

Let us start off with the innermost expression, the sum of $\bar{x}$ and
$\bar{y}$.

\begin{align*}
  \sumecn{\varec{x}}{\varec{y}}
& = \sumecn{(\ap{\op{get}!}{\bar{x}})}
           {(\ap{\op{get}!}{\bar{y}})} \\
& = \app{(\lam{X Y}{X \hsbind (\lam{x}{Y \hsbind (\lam{y}{\ap{\eta}{(x + y)}})})})}
        {(\ap{\op{get}!}{\bar{x}})}
        {(\ap{\op{get}!}{\bar{y}})} \\
& \to_{\beta,\beta} (\ap{\op{get}!}{\bar{x}}) \hsbind (\lam{x}{
                       (\ap{\op{get}!}{\bar{y}}) \hsbind (\lam{y}{
                        \ap{\eta}{(x + y)}})}) \\
& = (\ap{(\lam{x}{\app{\op{get}}{x}{\eta}})}{\bar{x}}) \hsbind (\lam{x}{
    (\ap{\op{get}!}{\bar{y}}) \hsbind (\lam{y}{
     \ap{\eta}{(x + y)}})}) \\
& \to_{\beta} (\app{\op{get}}{\bar{x}}{\eta}) \hsbind (\lam{x}{
    (\ap{\op{get}!}{\bar{y}}) \hsbind (\lam{y}{
     \ap{\eta}{(x + y)}})}) \\
& = \ap{\banana{\onto{\eta}{(\lam{x}{(\ap{\op{get}!}{\bar{y}}) \hsbind (\lam{y}{
    \ap{\eta}{(x + y)}})})}}}{(\app{\op{get}}{\bar{x}}{\eta})} \\
& \to_{\banana{\op{op}'}} \app{\op{get}}{\bar{x}}
        {(\lam{x}{\ap{\banana{\onto{\eta}{(\lam{x}{(\ap{\op{get}!}{\bar{y}})
                    \hsbind (\lam{y}{\ap{\eta}{(x + y)}})})}}}{(\ap{\eta}{x})}})} \\
& \to_{\banana{\eta}} \app{\op{get}}{\bar{x}}
        {(\lam{x}{\ap{(\lam{x}{(\ap{\op{get}!}{\bar{y}})
                    \hsbind (\lam{y}{\ap{\eta}{(x + y)}})})}{x}})} \\
& \to_{\beta} \app{\op{get}}{\bar{x}}
        {(\lam{x}{(\ap{\op{get}!}{\bar{y}})
                    \hsbind (\lam{y}{\ap{\eta}{(x + y)}})})} \\
& \to_{\ldots} \app{\op{get}}{\bar{x}}
        {(\lam{x}{\app{\op{get}}{\bar{y}}{(\lam{y}{\ap{\eta}{(x + y)}})}})} \\
\end{align*}

The first three lines are the usual. First, we expand the definitions of
the operands (the $\vare$ operator). Then we do the same for the operation
$\sume$ and we $\beta$ reduce.

On line 4, we substitute $\op{get}!$ with the definition of the exclamation
mark that we provided in~\ref{ssec:operations-and-handlers}. Similarly, on
line 6, we expand the definition of $\hsbind$\footnote{$\hsbind$ is defined
  in terms of $^*$, which is itself defined in terms of $\banana{}$. Here,
  we expand both in one go, i.e.\ $(x \hsbind f) =
  (\ap{\banana{\onto{\eta}{f}}}{x})$}.

If you compare line 3 and line 9 (the second to last line), you will
notice that they are practically identical. The $(\ap{\op{get}!}{\bar{x}})
\hsbind k$ on line 3 has been replaced with
$\app{\op{get}}{\bar{x}}{k}$. This is encapsulated in a rule named
$\op{op}!.\hsbind$ that will be introduced in Chapter~\ref{chap:properties}
and which lets us replace $(\ap{\op{op}!}{x}) \hsbind k$ with
$\app{\op{op}}{x}{k}$.

On the last line, we simply repeat the last 6 steps (which we now packaged
into the $\op{op}!.\hsbind$ rule) for $\ap{\op{get}!}{\bar{y}}$.

We have our result, a computation that asks for the values of the variables
$\bar{x}$ and $\bar{y}$ to compute their sum. Now we will compute the value
of $\app{\lete}{\bar{y}}{\litec{2}}$.

\begin{align*}
  \app{\lete}{\bar{y}}{\litec{2}}
&= \app{\lete}{\bar{y}}{(\ap{\eta}{2})} \\
&= \appp{(\lam{v X Y}{X \hsbind (\lam{x}{\ap{\banana{\onto{\op{get}} {(\lam{u k}{\ifte{u = v}{(\ap{k}{x})} {(\app{\op{get}}{u}{k})}})}}}{Y}})})} {\bar{y}}{(\ap{\eta}{2})} \\
&\to_{\beta,\beta} \lam{Y}{(\ap{\eta}{2}) \hsbind (\lam{x}{\ap{\banana{\onto{\op{get}}{(\lam{u k}{\ifte{u = \bar{y}}{(\ap{k}{x})}{(\app{\op{get}}{u}{k})}})}}}{Y}})} \\
&\to_{\eta.\hsbind} \lam{Y}{\ap{(\lam{x}{\ap{\banana{\onto{\op{get}}{(\lam{u k}{\ifte{u = \bar{y}}{(\ap{k}{x})}{(\app{\op{get}}{u}{k})}})}}}{Y}})}{2}} \\
&\to_{\beta} \lam{Y}{\ap{\banana{\onto{\op{get}}{(\lam{u k}{\ifte{u = \bar{y}}{(\ap{k}{2})}{(\app{\op{get}}{u}{k})}})}}}{Y}} \\
&\to_{\eta} \banana{\onto{\op{get}}{(\lam{u k}{\ifte{u = \bar{y}}{(\ap{k}{2})}{(\app{\op{get}}{u}{k})}})}}
\end{align*}

Lines 1 and 2 are our interpretations of $\lite$ and $\lete$ and the rest
are applications of rules we have seen before. We find out that the
semantic action of $\app{\lete}{\bar{y}}{\litec{2}}$ is a handler. This
handler is only interested in $\op{get}$ operations that act on the
variable $\bar{y}$. It interprets such operations by returning the value
$2$, otherwise it leaves them untouched.

We will now apply this handler to the meaning of
$\sumecn{\varec{x}}{\varec{y}}$ to get the meaning of
$\letecn{y}{\litec{2}}{\sumec{\varec{x}}{\varec{y}}}$.

\begin{align*}
  & \letecn{y}{\litec{2}}{\sumec{\varec{x}}{\varec{y}}} \\
= & \letecn{y}{\litec{2}}
       {(\app{\op{get}}{\bar{x}} {(\lam{x}{
         \app{\op{get}}{\bar{y}}{(\lam{y}{\ap{\eta}{(x + y)}})}})})} \\
= & \ap{\banana{\onto{\op{get}}{(\lam{u k}{\ifte{u = \bar{y}}
        {(\ap{k}{2})}{(\app{\op{get}}{u}{k})}})}}}
       {(\app{\op{get}}{\bar{x}} {(\lam{x}{
         \app{\op{get}}{\bar{y}}{(\lam{y}{\ap{\eta}{(x + y)}})}})})} \\
\to_{\banana{\op{op}},\beta} &
  \app{(\lam{u k}{\ifte{u = \bar{y}}{(\ap{k}{2})}{(\app{\op{get}}{u}{k})}})}
      {\bar{x}}
      {(\lam{x}{\appp{\lete}{\bar{y}}{\litec{2}}
                   {(\app{\op{get}}{\bar{y}}{(\lam{y}{\ap{\eta}{(x + y)}})})}})} \\
\to_{\beta} &
  \ap{(\lam{k}{\ifte{\bar{x} = \bar{y}}{(\ap{k}{2})}{(\app{\op{get}}{\bar{x}}{k})}})}
     {(\lam{x}{\appp{\lete}{\bar{y}}{\litec{2}}
                  {(\app{\op{get}}{\bar{y}}{(\lam{y}{\ap{\eta}{(x + y)}})})}})} \\
\to_= &
  \ap{(\lam{k}{\ifte{\false}{(\ap{k}{2})}{(\app{\op{get}}{\bar{x}}{k})}})}
     {(\lam{x}{\appp{\lete}{\bar{y}}{\litec{2}}
                  {(\app{\op{get}}{\bar{y}}{(\lam{y}{\ap{\eta}{(x + y)}})})}})} \\
\to_{\textbf{if}.\false} &
  \ap{(\lam{k}{(\app{\op{get}}{\bar{x}}{k})})}
     {(\lam{x}{\appp{\lete}{\bar{y}}{\litec{2}}
                  {(\app{\op{get}}{\bar{y}}{(\lam{y}{\ap{\eta}{(x + y)}})})}})} \\
\to_{\beta} &
  \app{\op{get}}{\bar{x}}
     {(\lam{x}{\appp{\lete}{\bar{y}}{\litec{2}}
                  {(\app{\op{get}}{\bar{y}}{(\lam{y}{\ap{\eta}{(x + y)}})})}})} \\
= &
  \app{\op{get}}{\bar{x}}
     {(\lam{x}{\ap{\banana{\onto{\op{get}}{(\lam{u k}{\ifte{u = \bar{y}}
                                          {(\ap{k}{2})}{(\app{\op{get}}{u}{k})}})}}}
                  {(\app{\op{get}}{\bar{y}}{(\lam{y}{\ap{\eta}{(x + y)}})})}})} \\
\to_{\banana{\op{op}},\beta} &
  \app{\op{get}}{\bar{x}}
     {(\lam{x}{(\app{(\lam{u k}{\ifte{u = \bar{y}}
                                     {(\ap{k}{2})}{(\app{\op{get}}{u}{k})}})}
                    {\bar{y}}{(\lam{y}{\appp{\lete}{\bar{y}}{\litec{2}}{(\ap{\eta}{(x + y)})}})})})} \\
\to_{\beta} &
  \app{\op{get}}{\bar{x}}
     {(\lam{x}{(\ap{(\lam{k}{\ifte{\bar{y} = \bar{y}}
                                     {(\ap{k}{2})}{(\app{\op{get}}{\bar{y}}{k})}})}
                    {(\lam{y}{\appp{\lete}{\bar{y}}{\litec{2}}{(\ap{\eta}{(x + y)})}})})})} \\
\to_= &
  \app{\op{get}}{\bar{x}}
     {(\lam{x}{(\ap{(\lam{k}{\ifte{\true}
                                     {(\ap{k}{2})}{(\app{\op{get}}{\bar{y}}{k})}})}
                    {(\lam{y}{\appp{\lete}{\bar{y}}{\litec{2}}{(\ap{\eta}{(x + y)})}})})})} \\
\to_{\textbf{if}.\true} &
  \app{\op{get}}{\bar{x}}
     {(\lam{x}{(\ap{(\lam{k}{\ap{k}{2}})}
                   {(\lam{y}{\appp{\lete}{\bar{y}}{\litec{2}}{(\ap{\eta}{(x + y)})}})})})} \\
\to_{\beta} &
  \app{\op{get}}{\bar{x}}
     {(\lam{x}{\ap{(\lam{y}{\appp{\lete}{\bar{y}}{\litec{2}}{(\ap{\eta}{(x + y)})}})}
                  {2}})} \\
\to_{\beta} &
  \app{\op{get}}{\bar{x}}
     {(\lam{x}{\appp{\lete}{\bar{y}}{\litec{2}}{(\ap{\eta}{(x + 2)})}})} \\
= &
  \app{\op{get}}{\bar{x}}
     {(\lam{x}{\ap{\banana{\onto{\op{get}}{(\lam{u k}{\ifte{u = \bar{y}}
                                          {(\ap{k}{2})}{(\app{\op{get}}{u}{k})}})}}}
                  {(\ap{\eta}{(x + 2)})}})} \\
\to_{\banana{\eta}} & \app{\op{get}}{\bar{x}} {(\lam{x}{\ap{\eta}{(x + 2)}})} \\
\end{align*}

This one is a bit more complicated and so we will go through it line by
line:


\begin{description}

  \item[Line 2 ($=$)] We replace the expression
    $\sumecn{\varec{x}}{\varec{y}}$ with its denotation that we computed
    earlier.

  \item[Line 3 ($=$)] We do the same for the construction
    $\app{\lete}{\bar{y}}{\litec{2}}$.

  \item[Line 4 ($\to_{\banana{\op{op}},\beta}$)] The let-binder is a
    handler with a clause for $\op{get}$ and its argument is a computation
    that performs $\op{get}$. This means we do two things: we replace
    $\op{get}$ with the clause ($\lam{u k}{\ldots}$) from the handler and
    we move the handler inside the continuation. In order to save space, we
    contract the handler back to $\app{\lete}{\bar{y}}{\litec{2}}$.

    Looking at the subscript on the arrow, you can see that we have also
    performed a $\beta$ reduction. This is part of a pattern we will
    repeat often. The rule $\banana{\op{op}}$ allows us to rewrite
    $\ap{\banana{\onto{\op{op}}{f}}}{(\app{\op{op}}{x}{k})}$ into
    $\app{f}{x}{(\lam{y}{\ap{\banana{\onto{\op{op}}{f}}}{(\ap{k}{y})}})}$. We
    see that we had to introduce a new $\lambda$ binder in order to compose
    $k$ and $\banana{\onto{\op{op}}{f}}$. However, quite often it will be
    the case that $k$ will already be of the form $(\lam{y}{M})$. In that
    case, the $\banana{\op{op}}$ rule would translate
    $\ap{\banana{\onto{\op{op}}{f}}}{(\app{\op{op}}{x}{(\lam{y}{M})})}$ to
    $\app{f}{x}{(\lam{y}{\ap{\banana{\onto{\op{op}}{f}}}{(\ap{(\lam{y}{M})}{y})}})}$. Now
    we have two $\lambda$ binders in the continuation but we can
    immediately $\beta$ reduce away the second one to get
    $\app{f}{x}{(\lam{y}{\ap{\banana{\onto{\op{op}}{f}}}{M}})}$. By
    composing the $\banana{\op{op}}$ rule and the $\beta$ rule, we can
    therefore get the following useful rule:

    $$
    \ap{\banana{\onto{\op{op}}{f}}}{(\app{\op{op}}{x}{(\lam{y}{M})})}
    \to_{\banana{\op{op}},\beta}
    \app{f}{x}{(\lam{y}{\ap{\banana{\onto{\op{op}}{f}}}{M}})}
    $$

  \item[Line 5 ($\beta$)] Here we apply the handler clause to the
    operation's argument, the variable $\bar{x}$. We do not do the same for
    the continuation argument $k$ since we would have to copy it twice and
    we can avoid that by reducing the conditional expression first.

  \item[Line 6 ($=$)] Since $\bar{x}$ and $\bar{y}$ are different
    variables, we expect the equality predicate to return false ($\false$).

  \item[Line 7 ($\textbf{if}.\false$)] Since the condition evaluated to
    $\false$, we choose the second branch of the conditional. We are
    invoking a derived rule that we have not shown before. The rule
    consists simply of expanding the definitions of $\false$ and
    $\textbf{if}$ and performing a $\beta.+_2$ reduction. You can find the
    rule in Chapter~\ref{chap:properties}.

  \item[Line 8 ($\beta$)] We have simplified the handler clause and so
    we are ready to substitute in the continuation. Remark the similarity
    between this line and line 2. The only difference is the position of
    the $\app{\lete}{\bar{y}}{\litec{2}}$ handler. It has moved from the
    first $\op{get}$ down to the second $\op{get}$.

  \item[Line 9 ($=$)] We will proceed as we did from line 3 onwards. We
    expand $\app{\lete}{\bar{y}}{\litec{2}}$ to reveal the handler.

  \item[Line 10 ($\banana{\op{op}},\beta$)] We have another $\op{get}$
    computation and so we pull out the handler clause for $\op{get}$. We
    contract the handler back into $\app{\lete}{\bar{y}}{\litec{2}}$ and
    move it down, applying it to the next step of the computation in the
    same way as was described at line 4.

  \item[Line 11 ($\beta$)] We pass the variable $\bar{y}$ into the
    handler clause so we can decide the condition.

  \item[Line 12 ($=$)] This time, we are comparing a variable to itself and
    so the equality predicate will yield true ($\true$).

  \item[Line 13 ($\textbf{if}.\true$)] Since the condition is true, we
    choose the first branch of the conditional. The rule
    $\textbf{if}.\true$ is analogous to the $\textbf{if}.\false$ described
    at line 7.

  \item[Line 14 ($\beta$)] Having simplified the handler clause, we can
    pass in the continuation so that we can supply it with the value of the
    variable $bar{y}$ that it was looking for.

  \item[Line 15 ($\beta$)] We pass $2$ to the continuation.

  \item[Line 16 ($=$)] Now we will use the handler one last time so we
    reveal it first by expanding $\app{\lete}{\bar{y}}{\litec{2}}$.

  \item[Line 17 ($\banana{\eta}$)] The handler has no clause for
    $\eta$. In~\ref{ssec:operations-and-handlers}, we have declared that
    this is just a syntactic shortcut for the handler having a default
    clause $\onto{\eta}{\eta}$. Applying the $\banana{\eta}$ rule has two
    effects: we replace $\eta$ with its handler clause, $\eta$, and we
    discard the handler since there is no continuation.
\end{description}

We have started with $\app{\op{get}}{\bar{x}}
{(\lam{x}{\app{\op{get}}{\bar{y}}{(\lam{y}{\ap{\eta}{(x + y)}})}})}$ as the
denotation of $\sumecn{\varec{x}}{\varec{y}}$. By applying the denotation
of $\app{\lete}{\bar{y}}{\litec{2}}$, we have interpreted away the
$\ap{\op{get}}{\bar{y}}$ request by providing the value $2$ as the
response, leaving us with $\app{\op{get}}{\bar{x}} {(\lam{x}{\ap{\eta}{(x +
      2)}})}$. We will resolve the variable $\bar{x}$ by analogy.

\begin{align*}
  \app{\lete}{\bar{x}}{\litec{1}}
&\to_{\ldots} \banana{\onto{\op{get}}{(\lam{u k}{\ifte{u = \bar{x}}{(\ap{k}{1})}{(\app{\op{get}}{u}{k})}})}}
\end{align*}

This computation goes down the same way as the one for
$\app{\lete}{\bar{x}}{\litec{1}}$, only the number and the variable name
are different.


\begin{align*}
\letecn{x}{\litec{1}}{\letec{y}{\litec{2}}{\sumec{\varec{x}}{\varec{y}}}}
& = \letecn{x}{\litec{1}}{(\app{\op{get}}{\bar{x}} {(\lam{x}{\ap{\eta}{(x + 2)}})})} \\
& \to_{\ldots} \ap{\eta}{(1 + 2)} \\
& \to_{+} \ap{\eta}{3}
\end{align*}

We can perform the reduction from the first line to the second line by
repeating the exact same steps as we did from line 8 to line 17 when
simplifying $\appp{\lete}{\bar{y}}{\litec{2}}
{(\app{\op{get}}{\bar{y}}{(\lam{y}{\ap{\eta}{(x + y)}})})}$.

Finally, we can conclude that according to our interpretations, the
value of the expression we started off with is $3$. In other words, the
following holds:

$$
\letecn{x}{\litec{1}}{\letec{y}{\litec{2}}{\sumec{\varec{x}}{\varec{y}}}}
\quad \to \quad
\ap{\eta}{3}
$$


\subsection{Treating Variables without Computations}

Before we proceed onto the next chapter, let us compare the calculator
semantics that we have developed so far with an alternative that does not
rely on the computation abstraction.

As a reminder, this is our calculator semantics
from~\ref{ssec:raising-the-type-of-exp}, along with a semantics for
$\trye$.

\begin{align*}
  exp &= \NN + 1 \\
  \dive &= \lam{X Y}{\case{X}{x}{\case{Y}{y}{x / y}{\_}{\ap{\inr}{\star}}}{\_}{\ap{\inr}{\star}}} \\ 
  \trye &= \lam{X Y}{\case{X}{x}{\case{Y}{y}{\ap{\inl}{y}}{\_}{\ap{\inl}{x}}}{\_}{\ap{\inr}{\star}}} \\ 
  \sume &= \lam{X Y}{\case{X}{x}{\case{Y}{y}{\ap{\inl}{(x + y)}}{\_}{\ap{\inr}{\star}}}{\_}{\ap{\inr}{\star}}} \\ 
  \prode &= \lam{X Y}{\case{X}{x}{\case{Y}{y}{\ap{\inl}{(x \times y)}}{\_}{\ap{\inr}{\star}}}{\_}{\ap{\inr}{\star}}} \\ 
  \lite &= \lam{x}{\ap{\inl}{x}}
\end{align*}

What denotation should we assign to $\varecn{x}$? It does not correspond to
any particular natural nor is it an expression that should always fail. We
can find an answer to our question by expanding the domain from $\NN + 1$
to $\gamma \to (\NN + 1)$ where $\gamma = var \to (\NN + 1)$ is the type of
environments\footnote{Environments are partial functions from variable
  names to naturals, hence the $+ 1$ in the result type.}.

\begin{align*}
  exp &= \gamma \to \NN + 1 \\
  \gamma &= var \to \NN + 1 \\
  \lete &= \lam{v X Y e}{\case{\ap{X}{e}}{x}{\ap{Y}{(\lam{u}{\ifte{u = v}{\ap{\inl}{x}}{\ap{e}{u}}})}}{\_}{\ap{\inr}{\star}}} \\
  \vare &= \lam{v e}{\ap{e}{v}} \\
  \dive &= \lam{X Y e}{\case{\ap{X}{e}}{x}{\case{\ap{Y}{e}}{y}{x / y}{\_}{\ap{\inr}{\star}}}{\_}{\ap{\inr}{\star}}} \\ 
  \trye &= \lam{X Y e}{\case{\ap{Y}{e}}{y}{\ap{\inl}{y}}{\_}{\ap{X}{e}}} \\ 
  \sume &= \lam{X Y e}{\case{\ap{X}{e}}{x}{\case{\ap{Y}{e}}{y}{\ap{\inl}{(x + y)}}{\_}{\ap{\inr}{\star}}}{\_}{\ap{\inr}{\star}}} \\ 
  \prode &= \lam{X Y e}{\case{\ap{X}{e}}{x}{\case{\ap{Y}{e}}{y}{\ap{\inl}{(x \times y)}}{\_}{\ap{\inr}{\star}}}{\_}{\ap{\inr}{\star}}} \\ 
  \lite &= \lam{x e}{\ap{\inl}{x}}
\end{align*}

Since we needed to expand the type $exp$ into $\gamma \to \NN + 1$, we need
to make appropriate changes to the previous denotations. When producing
values of type $exp$, we now abstract over an argument $e$ of type
$\gamma$. When consuming values of type $exp$, we apply to them an
environment of type $\gamma$ to get a value of the old type $\NN + 1$.

In the rule for $\vare$, we just lookup the value of the variable in the
environment. If the environment does not contain a value for this variable,
it returns $\inr{\star}$ of type $\NN + 1$. In that case, $\vare$ returns
the same to indicate that the expression has no reference. In $\lete$,
evaluate $X$ in the current environment $e$ and if it has a value, we also
evaluate $Y$. However, $Y$ itself is evaluated in an environment in which
$v$ is bound to $x$, the value of $X$.


\begin{align*}
  exp &= \FF_{\{\typedop{error}{\chi}{0},\typedop{get}{var}{\NN}\}}(\NN) \\
  \lete &= \lam{v X Y}{X \hsbind
    (\lam{x}{\ap{\banana{\onto{\op{get}}{(\lam{u k}{\ifte{u = v}{(\ap{k}{x})}{(\app{\op{get}}{u}{k})}})}}}{Y}})} \\
  \vare &= \op{get}! \\
  \dive &= \lam{X Y}{X \hsbind (\lam{x}{Y \hsbind (\lam{y}{\case{(x / y)}{z}{\ap{\eta}{z}}{\_}{\dbze}})})} \\
  \trye &= \lam{X Y}{\ap{\banana{\onto{\op{error}}{(\lam{e k}{X})}}}{Y}} \\
  \sume &= \lam{X Y}{X \hsbind (\lam{x}{Y \hsbind (\lam{y}{\ap{\eta}{(x + y)}})})} \\
  \prode &= \lam{X Y}{X \hsbind (\lam{x}{Y \hsbind (\lam{y}{\ap{\eta}{(x \times y)}})})} \\
  \lite &= \lam{x}{\etaE{x}}
\end{align*}

We notice that $\op{get}$ is used in $\lete$ and $\vare$ but it is not
mentioned in other parts of the fragment. $\op{error}$ is used in $\dive$
(through $\dbze$) and in $\trye$ and again it is absent in the rest of the
fragment. Compare this with the approach that models $exp$ as $\gamma \to
\NN + 1$. In all of the entries, we abstract over a variable of type
$\gamma$ for the current environment and we explicitly pass it around. We
also do explicit case analysis on all intermediate results and we wrap all
our results in $\inl$. Furthermore, changing the type $exp$ to accomodate
$\lete$ and $\vare$ forced us to change the definitions of $\dive$,
$\trye$, $\sume$, $\prode$ and $\lite$ (i.e.\ all the other constructions
in our fragment).

We could mitigate both of these problems using the same technique we have
seen in~\ref{ssec:refactoring-with-monads}. We can find a monad whose type
constructor is $T(\alpha) = \gamma \to \alpha + 1$, whose unit is $\eta =
(\lam{x e}{\ap{\inl}{x}})$ and whose bind simultaneously takes care of
passing around the environment and performing case analysis on intermediate
results. However, as we will see in~\ref{sec:monads}, we will still be
motivated to recreate something like the abstract operations that we are
using in our proposed solution.


\section{Summary}

In this chapter, we have hoped to convey two things:

\begin{itemize}
\item Some sense of familiarity with our calculus. 

  We have performed long reduction chains which illustrated a large part of
  the reduction rules and syntax introduced in
  Chapter~\ref{chap:definitions}. The goal of that was to build up our
  intuition of how terms written in our calculus actually look like and how
  to use them to write computations.

\item A glimpse of the methodology we would like to adopt.

  As a running example in this chapter, we have studied a simple formal
  language and gave it a compositional semantics. In the process of doing
  that, we have seen some advantages of using computations as opposed to
  using simpler terms. Notably the fact that we could extend the fragment
  with new constructions that demonstrate new semantic phenomena, such as
  errors or variables, without having to rewrite existing denotations. One
  of the principal motivations of this thesis is to apply this technique to
  the problem of natural language semantics and verify whether we can enjoy
  the same properties.
\end{itemize}
