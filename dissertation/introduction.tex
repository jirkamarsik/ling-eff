\chapter*{Introduction}

The motivation behind this thesis is to advance the automatic process of
translating natural language into logical representations. The interest
behind this translation procedure is two-fold: linguists use it to give a
formal semantics to natural languages and computer scientists use it to
perform automated reasoning on natural language data.

There has been considerable research into translating parts of English and
other natural languages. When looking at a sentence of English, we can
identify many of the problems inherent in the translation and point to
papers that have proposed solutions.

\begin{exe}
  \ex She must still be dating that idiot.
\end{exe}

\begin{enumerate}
  \item \label{item:first-feature} We have anaphoric expressions,
    \emph{she} and \emph{that}. We know that we can translate sentences
    containing those into Discourse Representation Theory (DRT)
    \cite{kamp1993discourse} structures, Dynamic Predicate Logic (DPL)
    \cite{groenendijk1991dynamic} formulas or continuation-passing style
    $\lambda$-terms \cite{de2006towards}.
  \item We have the modal auxiliary \emph{must} that we can translate into
    an operator of a modal logic or directly to some universal
    quantification over possible worlds.
  \item We have the progressive present tense in \emph{be
    dating}. Similarly to the modal, we can translate this into an operator
    of a temporal logic or directly posit the existence of some interval of
    time during which the dating takes place and assert that the time of
    utterance lies in that interval.
  \item We have the presupposition trigger \emph{still} that will tell us
    that the dating must have been going on for some time already. We will
    have to possess some mechanism to project this contribution outside the
    scope of any of the logical operators present (including \emph{must} in
    this example\footnote{Consider the same example with \emph{should}
      instead of \emph{must}.}). We can adopt the strategy of Lebedeva
    \cite{lebedeva2012expression} and raise exceptions to perform the
    projection.
  \item \label{item:last-feature} We have the expressive epithet
    \emph{idiot}. Following the theory of conventional implicatures of
    Potts \cite{potts2005logic}, elaborated by Gutzmann
    \cite{gutzmann2015use}, we know that we should introduce a second
    dimension into which we tuck away the speaker's negative attitude
    towards the date-ee so that it does not interfere with the at-issue
    content.
\end{enumerate}

All of the advice given above seems sound. We could follow these guidelines
to intuitively arrive at some reasonable logical representation. Now how do
we write down this joint translation process? Most of the theories
mentioned above come with their own language: their sets of definitions,
notations and operations.

DRT introduces its own encoding of logical formulas and states an algorithm
that builds them up incrementally from the input sentence. Potts's logic of
conventional implicatures introduces two-dimensional logical formulas and
defines new modes of combination to compute with them. Compositional
treatments of intensionality or tense tend to use the simply-typed lambda
calculus, as is the case in de Groote's treatment of anaphora
\cite{de2006towards}. Lebedeva then refines the calculus of de Groote and
integrates exceptions to the semantics of the calculus
\cite{lebedeva2012expression}.

It seems clear that in order to arrive at a precise notion of what it means
do all of \ref{item:first-feature}--\ref{item:last-feature}, we will first
have to be able to express the theories behind
\ref{item:first-feature}--\ref{item:last-feature} using a single formal
language.

\section{Enter Monads}

We will base our universal language on the $\lambda$-calculus. Thanks to
Richard Montague's hugely influential work \cite{montague1973proper}, the
$\lambda$-calculus is already a very popular formalism in formal
compositional semantics\footnote{Frege's compositionality principle states
  that the meanings of complex expressions should be determined by
  (i.e.\ be functions of) the meanings of its constituents. If meanings are
  defined in terms of meaning functions, it makes sense to use a calculus
  of functions.}. Many phenomena are analyzed using the $\lambda$-calculus
and the rest tend to get translated to $\lambda$-calculus as well (see, for
example, $\lambda$-DRT \cite{kuschert1995type} or de Groote's
continuation-based approach to dynamics \cite{de2006towards}).

However, even though we have several theories which are all formalised in
$\lambda$-calculus, it does not necessarily mean that they are compatible
or that we know how to combine them together. A theory of intensionality
might state that sentences ought to be translated to terms that have the
type $\sigma \to o$, the type of functions from possible worlds to truth
values. On the other hand, an account of expressives would argue that
sentences ought to correspond to terms of type $o \times \epsilon$, the
types of pairs of truth values (propositional content) and expressive
markers (expressive content). The two theories would be compatible at the
calculus-level but not at the term-level. A function operating with
intensional propositions would not be directly applicable to an expressive
proposition.

To continue our quest for uniformity and compatibility of semantic
operations, we will look at the terms and types used by the different
$\lambda$-theoretic treatments of semantics and try to find a common
structure underneath. We notice that all such approaches share at least the
following structure:

\begin{enumerate}
\item \label{item:type-transformation} The types of some of the denotations
  get expanded. For example, when dealing with quantifiers, the type of
  denotations of noun phrases goes from $\iota$ (single individuals) to
  $(\iota \to o) \to o$ (generalized quantifiers over individuals); in
  intensional semantics, the type of denotations of sentences goes from $o$
  (truth values) to $\sigma \to o$ (functions from possible worlds to truth
  values, i.e.\ sets of possible worlds); and with expressives, the type of
  denotations of sentences goes from $o$ to $o \times \epsilon$ (truth
  values coupled with expressive markers).
\item \label{item:monad-eta} There is a process that can lift denotations
  of the old type into denotations of the new type. In the quantifier
  example, this is the famous type raising operation. In the intensional
  example, this is the $\textbf{K}$ combinator that turns a truth value
  into a constant function that assigns that truth value to all worlds. In
  the expressive example, this is the function that couples a proposition
  with a neutral/empty expressive marker.
\item Then there are of course other inhabitants of the expanded type that
  are not found by using the lifting function described above. Those are
  the ones that we expanded the type for. Quantificational noun phrases
  such as \emph{everyone} are not the result of type raising any specific
  individual. Intensional propositions such as \emph{Hespherus is
    Phosphorus} have extensions that vary from world to world. Expressives
  such as the diminutive \emph{Wolfie} will point to some individual and
  also carry some expressive marker that conveys the speaker's attitude
  towards the referent.
\item \label{item:monad-mu} Finally, these approaches also have some
  general way of composing smaller denotations into larger ones and dealing
  with the added complexity caused by the more elaborate types. When
  applying a transitive verb to a quantificational subject and object, we
  let one (often the subject) first take scope and then we let the other
  take scope. When applying the verb to a intensional arguments, we pass
  down the world at which we are evaluating the sentence down to both the
  subject and the object. When applying it to expressive arguments, we
  apply the verb to the referents of both the subject and the object and on
  the side we collect the expressive content of both.
\end{enumerate}

This kind of structure is very commonly seen in functional programming and
in denotational semantics of programming languages. It is the structure of
an \emph{applicative functor} \cite{mcbride2008applicative} (or
\emph{strong lax monoidal functor}, for the categorically-inclined). The
above examples are also instances of a more special structure called a
\emph{monad} \cite{moggi1991notions}.

We will not go into the minutiae of the definition of a monad here but we
will give a rough sketch nonetheless. A monad is a triple $(T, \eta, \mu)$
where $T$ is a function on types (the type expansion we saw in
\ref{item:type-transformation}), $\eta$ is some way of lifting simple
values into expanded values (the lifting functions in \ref{item:monad-eta})
and $\mu$ gives us a general way of combining values of this expanded type
(similar to the examples given in \ref{item:monad-mu}). The triple also has
to satisfy some algebraic properties which guarantee that composing
functions of expanded types is associative and that the lifting function
serves as a unit for this composition operator.

The examples given above are all instances of monads. This has al


\TODO{David Lewis --- General Semantics --- ``In order to say what a meaning is, we may first ask what a meaning does and then find something that does that.''}

\TODO{Also speak about monad transformers (and free monads?) here as we are
not making an individual section in the Connections chapter about monads.}
