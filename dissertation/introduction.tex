\chapter*{Introduction}

The motivation behind this thesis is to advance the automatic process of
translating natural language into logical representations. The interest
behind this translation procedure is two-fold: linguists use it to give a
formal semantics to natural languages and computer scientists use it to
perform automated reasoning on natural language data.

There has been considerable research into translating parts of English and
other natural languages. When looking at a sentence of English, we can
identify many of the problems inherent in the translation and point to
papers that have proposed solutions.

\begin{exe}
  \ex She might still be dating that idiot. \label{ex:intro}
\end{exe}

\begin{enumerate}
\item \label{item:first-feature} We have anaphoric expressions, such as the
  pronoun \emph{she}. We know that we can translate sentences with anaphora
  into Discourse Representation Theory~(DRT)~\cite{kamp1993discourse}
  structures, Dynamic Predicate Logic~(DPL)~\cite{groenendijk1991dynamic}
  formulas or continuation-passing style
  $\lambda$-terms~\cite{de2006towards}.
  \item We have the modal auxiliary \emph{might} that we can translate into
    an operator of modal logic or directly to some existential
    quantification over possible worlds.
  \item We have the progressive present tense in \emph{be
      dating}. Similarly to the modal, we can translate this into an
    operator of temporal logic or directly posit the existence of some
    interval of time during which the dating takes place and assert that
    the time of utterance (possibly) lies in that interval.
  \item We have the presupposition trigger \emph{still} that tells us that
    the two subjects must have been dating in the past already. We will
    have to possess some mechanism to project this contribution outside the
    scope of any of the logical operators present.\footnote{While in the
      present, we can only infer that they \emph{might} be dating, by
      accommodating the presupposition, we can infer that sometime in the
      past, they \emph{must have been} dating.} We can adopt the strategy
    of Lebedeva~\cite{lebedeva2012expression} and raise exceptions to
    perform the projection.
  \item \label{item:last-feature} We have the expressive epithet
    \emph{idiot}. Following the theory of conventional implicatures of
    Potts~\cite{potts2005logic}, elaborated by
    Gutzmann~\cite{gutzmann2015use}, we know that we should introduce a
    second dimension into which we tuck away the speaker's negative
    attitude towards the date-ee so that it does not interfere with the
    at-issue content.
\end{enumerate}

All of the advice given above seems sound. We could follow these guidelines
to intuitively arrive at some reasonable logical representation. Now how do
we write down this joint translation process? Most of the theories
mentioned above come with their own language: their own definitions,
notation and operations.

DRT introduces its own encoding of logical formulas and states an algorithm
that builds them up incrementally from the input
sentence~\cite{kamp1993discourse}. Potts's logic of conventional
implicatures introduces two-dimensional logical formulas and defines new
modes of combination to compute with
them~\cite{potts2005logic}. Compositional treatments of intensionality or
tense tend to use the simply-typed lambda
calculus~\cite{ben2007semantics,de2013note}, as is also the case in de
Groote's treatment of anaphora~\cite{de2006towards}. In studying
presuppositions, Lebedeva uses a modified version of de Groote's calculus
which includes exceptions~\cite{lebedeva2012expression}.

It seems clear that in order to arrive at a precise notion of what it means
do all of~\ref{item:first-feature}--\ref{item:last-feature}, we will first
have to be able to express the theories behind
\ref{item:first-feature}--\ref{item:last-feature} using a single formal
language.

\section*{Enter Monads}

We will base our universal language on the $\lambda$-calculus. Thanks to
Richard Montague's hugely influential work~\cite{montague1973proper}, the
$\lambda$-calculus is already a very popular formalism in formal
compositional semantics.\footnote{Frege's compositionality principle states
  that the meanings of complex expressions should be determined by (i.e.\
  be functions of) the meanings of its constituents. If complex meanings
  are to be seen as functions of other meanings, it makes sense to use a
  calculus of functions, i.e.\ the $\lambda$-calculus.} Many phenomena are
analyzed using the $\lambda$-calculus and the rest tend to get translated
to $\lambda$-calculus as well (see, for example,
$\lambda$-DRT~\cite{kuschert1995type} or de Groote's continuation-based
approach to dynamics~\cite{de2006towards}).

However, even though we have several theories which are all formalised in
$\lambda$-calculus, it does not necessarily mean that they are compatible
or that we know how to combine them together. A theory of intensionality
might state that sentences ought to be translated to terms that have the
type $\sigma \to o$, the type of functions from possible worlds to truth
values. On the other hand, an account of expressives would argue that
sentences ought to correspond to terms of type $o \times \epsilon$, the
type of pairs of truth values (propositional content) and expressive
markers (expressive content). The two theories would be compatible at the
calculus-level but not at the term-level. A function operating with
intensional propositions would not be directly applicable to an expressive
proposition.

To continue our quest for uniformity and compatibility of semantic
operations, we will look at the terms and types used by the different
$\lambda$-theoretic treatments of semantics and try to find a common
structure underneath. We notice that all such approaches share at least the
following structure:

\begin{enumerate}
\item \label{item:type-transformation} The types of some of the denotations
  get expanded. For example, when dealing with quantifiers, the type of
  denotations of noun phrases goes from $\iota$ (single individuals) to
  $(\iota \to o) \to o$ (generalized quantifiers over individuals); in
  intensional semantics, the type of denotations of sentences goes from $o$
  (truth values) to $\sigma \to o$ (functions from possible worlds to truth
  values, i.e.\ sets of possible worlds); and with expressives, the type of
  denotations of sentences goes from $o$ to $o \times \epsilon$ (truth
  values coupled with expressive markers).
\item \label{item:monad-eta} There is a process that can lift denotations
  of the old type into denotations of the new type. In the quantifier
  example, this is the famous type raising operation. In the intensional
  example, this is the $\textbf{K}$ combinator that turns a truth value
  into a constant function that assigns that truth value to all worlds, a
  rigid intension. In the expressive example, this is the function that
  couples a proposition with a neutral/empty expressive marker.
\item Then there are other inhabitants of the expanded type that are not
  found by using the lifting function described above; those are the ones
  for which we expanded the type. Quantificational noun phrases such as
  \emph{everyone} are not the result of type raising any specific
  individual. Intensional propositions such as \emph{Hespherus is
    Phosphorus} have extensions that vary from world to world. Expressives
  such as the diminutive \emph{Wolfie} will point to some individual and
  also carry some expressive marker that conveys the speaker's attitude
  towards the referent.
\item \label{item:monad-mu} Finally, these approaches also have some
  general way of composing smaller denotations into larger ones and dealing
  with the added complexity caused by the more elaborate types. When
  applying a transitive verb to a quantificational subject and object, we
  let one (often the subject) first take scope and then we let the other
  take scope. When applying the verb to intensional arguments, we pass the
  world at which we are evaluating the sentence down to both the subject
  and the object. When applying it to expressive arguments, we apply the
  verb to the referents of both the subject and the object and on the side
  we collect the expressive content of both.
\end{enumerate}

This kind of structure is very commonly seen in functional programming and
in denotational semantics of programming languages. It is the structure of
an \emph{applicative functor}~\cite{mcbride2008applicative} (or
\emph{strong lax monoidal functor}, for the categorically-inclined). The
above examples are also instances of a more special structure called a
\emph{monad}~\cite{moggi1991notions}.

We will not go into the minutiae of the definition of a monad here but we
will give a rough sketch nonetheless. A monad is a triple
$(T, \eta, \hsbind)$ where $T$ is a function on types (the type expansion
we saw in~\ref{item:type-transformation}), $\eta$ is some way of lifting
simple values into expanded values (the lifting functions
in~\ref{item:monad-eta}) and $\hsbind$ gives us a general way of combining
values of this expanded type (similar to the examples given
in~\ref{item:monad-mu}).\footnote{This way of presenting a monad (a type
  constructor, $\eta$ and $\hsbind$) is particular to functional
  programming. Note that this presentation differs from the
  category-theoretical one which replaces $\hsbind$ with a natural
  transformation $\mu$ \cite{mac1978categories}.} The triple also has to
satisfy some algebraic properties which guarantee that composing functions
of expanded types is associative and that the lifting function serves as a
unit for this composition operator.

The examples given above are all instances of monads. The prevalence of
monads in natural language semantics has been already discovered by Shan
in~\cite{shan2002monads}. However, the challenge lies in trying to use
several monads at the same time.


\section*{Linguistic Side Effects}

Monads often appear in denotational semantics of programming languages to
account for notions of computation commonly referred to as \emph{side
  effects}~\cite{moggi1991notions}. We can base ourselves on this
correspondence and regard the monadic structure in natural language as
linguistic side effects. This analogy was pursued by
Shan~\cite{shan2005linguistic,shan2005thesis} and
Kiselyov~\cite{kiselyov2008call} and is present in recent work on monads in
natural language
semantics~\cite{giorgolo2012monads,charlow2014semantics}. However, the idea
itself stretches back before monads were even introduced to computer
science. In their 1977 paper~\cite{hobbs1977making}, Hobbs and Rosenschein
take a computational perspective on the intensional logic of
Montague~\cite{montague1973proper}: intensions correspond to programs and
extensions correspond to values. A program can read the value of global
variables that describe the state of the world.\footnote{Dependence on an
  environment of some type $\sigma$ is a side effect that can be described
  using the reader monad. This monad lifts the type $\alpha$ to the type
  $\sigma \to \alpha$. This is exactly the change of types that is
  prescribed by theories of
  intensionalization~\cite{ben2007semantics,de2013note}.} The operators
$\uparrow$ and $\downarrow$, which map between extension-denoting
expressions and intension-denoting expressions, then correspond to the
Lisp-style operators $\texttt{quote}$ and $\texttt{eval}$ respectively.

The idea of treating linguistic expressions as effectful actions or
programs is also very relevant to dynamic semantics, which treats the
meanings of sentences as instructions to update some common ground or other
linguistic context.\footnote{The use of monads to encode dynamic effects
  (anaphora) dates back to 2009 and the work of Giorgolo and
  Unger~\cite{giorgolo2009coreference,unger2012dynamic}} Dynamic semantics
and anaphora are sometimes classified as belonging to both semantics and
pragmatics. This is also the case for other phenomena that we will treat as
side effects in our dissertation: deixis, presupposition,
implicature. Pragmatics studies the way a language fits into the community
of its users, i.e.\ how it is actually used by its speakers to achieve
their goals. It might then come as no surprise that pragmatics align well
with side effects in programming languages since side effects themselves
concern the ways that programs can interact with the world of their users
(e.g.\ by making things appear on screen or by listening for the user's
input).


\section*{Effects and Handlers}

By looking at the different monadic structures of natural language
semantics as side effects, we can apply theories that combine side effects
to find a formalism that can talk about all the aspects of language at the
same time. One such theoretical framework are effects and handlers. In this
framework, programs are interpreted as sequences of instructions (or more
generally as decision trees).\footnote{More precisely, we are interpreting
  programs in a free monad~\cite{swierstra2008data}.} The instructions are
symbols called \emph{operations}, which stand for the different effects,
the different ways that programs can interact with their contexts. In our
application to natural language semantics, here are some examples of
operations that will feature in our demonstrations, along with their
intended semantics:\footnote{The operations are just symbols and so have no
  inherent meaning.}

\begin{itemize}
\item $\op{introduce}$ introduces a new discourse referent to the
  context. This is the kind of operation used by noun phrases such as the
  indefinite \emph{a man}.
\item $\ap{\op{presuppose}}{P}$ presupposes the existence of an entity
  satisfying the predicate $P$. This is used by definite descriptions
  \emph{the $P$} and by proper nouns.
\item $\ap{\op{implicate}}{i}$ states that $i$ is an implicature. This
  operation is used by appositive constructions such as \emph{John, who is
    my neighbor}.
\item $\op{speaker}$ asks the context for the identity of the speaker. This
  is used by the first-person pronoun to find its referent.
\end{itemize}

The process of calculating the denotation of a linguistic expression is
broken down to these operations. When expressions combine to form sentences
and discourses, these operations end up being concatenated into a large
program which will want to perform a series of interactions with its
context. This is when handlers come into play. A \emph{handler} is an
interpreter that gives a definition to the operation symbols in a
program. Handlers can be made modular~\footnote{In a similar way that
  monads can be turned into monad transformers (monad morphisms) and then
  composed~\cite{shan2002monads,wu2015transformers}.} so that the
interpreter for our vocabulary of context interactions can be defined as
the composition of several smaller handlers, each treating a different
aspect of language (dynamicity, implicature, deixis\ldots).

When using effects and handlers, we therefore start by enumerating the set
of interactions that programs (i.e.\ linguistic expressions in our
application) can have with their contexts. Then, we can interpret
linguistic expressions as sequences of such instructions. Finally, we write
handlers which implement these instructions and produce a suitable semantic
representation. This approach thus closely follows the mantra given by
Lewis:

\begin{quote}
  In order to say what a meaning is, we may first ask what a meaning does
  and then find something that does that.

  \begin{flushright}
    General Semantics, David Lewis~\cite{lewis1970general}
  \end{flushright}
\end{quote}

We can trace the origins of effects and handlers to two strands of
work. One is Cartwright and Felleisen's work on Extensible Denotational
Language Specifications~\cite{cartwright1994extensible}, in which a
technique for building semantics is developed such that when a
(programming) language is being extended with new constructions (and new
side effects), the existing denotations remain compatible and can be
reused. The other precursor is Hyland's, Plotkin's and Power's work on
algebraic effects~\cite{hyland2006combining}, a categorical technique for
studying effectful computations, which was later extended by Plotkin and
Pretnar to include
handlers~\cite{plotkin2009handlers,pretnar2010logic,plotkin2013handling}. The
technique has gained in popularity in recent years (2012 and onward). It
finds applications both in the encoding of effects in pure functional
programming
languages~\cite{kiselyov2013extensible,kiselyov2015freer,kammar2013handlers,brady2013programming}
and in the design of programming
languages~\cite{bauer2012programming,lindley2016dobedobedo,dolan2015effective,kiselyov2016eff,hillerstrom2016compiling}. Our
thesis will explore the applicability of effects and handlers to natural
language semantics.


\section*{Plan}

The manuscript of this thesis is split into two parts. In the first, we
develop a formal calculus that extends the simply-typed lambda calculus
with a free monad for effects and handlers. We prove some key properties,
such as subject reduction, confluence and, to an extent, termination, and
show how the calculus relates to other similar notions such as
continuations or monads. In the second part, we analyze some of the aspects
of linguistic meaning as side effects: deixis, conventional implicature,
quantification, anaphora and presupposition. We then incrementally build up
a fragment that features all of those features and demonstrates some of
their interactions.
