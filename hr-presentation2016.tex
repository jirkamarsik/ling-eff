\documentclass{beamer}

\usepackage[utf8]{inputenc}
\usepackage{pgfpages}
\usepackage{stmaryrd}
\usepackage{amsmath}
\usepackage{graphicx}
\usepackage{tikz}
\usepackage{listings}
\usepackage{xcolor}
\usepackage[normalem]{ulem}
\usepackage{appendixnumberbeamer}
\usepackage{url}

\usepackage[export]{adjustbox}
\setbeamercovered{transparent}

\hypersetup{pdfstartview={Fit}}
\def\limp {\mathbin{{-}\mkern-3.5mu{\circ}}}

\usepackage{pifont}% http://ctan.org/pkg/pifont
\newcommand{\cmark}{\ding{51}}%
\newcommand{\xmark}{\ding{55}}%




\setbeamertemplate{navigation symbols}{}
\setbeamertemplate{footline}
{\hfill {\normalsize \insertframenumber{}/\inserttotalframenumber{}}}
\hypersetup{pdfstartview={Fit}}

\AtBeginSection[]
{
\begin{frame}{Outline}
  \tableofcontents[currentsection]
\end{frame}
}



\definecolor{mygreen}{RGB}{30, 176, 37}

\definecolor{myColor}{rgb}{0.50,0.70,0.10}%
\definecolor{myColor2}{rgb}{0.50,0.20,0.80}%
%\definecolor{myColor}{rgb}{1.00,0.00,0.10}%
%\definecolor{myGreen}{rgb}{0.00,0.60,0.00}%
%\definecolor{myOrange}{rgb}{1.00,0.40,0.00}%
%\definecolor{myOtherOrange}{rgb}{0.90,0.50,0.00}%
%\definecolor{myBlue}{rgb}{0.20,0.20,0.70}%
%\definecolor{myGray}{rgb}{0.84,0.84,0.94}%
%\newcommand{\green}[1]{\textcolor{myGreen}{#1}}
%\newcommand{\blue}[1]{\textcolor{myBlue}{#1}} 
%\newcommand{\red}[1]{\textcolor{red}{#1}} 
%\newcommand{\marron}[1]{\textcolor{brown!65!black!80!white}{#1}}
%\newcommand{\myorange}[1]{\textcolor{myOrange}{#1}}
%\newcommand{\myotherorange}[1]{\textcolor{myOtherOrange}{#1}}
\newcommand{\mycolor}[1]{\textcolor{myColor}{#1}}
\newcommand{\mycolord}[1]{\textcolor{myColor2}{#1}}



\newcommand{\hastype}{\mathop{:}}

\newcommand{\dand}{\mathbin{\overline{\land}}}
\newcommand{\dnot}{\mathop{\overline{\lnot}}}
\newcommand{\dimpl}{\mathbin{\overline{\to}}}
\newcommand{\dexists}{\mathop{\overline{\exists}}}
\newcommand{\dforall}{\mathop{\overline{\forall}}}

\newcommand{\hsbind}{\mathbin{\texttt{>>=}}}
\newcommand{\hsrevbind}{\mathbin{\texttt{=<<}}}
\newcommand{\hsseq}{\mathbin{\texttt{>>}}}
\newcommand{\occons}{\mathbin{::}}

\newcommand{\statecps}[3]{\llbracket #3 \rrbracket^{#2}_{#1}}
\newcommand{\cps}[2]{\llbracket #2 \rrbracket^{#1}}

\newcommand{\sem}[1]{\llbracket #1 \rrbracket}
\newcommand{\intens}[1]{\overline{#1}}

\newcommand{\obj}[1]{\text{Obj}(#1)}
\newcommand{\inl}[1]{\text{inl}(#1)}
\newcommand{\inr}[1]{\text{inr}(#1)}
\newcommand{\id}[1]{\text{id}_{#1}}

\newcommand{\keyword}[1]{\texttt{#1}}
\newcommand{\effect}[1]{\textbf{#1}}
\newcommand{\semdom}[1]{\textbf{#1}}
\newcommand{\handle}[2]{\keyword{with}\ #1\ \keyword{handle}\ #2}


\newcommand{\highlight}[2]{\colorbox{#1}{$\displaystyle #2$}}




%% \setbeamercolor{block title}{fg=ngreen,bg=white} % Colors of the block titles
%% \setbeamercolor{block body}{fg=black,bg=white} % Colors of the body of blocks
%% \setbeamercolor{block alerted title}{fg=white,bg=dblue!70} % Colors of the highlighted block titles
%% \setbeamercolor{block alerted body}{fg=black,bg=dblue!10} % Colors of the body of highlighted blocks
%% % Many more colors are available for use in beamerthemeconfposter.sty












\title{Effects and Handlers \\ in Natural Language Semantics}
\author{Jirka Maršík}
\institute[LORIA, Université de Lorraine, Inria]
{
Équipe Sémagramme
\\
LORIA, UMR 7503, Université de Lorraine, CNRS, Inria, Campus Scientifique, \\
F-54506 Vand\oe uvre-lès-Nancy, France
}
\date{June 7, 2016}







\begin{document}

\begin{frame}
  \titlepage
\end{frame}

\begin{frame}
  \tableofcontents
\end{frame}

\section{New Results}

\begin{frame}{Formal Definition of a Calculus}
  \begin{itemize}
  \item syntax and notation
  \item type system
  \item reduction semantics
  \item basic combinators
  \end{itemize}
\end{frame}

\begin{frame}{Proof of Fundamental Properties}
  \begin{itemize}
  \item Confluence
  \item Termination
  \end{itemize}
\end{frame}

\begin{frame}{Connection to Continuations}
  \begin{itemize}
  \item simulating call-by-value lambda calculus
  \item adding delimited continuations to the simulation
  \item adapting an existing type system of delimited continuations\ldots
  \item and fitting it to the simulation
  \end{itemize}
\end{frame}

\section{Thesis Plan}

\begin{frame}{Timeline}
  Content-complete manuscript:
  \begin{itemize}
  \item end of July
  \end{itemize}
  
  Final version of manuscript:
  \begin{itemize}
  \item end of August
  \end{itemize}
  
  Defense:
  \begin{itemize}
  \item end of October
  \end{itemize}
\end{frame}

\begin{frame}{Part I --- Calculus for Effects and Handlers}
  Chapters:
  \begin{itemize}
  \item Definitions \cmark
  \item Examples \cmark
  \item Properties \ldots
  \item Continuations \cmark
  \end{itemize}
\end{frame}

\begin{frame}{Chapter --- Definitions}
  \begin{itemize}
  \item Sketching Out the Calculus \cmark
  \item Terms \cmark
  \item Types and Typing Rules \cmark
  \item Reduction Rules \cmark
  \item Sums and Products \cmark
  \item Common Combinators \cmark
  \end{itemize}
\end{frame}

\begin{frame}{Chapter --- Examples}
  \begin{itemize}
  \item Introducing Our Running Example \cmark
  \item Adding Errors \cmark
  \item Enriching the Context with Variables \cmark
  \item Summary \cmark
  \end{itemize}
\end{frame}

\begin{frame}{Chapter --- Properties}
  \begin{itemize}
  \item Algebraic Properties \ldots
  \item Subject Reduction \ldots
  \item Confluence \cmark
  \item Termination \ldots
  \end{itemize}
\end{frame}

\begin{frame}{Chapter --- Continuations}
  \begin{itemize}
  \item Introducing Call-by-Value \cmark
  \item Simulating Call-by-Value \cmark
  \item Introducing Control Operators \cmark
  \item Simulating $\texttt{shift0}$ and $\texttt{reset0}$ \cmark
  \item Turning to $\texttt{shift}$ and $\texttt{reset}$ \cmark
  \item Considering Types \cmark
  \item Other Control Operators \ldots
  \end{itemize}
\end{frame}

\begin{frame}{Part II --- Effects and Handlers in Natural Language}
  Chapters:
  \begin{itemize}
  \item Introduction to Formal Semantics \ldots
  \item Introducing the Effects \ldots
  \item Composing the Effects \ldots
  \end{itemize}
\end{frame}

\begin{frame}{Chapter --- Introduction to Formal Semantics}
  \begin{itemize}
  \item Montague Semantics
  \item Abstract Categorial Grammars
  \item Type-Theoretical Dynamic Logic
  \end{itemize}
\end{frame}

\begin{frame}{Chapter --- Introducing the Effects}
  \begin{itemize}
  \item Deixis
  \item Quantifiers
  \item Dynamics
  \item Presuppositions
  \item Conventional Implicatures
  \item Double Negation
  \end{itemize}
\end{frame}

\begin{frame}{Chapter --- Composing the Effects}
  \begin{itemize}
  \item Extending Fragmetns with ACG Transformers
  \item The Common Base
  \item ACG Transformers Introducing Effects
  \item Evaluating the Resulting Fragment
  \end{itemize}
\end{frame}

\section{Publications}

\begin{frame}{Formal Grammar}
\begin{center}
  Jirka Maršík and Maxime Amblard \\
  Introducing a Calculus of Effects and Handlers \\
  for Natural Language Semantics \\
  Formal Grammar
\end{center}
\end{frame}

\begin{frame}{SIGdial}
\begin{center}
  Stéphane Tiv, Jirka Maršík and Maxime Amblard \\
  Modeling Dialogue with Type-Theoretical Continuation Semantics \\
  17th Annual SIGdial Meeting on Discourse and Dialogue
\end{center}
\end{frame}

\section{Courses}

\begin{frame}{Professional Modules}
  Before:
  \begin{itemize}
  \item Prévention et secours civiques niveau 1
  \item Formation à la communication orale et corporelle en milieu
    professionnel 
  \item Résidentiel DCCE 1ère année
  \item Résidentiel DCCE 2eme année
  \end{itemize}
  
  This Year:
  \begin{itemize}
  \item Insertion professionnelle et techniques de recherche d'emploi 
  \end{itemize}
\end{frame}

\begin{frame}{Scientific Modules}
  Before:
  \begin{itemize}
  \item ESSLLI 2013
  \item ESSLLI 2014
  \item ESSLLI 2015
  \item ED IAEM --- Journée d'automne, 15 octobre 2014
  \item Lambda Calculus
  \end{itemize}
  
  This Year:
  \begin{itemize}
  \item ESSLLI 2016
  \item ED IAEM --- Journée d'automne, 15 octobree 2015
  \end{itemize}
\end{frame}

\section{After PhD}

\begin{frame}{The Next Step in My Career}
  From the Void
  
  Role:
  \begin{itemize}
  \item head of digital development
  \item developing mobile and desktop applications
  \end{itemize}

  Perks:
  \begin{itemize}
  \item start up (autonomy, risk taking)
  \item functional programming (methods fresh from academia)
  \item challenging domain (real-time computing, complex interactions)
  \end{itemize}
\end{frame}

\end{document}
