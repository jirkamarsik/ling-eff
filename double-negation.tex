\documentclass[a4paper,11pt,DIV=12]{scrartcl}

\usepackage[utf8]{inputenc}
\usepackage{listings}
\usepackage{color}
\usepackage{amsmath}
\usepackage{stmaryrd}
\usepackage{verbatim}
\usepackage{graphicx}
\usepackage{gb4e}
\usepackage{bussproofs}


\newcommand{\dand}{\mathbin{\overline{\land}}}
\newcommand{\dnot}{\mathop{\overline{\lnot}}}
\newcommand{\dimpl}{\mathbin{\overline{\to}}}
\newcommand{\dexists}{\mathop{\overline{\exists}}}
\newcommand{\dforall}{\mathop{\overline{\forall}}}

\newcommand{\hsbind}{\mathbin{\texttt{>>=}}}
\newcommand{\hsrevbind}{\mathbin{\texttt{=<<}}}
\newcommand{\hsseq}{\mathbin{\texttt{>>}}}
\newcommand{\occons}{\mathbin{::}}

\newcommand{\statecps}[3]{\llbracket #3 \rrbracket^{#2}_{#1}}
\newcommand{\cps}[2]{\llbracket #2 \rrbracket^{#1}}

\newcommand{\sem}[1]{\llbracket #1 \rrbracket}
\newcommand{\intens}[1]{\overline{#1}}

\newcommand{\keyword}[1]{\texttt{#1}}
\newcommand{\effect}[1]{\textbf{#1}}
\newcommand{\semdom}[1]{\textbf{#1}}
\newcommand{\handle}[2]{\keyword{with}\ #1\ \keyword{handle}\ #2}

\def\limp {\mathbin{{-}\mkern-3.5mu{\circ}}}


\title{Double Negation with Effects}
\date{}

\begin{document}

\maketitle

\begin{description}
  \item[The pragmatic effects hypothesis:] The integration of pragmatic
    phenomena into semantics can be adequately (when compared to existing
    theories) modelled using a theory of computational side effects, namely
    using effects and handlers.

  \item[The general effects hypothesis:] The type generalizations proposed
    in formal semantics of natural language (for in-situ quantification,
    interrogatives, focus\ldots) can be replaced with the introduction of
    effects and handlers in some restricted set of lexical items.
\end{description}

\section{Introduction}

In this document, I will examine the case of the general effects hypothesis
with respect to the type generalization introduced in
\cite{saiaccessibilite} to handle double negation.

Qian's proposal takes de Groote's type of dynamic proposition $\Omega =
\gamma \to (\gamma \to o) \to o$ and replaces it with the type $\Omega^{dn}
= \Omega \times \Omega$. ...

We will look at two standard monads which are used to describe side effects
and which fit the type generalization proposed here and we will see that
their semantics are not fit for the purposes of our theory. To evaluate the
fitness of a monad w.r.t. our theory, we will first have to pin down the
crux of the argument of using side effects in semantics.

\subsection{Raising Functions}

Monads are a special case of (lax) monoidal functors (i.e. applicative
functors), which are in turn a special case of functors. The latter gives
us an operation $\texttt{fmap} : (\alpha \to \beta) \to F \alpha \to F
\beta$ while the former gives us $\otimes : F \alpha \to F \beta \to F
(\alpha \times \beta)$, both of which satisfy certain
laws\footnote{$\texttt{fmap}$ is homomorphic w.r.t. to composition and
  identities, $\otimes$ satisfies the monoid laws.}.

Intuitively, $\texttt{fmap}$ tells us how to map a function over the
contents of some type construction, thereby ``upgrading'' a function over
simple values to a function over the more elaborate values. The $\otimes$
operator tells us how to combine the extra material introduced by the type
construction from multiple sources. Together, these two function let us
lift functions of arbitrary arity from simple values to the more elaborate
ones (such as in $(\tau_1 \to \ldots \to \tau_n) \to (F \tau_1 \to \ldots
\to F \tau_n)$).

The central premise of the side-effects-in-linguistics program is that when
more complicated types of denotations are introduced for the sake of some
lexical items (e.g., generalized quantifiers for QNPs, dynamic propositions
for anaphora, \ldots), raising the types of existing
denotations \footnote{Lexical items that play a vital role in the
  phenomenon covered by the new denotation type are an exception since
  their existing description given in the simpler types lacked an account
  of their interaction with this new phenomenon. For example, in the case
  of quantification, if we assume that tensed clauses form scope islands,
  the denotation of tensed verbs will need to change so that it will
  actually stop the scope of the quantifier. However, all the other
  denotations whose lexical items do not interact with the phenomenon
  (e.g., prepositions) are only raised to the new type without any other
  change.} using the combinator described above \footnote{Note that the
  raising combinator depends not only on the type construction itself but
  also on a specific monad which gives rise to the $\texttt{fmap}$ and
  $\otimes$ operators it is based on.} produces a valid solution. The extra
kicker of the general effects hypothesis is the use of effects and
handlers, which rely on a single parameterized monad, so that existing
denotations need not be repeatedly raised and stay stable.

The stability condition outlined above has been partly validated by several
researchers: Barker treated quantification as a side-effect and Lebedeva
did the same for presuppositions; Marlow and Champollion have been using
monads for anaphora. In my work, I have ...

\section{Applying Monads}

Let us look at some of the common monads whose type constructions could
explain the type generalization from $\Omega$ to $\Omega \times \Omega$.

\subsection{Reader}

Let us consider the reader monad. We will focus only on the types it
uses and the its functor/applicative functor operations.

\begin{align*}
  F \alpha &= \gamma \to \alpha \\
  \texttt{pure} &: \alpha \to F \alpha \\
  \texttt{pure}\ x &= \lambda e. x \\
  \texttt{fmap}\ f\ m &= \lambda e. f\ (m\ e) \\
  m_1 \otimes m_2 &= \lambda e. \left<m_1\ e, m_2\ e\right>
\end{align*}

We can see that the type of $F \alpha$ corresponds to a family of values of
type $\alpha$ indexed by values of some type $\gamma$. How does this relate
to the type of $\Omega \times \Omega$? We can see it as a family of two
values of type $\Omega$. The index then corresponds to \emph{polarity} and
lets us choose between the positive and negative version of a value
(proposition).

However, if we look beyond the types at the definition of some of the
operations of the monad, we see that the similarities do not go
far. Raising a simple value consists of taking a family 

\subsection{Writer}

We can take the writer monad that takes the type 

\bibliography{refs}
\bibliographystyle{plain}

\end{document}

