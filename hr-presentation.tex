\documentclass{beamer}

\usepackage[utf8]{inputenc}
\usepackage{pgfpages}
\usepackage{stmaryrd}
\usepackage{amsmath}
\usepackage{graphicx}
\usepackage{tikz}
\usepackage{listings}
\usepackage{xcolor}
\usepackage[normalem]{ulem}
\usepackage{appendixnumberbeamer}
\usepackage{url}

\usepackage[export]{adjustbox}
\setbeamercovered{transparent}

\hypersetup{pdfstartview={Fit}}
\def\limp {\mathbin{{-}\mkern-3.5mu{\circ}}}

\usepackage{pifont}% http://ctan.org/pkg/pifont
\newcommand{\cmark}{\ding{51}}%
\newcommand{\xmark}{\ding{55}}%




\setbeamertemplate{navigation symbols}{}
\setbeamertemplate{footline}
{\hfill {\normalsize \insertframenumber{}/\inserttotalframenumber{}}}

\definecolor{mygreen}{RGB}{30, 176, 37}

\definecolor{myColor}{rgb}{0.50,0.70,0.10}%
\definecolor{myColor2}{rgb}{0.50,0.20,0.80}%
%\definecolor{myColor}{rgb}{1.00,0.00,0.10}%
%\definecolor{myGreen}{rgb}{0.00,0.60,0.00}%
%\definecolor{myOrange}{rgb}{1.00,0.40,0.00}%
%\definecolor{myOtherOrange}{rgb}{0.90,0.50,0.00}%
%\definecolor{myBlue}{rgb}{0.20,0.20,0.70}%
%\definecolor{myGray}{rgb}{0.84,0.84,0.94}%
%\newcommand{\green}[1]{\textcolor{myGreen}{#1}}
%\newcommand{\blue}[1]{\textcolor{myBlue}{#1}} 
%\newcommand{\red}[1]{\textcolor{red}{#1}} 
%\newcommand{\marron}[1]{\textcolor{brown!65!black!80!white}{#1}}
%\newcommand{\myorange}[1]{\textcolor{myOrange}{#1}}
%\newcommand{\myotherorange}[1]{\textcolor{myOtherOrange}{#1}}
\newcommand{\mycolor}[1]{\textcolor{myColor}{#1}}
\newcommand{\mycolord}[1]{\textcolor{myColor2}{#1}}



\newcommand{\hastype}{\mathop{:}}

\newcommand{\dand}{\mathbin{\overline{\land}}}
\newcommand{\dnot}{\mathop{\overline{\lnot}}}
\newcommand{\dimpl}{\mathbin{\overline{\to}}}
\newcommand{\dexists}{\mathop{\overline{\exists}}}
\newcommand{\dforall}{\mathop{\overline{\forall}}}

\newcommand{\hsbind}{\mathbin{\texttt{>>=}}}
\newcommand{\hsrevbind}{\mathbin{\texttt{=<<}}}
\newcommand{\hsseq}{\mathbin{\texttt{>>}}}
\newcommand{\occons}{\mathbin{::}}

\newcommand{\statecps}[3]{\llbracket #3 \rrbracket^{#2}_{#1}}
\newcommand{\cps}[2]{\llbracket #2 \rrbracket^{#1}}

\newcommand{\sem}[1]{\llbracket #1 \rrbracket}
\newcommand{\intens}[1]{\overline{#1}}

\newcommand{\obj}[1]{\text{Obj}(#1)}
\newcommand{\inl}[1]{\text{inl}(#1)}
\newcommand{\inr}[1]{\text{inr}(#1)}
\newcommand{\id}[1]{\text{id}_{#1}}

\newcommand{\keyword}[1]{\texttt{#1}}
\newcommand{\effect}[1]{\textbf{#1}}
\newcommand{\semdom}[1]{\textbf{#1}}
\newcommand{\handle}[2]{\keyword{with}\ #1\ \keyword{handle}\ #2}


\newcommand{\highlight}[2]{\colorbox{#1}{$\displaystyle #2$}}

\newcommand{\includepicturescale}[2]{
    \includegraphics[scale=#2]{dias/#1.eps}
}

\newcommand{\includepicturem}[1]{
    \includegraphics[scale=0.48]{dias/#1.eps}
}




%% \setbeamercolor{block title}{fg=ngreen,bg=white} % Colors of the block titles
%% \setbeamercolor{block body}{fg=black,bg=white} % Colors of the body of blocks
%% \setbeamercolor{block alerted title}{fg=white,bg=dblue!70} % Colors of the highlighted block titles
%% \setbeamercolor{block alerted body}{fg=black,bg=dblue!10} % Colors of the body of highlighted blocks
%% % Many more colors are available for use in beamerthemeconfposter.sty












\title{Side Effects in Natural Language}
\author{Jirka Maršík}
\institute[LORIA, Université de Lorraine, Inria]
{
Équipe Sémagramme
\\
LORIA, UMR 7503, Université de Lorraine, CNRS, Inria, Campus Scientifique, \\
F-54506 Vand\oe uvre-lès-Nancy, France
}
\date{May 26, 2015}







\begin{document}

\begin{frame}
\titlepage
\end{frame}


\begin{frame}{Our Setting}
Context:
\begin{itemize}
\item denotational semantics of a natural language
\item using the $\lambda$-calculus
\end{itemize}
\vfill

Objective:
\begin{itemize}
\item cover more complicated fragments of language
\end{itemize}
\vfill

Challenge:
\begin{itemize}
\item divergent research
\begin{itemize}
\item same start, different questions $\rightarrow$ different developments
\item need a unified framework to integrate existing solutions
\end{itemize}
\end{itemize}
\end{frame}

\begin{frame}[fragile]{What is Denotational Semantics?}
\begin{columns}
\begin{column}{0.3\textwidth}
  \begin{block}{Code}
\begin{lstlisting}
if (x >= 0) {
  x--;
} else {
  neg = true;
  x = 0;
}
\end{lstlisting}
  \end{block}
\end{column}
\begin{column}{0.7\textwidth}
 \begin{block}{Abstract Syntax Tree}
  \includegraphics[width=\textwidth]{AST.pdf}
 \end{block}
\end{column}
\end{columns}
\end{frame}

\begin{frame}{This is Denotational Semantics!}
\vspace{-7mm}
\begin{align*}
\sem{exp_{\tau}} &= \gamma \to \sem{\tau} \times \gamma \\
\sem{var} &= symbol \\
\\ \pause
\texttt{0} &: exp_{int} \\
\texttt{true} &: exp_{bool} \\
\sem{\texttt{0}} &= \lambda s. (0, s)  \\
\sem{\texttt{true}} &= \lambda s. (\top, s) \\
\\ \pause
\texttt{x}, \texttt{neg} &: var \\
var &: var \to exp_{\tau} \\
= &: var \to \exp_{\tau} \to \exp_{\tau} \\
\sem{\texttt{x}} &= "x" \\
\sem{\texttt{neg}} &= "neg" \\
\sem{var} &= \lambda v. \lambda s. (s[v], s) \\
\sem{\texttt{=}} &= \lambda v e. \lambda s_1. \text{let}\ (r, s_2) = e\ s_1\ \text{in}\ (r, s_2[v := r])
\end{align*}
\end{frame}

\begin{frame}{What about Natural Language?}
\begin{center}
\begin{quotation}
  There is in my opinion no important theoretical difference between
  natural languages and the artifical languages of logicians; [\ldots]

  \raggedleft--- \textup{Richard Montague}, Universal Grammar (1970)
\end{quotation}
\end{center}
\end{frame}

\begin{frame}{What about Natural Language?}
\begin{columns}
\begin{column}{0.5\textwidth}
\begin{block}{Sentence}
  Every driver owns a car.
\end{block}
\end{column}
\begin{column}{0.5\textwidth}
  \begin{block}{Syntactic Tree}
  \includegraphics[width=\textwidth]{NLAST.pdf}
  \end{block}
\end{column}
\end{columns}
\end{frame}

\begin{frame}{Yes, We Can!}
\vspace{-5mm}
\begin{align*}
\textsc{driver}, \textsc{car} &: n \\
\sem{n} &= \iota \to o \\
\sem{\textsc{driver}} &= \lambda x. \semdom{driver}(x) \\
\sem{\textsc{car}} &= \lambda x. \semdom{car}(x) \\
\\ \pause
\textsc{every}, \textsc{a} &: n \to np \\
\sem{np} &= (\iota \to o) \to o \\
\sem{\textsc{every}} &= \lambda n. \lambda P. \forall x. (n\ x) \rightarrow (P\ x) \\
\sem{\textsc{a}} &= \lambda n. \lambda P. \exists x. (n\ x) \land (P\ x) \\
\\ \pause
\textsc{bought} &: np \to np \to s \\
\sem{s} &= o \\
\sem{\textsc{bought}} &= \lambda S O. S (\lambda x. O (\lambda y. \semdom{bought}(x, y))) \\
\pause
\sem{\text{Every driver bought a car}} &= \forall x. \semdom{driver}(x) \rightarrow \exists y. \semdom{car}(y) \land \semdom{bought}(x, y)
\end{align*}
\end{frame}


\begin{frame}{Beyond Single Sentences}
\framesubtitle{de Groote -- Type-Theoretic Dynamic Logic}
\begin{columns}
\begin{column}{0.5\textwidth}
  \begin{block}{Montague}
    \begin{align*}
      \sem{s} &= o \\
      \sem{n} &= \iota \to \sem{s} \\
      \sem{np} &= (\iota \to \sem{s}) \to \sem{s}
    \end{align*}
  \end{block}
\end{column}
\begin{column}{0.5\textwidth}
  \begin{block}{de Groote}
     \begin{align*}
      \sem{s} &= \gamma \to (\gamma \to o) \to o \\
      \sem{n} &= \iota \to \sem{s} \\
      \sem{np} &= (\iota \to \sem{s}) \to \sem{s}
    \end{align*}
  \end{block}
\end{column}
\end{columns}
\vfill
\pause
$$
\sem{\textit{He bought a car}} = \lambda e \phi.\ \exists
x. car(x)\ \land\ bought(\texttt{sel}_{he}(e), x)\ \land\ \phi (x \occons e)
$$
\end{frame}


\begin{frame}
  \begin{center}
    \begin{quotation}
      There is in my opinion no important theoretical difference between
      natural languages and the \sout{artificial} \textbf{programming}
      languages of \sout{logicians} \textbf{computer scientists}; [\ldots]

      \raggedleft \textup{Jirka Maršík}, PhD Thesis (2016)
    \end{quotation}
  \end{center}
\end{frame}


\begin{frame}{Side Effects and Pragmatics}
  \begin{columns}
    \begin{column}{0.5\textwidth}
   \begin{block}{Side Effects}
   Account for:
  \pause
  \vfill
  \begin{itemize}
  \item a program's interaction with the world of its users
    \begin{itemize}
    \item e.g., makings sounds, printing documents, moving robotic limbs\ldots
    \end{itemize}
  \end{itemize}
  \pause
  \vfill
  \begin{itemize}
  \item non-local interactions between parts of a program
    \begin{itemize}
    \item e.g., writing to and reading from variables, throwing and
      catching exceptions\ldots
    \end{itemize}
  \end{itemize}
  \end{block}
   \pause
    \end{column}
    \begin{column}{0.5\textwidth}
      \begin{block}{Pragmatics}
        Account for:
        \pause
        \vfill
        \begin{itemize}
        \item how language fits into the world of its users
          \begin{itemize}
          \item e.g., asking a colleague to print some documents\ldots
          \end{itemize}
        \pause \vfill
        \item phenomena that transcend the bounds of an isolated sentence
          \begin{itemize}
          \item e.g., the link between antecedent and pronoun, making and
            cancelling presuppositions\ldots
          \end{itemize}
        \end{itemize}
      \end{block}
    \end{column}
  \end{columns}
  \vfill
  \pause
  \alert{Side effects align with pragmatics in their purpose.}
\end{frame}

\begin{frame}{Type Raising}
  Side effects and pragmatics align also in their theories.
  \pause
  \vfill
  Most famous example: Montague's generalized quantifiers
  \begin{itemize}
  \item $\sem{np} = (\iota \to o) \to o$
  \item e.g., the meaning of a transitive verb such as \textit{bought}
    $$ \sem{\textsc{bought}} = \lambda S O. S (\lambda x. O (\lambda y. \semdom{bought}(x, y))) $$
  \end{itemize}
  \vfill
  \pause
  In computer science, discovered as continuations
  \begin{itemize}
  \item $\sem{exp_{\tau}} = (\tau \to \omega) \to \omega$
  \item e.g., applying a function $f$ to two arguments $S$ and $O$ in
    continuation-passing style
    $$ \lambda P. S (\lambda x. O (\lambda y. P\ (f\ x\ y))) $$
  \end{itemize}
\end{frame}

\begin{frame}{Generalizing Denotations}
  ``Upgrading'' the types of denotations in order to keep a compositional
  semantics seems like a common strategy.
  \vfill
  \begin{tabular}{llr}
    Natural Languages & Prog. Languages & Type $\alpha$ becomes \\
    \hline
    Quantification & Control &
    $(\alpha \to \omega) \to \omega$ \\
    Anaphora & State &
    $\gamma \to \alpha \times \gamma$ \\
    Intensionality & Environment &
    $\delta \to \alpha$ \\
    Presuppositions & Exceptions &
    $\alpha \oplus \chi$ \\
    Questions & Non-determinism &
    $\alpha \to o$ \\
    Focus & &
    $\alpha \times (\alpha \to o)$ \\
    Expressives & Output &
    $\alpha \times \epsilon$ \\
    Prob. semantics & Prob. programming &
    $[\mathbb{R} \times \alpha]$ \\
  \end{tabular}
\end{frame}

\begin{frame}{How to Avoid Changing Denotations?}
  Different pragmasemantic phenomena, all in one theory \\
    $\rightarrow$ more and more elaborate types
  \vfill
  \pause
  We often have to change our minds on what is meaning
  \begin{itemize}
  \item old denotations $\rightarrow$ outdated
  \item denotations from other strands of work $\rightarrow$ incompatible
  \end{itemize}
  \pause
  \vfill
  Some solutions to this problem exist already in computer science.
\end{frame}


\begin{frame}{Our Hypotheses}
  
  \begin{block}{The pragmatic effects hypothesis:}
  The integration of pragmatic phenomena into semantics can be adequately
  (when compared to existing theories) modelled using a theory of
  computational side effects.
  \end{block}
  \vfill
  \begin{block}{The general effects hypothesis:}
  The type generalizations proposed in formal semantics of natural language
  (in-situ quantification, interrogatives, alternative semantics\ldots) can
  be modelled using a theory of computational side effects.
  \end{block}
\end{frame}



\newcommand{\includepicture}[1]{
    \includegraphics[scale=0.4]{dias/#1.eps}
}
\begin{frame}{Algebraic Effects\ldots}

  I have been using a framework developed in PL research.

  \vfill
  
  In it:
  \begin{itemize}
  \item interacting with the context $=$ throwing an exception
  \item the exception contains a response for every possible outcome of the
    operation
  \end{itemize}
 
  \pause
  \vfill
  
  \begin{columns}
    \begin{column}{0.5\textwidth}
   Denotations are:
  \begin{itemize}
  \item algebraic expressions (drawn as trees)
  \item \textcolor{mygreen}{generators} $=$ values
  \item \textcolor{red}{operators} $=$ possible interactions with the context
  \item arity $=$ the number of possible outcomes
  \item type $= \mathcal{F}_{\textcolor{red}{\Sigma}}(\textcolor{mygreen}{\tau})$
  \end{itemize}
    \end{column}
    \begin{column}{0.5\textwidth}
      \only<2>{\includepicturescale{example}{0.4}}
    \end{column}
  \end{columns}
\end{frame}

\begin{frame}{\ldots and Handlers}
  Handlers give scope and interpretation to (some of) the effects in a
  computation.

  \begin{itemize}
  \item Practically, they are like exception handlers in programming languages.
  \item Technically, they are catamorphisms (folds) on the algebra of effects.
  \end{itemize}
  

  \pause
  \vfill 

  Examples from Computer Science:
  \begin{itemize}
  \item exception handlers and resumable conditions
  \item STM transactions
  \item redirecting stdin/stdout to strings or files
  \end{itemize}

  \vfill
  \pause
  
  Examples from Linguistics:
  \begin{itemize}
  \item hypotheseses can cancel presuppositions in their scope
  \item presuppositions are accomodated if they have not been cancelled
  \end{itemize}
\end{frame}

\begin{frame}{Example from Computer Science}
  \begin{adjustbox}{center}
   \begin{tabular}{|c|c|c|} \hline
    \textbf{3} & \visible<2->{\textbf{x + 3}} & \visible<3->{\textbf{print("hello")}} \\ \hline
    $3$ & & \\ \hline
    \visible<2->{$\lambda s. \left<3, s\right>$} & \visible<2->{$\lambda s. \left<s("x") + 3, s\right>$} & \\ \hline
    \visible<3->{$\lambda s. \left<3, s, ""\right>$} & \visible<3->{$\lambda s. \left<s("x") + 3, s, ""\right>$} & \visible<3->{$\lambda s. \left<(), s, "hello"\right>$} \\ \hline
    \visible<4->{\includepicture{3}} & & \\ \hline
    \visible<5->{\includepicture{3}} & \visible<5->{\includepicture{x+3}} & \\ \hline
    \visible<6->{\includepicture{3}} & \visible<6->{\includepicture{x+3}} & \visible<6->{\includepicture{print}} \\ \hline
  \end{tabular}
  \end{adjustbox}
\end{frame}



\begin{frame}{Proof of Concept}
  \vfill

  I have used Eff\footnote{\url{http://www.eff-lang.org /}} to build a
  small prototype to test and explore this approach.
  
  \vfill
  \begin{itemize}
  \item in-situ quantification
    \begin{itemize}
    \item \textit{``Every driver bought a car.''}
    \end{itemize}
  \item discourse anaphora
    \begin{itemize}
    \item \textit{``John bought a car. He likes it.''}
    \end{itemize}
  \item presuppositions (of referentials)
    \begin{itemize}
    \item \textit{``John doesn't like my car.''}
    \end{itemize}
  \item their interactions
    \begin{itemize}
    \item e.g., binding problem
    \item \textit{``Every driver sold his car.''}
    \end{itemize}
  \vfill
  \pause
  \item \mycolord{double negation}
    \begin{itemize}
    \item \textit{``Either John doesn't have a car or he keeps it in a
      funny place.''}
    \end{itemize}
  \end{itemize}
\end{frame}



\begin{frame}{Double Negation as an Effect}
  $$\textcolor{red}{\texttt{get\_polarity}} : 1 \to 2$$
  \vfill
  \begin{center}
  \includepicturescale{double_negation}{0.4}
  \end{center}
\end{frame}

\begin{frame}{Stability Jeopardized}
  $$X \dand Y = X \hsbind \left(\lambda x. Y \hsbind \left(\lambda y. \begin{array}{l}\includepicturescale{conj}{0.5}\end{array}\right)\right)$$
  \vfill
  \begin{adjustbox}{center}
  $$\begin{array}{l}\includepicturescale{double_negation}{0.27}\end{array} \dand \begin{array}{l}\includepicturescale{A}{0.27}\end{array} = \begin{array}{l}\includepicturescale{double_negation_A}{0.27}\end{array}$$
  \end{adjustbox}
\end{frame}

\begin{frame}{Stability Jeopardized}
  $$X \dand_{DN} Y = (\texttt{unpolarize}\ X) \dand (\texttt{unpolarize}\ Y)$$ 
  \vfill
  \begin{adjustbox}{center}
  $$\begin{array}{l}\includepicturescale{double_negation}{0.33}\end{array} \dand_{DN} \begin{array}{l}\includepicturescale{A}{0.33}\end{array} = \begin{array}{l}\includepicturescale{negation}{0.33}\end{array}$$
  \end{adjustbox}
\end{frame}

\begin{frame}{Why is Double Negation Problematic?}
  \framesubtitle{These are monads\ldots}

  \begin{align*}
    \mathcal{F} &: \text{Type} \to \text{Type} \\
    \eta &: \alpha \to \mathcal{F}(\alpha) \\
    (\hsbind) &: \mathcal{F}(\alpha) \to (\alpha \to \mathcal{F}(\beta)) \to \mathcal{F}(\beta)
  \end{align*}
  \vfill
  \pause
  \begin{align*}
    (\kleislicomp) &: (\alpha \to \mathcal{F}(\beta)) \to (\beta \to \mathcal{F}(\gamma)) \to (\alpha \to \mathcal{F}(\gamma)) \\
    f \kleislicomp g &= \lambda x.\ (f\ x) \hsbind g
  \end{align*}
  \vfill
  \pause
  \begin{align*}
    \eta \kleislicomp f &= f \\
    f \kleislicomp \eta &= f \\
    f \kleislicomp (g \kleislicomp h) &= (f \kleislicomp g) \kleislicomp h
  \end{align*}
\end{frame}

\begin{frame}{Why is Double Negation Problematic?}
  \framesubtitle{\ldots but double negation\ldots}
  
  \begin{align*}
    \mathcal{F}(\alpha) &= \alpha \times \alpha \\
    \eta &= \lambda A. \left< A, \dnot A \right> \\
    (\hsbind) &= \lambda M f. f (\pi_1 M)
  \end{align*}
  \vfill
  \pause
  \begin{align*}
    \texttt{raise}_{\alpha \to \beta \to \gamma} &: (\alpha \to \beta \to \gamma) \to (\mathcal{F}(\alpha) \to \mathcal{F}(\beta) \to \mathcal{F}(\gamma)) \\
    \texttt{raise}_{\alpha \to \beta \to \gamma} &= \lambda f X Y.\ X \hsbind (\lambda x. Y \hsbind (\lambda y. \eta\ (f\ x\ y))) \\
    \texttt{raise}_{\alpha \to \beta \to \gamma}(\dand) &= \lambda X Y. \left< (\pi_1X) \dand (\pi_1Y), \dnot ((\pi_1 X) \dand (\pi_1 Y)) \right>
  \end{align*}
\end{frame}

\begin{frame}{Why is Double Negation Problematic?}
  \framesubtitle{\ldots is not a monad!}

  \begin{align*}
    f \kleislicomp \eta &= \lambda x. \left< \pi_1 (f\ x), \dnot (\pi_1 (f
    \ x))\right> \\
    f &= \lambda x. \left< \pi_1 (f\ x), \pi_2 (f\ x) \right>
  \end{align*}
  \vfill
  \pause
  \begin{align*}
    f \kleislicomp \eta &\neq f
  \end{align*}
  \vfill
  \pause
  \begin{align*}
    \mathcal{F} &: (\alpha \to \beta) \to (\mathcal{F}(\alpha) \to \mathcal{F}(\beta)) \\
    \mathcal{F} &= \lambda f M. \left< f (\pi_1 M), \dnot (f (\pi_1 M)) \right> \\
    \mathcal{F}(\texttt{id}_{\Omega}) &= \lambda M. \left< \pi_1 M, \dnot (\pi_1 M) \right> \\
    \mathcal{F}(\texttt{id}_{\Omega}) &\neq \texttt{id}_{\mathcal{F}(\Omega)}
  \end{align*}
\end{frame}

\begin{frame}{Effect on our Hypotheses}
  \begin{block}{The pragmatic effects hypothesis:}
  The integration of pragmatic phenomena into semantics can be adequately
  (when compared to existing theories) modelled using a theory of
  computational side effects.
  
  \uncover<2->{\begin{flushright}\textcolor{green}{\cmark}\end{flushright}}
  \end{block}
  \vfill
  \begin{block}{The general effects hypothesis:}
  The type generalizations proposed in formal semantics of natural language
  (in-situ quantification, interrogatives, alternative semantics\ldots) can
  be modelled using a theory of computational side effects.

  \uncover<3->{\begin{flushright}\textcolor{red}{\xmark}\end{flushright}}
  \end{block}
\end{frame}





\begin{frame}{Summary}
  \begin{itemize}
  \item parallels between PLT and natural language semantics
    \begin{itemize}
    \item run deep and are not incidental
    \end{itemize}
   \vfill
   \pause
  \item a theory of effects could ease work on multiple phenomena
    \begin{itemize}
    \item our driving motivation (empirical coverage)
    \item stable denotations help avoid generalizing to the worst case
    \item captures parameters, mutable state, continuations, projections
      and their filtering/cancelling both flexibly and compositionally
      \begin{itemize}
      \item used in PLT research and functional programming too
      \end{itemize}
    \end{itemize}
    \vfill
    \pause
  \item next up:
    \begin{itemize}
    \item look for another solution to the double negation problem
    \item keep growing the system; either it breaks and will be shown
      insufficient or it will provide enough motivation for adoption in
      linguistic community
    \end{itemize}
  \end{itemize}
\end{frame}



\end{document}
